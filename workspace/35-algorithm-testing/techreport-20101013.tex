% Technical Report: Algorithm Testing

% The Clever Algorithms Project: http://www.CleverAlgorithms.com
% (c) Copyright 2010 Jason Brownlee. Some Rights Reserved. 
% This work is licensed under a Creative Commons Attribution-Noncommercial-Share Alike 2.5 Australia License.

\documentclass[a4paper, 11pt]{article}
\usepackage{tabularx}
\usepackage{booktabs}
\usepackage{url}
\usepackage[pdftex,breaklinks=true,colorlinks=true,urlcolor=blue,linkcolor=blue,citecolor=blue,]{hyperref}
\usepackage{geometry}
\usepackage[ruled, linesnumbered]{../algorithm2e}
\usepackage{listings} 
\usepackage{textcomp}
\ifx\pdfoutput\@undefined\usepackage[usenames,dvips]{color}
\else\usepackage[usenames,dvipsnames]{color}
\lstset{basicstyle=\footnotesize\ttfamily,numbers=left,numberstyle=\tiny,frame=single,columns=flexible,upquote=true,showstringspaces=false,tabsize=2,captionpos=b,breaklines=true,breakatwhitespace=true,keywordstyle=\color{blue},stringstyle=\color{ForestGreen}}
\geometry{verbose,a4paper,tmargin=25mm,bmargin=25mm,lmargin=25mm,rmargin=25mm}

% Dear template user: fill these in
\newcommand{\myreporttitle}{Algorithm Testing}
\newcommand{\myreportsubtitle}{Unit tests and writing for testability}
\newcommand{\myreportauthor}{Jason Brownlee}
\newcommand{\myreportemail}{jasonb@CleverAlgorithms.com}
\newcommand{\myreportproject}{The Clever Algorithms Project\\\url{http://www.CleverAlgorithms.com}}
\newcommand{\myreportdate}{20101013}
\newcommand{\myreportfulldate}{October 13, 2010}
\newcommand{\myreportversion}{1}
\newcommand{\myreportlicense}{\copyright\ Copyright 2010 Jason Brownlee. Some Rights Reserved. This work is licensed under a Creative Commons Attribution-Noncommercial-Share Alike 2.5 Australia License.}

% leave this alone, it's templated baby!
\title{{\myreporttitle}: {\myreportsubtitle}\footnote{\myreportlicense}}
\author{\myreportauthor\\{\myreportemail}\\\small\myreportproject}
\date{\myreportfulldate\\{\small{Technical Report: CA-TR-{\myreportdate}-\myreportversion}}}
\begin{document}
\maketitle

% write a summary sentence for each major section
\section*{Abstract} 
This is the abstract. Consider writing a one sentence summary of each major section in the report.

\begin{description}
	\item[Keywords:] {\small\texttt{Algorihtm, Test, Unit Test, Testability}}
\end{description} 

% summarise the document breakdown with cross references
\section{Introduction}
\label{sec:introduction}
% project
The Clever Algorithms project aims to describe a large number of algorithms from the fields of Computational Intelligence, Biologically Inspired Computation, and Metaheuristics in a complete, consistent and centralized manner \cite{Brownlee2010}.

\section{Background}
\label{sec:testing}

code quality, trust

\subsection{Unit Testing}
asdasd

\subsection{System Testing}
asdasd

\subsection{Testing Algorithms}
asdasd


\section{Unit Testing Example}
\label{sec:example}
This section provides an example of an algorithm and its associated unit tests as an illustration of the presented concepts. Section~\ref{subsec:algorithm} presents the Genetic Algorithm and Section~\ref{subsec:tests} presents the unit tests for the Genetic Algorithm, both written in the Ruby Programming Language.

\subsection{Algorithm}
\label{subsec:algorithm}
The source code for Genetic Algorithm was presented in a previous work that described the algorithm for the Clever Algorithms project \cite{Brownlee2010p}. The code description in Listing~\ref{ga} presents the same code although with a few bug fixes and updated to ensure the script can be run standalone and the functions can be imported by another script for testing without running the script (the \texttt{if \_\_FILE\_\_ == \$0} line). 

The Genetic Algorithm implementation is an excellent case for demonstrating unit testing methodology. The algorithm is modularized, partitioned into well-contained functions - most which are independently testable. 

The \texttt{reproduce} has some dependencies although is still testable. The \texttt{search} function is the only monolithic function, which both depends on all other functions in the implementation (directly or indirectly) and is difficult to unit test. At best the \texttt{search} function may be a case for system testing addressing functional requirements, such as ``\emph{does the algorithm deliver optimized solutions}''.

% the listing
\lstinputlisting[firstline=7,language=ruby,caption=Genetic Algorithm in the Ruby Programming Language, label=ga]{../../src/algorithms/evolutionary/genetic_algorithm.rb}

\subsection{Unit Tests}
\label{subsec:tests}
Listing~\ref{gatests} provides the \texttt{TC\_GeneticAlgorithm} class that makes use of the built-in Ruby unit testing framework by extending the \texttt{Test::Unit::TestCase} class.

The listing provides an example of 10 unit tests for 6 of the 7 functions in the Genetic Algorithm implementation. Two types of unit tests are provided: 

\begin{itemize}
	\item Tests that directly test the function in question, addressing questions such as: does \texttt{onemax} add correctly? and does \texttt{point\_mutation} behave correctly?
	\item Tests that test the probabilistic properties of the function in question, addressing questions such as: does \texttt{random\_bitstring} provide an expected 50/50 mixture of $1$'s and $0$'s over a large number of cases? and does \texttt{point\_mutation} make an expected number of changes over a large number of cases?
\end{itemize}

This second type of testing - the testing of probabilistic expectations - is a weaker form of unit testing that can be used to either provide additional confidence to directly tested functions, or to be used as a last resort where direct methods cannot be used.

% the listing
\lstinputlisting[firstline=10,language=ruby,caption=Unit Tests for the Genetic Algorithm the Ruby Programming Language, label=gatests]{../../src/algorithms/evolutionary/tests/tc_genetic_algorithm.rb}

\section{Suggestions}
\label{sec:suggestions}

\subsection{Do's}
\begin{itemize}
	\item write for testability
	\item function independence
\end{itemize}

\subsection{Do not's}
\begin{itemize}
	\item do not test the random number generator (any other infrastructure)
	\item do not try and unit test highly-dependent functions - black-box system test the algorithm
\end{itemize}


% summarise the document message and areas for future consideration
\section{Conclusions}
\label{sec:conclusions}
This is the conclusion. Consider summarizing the message of the document once again, and highlighting areas for future consideration.

% bibliography
\bibliographystyle{plain}
\bibliography{../bibtex}

\end{document}
% EOF