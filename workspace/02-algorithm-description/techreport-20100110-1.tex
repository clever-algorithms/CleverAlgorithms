% The Clever Algorithms Project: Algorithm Descriptions

% The Clever Algorithms Project: http://www.CleverAlgorithms.com
% (c) Copyright 2010 Jason Brownlee. All Rights Reserved. 
% This work is licensed under a Creative Commons Attribution-Noncommercial-Share Alike 2.5 Australia License.

\documentclass[a4paper, 11pt]{article}
\usepackage{tabularx}
\usepackage{booktabs}
\usepackage{url}
\usepackage[pdftex,breaklinks=true,colorlinks=true,urlcolor=blue,linkcolor=blue,citecolor=blue,]{hyperref}
\usepackage{geometry}
\geometry{verbose,a4paper,tmargin=25mm,bmargin=25mm,lmargin=25mm,rmargin=25mm}

% Dear template user: fill these in
\newcommand{\myreporttitle}{A Template for Standardized Algorithm Descriptions}
\newcommand{\myreportauthor}{Jason Brownlee}
\newcommand{\myreportemail}{jasonb@CleverAlgorithms.com}
\newcommand{\myreportproject}{The Clever Algorithms Project\\\url{http://www.CleverAlgorithms.com}}
\newcommand{\myreportdate}{20100110}
\newcommand{\myreportversion}{1}
\newcommand{\myreportlicense}{\copyright\ Copyright 2010 Jason Brownlee. All Rights Reserved. This work is licensed under a Creative Commons Attribution-Noncommercial-Share Alike 2.5 Australia License.}

% leave this alone, it's templated baby!
\title{{\myreporttitle}\footnote{\myreportlicense}}
\author{\myreportauthor\\{\myreportemail}\\\small\myreportproject}
\date{\today\\{\small{Technical Report: CA-TR-{\myreportdate}-\myreportversion}}}
\begin{document}
\maketitle

% write a summary sentence for each major section
\section*{Abstract} 
todo

\begin{description}
	\item[Keywords:] {\small\texttt{Clever, Algorithms, Standard, Description, Procedure, Template, Algorithm}}
\end{description} 

% summarise the document breakdown with cross references
\section{Introduction}
\label{sec:introduction}
%  project
The Clever Algorithm Project is concerned with the complete, consistent, and centralized description of algorithms from the fields of Computational Intelligence and Biologically Inspired Computation to ensure that they are accessible, usable, and understandable \cite{Brownlee2010}.
% this report
This report provides an exploration and definition of the standardized structure for algorithm descriptions in the project.

% breakdown
Section~\ref{sec:elements} provides a summary of algorithm description elements that may be used in a standardized algorithm description template. Section~\ref{sec:examples} demonstrates the usage of the description elements against three popular algorithmic techniques. Section~\ref{sec:template} proposes a template with specific description elements and expectations as to how the template should be adopted and refined.

\section{Descriptive Elements}
\label{sec:elements}
This section provides a summary of descriptive elements that may form apart of a standardized algorithm description. Each element is 1) defined, 2) specifies the intended nature of an instantiated description (such as size, scope and form), and 3) specifies two questions that motivate the content for the element.

\subsection{Name}
The algorithm name defines the canonical name used to refer to the technique, in addition to common aliases, abbreviations, and acronyms. The name is described in terms of the heading and sub-headings of an algorithm.

\begin{itemize}
	% names
	\item \emph{What is the canonical name and common aliases for a technique?}
	% short names
	\item \emph{What are the common abbreviations and acronyms for a technique?}
\end{itemize}

\subsection{Taxonomy: Lineage and locality}
The algorithm taxonomy defines where a techniques fits into the field, both the specific subfields of Computational Intelligence and Biologically Inspired Computation as well as the broader field of Artificial Intelligence. The taxonomy also provides a context for determining the relationships between algorithms. The taxonomy may be described in terms of a series of relationship statements or pictorially as a venn diagram or a graph with hierarchical structure.

\begin{itemize}
	% membership
	\item \emph{To what fields of study does a technique belong?}
	% inter-relationship
	\item \emph{What are closely related techniques to a technique?}
\end{itemize}

\subsection{Inspiration: Motivating system}
The inspiration describes the specific system or process that provoked the inception of the algorithm. The inspiring system may be natural, biological, physical, or social. The description of the inspiring system may include relevant domain specific theory, observation, nomenclature, and most important must include those salient attributes of the system that are somehow abstractly or conceptually manifest in the technique. The inspiration is described textually with source and may include diagrams to highlight features and relationships within the inspiring system.

\begin{itemize}
	% description
	\item \emph{What is the system that motivated the development of a technique?}
	% scope
	\item \emph{Which features of the motivating system are relevant to a technique?}
\end{itemize}

\subsection{Metaphor: Explanation via analogy}
The metaphor is a description of the technique in the context of the inspiring system or a different suitable system. The features of the technique are made apparent through an analogous description of the features of the inspiring system. The explanation through analogy is not expected to be literal scientific truth, rather the method is used as an allegorical communication tool. The inspiring system is not explicitly described, this is the role of the `inspiration' element, which represents a loose dependency for this element. The explanation is textual and uses the nomenclature of the metaphorical system. 

\begin{itemize}
	% metaphor (how does it work)
	\item \emph{What is the explanation of a technique in the context of the inspiring system?}
	% analogy (what can it do)
	\item \emph{What are the functionalities inferred for a technique from the analogous inspiring system?}
\end{itemize}

\subsection{Strategy: Problem solving plan}
The strategy is an abstract description of the computational model. The strategy describes the information processing actions a technique shall take in order to achieve an objective. The strategy provides a logical separation between a computational realization (procedure) and a analogous system (metaphor). A given problem solving strategy may be realized as one of a number specific algorithms or problem solving systems. The strategy description is textual using objective information processing and algorithmic terminology.

\begin{itemize}
	\item \emph{What is a techniques plan of action?}
	\item \emph{What are the information processing properties of a technique?}
\end{itemize}

\subsection{Procedure: Abstract computation}
The algorithm procedure summarizes the specifics of realizing the strategy as a systemized and parameterized computation. It outlines how the algorithm is organized in terms of the data structures and adopted representations. The procedure may be described in terms of software engineering and computer science artifacts such as pseudo code, design diagrams, and relevant mathematical equations.

\begin{itemize}
	% recipe
	\item \emph{What is the computational recipe for a technique?}
	% data
	\item \emph{What are the data structures and representations used in a technique?}
\end{itemize}

\subsection{Heuristics: Usage guidelines}
The heuristics describe the commonsense, best practice, and demonstrated rules for applying and configuring a parameterized algorithm. The heuristics relate to technical details of the techniques procedure and data structures for general classes of application (neither specific implementations not specific problem instances). The heuristics are described textually, such as a series of guidelines in a bullet-point structure.

\begin{itemize}
	% parameters
	\item \emph{What are the suggested configurations for a technique?}
	% application
	\item \emph{What are the guidelines for the application of a technique to a problem instance?}
\end{itemize}

\subsection{Applications: Demonstrated usage}
The applications describes the general classes problem problem to which the technique has been applied or to which the technique is well suited. The description may also include descriptions of specific problem instances that represent archetypal applications of the technique. The description of the applications is textual, likely with supporting references, potentially in a table structure. 

\begin{itemize}
	\item \emph{What are the suggested classes of problems for which a technique is suited?}
	\item \emph{What are some exemplar problem instances to which a technique has been applied?}
\end{itemize}

\subsection{History: Inception and development}
The history describes the inception of the technique including the original sources and the authors and contributors. The description also covers the subsequent and/or parallel development of the technique and the relevant original sources and contributors. The description may be textual with citations, and a timeline structure may be appropriate.

\begin{itemize}
	\item \emph{Under what conditions was a technique originally proposed?}
	\item \emph{What were the significant milestones in the development of a technique?}
\end{itemize}

\subsection{References: Deeper understanding}
The references description includes a listing of both primary sources of information about the technique as well as useful introductory sources for novices to gain a deeper understanding of the theory and application of the technique. The description consists of hand-selected reference material including books, peer reviewed conference papers, journal articles, and potentially websites. A bullet-pointed structure is well suited. 

\begin{itemize}
	\item \emph{What are the primary sources for a technique?}
	\item \emph{What are the suggested reference sources for learning more about a technique?}
\end{itemize}

\subsection{Code: Operational realization}
The code description provides a minimal but functional version of the technique implemented with a programming language. The code description must be able to be typed into an appropriate computer, compiled or interpreted as need be, and provide a working execution of the technique. The technique implementation also includes a minimal problem instance to which it is applied, and both the problem and algorithm implementations are complete enough to demonstrate the techniques procedure. The description is presented as a programming source code listing.

\begin{itemize}
	\item \emph{How is a technique implemented as an executable program?}
	\item \emph{How is a technique applied to an actual problem instance?}
\end{itemize}

\subsection{Tutorial: Guided implementation}
The tutorial description provides a guide to realizing the technique using a programming language. The result of completing the tutorial is a minimal yet complete implementation of the technique applied to a problem, similar or the same to the source code description. The tutorial description provides explanations as to the design decisions and rationale for the way the technique is implemented. The tutorial description is textual, providing a narrative with an objective, a series of steps, and outcome that may be interspersed with source code examples.

\begin{itemize}
	\item \emph{What is the rationale when implementing a technique as an executable program?}
	\item \emph{What is the rationale when applying a technique to an actual problem instance?}
\end{itemize}

\section{Standardized Description} 
\label{sec:template}
This section proposes a standardized algorithm template that includes some of the elements drawn from Section~\ref{sec:elements}. The template is separate from the elements listed in the previous section to allow for independent on going refinement and modification.

\subsection{The Template}
% requirements
A standardized algorithm description template needs a suitable mixture of softer narrative descriptions, programmatic descriptions both abstract and concrete, and most importantly useful sources for finding out more information about the technique.  
% living standard
The proposed template should be considered a first draft of a standard that is expected to be refined through adoption and use throughout the execution of the Clever Algorithms project. The final version of this standard that the project converges to will be used for all algorithms described in the projects compendium to ensure the objective of consistency is met. This constraint is suggested, but not required for the algorithms described workspace during content development.

% template
Table~\ref{tab:template} contains the standardized algorithm description template to be completed by all algorithms described in the Clever Algorithms project. The table describes the specific elements included in the template as well as their suggested ordering and their inclusion state as either required or optional. 

\begin{table}[ht]
	\centering
		\begin{tabularx}{\textwidth}{llX}
		\toprule
		\textbf{Element} & \textbf{Inclusion} &\textbf{Description} \\ 
		\toprule
		\emph{Name} & Required & The heading and alternate headings for the algorithm description.  \\ 
		\midrule
		\emph{Taxonomy} & Required & A small tree diagram showing related fields and algorithms. \\
		\midrule
		\emph{Inspiration} & Optional & A textual description of the inspiring system. \\
		\midrule
		\emph{Metaphor} & Optional & A textual description of the algorithm by analogy. \\
		\midrule
		\emph{Strategy} & Required & A textual description of the information processing strategy. \\
		\midrule
		\emph{Procedure} & Required & A pseudo code description of the algorithms procedure. \\
		\midrule
		\emph{Heuristics} & Required & A bullet-point listing of best practice usage. \\
		\midrule
		\emph{Tutorial} & Required & A textural narrative for realizing the algorithm with complete source code. \\
		\midrule
		\emph{References} & Required & An bullet-point annotated reference list of primary sources and useful resources. \\
		\bottomrule
		\end{tabularx}	
	\caption{Standardized algorithm description template.}
	\label{tab:template}
\end{table}

% usage
The template describes the intended usage of each descriptive element and the arrangements of the elements. It is suggested that each element represent a heading in a complete description of an algorithm, except the name element that becomes the name of the description itself. Very little algorithms will have the information required for all of the elements in a ready to use format. Elements such as the metaphor and strategy will have to be deduced from the primary sources. The procedure will have to be mapped onto a common pseudo code standard adopted for all algorithms described in the project to ensure the consistency objective is observed. The tutorial element will also have to be prepared anew for each technique, ensuring that a minimal although complete implementation is prepared and explained. It is suggested that standalone source code is prepared and tested separate from the tutorial elements in the description to ensure that source code correctness can be maintained. These code listing may be associated with the project but (as of the current version of the template) will not directly appear as an element in the description.  

For many techniques, the inspiration and the metaphor will be similar or even identical, with differentiation between the techniques occurring at the strategy or maybe even the procedure elements. In such cases, these algorithms are likely to represent candidates for siblings in the taxonomy description.

\section{Examples} 
\label{sec:examples}
This section supports the standardized algorithm description template presented in Section~\ref{sec:template} by briefly summarizing how three popular algorithms from the fields of Computational Intelligence and Biologically Inspired Computation may be described. The presented algorithm descriptions are not complete, rather they suggest at the content that may be provided in each description element for a given specific algorithm. 

\subsection{Genetic Algorithm}
An example of the standardized algorithm description template applied to the genetic algorithm.

\begin{description}
	\item[Name]
	\item[Taxonomy] 
	\item[Inspiration]
	\item[Metaphor]
	\item[Strategy]
	\item[Heuristics]
	\item[Procedure]
	\item[Tutorial]
	\item[References]
\end{description}

\subsection{Simulated Annealing}
An example of the standardized algorithm description template applied to the simulated annealing algorithm.

\begin{description}
	\item[Name]
	\item[Taxonomy] 
	\item[Inspiration]
	\item[Metaphor]
	\item[Strategy]
	\item[Heuristics]
	\item[Procedure]
	\item[Tutorial]
	\item[References]
\end{description}

\subsection{Particle Swarm Optimization}
An example of the standardized algorithm description template applied to the particle swarm optimization algorithm.

\begin{description}
	\item[Name]
	\item[Taxonomy] 
	\item[Inspiration]
	\item[Metaphor]
	\item[Strategy]
	\item[Heuristics]
	\item[Procedure]
	\item[Tutorial]
	\item[References]
\end{description}


% summarise the document message and areas for future consideration
\section{Conclusions}
\label{sec:conclusions}
todo

% bibliography
\bibliographystyle{plain}
\bibliography{../bibtex}

\end{document}
% EOF