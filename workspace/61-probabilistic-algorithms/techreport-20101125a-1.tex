% Probabilistic Algorithms

% The Clever Algorithms Project: http://www.CleverAlgorithms.com
% (c) Copyright 2010 Jason Brownlee. Some Rights Reserved. 
% This work is licensed under a Creative Commons Attribution-Noncommercial-Share Alike 2.5 Australia License.

\documentclass[a4paper, 11pt]{article}
\usepackage{tabularx}
\usepackage{booktabs}
\usepackage{url}
\usepackage[pdftex,breaklinks=true,colorlinks=true,urlcolor=blue,linkcolor=blue,citecolor=blue,]{hyperref}
\usepackage{geometry}
\geometry{verbose,a4paper,tmargin=25mm,bmargin=25mm,lmargin=25mm,rmargin=25mm}

% Dear template user: fill these in
\newcommand{\myreporttitle}{Probabilistic Algorithms}
\newcommand{\myreportauthor}{Jason Brownlee}
\newcommand{\myreportemail}{jasonb@CleverAlgorithms.com}
\newcommand{\myreportproject}{The Clever Algorithms Project\\\url{http://www.CleverAlgorithms.com}}
\newcommand{\myreportdate}{20101125a}
\newcommand{\myreportfulldate}{November 24, 2010}
\newcommand{\myreportversion}{1}
\newcommand{\myreportlicense}{\copyright\ Copyright 2010 Jason Brownlee. Some Rights Reserved. This work is licensed under a Creative Commons Attribution-Noncommercial-Share Alike 2.5 Australia License.}

% leave this alone, it's templated baby!
\title{{\myreporttitle}\footnote{\myreportlicense}}
\author{\myreportauthor\\{\myreportemail}\\\small\myreportproject}
\date{\myreportfulldate\\{\small{Technical Report: CA-TR-{\myreportdate}-\myreportversion}}}
\begin{document}
\maketitle

% write a summary sentence for each major section
\section*{Abstract} 
% project
The Clever Algorithms project aims to describe a large number of Artificial Intelligence algorithms in a complete, consistent, and centralized manner, to improve their general accessibility. 
% template
The project makes use of a standardized algorithm description template that uses well-defined topics that motivate the collection of specific and useful information about each algorithm described.
% type
A collection of algorithms for the project have been described, all of which are classified as Probabilistic Algorithms under the adopted algorithm taxonomy.
% best practices
This report provides a point of reflection on the preparation of these algorithms.

\begin{description}
	\item[Keywords:] {\small\texttt{Clever, Algorithms, Project, Probabilistic, Findings}}
\end{description} 

% summarise the document breakdown with cross references
\section{Introduction}
\label{sec:introduction}
% project
The Clever Algorithms project aims to describe a large number of algorithms from the fields of Computational Intelligence, Biologically Inspired Computation, and Metaheuristics in a complete, consistent and centralized manner \cite{Brownlee2010}.
% description
The project requires all algorithms to be described using a standardized template that includes a fixed number of sections, each of which is motivated by the presentation of specific information about the technique \cite{Brownlee2010a}.
% this report
This report provides an overview of the Probabilistic Algorithms in the Clever Algorithms project. 
Section~\ref{sec:algorithms} provides background information and reviews common themes for the general class of algorithm and summarizes those probabilistic algorithms that have been described for the Clever Algorithms Project.

% 
% Described Algorithms
% 
\section{Probabilistic Algorithms}
\label{sec:algorithms}

% 
% Background
% 
\subsection{Background}
% broadly
The algorithms to be described in the Clever Algorithms project are drawn from a diverse set of subfields of Artificial Intelligence, such as Computational Intelligence, Biologically Inspired Computation, and Metaheuristics. The majority of the algorithms selected for description in the project are optimization algorithms \cite{Brownlee2010b}. 
% specific
The recently completed algorithms that have been described for the Clever Algorithms project are referred to as Probabilistic Algorithms. They are differentiated from the remainder of the algorithms described in the project that have been designated a taxonomy including swarm, stochastic, immune, physical, neural, and evolutionary algorithms \cite{Brownlee2010b}. 

% probabilistic
\subsubsection{Probabilistic Models}
Probabilistic Algorithms are those algorithms that model a problem or search a problem space using an probabilistic model of candidate solutions. Many Metaheuristics and Computational Intelligence algorithms may be considered probabilistic, although the difference with these so-called probabilistic algorithms is the explicit use of the tools of probability in problem solving. The vast majority of such algorithms are from the field of Estimation of Distribution Algorithms.

% EDA
\subsubsection{Estimation of Distribution Algorithms}
Estimation of Distribution Algorithms (EDA) also called Probabilistic Model-Building Genetic Algorithms (PMBGA) are an extension of the field of Evolutionary Computation that model a population of candidate solutions as a probabilistic model. The generally involve iterations that alternate between creating candidate solutions in the problem space from a probabilistic model, and reducing a collection of generated candidate solutions into a probabilistic model. 

The model at the heart of an EDA typically provides a the probabilistic expectation of a component or component configuration comprising part of an optimal solution. This estimation is typically based on the observed frequency of use of the component in better than average candidate solutions. The probabilistic model is used to generate candidate solutions in the problem space, typically in a component-wise or step-wise manner using a domain specific construction method to ensure validity.

% papers
Pelikan, et al. provide a comprehensive summary of the field of probabilistic optimization algorithms, summarizing the core approaches and their differences \cite{Pelikan2002b}.
% books
The edited volume by Pelikan, Sastry, and Cantu-Paz provides a collection of studies on the popular Estimation of Distribution algorithms s well is methodology for designing algorithms and application demonstration studies \cite{Pelikan2006}.
An edited volume on studies of EDAs by Larrañaga and Lozano \cite{Larranaga2002} and the follow-up volume by Lozano et al. \cite{Lozano2006}  provides an applied foundation for the field.


% 
% Described Algorithms
% 
\subsection{Described Algorithms}
\label{subsec:algorithms}
% overview
This section lists the Probabilistic Algorithms currently described for inclusion in the Clever Algorithms project. It is proposed that these algorithms will collectively comprise a chapter on `Probabilistic Algorithms' in the Clever Algorithms book. 

\begin{enumerate}
	\item \textbf{Population-Based Incremental Learning}: \cite{Brownlee2010ap}
	\item \textbf{Univariate Marginal Distribution Algorithm}: \cite{Brownlee2010aq}
	\item \textbf{Compact Genetic Algorithm}: \cite{Brownlee2010ar}
	\item \textbf{Bayesian Optimization Algorithm}: \cite{Brownlee2010as}
	\item \textbf{Cross-Entropy Method}: \cite{Brownlee2010at}
\end{enumerate}

% 
% Extensions
% 
\section{Extensions}
\label{sec:extensions}
There are other algorithms and classes of algorithm that were not described from the field of Probabilistic Intelligence. Some areas that may be considered for algorithm description in follow up works include:

\begin{itemize}
	\item \textbf{Extensions to UMDA}: Extensions to the Univariate Marginal Distribution Algorithm such as the Bivariate MArginal Distribution Algorithm (BMDA) \cite{Pelikan1998, Pelikan1999} and the Factorized Distribution Algorithm (FDA) \cite{Muhlenbein1999}.
	\item \textbf{Extensions to cGA}: Extensions to the Compact Genetic Algorithm such as the Extended Compact Genetic Algorithm (ECGA) \cite{Harik1999a, Harik2006}.
	\item \textbf{Extensions to BOA}: Extensions to the Bayesian Optimization Algorithm such as the Hierarchal Bayesian Optimization Algorithm (hBOA) \cite{Pelikan2000, Pelikan2001b} and the Incremental Bayesian Optimization Algorithm (iBOA) \cite{Pelikan2008}.
	\item \textbf{Bayesian Network Algorithms}: Other Bayesian network algorithms such as The Estimation of Bayesian Network Algorithm \cite{Etxeberria1999}, and the Learning Factorized Distribution Algorithm (LFDA) \cite{Muehlenbein1999}.
	\item \textbf{PIPE}: The Probabilistic Incremental Program Evolution that uses EDA methods for constructing programs \cite{Salustowicz1997}. 
	\item \textbf{SHCLVND}: The Stochastic Hill-Climbing with Learning by Vectors of Normal Distributions algorithm \cite{Rudlof1996}.
\end{itemize}

% 
% Conclusions
% 
\section{Conclusions}
\label{sec:conclusions}
% overview
This report provided a point of reflection for the batch of Probabilistic Algorithm descriptions prepared for the Clever Algorithms project. All described algorithms were assigned to the `Probabilistic Algorithms' kingdom in the chosen algorithm taxonomy. This report highlighted the commonality for all described Probabilistic Algorithms and provided a definition suitable for use in the proposed book and website.

% bibliography
\bibliographystyle{plain}
\bibliography{../bibtex}

\end{document}
% EOF