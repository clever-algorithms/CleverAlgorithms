% A Gentle Introduction to Artificial Intelligence: Natural Computation, Computational Intelligence and Metaheuristics

% The Clever Algorithms Project: http://www.CleverAlgorithms.com
% (c) Copyright 2010 Jason Brownlee. Some Rights Reserved. 
% This work is licensed under a Creative Commons Attribution-Noncommercial-Share Alike 2.5 Australia License.

\documentclass[a4paper, 11pt]{article}
\usepackage{tabularx}
\usepackage{booktabs}
\usepackage{url}
\usepackage[pdftex,breaklinks=true,colorlinks=true,urlcolor=blue,linkcolor=blue,citecolor=blue,]{hyperref}
\usepackage{geometry}
\geometry{verbose,a4paper,tmargin=25mm,bmargin=25mm,lmargin=25mm,rmargin=25mm}

% Dear template user: fill these in
\newcommand{\myreporttitle}{A Gentle Introduction to AI}
\newcommand{\myreportsubtitle}{Natural Computation, Computational Intelligence and Metaheuristics}
\newcommand{\myreportauthor}{Jason Brownlee}
\newcommand{\myreportemail}{jasonb@CleverAlgorithms.com}
\newcommand{\myreportproject}{The Clever Algorithms Project\\\url{http://www.CleverAlgorithms.com}}
\newcommand{\myreportdate}{20100115}
\newcommand{\myreportfulldate}{January 15, 2010}
\newcommand{\myreportversion}{1}
\newcommand{\myreportlicense}{\copyright\ Copyright 2010 Jason Brownlee. Some Rights Reserved. This work is licensed under a Creative Commons Attribution-Noncommercial-Share Alike 2.5 Australia License.}

% leave this alone, it's templated baby!
\title{{\myreporttitle}: {\myreportsubtitle}\footnote{\myreportlicense}}
\author{\myreportauthor\\{\myreportemail}\\\small\myreportproject}
\date{\myreportfulldate\\{\small{Technical Report: CA-TR-{\myreportdate}-\myreportversion}}}
\begin{document}
\maketitle

% write a summary sentence for each major section
\section*{Abstract} 
The Clever Algorithms project is concerned with the complete, consistent, and centralized description of a number of algorithms from the field of Artificial Intelligence.
% AI
Artificial Intelligence is a large and diverse field concerned with the study of intelligent systems, and as such, may be considered a confluence of may other disciplines.
% report
This report provides a gentle introduction to the field of AI focusing on the classical `neat' and the newer `scruffy' pursuits. These two streams provide a context for a high-level introduction to the fields from which the techniques from the Clever Algorithms are drawn, namely: Natural Computation, Computational Intelligence, and Metaheuristics.
% future
The report ends with a discussion on the selection and intention of the phrase `clever algorithms' used in the project as well as some ares for future consideration.

\begin{description}
	\item[Keywords:] {\small\texttt{Clever, Algorithms, Artificial, Intelligence, Computational, \\
	Metaheuristics, Natural, Computation, Biologically, Inspired}}
\end{description} 

% summarise the document breakdown with cross references
\section{Introduction}
\label{sec:introduction}
% project
The Clever Algorithms project aims to describe a large number of `unconventional optimization algorithms' in a complete, consistent, and centralized way from the fields of Computational Intelligence and Biologically Inspired Computation \cite{Brownlee2010}.
% report
This report provides background for the Clever Algorithms project by defining a taxonomy of the related sub-fields of Artificial Intelligence from which the algorithms are drawn.
% sections
Section~\ref{sec:artificial_intelligence} introduces the field of Artificial Intelligence in terms of the classical symbolic efforts called \emph{neat AI} and the sub-symbolic and descriptive methods referred to as \emph{scruffy AI}. Section~\ref{sec:natural_computation} presents an orthogonal field to AI called Natural Computing that intersects AI with the field of \emph{Biologically Inspired Computation}. Section~\ref{sec:computationl_intelligence} introduces the field of \emph{Computational Intelligence} as an effort that unifies a number of descriptive AI sub-fields such as evolutionary computation, fuzzy logic, and neural networks. Section~\ref{sec:metaheuristics} considers the heuristic perspective on Computational Intelligence methods and the more recent fields of \emph{Metaheuristics} and \emph{Hyperheuristics}. Finally, Section~\ref{sec:clever_algorithms} introduces so-named \emph{clever algorithms} from the Clever Algorithms project and how they relate to the canonical taxonomy of Artificial Intelligence sub-fields introduced in this report.

% 
% Artificial Intelligence (based on copy from my thesis)
% 
\section{Artificial Intelligence}
\label{sec:artificial_intelligence}
The field of classical \emph{Artificial Intelligence} (AI) coalesced after World War II in the 1950s drawing on an understanding of the brain from neuroscience, the new mathematics of information theory, control theory referred to as cybernetics, and the dawn of the digital computer. AI is a cross-disciplinary field of research generally concerned with developing and investigating systems that operate or act intelligently. It is generally considered a discipline in the field of computer science given the strong focus on computation.

Russell and Norvig provide a perspective that defines Artificial Intelligence in four categories: (1) systems that think like humans, (2) systems that act like humans, (3) systems that think rationally, (4) systems that act rationally \cite{Russell2009}. In their definition, acting like a human suggests that a system can do some specific things humans can do, this includes fields such as the Turing test, natural language processing, automated reasoning, knowledge representation, machine learning, computer vision, and robotics. Thinking like a human suggests systems that model the cognitive information processing properties of humans, for example a general problem solver and systems that build internal models of their world. Thinking rationally suggests laws of rationalism and structured thought, such as syllogisms and formal logic. Finally, acting rationally suggests systems that do rational things such as expected utility maximization and rational agents. 

Luger and Stubblefield suggest that AI is a sub-field of computer science concerned with the automation of intelligence, and like other sub-fields of computer science has both theoretical concerns (\emph{how and why do the systems work?}) and application concerns (\emph{where and when can the systems be used?}) \cite{Luger1993}. They suggest a strong empirical focus to research, because although there may be a strong desire for mathematical analysis, the systems themselves defy analysis due to their complexity. The machines and software investigated in AI are not black boxes, rather analysis proceeds by observing the systems interactions with their environment, followed by an internal assessment of the system to relate its structure back to their behavior.

Artificial Intelligence is therefore concerned with investigating mechanisms that underlie intelligence and intelligence behavior. The traditional approach toward designing and investigating AI (the so-called `good old fashioned' AI) has been to employ a symbolic basis for these mechanisms. A newer approach historically referred to as scruffy artificial intelligence or or soft computing does not necessarily use a symbolic basis, instead patterning these mechanisms after biological or natural processes. This represents a modern paradigm shift in interest from symbolic knowledge representations, to inference strategies for adaptation and learning, and has been referred to as neat versus scruffy approaches to AI. The neat philosophy is concerned with formal symbolic models of intelligence that can explain \emph{why} they work, whereas the scruffy philosophy is concerned with intelligent strategies that explain \emph{how} they work \cite{Sloman1990}.

\subsection{Neat AI}
The traditional stream of AI involves a top down perspective of problem solving, generally involving symbolic representations and logic processes that most importantly can explain why they work. The successes of this prescriptive stream include a multitude of specialist approaches such as rule-based expert systems, automatic theorem provers, and operations research techniques that underly modern planning and scheduling software. Although traditional approaches have resulted in significant success they have their limits, most notably scalability. Increases in problem size result in an unmanageable increase in the complexity of such problems meaning that although traditional techniques can guarantee an optimal, precise, or true solution, the computational execution time or computing memory required can be fantastically unreasonable.

\subsection{Scruffy AI}
There have been a number of thrusts in the field of AI toward less crisp techniques that are able to locate approximate, imprecise, or partially-true solutions to problems with a reasonable cost of resources. Such approaches are typically \emph{descriptive} rather than \emph{prescriptive}, describing a process for achieving a solution (how), but not explaining why they work (like the neater approaches). 

Scruffy AI approaches are defined as relatively simple procedures that result in complex emergent and self-organizing behavior that can defy traditional reductionist analyses, the effects of which can be exploited for quickly locating approximate solutions to intractable problems. A common characteristic of such techniques is the incorporation of randomness in their processes resulting in robust probabilistic and stochastic decision making contrasted to the sometimes more fragile crisp approaches. Another important common attribute is the adoption of an inductive rather than deductive approach to problem solving, generalizing solutions or decisions from sets of specific observations made by the system.

% 
% Natural Computation - based on copy from my thesis
% 
\section{Natural Computation}
\label{sec:natural_computation}
An important perspective on scruffy Artificial Intelligence is the motivation and inspiration for the core information processing strategy of a given technique. Computers can only do what they are instructed, therefore a consideration is to distill information principles and strategies from other fields of study, such as the physical world and biology. The study of biologically motivated computation is called Biologically Inspired Computing \cite{Castro2005a}, and is one of three related fields of Natural Computing \cite{Forbes2000, Forbes2005, Paton1994}. 
% more
Natural Computing is an interdisciplinary field concerned with the relationship of computation and biology, which in addition to Biologically Inspired Computing is also comprised of Computationally Motivated Biology and Computing with Biology \cite{Paun2005, Marrow2000}.

\subsection{Biologically Inspired Computation}
Biologically Inspired Computation is computation inspired by biological metaphor, also referred to as \emph{Biomimicry}, and \emph{Biomemetics} in other engineering disciplines \cite{Castro2005, Benyus1998}. The intent of this field is to devise mathematical and engineering tools to generate solutions to computation problems. Biologically Inspired Computation fits into this category, as do other non-computational areas of problem solving not discussed. At its simplest, the field involves using procedures for finding solutions found in the biological environment for addressing computationally phrased problems.

\subsection{Computationally Motivated Biology}
Computationally Motivated Biology involves investigating biology with computers. The intent of this area is to use information sciences and simulation to model biological systems in digital computers with the aim to replicate and better understand behaviors in biological systems. The field facilitates the ability to better understand life-as-it-is and investigate life-as-it-could-be. Typically, work in this sub-field is not concerned with the construction of mathematical and engineering tools, rather it is focused on simulating natural phenomena. Common examples include Artificial Life, Fractal Geometry (L-systems, Iterative Function Systems, Particle Systems, Brownian motion), and Cellular Automata. A related field is that of Computational Biology generally concerned with modeling biological systems and the application of statistical methods such as in the sub-field of Bioinformatics.

\subsection{Computation with Biology}
Computation with Biology is the investigation of substrates other than silicon in which to implement computation \cite{Aaronson2005}. Common examples include molecular or DNA Computing and Quantum Computing.

% 
% Computational Intelligence - based on copy from my thesis
% 
\section{Computational Intelligence}
\label{sec:computationl_intelligence}
Computational Intelligence is a modern name for the sub-field of AI concerned with sub-symbolic (messy, scruffy, soft) techniques. Generally, Computational Intelligence describes techniques that focus on \emph{strategy} and \emph{outcome}. 
% examples
Computational Intelligence broadly covers sub-disciplines that focus on adaptive and intelligence systems, not limited to: evolutionary computation, Swarm Intelligence (Particle Swarm and Ant Colony Optimization), Fuzzy Systems, Artificial Immune Systems, and Artificial Neural Networks \cite{Engelbrecht2007, Pedrycz1997}. This section provides a brief summary of the each of the five primary areas of study.

\subsection{Evolutionary Computation} 
A paradigm that is concerned with the investigation of systems inspired by the neo-Darwinian theory of evolution by means of natural selection. Popular evolutionary algorithms include the Genetic Algorithm, Evolution Strategy, Genetic and Evolutionary Programming, and Differential Evolution \cite{Baeck2000, Baeck2000a}. The evolutionary process is considered an adaptive strategy and is typically applied to search and optimization domains \cite{Goldberg1989, Holland1975}.

\subsection{Swarm Intelligence} 
A paradigm that considers collective intelligence as a behavior that emerges through the interaction and cooperation of large numbers of lesser intelligent agents. The paradigm consists of two dominant sub-fields (1) Ant Colony Optimization that investigates probabilistic algorithms inspired by the stigmergy and foraging behavior of ants \cite{Bonabeau1999, Dorigo2004}, and (2) Particle Swarm Optimization that investigates probabilistic algorithms inspired by the flocking and foraging behavior of birds and fish \cite{Shi2001}. Like evolutionary computation, swarm intelligences are considered adaptive strategies and are typically applied to search and optimization domains.

\subsection{Artificial Neural Networks}
A paradigm that is concerned with the investigation of architectures and learning strategies inspired by the modeling of neurons in the brain \cite{Bishop1995}. Learning strategies are typically divided into supervised and unsupervised which manage environmental feedback in different ways. Neural network learning processes are considered adaptive learning and are typically applied to function approximation and pattern recognition domains.

\subsection{Fuzzy Intelligence}
A paradigm that is concerned with the investigation of fuzzy logic which is a form of logic that is not constrained to true and false like propositional logic, but rather functions which define approximate truth or degree’s of truth \cite{Zadeh1996}. Fuzzy logic and fuzzy systems are a logic system used as a reasoning strategy and are typically applied to expert system and control system domains.

\subsection{Artificial Immune Systems}
A collection of approaches inspired by the structure and function of the acquired immune system of vertebrates. Popular approaches include clonal selection, negative selection, dendritic cell algorithm, and immune network algorithms. The immune-inspired adaptive processes vary in strategy and show similarities to the fields evolutionary computation and artificial neural networks, and are typically used for optimization and pattern recognition domains \cite{Castro2002}.  

% 
% Metaheuristics
% 
\section{Metaheuristics}
\label{sec:metaheuristics}
Another popular and general name for the strategy-outcome perspective of scruffy AI is \emph{Metaheuristics}. 
% heuristics
A heuristic is an algorithm that locates `good enough' solutions to a problem without concern for whether the solution can be proven to be correct \cite{Michalewicz2004}. Heuristic methods trade off concerns such as precision, quality, and accuracy in favor of computational effort (space and time efficiency). Some examples of heuristic methods include enumerative and greedy search procedures.

% meta
Like heuristics, metaheuristic may be considered a general algorithmic framework that can be applied to different optimization problems with relative few modifications to make them adapted to a specific problem \cite{Glover2003, Talbi2009}. The difference is that Metaheuristics are intended to extend the capabilities of heuristics by combining one or more heuristic methods (referred to as procedures) using a higher-level strategy (hence `meta'). A procedure in a metaheuristic is considered black-box in that little (if any) prior knowledge is known about it by the meta-heuristic, and as such it may be replaced with a different procedure. Procedures may be as simple as a the manipulation of a representation, to as complex as another metaheuristic. Some examples of metaheuristics include iterated local search, tabu search, the genetic algorithm, ant colony optimization, and simulated annealing.

Blum and Roli outline nine properties of metaheuristics \cite{Blum2003}, as follows: 
\begin{itemize}
	\item Metaheuristics are strategies that ``guide'' the search process.
	\item The goal is to efficiently explore the search space in order to find (near-)optimal solutions.
	\item Techniques which constitute metaheuristic algorithms range from simple local search procedures to complex learning processes.
	\item Metaheuristic algorithms are approximate and usually non-deterministic.
	\item They may incorporate mechanisms to avoid getting trapped in confined areas of the search space.
	\item The basic concepts of metaheuristics permit an abstract level description.
	\item Metaheuristics are not problem-specific.
	\item Metaheuristics may make use of domain-specific knowledge in the form of heuristics that are controlled by the upper level strategy.
	\item Todays more advanced metaheuristics use search experience (embodied in some form of memory) to guide the search.
\end{itemize}

% hyper
Hyperheuristics are yet another extension that focuses on heuristics that modify their parameters (online or offline) to improve the efficacy of solution or efficiency of the computation. Hyperheuristics provide high-level strategies that may employ machine learning and adapt their search behavior by modifying the application of the sub-procedures or even which procedures are used (operating on the space of heuristics which in turn operate within the problem domain) \cite{Burke2003a, Burke2003}. 

% 
% Clever Algorithms
% 
\section{Clever Algorithms}
\label{sec:clever_algorithms}
% algorithm soures
The Clever Algorithms project is concerned with algorithms drawn from across many sub-fields of Artificial Intelligence not limited to the scruff fields of Biologically Inspired Computation, Computational Intelligence and Metaheuristics. 
% focus
The current set of algorithms selected to be described in the project may generally be referred to as `unconventional optimization algorithms' (for example, see \cite{Corne1999}), as optimization is the main form of computation provided by the listed approaches \cite{Brownlee2010b}. A technically more appropriate name for these approaches is Stochastic Global Optimization (for example, see \cite{Weise2007} and \cite{Luke2009}).

% the name
The term \emph{Clever Algorithms} is intended to unify a collection of interesting and useful computational tools under a consistent and accessible banner. An alternative name (\emph{Inspired Algorithms}) was considered, although ultimately rejected given that not all of the algorithms to be described in the project have an inspiration (specifically a biological or physical inspiration) for their computational strategy. The term `Clever Algorithms' was chosen for accessibility and not as a new branch of study (a branch that perhaps already has too many names). It is general enough that it may be used to describe any so-called `intelligent systems', and sufficiently underutilized (from a marketing perspective) that it may be specialized as needed, such as its current application to unconventional optimization algorithms. The generality also means that project may be extended into a series and cover model-generating algorithms such as fuzzy systems and artificial neural networks without ambiguity.

% furure
This gentle introduction to Artificial Intelligence was not exhaustive, focusing only on the duality of scruffy and neat approaches as a context for discussing the three areas of interest for the Clever Algorithms project. A useful extension to this work would be an explicit listing (annotated bibliography) of reference books and articles that may be used by interested readers to gain a deeper understanding of each of the fields introduced. Additional future efforts may consider the relationship of Statistical Machine Learning to Artificial Intelligence and the difference of the perspective on intelligent systems compared to those considered in this report, especially considering that there exists some overlap of approaches. 

% bibliography
\bibliographystyle{plain}
\bibliography{../bibtex}

\end{document}
% EOF