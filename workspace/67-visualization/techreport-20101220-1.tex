% Clever Algorithms: Algorithm Visualization

% The Clever Algorithms Project: http://www.CleverAlgorithms.com
% (c) Copyright 2010 Jason Brownlee. Some Rights Reserved. 
% This work is licensed under a Creative Commons Attribution-Noncommercial-Share Alike 2.5 Australia License.

\documentclass[a4paper, 11pt]{article}
\usepackage{tabularx}
\usepackage{booktabs}
\usepackage{url}
\usepackage[pdftex,breaklinks=true,colorlinks=true,urlcolor=blue,linkcolor=blue,citecolor=blue,]{hyperref}
\usepackage{geometry}
\usepackage[ruled, linesnumbered]{../algorithm2e}
\usepackage{listings} 
\usepackage{textcomp}
\ifx\pdfoutput\@undefined\usepackage[usenames,dvips]{color}
\else\usepackage[usenames,dvipsnames]{color}
\lstset{basicstyle=\footnotesize\ttfamily,numbers=left,numberstyle=\tiny,frame=single,columns=flexible,upquote=true,showstringspaces=false,tabsize=2,captionpos=b,breaklines=true,breakatwhitespace=true,keywordstyle=\color{blue},stringstyle=\color{ForestGreen}}
\geometry{verbose,a4paper,tmargin=25mm,bmargin=25mm,lmargin=25mm,rmargin=25mm}


% Dear template user: fill these in
\newcommand{\myreporttitle}{Clever Algorithms}
\newcommand{\myreportsubtitle}{Algorithm Visualization}
\newcommand{\myreportauthor}{Jason Brownlee}
\newcommand{\myreportemail}{jasonb@CleverAlgorithms.com}
\newcommand{\myreportproject}{The Clever Algorithms Project\\\url{http://www.CleverAlgorithms.com}}
\newcommand{\myreportdate}{20101220}
\newcommand{\myreportfulldate}{December 20, 2010}
\newcommand{\myreportversion}{1}
\newcommand{\myreportlicense}{\copyright\ Copyright 2010 Jason Brownlee. Some Rights Reserved. This work is licensed under a Creative Commons Attribution-Noncommercial-Share Alike 2.5 Australia License.}

% leave this alone, it's templated baby!
\title{{\myreporttitle}: {\myreportsubtitle}\footnote{\myreportlicense}}
\author{\myreportauthor\\{\myreportemail}\\\small\myreportproject}
\date{\myreportfulldate\\{\small{Technical Report: CA-TR-{\myreportdate}-\myreportversion}}}
\begin{document}
\maketitle

% write a summary sentence for each major section
\section*{Abstract} 
% project
The Clever Algorithms project aims to describe a large number of Artificial Intelligence algorithms in a complete, consistent, and centralized manner, to improve their general accessibility. 
% template
The project makes use of a standardized algorithm description template that uses well-defined topics that motivate the collection of specific and useful information about each algorithm described.
% report
This report provides a description and examples of visualization as a form of exploration and weak algorithm testing. 

\begin{description}
	\item[Keywords:] {\small\texttt{Clever, Algorithms, GNUPlot, Optimization, Visualization}}
\end{description} 

% summarise the document breakdown with cross references
\section{Introduction}
\label{sec:introduction}
The Clever Algorithms project aims to describe a large number of algorithms from the fields of Computational Intelligence, Biologically Inspired Computation, and Metaheuristics in a complete, consistent and centralized manner \cite{Brownlee2010}.
% description
The project requires all algorithms to be described using a standardized template that includes a fixed number of sections, each of which is motivated by the presentation of specific information about the technique \cite{Brownlee2010a}.

% this report
This report considers the role of visualization in the development and application of algorithms from the fields of Metaheuristics, Computational Intelligence, and Biologically Inspired Computation.

% on visualization
Visualization is can be a powerful technique for exploration the spatial relationships between data (such as an algorithm's performance over time), and investigatory tool (such as plotting an objective problem domain or search space). Visualization can also provide a weak form of algorithm testing, providing metrics on an algorithms efficiency or efficacy that may be indicative of the expected algorithm behavior.

% on visualization in this report
This report provides a discussion of the techniques and the tools that may be used to explore and evaluate the problems and algorithms described in the Clever Algorithms project. 
% examples
A discussion of tools and libraries is provided, although all examples in this report are provided using GNUPlot and include both the resulting plot and the script used to generate it. This provides a useful starting point for applying the same methods to the problems and algorithms throughout the Clever Algorithms project.

The discussion and examples are primarily focused on function optimization problems, although the principles of visualization as exploration and weak algorithm testing are generally applicable to function approximation problem instances.

%
% Problems
%
\section{Problems}
The visualization of the problem under study is an excellent place to get started. A simple spatial representation of the search space or objective function can help to motivate the selection of an appropriate technique and/or help to configure that technique. 

The visualization method is specific to the problem type and instance being considered. 
This section provides examples of visualization problems from the fields of continuous and combinatorial function optimization, two types of problems that appear frequently in the Clever Algorithms project.

% Continuous Function Optimization
\subsection{Continuous Function Optimization}
A continuous function optimization problem is typically visualized in two dimensions as a line where $x=input, y=f(input)$ or three dimensions as a surface where $x,y=input, z=f(input)$. 

Some functions may have many more dimensions, which if the function is linearly separable can be visualized in lower dimensions. For example, visualize the first two dimensions of a function with 30 inputs. Functions that are not linearly-separable may be able to make use of projection techniques such as Principle Component Analysis (PCA). For example prepare a stratified sample of the search space as vectors with associated cost function for the PCA.

Similarly, the range of each variable input to the function may be large. This may mean that some of the complexity or detail may be lost when the function is visualized as a line or surface. An indication of this detail may be achieved by creating spot-sample plots of narrow sub-sections of the function. 

Figure~\ref{plot:basin1} provides an example of the basin function used through out the algorithm description in the Clever Algorithms in one dimension and Listing~\ref{basin_1d} provides the gnuplot script used to prepare the plot.

\begin{figure}[htp]
% GNUPLOT: LaTeX picture
\setlength{\unitlength}{0.240900pt}
\ifx\plotpoint\undefined\newsavebox{\plotpoint}\fi
\sbox{\plotpoint}{\rule[-0.200pt]{0.400pt}{0.400pt}}%
\begin{picture}(1200,900)(0,0)
\sbox{\plotpoint}{\rule[-0.200pt]{0.400pt}{0.400pt}}%
\put(150.0,82.0){\rule[-0.200pt]{4.818pt}{0.400pt}}
\put(130,82){\makebox(0,0)[r]{ 0}}
\put(1130.0,82.0){\rule[-0.200pt]{4.818pt}{0.400pt}}
\put(150.0,238.0){\rule[-0.200pt]{4.818pt}{0.400pt}}
\put(130,238){\makebox(0,0)[r]{ 5}}
\put(1130.0,238.0){\rule[-0.200pt]{4.818pt}{0.400pt}}
\put(150.0,393.0){\rule[-0.200pt]{4.818pt}{0.400pt}}
\put(130,393){\makebox(0,0)[r]{ 10}}
\put(1130.0,393.0){\rule[-0.200pt]{4.818pt}{0.400pt}}
\put(150.0,549.0){\rule[-0.200pt]{4.818pt}{0.400pt}}
\put(130,549){\makebox(0,0)[r]{ 15}}
\put(1130.0,549.0){\rule[-0.200pt]{4.818pt}{0.400pt}}
\put(150.0,704.0){\rule[-0.200pt]{4.818pt}{0.400pt}}
\put(130,704){\makebox(0,0)[r]{ 20}}
\put(1130.0,704.0){\rule[-0.200pt]{4.818pt}{0.400pt}}
\put(150.0,860.0){\rule[-0.200pt]{4.818pt}{0.400pt}}
\put(130,860){\makebox(0,0)[r]{ 25}}
\put(1130.0,860.0){\rule[-0.200pt]{4.818pt}{0.400pt}}
\put(250.0,82.0){\rule[-0.200pt]{0.400pt}{4.818pt}}
\put(250,41){\makebox(0,0){-4}}
\put(250.0,840.0){\rule[-0.200pt]{0.400pt}{4.818pt}}
\put(450.0,82.0){\rule[-0.200pt]{0.400pt}{4.818pt}}
\put(450,41){\makebox(0,0){-2}}
\put(450.0,840.0){\rule[-0.200pt]{0.400pt}{4.818pt}}
\put(650.0,82.0){\rule[-0.200pt]{0.400pt}{4.818pt}}
\put(650,41){\makebox(0,0){ 0}}
\put(650.0,840.0){\rule[-0.200pt]{0.400pt}{4.818pt}}
\put(850.0,82.0){\rule[-0.200pt]{0.400pt}{4.818pt}}
\put(850,41){\makebox(0,0){ 2}}
\put(850.0,840.0){\rule[-0.200pt]{0.400pt}{4.818pt}}
\put(1050.0,82.0){\rule[-0.200pt]{0.400pt}{4.818pt}}
\put(1050,41){\makebox(0,0){ 4}}
\put(1050.0,840.0){\rule[-0.200pt]{0.400pt}{4.818pt}}
\put(150.0,82.0){\rule[-0.200pt]{0.400pt}{187.420pt}}
\put(150.0,82.0){\rule[-0.200pt]{240.900pt}{0.400pt}}
\put(1150.0,82.0){\rule[-0.200pt]{0.400pt}{187.420pt}}
\put(150.0,860.0){\rule[-0.200pt]{240.900pt}{0.400pt}}
\put(150,860){\usebox{\plotpoint}}
\multiput(150.58,854.44)(0.491,-1.590){17}{\rule{0.118pt}{1.340pt}}
\multiput(149.17,857.22)(10.000,-28.219){2}{\rule{0.400pt}{0.670pt}}
\multiput(160.58,823.44)(0.491,-1.590){17}{\rule{0.118pt}{1.340pt}}
\multiput(159.17,826.22)(10.000,-28.219){2}{\rule{0.400pt}{0.670pt}}
\multiput(170.58,792.77)(0.491,-1.486){17}{\rule{0.118pt}{1.260pt}}
\multiput(169.17,795.38)(10.000,-26.385){2}{\rule{0.400pt}{0.630pt}}
\multiput(180.58,763.60)(0.491,-1.538){17}{\rule{0.118pt}{1.300pt}}
\multiput(179.17,766.30)(10.000,-27.302){2}{\rule{0.400pt}{0.650pt}}
\multiput(190.58,734.36)(0.492,-1.298){19}{\rule{0.118pt}{1.118pt}}
\multiput(189.17,736.68)(11.000,-25.679){2}{\rule{0.400pt}{0.559pt}}
\multiput(201.58,705.94)(0.491,-1.433){17}{\rule{0.118pt}{1.220pt}}
\multiput(200.17,708.47)(10.000,-25.468){2}{\rule{0.400pt}{0.610pt}}
\multiput(211.58,678.10)(0.491,-1.381){17}{\rule{0.118pt}{1.180pt}}
\multiput(210.17,680.55)(10.000,-24.551){2}{\rule{0.400pt}{0.590pt}}
\multiput(221.58,651.10)(0.491,-1.381){17}{\rule{0.118pt}{1.180pt}}
\multiput(220.17,653.55)(10.000,-24.551){2}{\rule{0.400pt}{0.590pt}}
\multiput(231.58,624.27)(0.491,-1.329){17}{\rule{0.118pt}{1.140pt}}
\multiput(230.17,626.63)(10.000,-23.634){2}{\rule{0.400pt}{0.570pt}}
\multiput(241.58,598.27)(0.491,-1.329){17}{\rule{0.118pt}{1.140pt}}
\multiput(240.17,600.63)(10.000,-23.634){2}{\rule{0.400pt}{0.570pt}}
\multiput(251.58,572.60)(0.491,-1.225){17}{\rule{0.118pt}{1.060pt}}
\multiput(250.17,574.80)(10.000,-21.800){2}{\rule{0.400pt}{0.530pt}}
\multiput(261.58,548.60)(0.491,-1.225){17}{\rule{0.118pt}{1.060pt}}
\multiput(260.17,550.80)(10.000,-21.800){2}{\rule{0.400pt}{0.530pt}}
\multiput(271.58,524.60)(0.491,-1.225){17}{\rule{0.118pt}{1.060pt}}
\multiput(270.17,526.80)(10.000,-21.800){2}{\rule{0.400pt}{0.530pt}}
\multiput(281.58,500.77)(0.491,-1.173){17}{\rule{0.118pt}{1.020pt}}
\multiput(280.17,502.88)(10.000,-20.883){2}{\rule{0.400pt}{0.510pt}}
\multiput(291.58,478.26)(0.492,-1.015){19}{\rule{0.118pt}{0.900pt}}
\multiput(290.17,480.13)(11.000,-20.132){2}{\rule{0.400pt}{0.450pt}}
\multiput(302.58,455.93)(0.491,-1.121){17}{\rule{0.118pt}{0.980pt}}
\multiput(301.17,457.97)(10.000,-19.966){2}{\rule{0.400pt}{0.490pt}}
\multiput(312.58,434.10)(0.491,-1.069){17}{\rule{0.118pt}{0.940pt}}
\multiput(311.17,436.05)(10.000,-19.049){2}{\rule{0.400pt}{0.470pt}}
\multiput(322.58,413.26)(0.491,-1.017){17}{\rule{0.118pt}{0.900pt}}
\multiput(321.17,415.13)(10.000,-18.132){2}{\rule{0.400pt}{0.450pt}}
\multiput(332.58,393.26)(0.491,-1.017){17}{\rule{0.118pt}{0.900pt}}
\multiput(331.17,395.13)(10.000,-18.132){2}{\rule{0.400pt}{0.450pt}}
\multiput(342.58,373.43)(0.491,-0.964){17}{\rule{0.118pt}{0.860pt}}
\multiput(341.17,375.22)(10.000,-17.215){2}{\rule{0.400pt}{0.430pt}}
\multiput(352.58,354.60)(0.491,-0.912){17}{\rule{0.118pt}{0.820pt}}
\multiput(351.17,356.30)(10.000,-16.298){2}{\rule{0.400pt}{0.410pt}}
\multiput(362.58,336.60)(0.491,-0.912){17}{\rule{0.118pt}{0.820pt}}
\multiput(361.17,338.30)(10.000,-16.298){2}{\rule{0.400pt}{0.410pt}}
\multiput(372.58,318.76)(0.491,-0.860){17}{\rule{0.118pt}{0.780pt}}
\multiput(371.17,320.38)(10.000,-15.381){2}{\rule{0.400pt}{0.390pt}}
\multiput(382.58,301.76)(0.491,-0.860){17}{\rule{0.118pt}{0.780pt}}
\multiput(381.17,303.38)(10.000,-15.381){2}{\rule{0.400pt}{0.390pt}}
\multiput(392.58,285.32)(0.492,-0.684){19}{\rule{0.118pt}{0.645pt}}
\multiput(391.17,286.66)(11.000,-13.660){2}{\rule{0.400pt}{0.323pt}}
\multiput(403.58,269.93)(0.491,-0.808){17}{\rule{0.118pt}{0.740pt}}
\multiput(402.17,271.46)(10.000,-14.464){2}{\rule{0.400pt}{0.370pt}}
\multiput(413.58,254.26)(0.491,-0.704){17}{\rule{0.118pt}{0.660pt}}
\multiput(412.17,255.63)(10.000,-12.630){2}{\rule{0.400pt}{0.330pt}}
\multiput(423.58,240.26)(0.491,-0.704){17}{\rule{0.118pt}{0.660pt}}
\multiput(422.17,241.63)(10.000,-12.630){2}{\rule{0.400pt}{0.330pt}}
\multiput(433.58,226.26)(0.491,-0.704){17}{\rule{0.118pt}{0.660pt}}
\multiput(432.17,227.63)(10.000,-12.630){2}{\rule{0.400pt}{0.330pt}}
\multiput(443.58,212.59)(0.491,-0.600){17}{\rule{0.118pt}{0.580pt}}
\multiput(442.17,213.80)(10.000,-10.796){2}{\rule{0.400pt}{0.290pt}}
\multiput(453.58,200.59)(0.491,-0.600){17}{\rule{0.118pt}{0.580pt}}
\multiput(452.17,201.80)(10.000,-10.796){2}{\rule{0.400pt}{0.290pt}}
\multiput(463.58,188.59)(0.491,-0.600){17}{\rule{0.118pt}{0.580pt}}
\multiput(462.17,189.80)(10.000,-10.796){2}{\rule{0.400pt}{0.290pt}}
\multiput(473.58,176.76)(0.491,-0.547){17}{\rule{0.118pt}{0.540pt}}
\multiput(472.17,177.88)(10.000,-9.879){2}{\rule{0.400pt}{0.270pt}}
\multiput(483.00,166.92)(0.495,-0.491){17}{\rule{0.500pt}{0.118pt}}
\multiput(483.00,167.17)(8.962,-10.000){2}{\rule{0.250pt}{0.400pt}}
\multiput(493.00,156.93)(0.611,-0.489){15}{\rule{0.589pt}{0.118pt}}
\multiput(493.00,157.17)(9.778,-9.000){2}{\rule{0.294pt}{0.400pt}}
\multiput(504.00,147.93)(0.553,-0.489){15}{\rule{0.544pt}{0.118pt}}
\multiput(504.00,148.17)(8.870,-9.000){2}{\rule{0.272pt}{0.400pt}}
\multiput(514.00,138.93)(0.626,-0.488){13}{\rule{0.600pt}{0.117pt}}
\multiput(514.00,139.17)(8.755,-8.000){2}{\rule{0.300pt}{0.400pt}}
\multiput(524.00,130.93)(0.626,-0.488){13}{\rule{0.600pt}{0.117pt}}
\multiput(524.00,131.17)(8.755,-8.000){2}{\rule{0.300pt}{0.400pt}}
\multiput(534.00,122.93)(0.721,-0.485){11}{\rule{0.671pt}{0.117pt}}
\multiput(534.00,123.17)(8.606,-7.000){2}{\rule{0.336pt}{0.400pt}}
\multiput(544.00,115.93)(0.852,-0.482){9}{\rule{0.767pt}{0.116pt}}
\multiput(544.00,116.17)(8.409,-6.000){2}{\rule{0.383pt}{0.400pt}}
\multiput(554.00,109.93)(0.852,-0.482){9}{\rule{0.767pt}{0.116pt}}
\multiput(554.00,110.17)(8.409,-6.000){2}{\rule{0.383pt}{0.400pt}}
\multiput(564.00,103.93)(1.044,-0.477){7}{\rule{0.900pt}{0.115pt}}
\multiput(564.00,104.17)(8.132,-5.000){2}{\rule{0.450pt}{0.400pt}}
\multiput(574.00,98.93)(1.044,-0.477){7}{\rule{0.900pt}{0.115pt}}
\multiput(574.00,99.17)(8.132,-5.000){2}{\rule{0.450pt}{0.400pt}}
\multiput(584.00,93.95)(2.025,-0.447){3}{\rule{1.433pt}{0.108pt}}
\multiput(584.00,94.17)(7.025,-3.000){2}{\rule{0.717pt}{0.400pt}}
\multiput(594.00,90.94)(1.505,-0.468){5}{\rule{1.200pt}{0.113pt}}
\multiput(594.00,91.17)(8.509,-4.000){2}{\rule{0.600pt}{0.400pt}}
\put(605,86.17){\rule{2.100pt}{0.400pt}}
\multiput(605.00,87.17)(5.641,-2.000){2}{\rule{1.050pt}{0.400pt}}
\put(615,84.17){\rule{2.100pt}{0.400pt}}
\multiput(615.00,85.17)(5.641,-2.000){2}{\rule{1.050pt}{0.400pt}}
\put(625,82.67){\rule{2.409pt}{0.400pt}}
\multiput(625.00,83.17)(5.000,-1.000){2}{\rule{1.204pt}{0.400pt}}
\put(635,81.67){\rule{2.409pt}{0.400pt}}
\multiput(635.00,82.17)(5.000,-1.000){2}{\rule{1.204pt}{0.400pt}}
\put(655,81.67){\rule{2.409pt}{0.400pt}}
\multiput(655.00,81.17)(5.000,1.000){2}{\rule{1.204pt}{0.400pt}}
\put(665,82.67){\rule{2.409pt}{0.400pt}}
\multiput(665.00,82.17)(5.000,1.000){2}{\rule{1.204pt}{0.400pt}}
\put(675,84.17){\rule{2.100pt}{0.400pt}}
\multiput(675.00,83.17)(5.641,2.000){2}{\rule{1.050pt}{0.400pt}}
\put(685,86.17){\rule{2.100pt}{0.400pt}}
\multiput(685.00,85.17)(5.641,2.000){2}{\rule{1.050pt}{0.400pt}}
\multiput(695.00,88.60)(1.505,0.468){5}{\rule{1.200pt}{0.113pt}}
\multiput(695.00,87.17)(8.509,4.000){2}{\rule{0.600pt}{0.400pt}}
\multiput(706.00,92.61)(2.025,0.447){3}{\rule{1.433pt}{0.108pt}}
\multiput(706.00,91.17)(7.025,3.000){2}{\rule{0.717pt}{0.400pt}}
\multiput(716.00,95.59)(1.044,0.477){7}{\rule{0.900pt}{0.115pt}}
\multiput(716.00,94.17)(8.132,5.000){2}{\rule{0.450pt}{0.400pt}}
\multiput(726.00,100.59)(1.044,0.477){7}{\rule{0.900pt}{0.115pt}}
\multiput(726.00,99.17)(8.132,5.000){2}{\rule{0.450pt}{0.400pt}}
\multiput(736.00,105.59)(0.852,0.482){9}{\rule{0.767pt}{0.116pt}}
\multiput(736.00,104.17)(8.409,6.000){2}{\rule{0.383pt}{0.400pt}}
\multiput(746.00,111.59)(0.852,0.482){9}{\rule{0.767pt}{0.116pt}}
\multiput(746.00,110.17)(8.409,6.000){2}{\rule{0.383pt}{0.400pt}}
\multiput(756.00,117.59)(0.721,0.485){11}{\rule{0.671pt}{0.117pt}}
\multiput(756.00,116.17)(8.606,7.000){2}{\rule{0.336pt}{0.400pt}}
\multiput(766.00,124.59)(0.626,0.488){13}{\rule{0.600pt}{0.117pt}}
\multiput(766.00,123.17)(8.755,8.000){2}{\rule{0.300pt}{0.400pt}}
\multiput(776.00,132.59)(0.626,0.488){13}{\rule{0.600pt}{0.117pt}}
\multiput(776.00,131.17)(8.755,8.000){2}{\rule{0.300pt}{0.400pt}}
\multiput(786.00,140.59)(0.553,0.489){15}{\rule{0.544pt}{0.118pt}}
\multiput(786.00,139.17)(8.870,9.000){2}{\rule{0.272pt}{0.400pt}}
\multiput(796.00,149.59)(0.611,0.489){15}{\rule{0.589pt}{0.118pt}}
\multiput(796.00,148.17)(9.778,9.000){2}{\rule{0.294pt}{0.400pt}}
\multiput(807.00,158.58)(0.495,0.491){17}{\rule{0.500pt}{0.118pt}}
\multiput(807.00,157.17)(8.962,10.000){2}{\rule{0.250pt}{0.400pt}}
\multiput(817.58,168.00)(0.491,0.547){17}{\rule{0.118pt}{0.540pt}}
\multiput(816.17,168.00)(10.000,9.879){2}{\rule{0.400pt}{0.270pt}}
\multiput(827.58,179.00)(0.491,0.600){17}{\rule{0.118pt}{0.580pt}}
\multiput(826.17,179.00)(10.000,10.796){2}{\rule{0.400pt}{0.290pt}}
\multiput(837.58,191.00)(0.491,0.600){17}{\rule{0.118pt}{0.580pt}}
\multiput(836.17,191.00)(10.000,10.796){2}{\rule{0.400pt}{0.290pt}}
\multiput(847.58,203.00)(0.491,0.600){17}{\rule{0.118pt}{0.580pt}}
\multiput(846.17,203.00)(10.000,10.796){2}{\rule{0.400pt}{0.290pt}}
\multiput(857.58,215.00)(0.491,0.704){17}{\rule{0.118pt}{0.660pt}}
\multiput(856.17,215.00)(10.000,12.630){2}{\rule{0.400pt}{0.330pt}}
\multiput(867.58,229.00)(0.491,0.704){17}{\rule{0.118pt}{0.660pt}}
\multiput(866.17,229.00)(10.000,12.630){2}{\rule{0.400pt}{0.330pt}}
\multiput(877.58,243.00)(0.491,0.704){17}{\rule{0.118pt}{0.660pt}}
\multiput(876.17,243.00)(10.000,12.630){2}{\rule{0.400pt}{0.330pt}}
\multiput(887.58,257.00)(0.491,0.808){17}{\rule{0.118pt}{0.740pt}}
\multiput(886.17,257.00)(10.000,14.464){2}{\rule{0.400pt}{0.370pt}}
\multiput(897.58,273.00)(0.492,0.684){19}{\rule{0.118pt}{0.645pt}}
\multiput(896.17,273.00)(11.000,13.660){2}{\rule{0.400pt}{0.323pt}}
\multiput(908.58,288.00)(0.491,0.860){17}{\rule{0.118pt}{0.780pt}}
\multiput(907.17,288.00)(10.000,15.381){2}{\rule{0.400pt}{0.390pt}}
\multiput(918.58,305.00)(0.491,0.860){17}{\rule{0.118pt}{0.780pt}}
\multiput(917.17,305.00)(10.000,15.381){2}{\rule{0.400pt}{0.390pt}}
\multiput(928.58,322.00)(0.491,0.912){17}{\rule{0.118pt}{0.820pt}}
\multiput(927.17,322.00)(10.000,16.298){2}{\rule{0.400pt}{0.410pt}}
\multiput(938.58,340.00)(0.491,0.912){17}{\rule{0.118pt}{0.820pt}}
\multiput(937.17,340.00)(10.000,16.298){2}{\rule{0.400pt}{0.410pt}}
\multiput(948.58,358.00)(0.491,0.964){17}{\rule{0.118pt}{0.860pt}}
\multiput(947.17,358.00)(10.000,17.215){2}{\rule{0.400pt}{0.430pt}}
\multiput(958.58,377.00)(0.491,1.017){17}{\rule{0.118pt}{0.900pt}}
\multiput(957.17,377.00)(10.000,18.132){2}{\rule{0.400pt}{0.450pt}}
\multiput(968.58,397.00)(0.491,1.017){17}{\rule{0.118pt}{0.900pt}}
\multiput(967.17,397.00)(10.000,18.132){2}{\rule{0.400pt}{0.450pt}}
\multiput(978.58,417.00)(0.491,1.069){17}{\rule{0.118pt}{0.940pt}}
\multiput(977.17,417.00)(10.000,19.049){2}{\rule{0.400pt}{0.470pt}}
\multiput(988.58,438.00)(0.491,1.121){17}{\rule{0.118pt}{0.980pt}}
\multiput(987.17,438.00)(10.000,19.966){2}{\rule{0.400pt}{0.490pt}}
\multiput(998.58,460.00)(0.492,1.015){19}{\rule{0.118pt}{0.900pt}}
\multiput(997.17,460.00)(11.000,20.132){2}{\rule{0.400pt}{0.450pt}}
\multiput(1009.58,482.00)(0.491,1.173){17}{\rule{0.118pt}{1.020pt}}
\multiput(1008.17,482.00)(10.000,20.883){2}{\rule{0.400pt}{0.510pt}}
\multiput(1019.58,505.00)(0.491,1.225){17}{\rule{0.118pt}{1.060pt}}
\multiput(1018.17,505.00)(10.000,21.800){2}{\rule{0.400pt}{0.530pt}}
\multiput(1029.58,529.00)(0.491,1.225){17}{\rule{0.118pt}{1.060pt}}
\multiput(1028.17,529.00)(10.000,21.800){2}{\rule{0.400pt}{0.530pt}}
\multiput(1039.58,553.00)(0.491,1.225){17}{\rule{0.118pt}{1.060pt}}
\multiput(1038.17,553.00)(10.000,21.800){2}{\rule{0.400pt}{0.530pt}}
\multiput(1049.58,577.00)(0.491,1.329){17}{\rule{0.118pt}{1.140pt}}
\multiput(1048.17,577.00)(10.000,23.634){2}{\rule{0.400pt}{0.570pt}}
\multiput(1059.58,603.00)(0.491,1.329){17}{\rule{0.118pt}{1.140pt}}
\multiput(1058.17,603.00)(10.000,23.634){2}{\rule{0.400pt}{0.570pt}}
\multiput(1069.58,629.00)(0.491,1.381){17}{\rule{0.118pt}{1.180pt}}
\multiput(1068.17,629.00)(10.000,24.551){2}{\rule{0.400pt}{0.590pt}}
\multiput(1079.58,656.00)(0.491,1.381){17}{\rule{0.118pt}{1.180pt}}
\multiput(1078.17,656.00)(10.000,24.551){2}{\rule{0.400pt}{0.590pt}}
\multiput(1089.58,683.00)(0.491,1.433){17}{\rule{0.118pt}{1.220pt}}
\multiput(1088.17,683.00)(10.000,25.468){2}{\rule{0.400pt}{0.610pt}}
\multiput(1099.58,711.00)(0.492,1.298){19}{\rule{0.118pt}{1.118pt}}
\multiput(1098.17,711.00)(11.000,25.679){2}{\rule{0.400pt}{0.559pt}}
\multiput(1110.58,739.00)(0.491,1.538){17}{\rule{0.118pt}{1.300pt}}
\multiput(1109.17,739.00)(10.000,27.302){2}{\rule{0.400pt}{0.650pt}}
\multiput(1120.58,769.00)(0.491,1.486){17}{\rule{0.118pt}{1.260pt}}
\multiput(1119.17,769.00)(10.000,26.385){2}{\rule{0.400pt}{0.630pt}}
\multiput(1130.58,798.00)(0.491,1.590){17}{\rule{0.118pt}{1.340pt}}
\multiput(1129.17,798.00)(10.000,28.219){2}{\rule{0.400pt}{0.670pt}}
\multiput(1140.58,829.00)(0.491,1.590){17}{\rule{0.118pt}{1.340pt}}
\multiput(1139.17,829.00)(10.000,28.219){2}{\rule{0.400pt}{0.670pt}}
\put(645.0,82.0){\rule[-0.200pt]{2.409pt}{0.400pt}}
\put(150.0,82.0){\rule[-0.200pt]{0.400pt}{187.420pt}}
\put(150.0,82.0){\rule[-0.200pt]{240.900pt}{0.400pt}}
\put(1150.0,82.0){\rule[-0.200pt]{0.400pt}{187.420pt}}
\put(150.0,860.0){\rule[-0.200pt]{240.900pt}{0.400pt}}
\end{picture}

\caption{Plot of the cities of the Basin function in 1-dimension.}
\label{plot:basin1}
\end{figure}

\begin{lstlisting}[caption=Basin Function in 1-dimension, label=basin_1d]
set xrange [-5:5]
plot x*x
\end{lstlisting}

Figure~\ref{plot:basin2} provides an example of the basin function in two-dimensions and Listing~\ref{basin_2d} provides the gnuplot script used to prepare the plot.

\begin{figure}[htp]
% GNUPLOT: LaTeX picture
\setlength{\unitlength}{0.240900pt}
\ifx\plotpoint\undefined\newsavebox{\plotpoint}\fi
\sbox{\plotpoint}{\rule[-0.200pt]{0.400pt}{0.400pt}}%
\begin{picture}(1200,900)(0,0)
\sbox{\plotpoint}{\rule[-0.200pt]{0.400pt}{0.400pt}}%
\multiput(179.00,251.58)(0.854,0.500){359}{\rule{0.783pt}{0.120pt}}
\multiput(179.00,250.17)(307.375,181.000){2}{\rule{0.391pt}{0.400pt}}
\multiput(1012.16,327.58)(-2.544,0.499){207}{\rule{2.130pt}{0.120pt}}
\multiput(1016.58,326.17)(-528.578,105.000){2}{\rule{1.065pt}{0.400pt}}
\put(179.0,251.0){\rule[-0.200pt]{0.400pt}{87.206pt}}
\put(488.0,432.0){\rule[-0.200pt]{0.400pt}{87.206pt}}
\put(1021.0,327.0){\rule[-0.200pt]{0.400pt}{87.206pt}}
\multiput(233.00,240.59)(0.786,0.489){15}{\rule{0.722pt}{0.118pt}}
\multiput(233.00,239.17)(12.501,9.000){2}{\rule{0.361pt}{0.400pt}}
\put(218,222){\makebox(0,0){-4}}
\multiput(537.82,420.93)(-0.844,-0.489){15}{\rule{0.767pt}{0.118pt}}
\multiput(539.41,421.17)(-13.409,-9.000){2}{\rule{0.383pt}{0.400pt}}
\multiput(339.00,219.59)(0.844,0.489){15}{\rule{0.767pt}{0.118pt}}
\multiput(339.00,218.17)(13.409,9.000){2}{\rule{0.383pt}{0.400pt}}
\put(325,202){\makebox(0,0){-2}}
\multiput(643.82,399.93)(-0.844,-0.489){15}{\rule{0.767pt}{0.118pt}}
\multiput(645.41,400.17)(-13.409,-9.000){2}{\rule{0.383pt}{0.400pt}}
\multiput(446.00,198.59)(0.844,0.489){15}{\rule{0.767pt}{0.118pt}}
\multiput(446.00,197.17)(13.409,9.000){2}{\rule{0.383pt}{0.400pt}}
\put(432,181){\makebox(0,0){ 0}}
\multiput(750.82,378.93)(-0.844,-0.489){15}{\rule{0.767pt}{0.118pt}}
\multiput(752.41,379.17)(-13.409,-9.000){2}{\rule{0.383pt}{0.400pt}}
\multiput(553.00,177.59)(0.844,0.489){15}{\rule{0.767pt}{0.118pt}}
\multiput(553.00,176.17)(13.409,9.000){2}{\rule{0.383pt}{0.400pt}}
\put(538,160){\makebox(0,0){ 2}}
\multiput(857.82,357.93)(-0.844,-0.489){15}{\rule{0.767pt}{0.118pt}}
\multiput(859.41,358.17)(-13.409,-9.000){2}{\rule{0.383pt}{0.400pt}}
\multiput(659.00,156.59)(0.844,0.489){15}{\rule{0.767pt}{0.118pt}}
\multiput(659.00,155.17)(13.409,9.000){2}{\rule{0.383pt}{0.400pt}}
\put(644,139){\makebox(0,0){ 4}}
\multiput(964.00,336.93)(-0.786,-0.489){15}{\rule{0.722pt}{0.118pt}}
\multiput(965.50,337.17)(-12.501,-9.000){2}{\rule{0.361pt}{0.400pt}}
\multiput(735.53,164.60)(-2.382,0.468){5}{\rule{1.800pt}{0.113pt}}
\multiput(739.26,163.17)(-13.264,4.000){2}{\rule{0.900pt}{0.400pt}}
\put(761,157){\makebox(0,0){-4}}
\multiput(210.00,267.94)(2.382,-0.468){5}{\rule{1.800pt}{0.113pt}}
\multiput(210.00,268.17)(13.264,-4.000){2}{\rule{0.900pt}{0.400pt}}
\multiput(797.53,200.60)(-2.382,0.468){5}{\rule{1.800pt}{0.113pt}}
\multiput(801.26,199.17)(-13.264,4.000){2}{\rule{0.900pt}{0.400pt}}
\put(822,193){\makebox(0,0){-2}}
\multiput(272.00,303.95)(3.588,-0.447){3}{\rule{2.367pt}{0.108pt}}
\multiput(272.00,304.17)(12.088,-3.000){2}{\rule{1.183pt}{0.400pt}}
\multiput(856.62,237.61)(-3.811,0.447){3}{\rule{2.500pt}{0.108pt}}
\multiput(861.81,236.17)(-12.811,3.000){2}{\rule{1.250pt}{0.400pt}}
\put(884,230){\makebox(0,0){ 0}}
\multiput(333.00,339.95)(3.811,-0.447){3}{\rule{2.500pt}{0.108pt}}
\multiput(333.00,340.17)(12.811,-3.000){2}{\rule{1.250pt}{0.400pt}}
\multiput(918.18,273.61)(-3.588,0.447){3}{\rule{2.367pt}{0.108pt}}
\multiput(923.09,272.17)(-12.088,3.000){2}{\rule{1.183pt}{0.400pt}}
\put(946,266){\makebox(0,0){ 2}}
\multiput(395.00,376.94)(2.382,-0.468){5}{\rule{1.800pt}{0.113pt}}
\multiput(395.00,377.17)(13.264,-4.000){2}{\rule{0.900pt}{0.400pt}}
\multiput(982.53,309.60)(-2.382,0.468){5}{\rule{1.800pt}{0.113pt}}
\multiput(986.26,308.17)(-13.264,4.000){2}{\rule{0.900pt}{0.400pt}}
\put(1007,302){\makebox(0,0){ 4}}
\multiput(457.00,412.95)(3.588,-0.447){3}{\rule{2.367pt}{0.108pt}}
\multiput(457.00,413.17)(12.088,-3.000){2}{\rule{1.183pt}{0.400pt}}
\put(139,372){\makebox(0,0)[r]{ 0}}
\put(179.0,372.0){\rule[-0.200pt]{4.818pt}{0.400pt}}
\put(139,420){\makebox(0,0)[r]{ 10}}
\put(179.0,420.0){\rule[-0.200pt]{4.818pt}{0.400pt}}
\put(139,468){\makebox(0,0)[r]{ 20}}
\put(179.0,468.0){\rule[-0.200pt]{4.818pt}{0.400pt}}
\put(139,516){\makebox(0,0)[r]{ 30}}
\put(179.0,516.0){\rule[-0.200pt]{4.818pt}{0.400pt}}
\put(139,564){\makebox(0,0)[r]{ 40}}
\put(179.0,564.0){\rule[-0.200pt]{4.818pt}{0.400pt}}
\put(139,613){\makebox(0,0)[r]{ 50}}
\put(179.0,613.0){\rule[-0.200pt]{4.818pt}{0.400pt}}
\multiput(178.59,610.59)(0.477,-0.599){7}{\rule{0.115pt}{0.580pt}}
\multiput(177.17,611.80)(5.000,-4.796){2}{\rule{0.400pt}{0.290pt}}
\multiput(183.00,605.93)(0.491,-0.482){9}{\rule{0.500pt}{0.116pt}}
\multiput(183.00,606.17)(4.962,-6.000){2}{\rule{0.250pt}{0.400pt}}
\multiput(189.59,598.59)(0.477,-0.599){7}{\rule{0.115pt}{0.580pt}}
\multiput(188.17,599.80)(5.000,-4.796){2}{\rule{0.400pt}{0.290pt}}
\multiput(194.00,593.93)(0.599,-0.477){7}{\rule{0.580pt}{0.115pt}}
\multiput(194.00,594.17)(4.796,-5.000){2}{\rule{0.290pt}{0.400pt}}
\multiput(200.59,587.59)(0.477,-0.599){7}{\rule{0.115pt}{0.580pt}}
\multiput(199.17,588.80)(5.000,-4.796){2}{\rule{0.400pt}{0.290pt}}
\multiput(205.00,582.93)(0.487,-0.477){7}{\rule{0.500pt}{0.115pt}}
\multiput(205.00,583.17)(3.962,-5.000){2}{\rule{0.250pt}{0.400pt}}
\multiput(210.00,577.93)(0.491,-0.482){9}{\rule{0.500pt}{0.116pt}}
\multiput(210.00,578.17)(4.962,-6.000){2}{\rule{0.250pt}{0.400pt}}
\multiput(216.00,571.93)(0.487,-0.477){7}{\rule{0.500pt}{0.115pt}}
\multiput(216.00,572.17)(3.962,-5.000){2}{\rule{0.250pt}{0.400pt}}
\multiput(221.00,566.93)(0.599,-0.477){7}{\rule{0.580pt}{0.115pt}}
\multiput(221.00,567.17)(4.796,-5.000){2}{\rule{0.290pt}{0.400pt}}
\multiput(227.00,561.93)(0.487,-0.477){7}{\rule{0.500pt}{0.115pt}}
\multiput(227.00,562.17)(3.962,-5.000){2}{\rule{0.250pt}{0.400pt}}
\multiput(232.00,556.93)(0.487,-0.477){7}{\rule{0.500pt}{0.115pt}}
\multiput(232.00,557.17)(3.962,-5.000){2}{\rule{0.250pt}{0.400pt}}
\multiput(237.00,551.93)(0.599,-0.477){7}{\rule{0.580pt}{0.115pt}}
\multiput(237.00,552.17)(4.796,-5.000){2}{\rule{0.290pt}{0.400pt}}
\multiput(243.00,546.94)(0.627,-0.468){5}{\rule{0.600pt}{0.113pt}}
\multiput(243.00,547.17)(3.755,-4.000){2}{\rule{0.300pt}{0.400pt}}
\multiput(248.00,542.93)(0.599,-0.477){7}{\rule{0.580pt}{0.115pt}}
\multiput(248.00,543.17)(4.796,-5.000){2}{\rule{0.290pt}{0.400pt}}
\multiput(254.00,537.94)(0.627,-0.468){5}{\rule{0.600pt}{0.113pt}}
\multiput(254.00,538.17)(3.755,-4.000){2}{\rule{0.300pt}{0.400pt}}
\multiput(259.00,533.93)(0.487,-0.477){7}{\rule{0.500pt}{0.115pt}}
\multiput(259.00,534.17)(3.962,-5.000){2}{\rule{0.250pt}{0.400pt}}
\multiput(264.00,528.94)(0.774,-0.468){5}{\rule{0.700pt}{0.113pt}}
\multiput(264.00,529.17)(4.547,-4.000){2}{\rule{0.350pt}{0.400pt}}
\multiput(270.00,524.94)(0.627,-0.468){5}{\rule{0.600pt}{0.113pt}}
\multiput(270.00,525.17)(3.755,-4.000){2}{\rule{0.300pt}{0.400pt}}
\multiput(275.00,520.93)(0.599,-0.477){7}{\rule{0.580pt}{0.115pt}}
\multiput(275.00,521.17)(4.796,-5.000){2}{\rule{0.290pt}{0.400pt}}
\multiput(281.00,515.94)(0.627,-0.468){5}{\rule{0.600pt}{0.113pt}}
\multiput(281.00,516.17)(3.755,-4.000){2}{\rule{0.300pt}{0.400pt}}
\multiput(286.00,511.95)(0.909,-0.447){3}{\rule{0.767pt}{0.108pt}}
\multiput(286.00,512.17)(3.409,-3.000){2}{\rule{0.383pt}{0.400pt}}
\multiput(291.00,508.94)(0.774,-0.468){5}{\rule{0.700pt}{0.113pt}}
\multiput(291.00,509.17)(4.547,-4.000){2}{\rule{0.350pt}{0.400pt}}
\multiput(297.00,504.94)(0.627,-0.468){5}{\rule{0.600pt}{0.113pt}}
\multiput(297.00,505.17)(3.755,-4.000){2}{\rule{0.300pt}{0.400pt}}
\multiput(302.00,500.94)(0.774,-0.468){5}{\rule{0.700pt}{0.113pt}}
\multiput(302.00,501.17)(4.547,-4.000){2}{\rule{0.350pt}{0.400pt}}
\multiput(308.00,496.95)(0.909,-0.447){3}{\rule{0.767pt}{0.108pt}}
\multiput(308.00,497.17)(3.409,-3.000){2}{\rule{0.383pt}{0.400pt}}
\multiput(313.00,493.94)(0.627,-0.468){5}{\rule{0.600pt}{0.113pt}}
\multiput(313.00,494.17)(3.755,-4.000){2}{\rule{0.300pt}{0.400pt}}
\multiput(318.00,489.95)(1.132,-0.447){3}{\rule{0.900pt}{0.108pt}}
\multiput(318.00,490.17)(4.132,-3.000){2}{\rule{0.450pt}{0.400pt}}
\multiput(324.00,486.95)(0.909,-0.447){3}{\rule{0.767pt}{0.108pt}}
\multiput(324.00,487.17)(3.409,-3.000){2}{\rule{0.383pt}{0.400pt}}
\multiput(329.00,483.95)(1.132,-0.447){3}{\rule{0.900pt}{0.108pt}}
\multiput(329.00,484.17)(4.132,-3.000){2}{\rule{0.450pt}{0.400pt}}
\multiput(335.00,480.95)(0.909,-0.447){3}{\rule{0.767pt}{0.108pt}}
\multiput(335.00,481.17)(3.409,-3.000){2}{\rule{0.383pt}{0.400pt}}
\multiput(340.00,477.95)(0.909,-0.447){3}{\rule{0.767pt}{0.108pt}}
\multiput(340.00,478.17)(3.409,-3.000){2}{\rule{0.383pt}{0.400pt}}
\multiput(345.00,474.95)(1.132,-0.447){3}{\rule{0.900pt}{0.108pt}}
\multiput(345.00,475.17)(4.132,-3.000){2}{\rule{0.450pt}{0.400pt}}
\multiput(351.00,471.95)(0.909,-0.447){3}{\rule{0.767pt}{0.108pt}}
\multiput(351.00,472.17)(3.409,-3.000){2}{\rule{0.383pt}{0.400pt}}
\put(356,468.17){\rule{1.300pt}{0.400pt}}
\multiput(356.00,469.17)(3.302,-2.000){2}{\rule{0.650pt}{0.400pt}}
\multiput(362.00,466.95)(0.909,-0.447){3}{\rule{0.767pt}{0.108pt}}
\multiput(362.00,467.17)(3.409,-3.000){2}{\rule{0.383pt}{0.400pt}}
\put(367,463.17){\rule{1.100pt}{0.400pt}}
\multiput(367.00,464.17)(2.717,-2.000){2}{\rule{0.550pt}{0.400pt}}
\multiput(372.00,461.95)(1.132,-0.447){3}{\rule{0.900pt}{0.108pt}}
\multiput(372.00,462.17)(4.132,-3.000){2}{\rule{0.450pt}{0.400pt}}
\put(378,458.17){\rule{1.100pt}{0.400pt}}
\multiput(378.00,459.17)(2.717,-2.000){2}{\rule{0.550pt}{0.400pt}}
\put(383,456.17){\rule{1.300pt}{0.400pt}}
\multiput(383.00,457.17)(3.302,-2.000){2}{\rule{0.650pt}{0.400pt}}
\put(389,454.17){\rule{1.100pt}{0.400pt}}
\multiput(389.00,455.17)(2.717,-2.000){2}{\rule{0.550pt}{0.400pt}}
\put(394,452.17){\rule{1.100pt}{0.400pt}}
\multiput(394.00,453.17)(2.717,-2.000){2}{\rule{0.550pt}{0.400pt}}
\put(399,450.17){\rule{1.300pt}{0.400pt}}
\multiput(399.00,451.17)(3.302,-2.000){2}{\rule{0.650pt}{0.400pt}}
\put(405,448.17){\rule{1.100pt}{0.400pt}}
\multiput(405.00,449.17)(2.717,-2.000){2}{\rule{0.550pt}{0.400pt}}
\put(410,446.67){\rule{1.445pt}{0.400pt}}
\multiput(410.00,447.17)(3.000,-1.000){2}{\rule{0.723pt}{0.400pt}}
\put(416,445.17){\rule{1.100pt}{0.400pt}}
\multiput(416.00,446.17)(2.717,-2.000){2}{\rule{0.550pt}{0.400pt}}
\put(421,443.67){\rule{1.204pt}{0.400pt}}
\multiput(421.00,444.17)(2.500,-1.000){2}{\rule{0.602pt}{0.400pt}}
\put(426,442.17){\rule{1.300pt}{0.400pt}}
\multiput(426.00,443.17)(3.302,-2.000){2}{\rule{0.650pt}{0.400pt}}
\put(432,440.67){\rule{1.204pt}{0.400pt}}
\multiput(432.00,441.17)(2.500,-1.000){2}{\rule{0.602pt}{0.400pt}}
\put(437,439.67){\rule{1.445pt}{0.400pt}}
\multiput(437.00,440.17)(3.000,-1.000){2}{\rule{0.723pt}{0.400pt}}
\put(443,438.67){\rule{1.204pt}{0.400pt}}
\multiput(443.00,439.17)(2.500,-1.000){2}{\rule{0.602pt}{0.400pt}}
\put(448,437.67){\rule{1.204pt}{0.400pt}}
\multiput(448.00,438.17)(2.500,-1.000){2}{\rule{0.602pt}{0.400pt}}
\put(453,436.67){\rule{1.445pt}{0.400pt}}
\multiput(453.00,437.17)(3.000,-1.000){2}{\rule{0.723pt}{0.400pt}}
\put(459,435.67){\rule{1.204pt}{0.400pt}}
\multiput(459.00,436.17)(2.500,-1.000){2}{\rule{0.602pt}{0.400pt}}
\put(464,434.67){\rule{1.445pt}{0.400pt}}
\multiput(464.00,435.17)(3.000,-1.000){2}{\rule{0.723pt}{0.400pt}}
\put(475,433.67){\rule{1.204pt}{0.400pt}}
\multiput(475.00,434.17)(2.500,-1.000){2}{\rule{0.602pt}{0.400pt}}
\put(470.0,435.0){\rule[-0.200pt]{1.204pt}{0.400pt}}
\put(480.0,434.0){\rule[-0.200pt]{1.445pt}{0.400pt}}
\put(486.0,434.0){\rule[-0.200pt]{1.204pt}{0.400pt}}
\put(491.0,434.0){\rule[-0.200pt]{1.445pt}{0.400pt}}
\put(497.0,434.0){\rule[-0.200pt]{1.204pt}{0.400pt}}
\put(502.0,434.0){\rule[-0.200pt]{1.204pt}{0.400pt}}
\put(507.0,434.0){\rule[-0.200pt]{1.445pt}{0.400pt}}
\put(513.0,434.0){\rule[-0.200pt]{1.204pt}{0.400pt}}
\put(524,433.67){\rule{1.204pt}{0.400pt}}
\multiput(524.00,433.17)(2.500,1.000){2}{\rule{0.602pt}{0.400pt}}
\put(518.0,434.0){\rule[-0.200pt]{1.445pt}{0.400pt}}
\put(534,434.67){\rule{1.445pt}{0.400pt}}
\multiput(534.00,434.17)(3.000,1.000){2}{\rule{0.723pt}{0.400pt}}
\put(540,435.67){\rule{1.204pt}{0.400pt}}
\multiput(540.00,435.17)(2.500,1.000){2}{\rule{0.602pt}{0.400pt}}
\put(529.0,435.0){\rule[-0.200pt]{1.204pt}{0.400pt}}
\put(550,436.67){\rule{1.445pt}{0.400pt}}
\multiput(550.00,436.17)(3.000,1.000){2}{\rule{0.723pt}{0.400pt}}
\put(556,437.67){\rule{1.204pt}{0.400pt}}
\multiput(556.00,437.17)(2.500,1.000){2}{\rule{0.602pt}{0.400pt}}
\put(561,438.67){\rule{1.445pt}{0.400pt}}
\multiput(561.00,438.17)(3.000,1.000){2}{\rule{0.723pt}{0.400pt}}
\put(567,440.17){\rule{1.100pt}{0.400pt}}
\multiput(567.00,439.17)(2.717,2.000){2}{\rule{0.550pt}{0.400pt}}
\put(572,441.67){\rule{1.204pt}{0.400pt}}
\multiput(572.00,441.17)(2.500,1.000){2}{\rule{0.602pt}{0.400pt}}
\put(577,442.67){\rule{1.445pt}{0.400pt}}
\multiput(577.00,442.17)(3.000,1.000){2}{\rule{0.723pt}{0.400pt}}
\put(583,444.17){\rule{1.100pt}{0.400pt}}
\multiput(583.00,443.17)(2.717,2.000){2}{\rule{0.550pt}{0.400pt}}
\put(588,445.67){\rule{1.445pt}{0.400pt}}
\multiput(588.00,445.17)(3.000,1.000){2}{\rule{0.723pt}{0.400pt}}
\put(594,447.17){\rule{1.100pt}{0.400pt}}
\multiput(594.00,446.17)(2.717,2.000){2}{\rule{0.550pt}{0.400pt}}
\put(599,449.17){\rule{1.100pt}{0.400pt}}
\multiput(599.00,448.17)(2.717,2.000){2}{\rule{0.550pt}{0.400pt}}
\put(604,451.17){\rule{1.300pt}{0.400pt}}
\multiput(604.00,450.17)(3.302,2.000){2}{\rule{0.650pt}{0.400pt}}
\put(610,453.17){\rule{1.100pt}{0.400pt}}
\multiput(610.00,452.17)(2.717,2.000){2}{\rule{0.550pt}{0.400pt}}
\put(615,455.17){\rule{1.300pt}{0.400pt}}
\multiput(615.00,454.17)(3.302,2.000){2}{\rule{0.650pt}{0.400pt}}
\put(621,457.17){\rule{1.100pt}{0.400pt}}
\multiput(621.00,456.17)(2.717,2.000){2}{\rule{0.550pt}{0.400pt}}
\multiput(626.00,459.61)(0.909,0.447){3}{\rule{0.767pt}{0.108pt}}
\multiput(626.00,458.17)(3.409,3.000){2}{\rule{0.383pt}{0.400pt}}
\put(631,462.17){\rule{1.300pt}{0.400pt}}
\multiput(631.00,461.17)(3.302,2.000){2}{\rule{0.650pt}{0.400pt}}
\put(637,464.17){\rule{1.100pt}{0.400pt}}
\multiput(637.00,463.17)(2.717,2.000){2}{\rule{0.550pt}{0.400pt}}
\multiput(642.00,466.61)(1.132,0.447){3}{\rule{0.900pt}{0.108pt}}
\multiput(642.00,465.17)(4.132,3.000){2}{\rule{0.450pt}{0.400pt}}
\multiput(648.00,469.61)(0.909,0.447){3}{\rule{0.767pt}{0.108pt}}
\multiput(648.00,468.17)(3.409,3.000){2}{\rule{0.383pt}{0.400pt}}
\put(653,472.17){\rule{1.100pt}{0.400pt}}
\multiput(653.00,471.17)(2.717,2.000){2}{\rule{0.550pt}{0.400pt}}
\multiput(658.00,474.61)(1.132,0.447){3}{\rule{0.900pt}{0.108pt}}
\multiput(658.00,473.17)(4.132,3.000){2}{\rule{0.450pt}{0.400pt}}
\multiput(664.00,477.61)(0.909,0.447){3}{\rule{0.767pt}{0.108pt}}
\multiput(664.00,476.17)(3.409,3.000){2}{\rule{0.383pt}{0.400pt}}
\multiput(669.00,480.61)(1.132,0.447){3}{\rule{0.900pt}{0.108pt}}
\multiput(669.00,479.17)(4.132,3.000){2}{\rule{0.450pt}{0.400pt}}
\multiput(675.00,483.60)(0.627,0.468){5}{\rule{0.600pt}{0.113pt}}
\multiput(675.00,482.17)(3.755,4.000){2}{\rule{0.300pt}{0.400pt}}
\multiput(680.00,487.61)(0.909,0.447){3}{\rule{0.767pt}{0.108pt}}
\multiput(680.00,486.17)(3.409,3.000){2}{\rule{0.383pt}{0.400pt}}
\multiput(685.00,490.61)(1.132,0.447){3}{\rule{0.900pt}{0.108pt}}
\multiput(685.00,489.17)(4.132,3.000){2}{\rule{0.450pt}{0.400pt}}
\multiput(691.00,493.60)(0.627,0.468){5}{\rule{0.600pt}{0.113pt}}
\multiput(691.00,492.17)(3.755,4.000){2}{\rule{0.300pt}{0.400pt}}
\multiput(696.00,497.61)(1.132,0.447){3}{\rule{0.900pt}{0.108pt}}
\multiput(696.00,496.17)(4.132,3.000){2}{\rule{0.450pt}{0.400pt}}
\multiput(702.00,500.60)(0.627,0.468){5}{\rule{0.600pt}{0.113pt}}
\multiput(702.00,499.17)(3.755,4.000){2}{\rule{0.300pt}{0.400pt}}
\multiput(707.00,504.60)(0.627,0.468){5}{\rule{0.600pt}{0.113pt}}
\multiput(707.00,503.17)(3.755,4.000){2}{\rule{0.300pt}{0.400pt}}
\multiput(212.00,583.93)(0.491,-0.482){9}{\rule{0.500pt}{0.116pt}}
\multiput(212.00,584.17)(4.962,-6.000){2}{\rule{0.250pt}{0.400pt}}
\multiput(218.59,576.59)(0.477,-0.599){7}{\rule{0.115pt}{0.580pt}}
\multiput(217.17,577.80)(5.000,-4.796){2}{\rule{0.400pt}{0.290pt}}
\multiput(223.00,571.93)(0.599,-0.477){7}{\rule{0.580pt}{0.115pt}}
\multiput(223.00,572.17)(4.796,-5.000){2}{\rule{0.290pt}{0.400pt}}
\multiput(229.59,565.59)(0.477,-0.599){7}{\rule{0.115pt}{0.580pt}}
\multiput(228.17,566.80)(5.000,-4.796){2}{\rule{0.400pt}{0.290pt}}
\multiput(234.59,559.59)(0.477,-0.599){7}{\rule{0.115pt}{0.580pt}}
\multiput(233.17,560.80)(5.000,-4.796){2}{\rule{0.400pt}{0.290pt}}
\multiput(239.00,554.93)(0.599,-0.477){7}{\rule{0.580pt}{0.115pt}}
\multiput(239.00,555.17)(4.796,-5.000){2}{\rule{0.290pt}{0.400pt}}
\multiput(245.00,549.93)(0.487,-0.477){7}{\rule{0.500pt}{0.115pt}}
\multiput(245.00,550.17)(3.962,-5.000){2}{\rule{0.250pt}{0.400pt}}
\multiput(250.00,544.93)(0.599,-0.477){7}{\rule{0.580pt}{0.115pt}}
\multiput(250.00,545.17)(4.796,-5.000){2}{\rule{0.290pt}{0.400pt}}
\multiput(256.59,538.59)(0.477,-0.599){7}{\rule{0.115pt}{0.580pt}}
\multiput(255.17,539.80)(5.000,-4.796){2}{\rule{0.400pt}{0.290pt}}
\multiput(261.00,533.93)(0.487,-0.477){7}{\rule{0.500pt}{0.115pt}}
\multiput(261.00,534.17)(3.962,-5.000){2}{\rule{0.250pt}{0.400pt}}
\multiput(266.00,528.94)(0.774,-0.468){5}{\rule{0.700pt}{0.113pt}}
\multiput(266.00,529.17)(4.547,-4.000){2}{\rule{0.350pt}{0.400pt}}
\multiput(272.00,524.93)(0.487,-0.477){7}{\rule{0.500pt}{0.115pt}}
\multiput(272.00,525.17)(3.962,-5.000){2}{\rule{0.250pt}{0.400pt}}
\multiput(277.00,519.93)(0.599,-0.477){7}{\rule{0.580pt}{0.115pt}}
\multiput(277.00,520.17)(4.796,-5.000){2}{\rule{0.290pt}{0.400pt}}
\multiput(283.00,514.93)(0.487,-0.477){7}{\rule{0.500pt}{0.115pt}}
\multiput(283.00,515.17)(3.962,-5.000){2}{\rule{0.250pt}{0.400pt}}
\multiput(288.00,509.94)(0.627,-0.468){5}{\rule{0.600pt}{0.113pt}}
\multiput(288.00,510.17)(3.755,-4.000){2}{\rule{0.300pt}{0.400pt}}
\multiput(293.00,505.93)(0.599,-0.477){7}{\rule{0.580pt}{0.115pt}}
\multiput(293.00,506.17)(4.796,-5.000){2}{\rule{0.290pt}{0.400pt}}
\multiput(299.00,500.94)(0.627,-0.468){5}{\rule{0.600pt}{0.113pt}}
\multiput(299.00,501.17)(3.755,-4.000){2}{\rule{0.300pt}{0.400pt}}
\multiput(304.00,496.94)(0.774,-0.468){5}{\rule{0.700pt}{0.113pt}}
\multiput(304.00,497.17)(4.547,-4.000){2}{\rule{0.350pt}{0.400pt}}
\multiput(310.00,492.94)(0.627,-0.468){5}{\rule{0.600pt}{0.113pt}}
\multiput(310.00,493.17)(3.755,-4.000){2}{\rule{0.300pt}{0.400pt}}
\multiput(315.00,488.94)(0.627,-0.468){5}{\rule{0.600pt}{0.113pt}}
\multiput(315.00,489.17)(3.755,-4.000){2}{\rule{0.300pt}{0.400pt}}
\multiput(320.00,484.94)(0.774,-0.468){5}{\rule{0.700pt}{0.113pt}}
\multiput(320.00,485.17)(4.547,-4.000){2}{\rule{0.350pt}{0.400pt}}
\multiput(326.00,480.94)(0.627,-0.468){5}{\rule{0.600pt}{0.113pt}}
\multiput(326.00,481.17)(3.755,-4.000){2}{\rule{0.300pt}{0.400pt}}
\multiput(331.00,476.94)(0.627,-0.468){5}{\rule{0.600pt}{0.113pt}}
\multiput(331.00,477.17)(3.755,-4.000){2}{\rule{0.300pt}{0.400pt}}
\multiput(336.00,472.95)(1.132,-0.447){3}{\rule{0.900pt}{0.108pt}}
\multiput(336.00,473.17)(4.132,-3.000){2}{\rule{0.450pt}{0.400pt}}
\multiput(342.00,469.94)(0.627,-0.468){5}{\rule{0.600pt}{0.113pt}}
\multiput(342.00,470.17)(3.755,-4.000){2}{\rule{0.300pt}{0.400pt}}
\multiput(347.00,465.95)(1.132,-0.447){3}{\rule{0.900pt}{0.108pt}}
\multiput(347.00,466.17)(4.132,-3.000){2}{\rule{0.450pt}{0.400pt}}
\multiput(353.00,462.94)(0.627,-0.468){5}{\rule{0.600pt}{0.113pt}}
\multiput(353.00,463.17)(3.755,-4.000){2}{\rule{0.300pt}{0.400pt}}
\multiput(358.00,458.95)(0.909,-0.447){3}{\rule{0.767pt}{0.108pt}}
\multiput(358.00,459.17)(3.409,-3.000){2}{\rule{0.383pt}{0.400pt}}
\multiput(363.00,455.95)(1.132,-0.447){3}{\rule{0.900pt}{0.108pt}}
\multiput(363.00,456.17)(4.132,-3.000){2}{\rule{0.450pt}{0.400pt}}
\multiput(369.00,452.95)(0.909,-0.447){3}{\rule{0.767pt}{0.108pt}}
\multiput(369.00,453.17)(3.409,-3.000){2}{\rule{0.383pt}{0.400pt}}
\multiput(374.00,449.95)(1.132,-0.447){3}{\rule{0.900pt}{0.108pt}}
\multiput(374.00,450.17)(4.132,-3.000){2}{\rule{0.450pt}{0.400pt}}
\multiput(380.00,446.95)(0.909,-0.447){3}{\rule{0.767pt}{0.108pt}}
\multiput(380.00,447.17)(3.409,-3.000){2}{\rule{0.383pt}{0.400pt}}
\put(385,443.17){\rule{1.100pt}{0.400pt}}
\multiput(385.00,444.17)(2.717,-2.000){2}{\rule{0.550pt}{0.400pt}}
\multiput(390.00,441.95)(1.132,-0.447){3}{\rule{0.900pt}{0.108pt}}
\multiput(390.00,442.17)(4.132,-3.000){2}{\rule{0.450pt}{0.400pt}}
\multiput(396.00,438.95)(0.909,-0.447){3}{\rule{0.767pt}{0.108pt}}
\multiput(396.00,439.17)(3.409,-3.000){2}{\rule{0.383pt}{0.400pt}}
\put(401,435.17){\rule{1.300pt}{0.400pt}}
\multiput(401.00,436.17)(3.302,-2.000){2}{\rule{0.650pt}{0.400pt}}
\put(407,433.17){\rule{1.100pt}{0.400pt}}
\multiput(407.00,434.17)(2.717,-2.000){2}{\rule{0.550pt}{0.400pt}}
\multiput(412.00,431.95)(0.909,-0.447){3}{\rule{0.767pt}{0.108pt}}
\multiput(412.00,432.17)(3.409,-3.000){2}{\rule{0.383pt}{0.400pt}}
\put(417,428.17){\rule{1.300pt}{0.400pt}}
\multiput(417.00,429.17)(3.302,-2.000){2}{\rule{0.650pt}{0.400pt}}
\put(423,426.17){\rule{1.100pt}{0.400pt}}
\multiput(423.00,427.17)(2.717,-2.000){2}{\rule{0.550pt}{0.400pt}}
\put(428,424.17){\rule{1.300pt}{0.400pt}}
\multiput(428.00,425.17)(3.302,-2.000){2}{\rule{0.650pt}{0.400pt}}
\put(434,422.17){\rule{1.100pt}{0.400pt}}
\multiput(434.00,423.17)(2.717,-2.000){2}{\rule{0.550pt}{0.400pt}}
\put(439,420.67){\rule{1.204pt}{0.400pt}}
\multiput(439.00,421.17)(2.500,-1.000){2}{\rule{0.602pt}{0.400pt}}
\put(444,419.17){\rule{1.300pt}{0.400pt}}
\multiput(444.00,420.17)(3.302,-2.000){2}{\rule{0.650pt}{0.400pt}}
\put(450,417.17){\rule{1.100pt}{0.400pt}}
\multiput(450.00,418.17)(2.717,-2.000){2}{\rule{0.550pt}{0.400pt}}
\put(455,415.67){\rule{1.445pt}{0.400pt}}
\multiput(455.00,416.17)(3.000,-1.000){2}{\rule{0.723pt}{0.400pt}}
\put(461,414.67){\rule{1.204pt}{0.400pt}}
\multiput(461.00,415.17)(2.500,-1.000){2}{\rule{0.602pt}{0.400pt}}
\put(466,413.17){\rule{1.100pt}{0.400pt}}
\multiput(466.00,414.17)(2.717,-2.000){2}{\rule{0.550pt}{0.400pt}}
\put(471,411.67){\rule{1.445pt}{0.400pt}}
\multiput(471.00,412.17)(3.000,-1.000){2}{\rule{0.723pt}{0.400pt}}
\put(477,410.67){\rule{1.204pt}{0.400pt}}
\multiput(477.00,411.17)(2.500,-1.000){2}{\rule{0.602pt}{0.400pt}}
\put(482,409.67){\rule{1.445pt}{0.400pt}}
\multiput(482.00,410.17)(3.000,-1.000){2}{\rule{0.723pt}{0.400pt}}
\put(488,408.67){\rule{1.204pt}{0.400pt}}
\multiput(488.00,409.17)(2.500,-1.000){2}{\rule{0.602pt}{0.400pt}}
\put(545.0,437.0){\rule[-0.200pt]{1.204pt}{0.400pt}}
\put(498,407.67){\rule{1.445pt}{0.400pt}}
\multiput(498.00,408.17)(3.000,-1.000){2}{\rule{0.723pt}{0.400pt}}
\put(504,406.67){\rule{1.204pt}{0.400pt}}
\multiput(504.00,407.17)(2.500,-1.000){2}{\rule{0.602pt}{0.400pt}}
\put(493.0,409.0){\rule[-0.200pt]{1.204pt}{0.400pt}}
\put(515,405.67){\rule{1.204pt}{0.400pt}}
\multiput(515.00,406.17)(2.500,-1.000){2}{\rule{0.602pt}{0.400pt}}
\put(509.0,407.0){\rule[-0.200pt]{1.445pt}{0.400pt}}
\put(520.0,406.0){\rule[-0.200pt]{1.204pt}{0.400pt}}
\put(525.0,406.0){\rule[-0.200pt]{1.445pt}{0.400pt}}
\put(531.0,406.0){\rule[-0.200pt]{1.204pt}{0.400pt}}
\put(536.0,406.0){\rule[-0.200pt]{1.445pt}{0.400pt}}
\put(542.0,406.0){\rule[-0.200pt]{1.204pt}{0.400pt}}
\put(552,405.67){\rule{1.445pt}{0.400pt}}
\multiput(552.00,405.17)(3.000,1.000){2}{\rule{0.723pt}{0.400pt}}
\put(547.0,406.0){\rule[-0.200pt]{1.204pt}{0.400pt}}
\put(563,406.67){\rule{1.445pt}{0.400pt}}
\multiput(563.00,406.17)(3.000,1.000){2}{\rule{0.723pt}{0.400pt}}
\put(558.0,407.0){\rule[-0.200pt]{1.204pt}{0.400pt}}
\put(574,407.67){\rule{1.204pt}{0.400pt}}
\multiput(574.00,407.17)(2.500,1.000){2}{\rule{0.602pt}{0.400pt}}
\put(579,408.67){\rule{1.445pt}{0.400pt}}
\multiput(579.00,408.17)(3.000,1.000){2}{\rule{0.723pt}{0.400pt}}
\put(585,409.67){\rule{1.204pt}{0.400pt}}
\multiput(585.00,409.17)(2.500,1.000){2}{\rule{0.602pt}{0.400pt}}
\put(590,410.67){\rule{1.445pt}{0.400pt}}
\multiput(590.00,410.17)(3.000,1.000){2}{\rule{0.723pt}{0.400pt}}
\put(596,411.67){\rule{1.204pt}{0.400pt}}
\multiput(596.00,411.17)(2.500,1.000){2}{\rule{0.602pt}{0.400pt}}
\put(601,412.67){\rule{1.204pt}{0.400pt}}
\multiput(601.00,412.17)(2.500,1.000){2}{\rule{0.602pt}{0.400pt}}
\put(606,413.67){\rule{1.445pt}{0.400pt}}
\multiput(606.00,413.17)(3.000,1.000){2}{\rule{0.723pt}{0.400pt}}
\put(612,415.17){\rule{1.100pt}{0.400pt}}
\multiput(612.00,414.17)(2.717,2.000){2}{\rule{0.550pt}{0.400pt}}
\put(617,416.67){\rule{1.445pt}{0.400pt}}
\multiput(617.00,416.17)(3.000,1.000){2}{\rule{0.723pt}{0.400pt}}
\put(623,418.17){\rule{1.100pt}{0.400pt}}
\multiput(623.00,417.17)(2.717,2.000){2}{\rule{0.550pt}{0.400pt}}
\put(628,420.17){\rule{1.100pt}{0.400pt}}
\multiput(628.00,419.17)(2.717,2.000){2}{\rule{0.550pt}{0.400pt}}
\put(633,421.67){\rule{1.445pt}{0.400pt}}
\multiput(633.00,421.17)(3.000,1.000){2}{\rule{0.723pt}{0.400pt}}
\put(639,423.17){\rule{1.100pt}{0.400pt}}
\multiput(639.00,422.17)(2.717,2.000){2}{\rule{0.550pt}{0.400pt}}
\put(644,425.17){\rule{1.300pt}{0.400pt}}
\multiput(644.00,424.17)(3.302,2.000){2}{\rule{0.650pt}{0.400pt}}
\put(650,427.17){\rule{1.100pt}{0.400pt}}
\multiput(650.00,426.17)(2.717,2.000){2}{\rule{0.550pt}{0.400pt}}
\multiput(655.00,429.61)(0.909,0.447){3}{\rule{0.767pt}{0.108pt}}
\multiput(655.00,428.17)(3.409,3.000){2}{\rule{0.383pt}{0.400pt}}
\put(660,432.17){\rule{1.300pt}{0.400pt}}
\multiput(660.00,431.17)(3.302,2.000){2}{\rule{0.650pt}{0.400pt}}
\put(666,434.17){\rule{1.100pt}{0.400pt}}
\multiput(666.00,433.17)(2.717,2.000){2}{\rule{0.550pt}{0.400pt}}
\multiput(671.00,436.61)(1.132,0.447){3}{\rule{0.900pt}{0.108pt}}
\multiput(671.00,435.17)(4.132,3.000){2}{\rule{0.450pt}{0.400pt}}
\put(677,439.17){\rule{1.100pt}{0.400pt}}
\multiput(677.00,438.17)(2.717,2.000){2}{\rule{0.550pt}{0.400pt}}
\multiput(682.00,441.61)(0.909,0.447){3}{\rule{0.767pt}{0.108pt}}
\multiput(682.00,440.17)(3.409,3.000){2}{\rule{0.383pt}{0.400pt}}
\multiput(687.00,444.61)(1.132,0.447){3}{\rule{0.900pt}{0.108pt}}
\multiput(687.00,443.17)(4.132,3.000){2}{\rule{0.450pt}{0.400pt}}
\multiput(693.00,447.61)(0.909,0.447){3}{\rule{0.767pt}{0.108pt}}
\multiput(693.00,446.17)(3.409,3.000){2}{\rule{0.383pt}{0.400pt}}
\multiput(698.00,450.61)(1.132,0.447){3}{\rule{0.900pt}{0.108pt}}
\multiput(698.00,449.17)(4.132,3.000){2}{\rule{0.450pt}{0.400pt}}
\multiput(704.00,453.61)(0.909,0.447){3}{\rule{0.767pt}{0.108pt}}
\multiput(704.00,452.17)(3.409,3.000){2}{\rule{0.383pt}{0.400pt}}
\multiput(709.00,456.61)(0.909,0.447){3}{\rule{0.767pt}{0.108pt}}
\multiput(709.00,455.17)(3.409,3.000){2}{\rule{0.383pt}{0.400pt}}
\multiput(714.00,459.61)(1.132,0.447){3}{\rule{0.900pt}{0.108pt}}
\multiput(714.00,458.17)(4.132,3.000){2}{\rule{0.450pt}{0.400pt}}
\multiput(720.00,462.60)(0.627,0.468){5}{\rule{0.600pt}{0.113pt}}
\multiput(720.00,461.17)(3.755,4.000){2}{\rule{0.300pt}{0.400pt}}
\multiput(725.00,466.61)(1.132,0.447){3}{\rule{0.900pt}{0.108pt}}
\multiput(725.00,465.17)(4.132,3.000){2}{\rule{0.450pt}{0.400pt}}
\multiput(731.00,469.60)(0.627,0.468){5}{\rule{0.600pt}{0.113pt}}
\multiput(731.00,468.17)(3.755,4.000){2}{\rule{0.300pt}{0.400pt}}
\multiput(736.00,473.61)(0.909,0.447){3}{\rule{0.767pt}{0.108pt}}
\multiput(736.00,472.17)(3.409,3.000){2}{\rule{0.383pt}{0.400pt}}
\multiput(741.00,476.60)(0.774,0.468){5}{\rule{0.700pt}{0.113pt}}
\multiput(741.00,475.17)(4.547,4.000){2}{\rule{0.350pt}{0.400pt}}
\multiput(247.59,566.59)(0.477,-0.599){7}{\rule{0.115pt}{0.580pt}}
\multiput(246.17,567.80)(5.000,-4.796){2}{\rule{0.400pt}{0.290pt}}
\multiput(252.00,561.93)(0.487,-0.477){7}{\rule{0.500pt}{0.115pt}}
\multiput(252.00,562.17)(3.962,-5.000){2}{\rule{0.250pt}{0.400pt}}
\multiput(257.00,556.93)(0.491,-0.482){9}{\rule{0.500pt}{0.116pt}}
\multiput(257.00,557.17)(4.962,-6.000){2}{\rule{0.250pt}{0.400pt}}
\multiput(263.59,549.59)(0.477,-0.599){7}{\rule{0.115pt}{0.580pt}}
\multiput(262.17,550.80)(5.000,-4.796){2}{\rule{0.400pt}{0.290pt}}
\multiput(268.00,544.93)(0.599,-0.477){7}{\rule{0.580pt}{0.115pt}}
\multiput(268.00,545.17)(4.796,-5.000){2}{\rule{0.290pt}{0.400pt}}
\multiput(274.59,538.59)(0.477,-0.599){7}{\rule{0.115pt}{0.580pt}}
\multiput(273.17,539.80)(5.000,-4.796){2}{\rule{0.400pt}{0.290pt}}
\multiput(279.00,533.93)(0.487,-0.477){7}{\rule{0.500pt}{0.115pt}}
\multiput(279.00,534.17)(3.962,-5.000){2}{\rule{0.250pt}{0.400pt}}
\multiput(284.00,528.93)(0.599,-0.477){7}{\rule{0.580pt}{0.115pt}}
\multiput(284.00,529.17)(4.796,-5.000){2}{\rule{0.290pt}{0.400pt}}
\multiput(290.00,523.93)(0.487,-0.477){7}{\rule{0.500pt}{0.115pt}}
\multiput(290.00,524.17)(3.962,-5.000){2}{\rule{0.250pt}{0.400pt}}
\multiput(295.00,518.93)(0.599,-0.477){7}{\rule{0.580pt}{0.115pt}}
\multiput(295.00,519.17)(4.796,-5.000){2}{\rule{0.290pt}{0.400pt}}
\multiput(301.00,513.93)(0.487,-0.477){7}{\rule{0.500pt}{0.115pt}}
\multiput(301.00,514.17)(3.962,-5.000){2}{\rule{0.250pt}{0.400pt}}
\multiput(306.00,508.93)(0.487,-0.477){7}{\rule{0.500pt}{0.115pt}}
\multiput(306.00,509.17)(3.962,-5.000){2}{\rule{0.250pt}{0.400pt}}
\multiput(311.00,503.93)(0.599,-0.477){7}{\rule{0.580pt}{0.115pt}}
\multiput(311.00,504.17)(4.796,-5.000){2}{\rule{0.290pt}{0.400pt}}
\multiput(317.00,498.94)(0.627,-0.468){5}{\rule{0.600pt}{0.113pt}}
\multiput(317.00,499.17)(3.755,-4.000){2}{\rule{0.300pt}{0.400pt}}
\multiput(322.00,494.93)(0.599,-0.477){7}{\rule{0.580pt}{0.115pt}}
\multiput(322.00,495.17)(4.796,-5.000){2}{\rule{0.290pt}{0.400pt}}
\multiput(328.00,489.94)(0.627,-0.468){5}{\rule{0.600pt}{0.113pt}}
\multiput(328.00,490.17)(3.755,-4.000){2}{\rule{0.300pt}{0.400pt}}
\multiput(333.00,485.93)(0.487,-0.477){7}{\rule{0.500pt}{0.115pt}}
\multiput(333.00,486.17)(3.962,-5.000){2}{\rule{0.250pt}{0.400pt}}
\multiput(338.00,480.94)(0.774,-0.468){5}{\rule{0.700pt}{0.113pt}}
\multiput(338.00,481.17)(4.547,-4.000){2}{\rule{0.350pt}{0.400pt}}
\multiput(344.00,476.94)(0.627,-0.468){5}{\rule{0.600pt}{0.113pt}}
\multiput(344.00,477.17)(3.755,-4.000){2}{\rule{0.300pt}{0.400pt}}
\multiput(349.00,472.94)(0.774,-0.468){5}{\rule{0.700pt}{0.113pt}}
\multiput(349.00,473.17)(4.547,-4.000){2}{\rule{0.350pt}{0.400pt}}
\multiput(355.00,468.94)(0.627,-0.468){5}{\rule{0.600pt}{0.113pt}}
\multiput(355.00,469.17)(3.755,-4.000){2}{\rule{0.300pt}{0.400pt}}
\multiput(360.00,464.94)(0.627,-0.468){5}{\rule{0.600pt}{0.113pt}}
\multiput(360.00,465.17)(3.755,-4.000){2}{\rule{0.300pt}{0.400pt}}
\multiput(365.00,460.95)(1.132,-0.447){3}{\rule{0.900pt}{0.108pt}}
\multiput(365.00,461.17)(4.132,-3.000){2}{\rule{0.450pt}{0.400pt}}
\multiput(371.00,457.94)(0.627,-0.468){5}{\rule{0.600pt}{0.113pt}}
\multiput(371.00,458.17)(3.755,-4.000){2}{\rule{0.300pt}{0.400pt}}
\multiput(376.00,453.95)(1.132,-0.447){3}{\rule{0.900pt}{0.108pt}}
\multiput(376.00,454.17)(4.132,-3.000){2}{\rule{0.450pt}{0.400pt}}
\multiput(382.00,450.94)(0.627,-0.468){5}{\rule{0.600pt}{0.113pt}}
\multiput(382.00,451.17)(3.755,-4.000){2}{\rule{0.300pt}{0.400pt}}
\multiput(387.00,446.95)(0.909,-0.447){3}{\rule{0.767pt}{0.108pt}}
\multiput(387.00,447.17)(3.409,-3.000){2}{\rule{0.383pt}{0.400pt}}
\multiput(392.00,443.95)(1.132,-0.447){3}{\rule{0.900pt}{0.108pt}}
\multiput(392.00,444.17)(4.132,-3.000){2}{\rule{0.450pt}{0.400pt}}
\multiput(398.00,440.94)(0.627,-0.468){5}{\rule{0.600pt}{0.113pt}}
\multiput(398.00,441.17)(3.755,-4.000){2}{\rule{0.300pt}{0.400pt}}
\multiput(403.00,436.95)(1.132,-0.447){3}{\rule{0.900pt}{0.108pt}}
\multiput(403.00,437.17)(4.132,-3.000){2}{\rule{0.450pt}{0.400pt}}
\multiput(409.00,433.95)(0.909,-0.447){3}{\rule{0.767pt}{0.108pt}}
\multiput(409.00,434.17)(3.409,-3.000){2}{\rule{0.383pt}{0.400pt}}
\put(414,430.17){\rule{1.100pt}{0.400pt}}
\multiput(414.00,431.17)(2.717,-2.000){2}{\rule{0.550pt}{0.400pt}}
\multiput(419.00,428.95)(1.132,-0.447){3}{\rule{0.900pt}{0.108pt}}
\multiput(419.00,429.17)(4.132,-3.000){2}{\rule{0.450pt}{0.400pt}}
\multiput(425.00,425.95)(0.909,-0.447){3}{\rule{0.767pt}{0.108pt}}
\multiput(425.00,426.17)(3.409,-3.000){2}{\rule{0.383pt}{0.400pt}}
\put(430,422.17){\rule{1.300pt}{0.400pt}}
\multiput(430.00,423.17)(3.302,-2.000){2}{\rule{0.650pt}{0.400pt}}
\multiput(436.00,420.95)(0.909,-0.447){3}{\rule{0.767pt}{0.108pt}}
\multiput(436.00,421.17)(3.409,-3.000){2}{\rule{0.383pt}{0.400pt}}
\put(441,417.17){\rule{1.100pt}{0.400pt}}
\multiput(441.00,418.17)(2.717,-2.000){2}{\rule{0.550pt}{0.400pt}}
\put(446,415.17){\rule{1.300pt}{0.400pt}}
\multiput(446.00,416.17)(3.302,-2.000){2}{\rule{0.650pt}{0.400pt}}
\put(452,413.17){\rule{1.100pt}{0.400pt}}
\multiput(452.00,414.17)(2.717,-2.000){2}{\rule{0.550pt}{0.400pt}}
\multiput(457.00,411.95)(1.132,-0.447){3}{\rule{0.900pt}{0.108pt}}
\multiput(457.00,412.17)(4.132,-3.000){2}{\rule{0.450pt}{0.400pt}}
\put(463,408.67){\rule{1.204pt}{0.400pt}}
\multiput(463.00,409.17)(2.500,-1.000){2}{\rule{0.602pt}{0.400pt}}
\put(468,407.17){\rule{1.100pt}{0.400pt}}
\multiput(468.00,408.17)(2.717,-2.000){2}{\rule{0.550pt}{0.400pt}}
\put(473,405.17){\rule{1.300pt}{0.400pt}}
\multiput(473.00,406.17)(3.302,-2.000){2}{\rule{0.650pt}{0.400pt}}
\put(479,403.17){\rule{1.100pt}{0.400pt}}
\multiput(479.00,404.17)(2.717,-2.000){2}{\rule{0.550pt}{0.400pt}}
\put(484,401.67){\rule{1.445pt}{0.400pt}}
\multiput(484.00,402.17)(3.000,-1.000){2}{\rule{0.723pt}{0.400pt}}
\put(490,400.17){\rule{1.100pt}{0.400pt}}
\multiput(490.00,401.17)(2.717,-2.000){2}{\rule{0.550pt}{0.400pt}}
\put(495,398.67){\rule{1.204pt}{0.400pt}}
\multiput(495.00,399.17)(2.500,-1.000){2}{\rule{0.602pt}{0.400pt}}
\put(500,397.67){\rule{1.445pt}{0.400pt}}
\multiput(500.00,398.17)(3.000,-1.000){2}{\rule{0.723pt}{0.400pt}}
\put(506,396.17){\rule{1.100pt}{0.400pt}}
\multiput(506.00,397.17)(2.717,-2.000){2}{\rule{0.550pt}{0.400pt}}
\put(511,394.67){\rule{1.445pt}{0.400pt}}
\multiput(511.00,395.17)(3.000,-1.000){2}{\rule{0.723pt}{0.400pt}}
\put(517,393.67){\rule{1.204pt}{0.400pt}}
\multiput(517.00,394.17)(2.500,-1.000){2}{\rule{0.602pt}{0.400pt}}
\put(569.0,408.0){\rule[-0.200pt]{1.204pt}{0.400pt}}
\put(527,392.67){\rule{1.445pt}{0.400pt}}
\multiput(527.00,393.17)(3.000,-1.000){2}{\rule{0.723pt}{0.400pt}}
\put(533,391.67){\rule{1.204pt}{0.400pt}}
\multiput(533.00,392.17)(2.500,-1.000){2}{\rule{0.602pt}{0.400pt}}
\put(522.0,394.0){\rule[-0.200pt]{1.204pt}{0.400pt}}
\put(543,390.67){\rule{1.445pt}{0.400pt}}
\multiput(543.00,391.17)(3.000,-1.000){2}{\rule{0.723pt}{0.400pt}}
\put(538.0,392.0){\rule[-0.200pt]{1.204pt}{0.400pt}}
\put(549.0,391.0){\rule[-0.200pt]{1.204pt}{0.400pt}}
\put(560,389.67){\rule{1.204pt}{0.400pt}}
\multiput(560.00,390.17)(2.500,-1.000){2}{\rule{0.602pt}{0.400pt}}
\put(554.0,391.0){\rule[-0.200pt]{1.445pt}{0.400pt}}
\put(565.0,390.0){\rule[-0.200pt]{1.204pt}{0.400pt}}
\put(570.0,390.0){\rule[-0.200pt]{1.445pt}{0.400pt}}
\put(581,389.67){\rule{1.445pt}{0.400pt}}
\multiput(581.00,389.17)(3.000,1.000){2}{\rule{0.723pt}{0.400pt}}
\put(576.0,390.0){\rule[-0.200pt]{1.204pt}{0.400pt}}
\put(587.0,391.0){\rule[-0.200pt]{1.204pt}{0.400pt}}
\put(597,390.67){\rule{1.445pt}{0.400pt}}
\multiput(597.00,390.17)(3.000,1.000){2}{\rule{0.723pt}{0.400pt}}
\put(603,391.67){\rule{1.204pt}{0.400pt}}
\multiput(603.00,391.17)(2.500,1.000){2}{\rule{0.602pt}{0.400pt}}
\put(592.0,391.0){\rule[-0.200pt]{1.204pt}{0.400pt}}
\put(614,392.67){\rule{1.204pt}{0.400pt}}
\multiput(614.00,392.17)(2.500,1.000){2}{\rule{0.602pt}{0.400pt}}
\put(619,393.67){\rule{1.204pt}{0.400pt}}
\multiput(619.00,393.17)(2.500,1.000){2}{\rule{0.602pt}{0.400pt}}
\put(624,394.67){\rule{1.445pt}{0.400pt}}
\multiput(624.00,394.17)(3.000,1.000){2}{\rule{0.723pt}{0.400pt}}
\put(630,395.67){\rule{1.204pt}{0.400pt}}
\multiput(630.00,395.17)(2.500,1.000){2}{\rule{0.602pt}{0.400pt}}
\put(635,396.67){\rule{1.445pt}{0.400pt}}
\multiput(635.00,396.17)(3.000,1.000){2}{\rule{0.723pt}{0.400pt}}
\put(641,398.17){\rule{1.100pt}{0.400pt}}
\multiput(641.00,397.17)(2.717,2.000){2}{\rule{0.550pt}{0.400pt}}
\put(646,399.67){\rule{1.204pt}{0.400pt}}
\multiput(646.00,399.17)(2.500,1.000){2}{\rule{0.602pt}{0.400pt}}
\put(651,401.17){\rule{1.300pt}{0.400pt}}
\multiput(651.00,400.17)(3.302,2.000){2}{\rule{0.650pt}{0.400pt}}
\put(657,402.67){\rule{1.204pt}{0.400pt}}
\multiput(657.00,402.17)(2.500,1.000){2}{\rule{0.602pt}{0.400pt}}
\put(662,404.17){\rule{1.300pt}{0.400pt}}
\multiput(662.00,403.17)(3.302,2.000){2}{\rule{0.650pt}{0.400pt}}
\put(668,406.17){\rule{1.100pt}{0.400pt}}
\multiput(668.00,405.17)(2.717,2.000){2}{\rule{0.550pt}{0.400pt}}
\put(673,408.17){\rule{1.100pt}{0.400pt}}
\multiput(673.00,407.17)(2.717,2.000){2}{\rule{0.550pt}{0.400pt}}
\put(678,410.17){\rule{1.300pt}{0.400pt}}
\multiput(678.00,409.17)(3.302,2.000){2}{\rule{0.650pt}{0.400pt}}
\put(684,412.17){\rule{1.100pt}{0.400pt}}
\multiput(684.00,411.17)(2.717,2.000){2}{\rule{0.550pt}{0.400pt}}
\put(689,414.17){\rule{1.300pt}{0.400pt}}
\multiput(689.00,413.17)(3.302,2.000){2}{\rule{0.650pt}{0.400pt}}
\put(695,416.17){\rule{1.100pt}{0.400pt}}
\multiput(695.00,415.17)(2.717,2.000){2}{\rule{0.550pt}{0.400pt}}
\multiput(700.00,418.61)(0.909,0.447){3}{\rule{0.767pt}{0.108pt}}
\multiput(700.00,417.17)(3.409,3.000){2}{\rule{0.383pt}{0.400pt}}
\put(705,421.17){\rule{1.300pt}{0.400pt}}
\multiput(705.00,420.17)(3.302,2.000){2}{\rule{0.650pt}{0.400pt}}
\multiput(711.00,423.61)(0.909,0.447){3}{\rule{0.767pt}{0.108pt}}
\multiput(711.00,422.17)(3.409,3.000){2}{\rule{0.383pt}{0.400pt}}
\put(716,426.17){\rule{1.300pt}{0.400pt}}
\multiput(716.00,425.17)(3.302,2.000){2}{\rule{0.650pt}{0.400pt}}
\multiput(722.00,428.61)(0.909,0.447){3}{\rule{0.767pt}{0.108pt}}
\multiput(722.00,427.17)(3.409,3.000){2}{\rule{0.383pt}{0.400pt}}
\multiput(727.00,431.61)(0.909,0.447){3}{\rule{0.767pt}{0.108pt}}
\multiput(727.00,430.17)(3.409,3.000){2}{\rule{0.383pt}{0.400pt}}
\multiput(732.00,434.61)(1.132,0.447){3}{\rule{0.900pt}{0.108pt}}
\multiput(732.00,433.17)(4.132,3.000){2}{\rule{0.450pt}{0.400pt}}
\multiput(738.00,437.61)(0.909,0.447){3}{\rule{0.767pt}{0.108pt}}
\multiput(738.00,436.17)(3.409,3.000){2}{\rule{0.383pt}{0.400pt}}
\multiput(743.00,440.61)(1.132,0.447){3}{\rule{0.900pt}{0.108pt}}
\multiput(743.00,439.17)(4.132,3.000){2}{\rule{0.450pt}{0.400pt}}
\multiput(749.00,443.60)(0.627,0.468){5}{\rule{0.600pt}{0.113pt}}
\multiput(749.00,442.17)(3.755,4.000){2}{\rule{0.300pt}{0.400pt}}
\multiput(754.00,447.61)(0.909,0.447){3}{\rule{0.767pt}{0.108pt}}
\multiput(754.00,446.17)(3.409,3.000){2}{\rule{0.383pt}{0.400pt}}
\multiput(759.00,450.61)(1.132,0.447){3}{\rule{0.900pt}{0.108pt}}
\multiput(759.00,449.17)(4.132,3.000){2}{\rule{0.450pt}{0.400pt}}
\multiput(765.00,453.60)(0.627,0.468){5}{\rule{0.600pt}{0.113pt}}
\multiput(765.00,452.17)(3.755,4.000){2}{\rule{0.300pt}{0.400pt}}
\multiput(770.00,457.60)(0.774,0.468){5}{\rule{0.700pt}{0.113pt}}
\multiput(770.00,456.17)(4.547,4.000){2}{\rule{0.350pt}{0.400pt}}
\multiput(776.00,461.60)(0.627,0.468){5}{\rule{0.600pt}{0.113pt}}
\multiput(776.00,460.17)(3.755,4.000){2}{\rule{0.300pt}{0.400pt}}
\multiput(281.59,563.59)(0.477,-0.599){7}{\rule{0.115pt}{0.580pt}}
\multiput(280.17,564.80)(5.000,-4.796){2}{\rule{0.400pt}{0.290pt}}
\multiput(286.00,558.93)(0.491,-0.482){9}{\rule{0.500pt}{0.116pt}}
\multiput(286.00,559.17)(4.962,-6.000){2}{\rule{0.250pt}{0.400pt}}
\multiput(292.59,551.59)(0.477,-0.599){7}{\rule{0.115pt}{0.580pt}}
\multiput(291.17,552.80)(5.000,-4.796){2}{\rule{0.400pt}{0.290pt}}
\multiput(297.00,546.93)(0.599,-0.477){7}{\rule{0.580pt}{0.115pt}}
\multiput(297.00,547.17)(4.796,-5.000){2}{\rule{0.290pt}{0.400pt}}
\multiput(303.59,540.59)(0.477,-0.599){7}{\rule{0.115pt}{0.580pt}}
\multiput(302.17,541.80)(5.000,-4.796){2}{\rule{0.400pt}{0.290pt}}
\multiput(308.00,535.93)(0.487,-0.477){7}{\rule{0.500pt}{0.115pt}}
\multiput(308.00,536.17)(3.962,-5.000){2}{\rule{0.250pt}{0.400pt}}
\multiput(313.00,530.93)(0.491,-0.482){9}{\rule{0.500pt}{0.116pt}}
\multiput(313.00,531.17)(4.962,-6.000){2}{\rule{0.250pt}{0.400pt}}
\multiput(319.00,524.93)(0.487,-0.477){7}{\rule{0.500pt}{0.115pt}}
\multiput(319.00,525.17)(3.962,-5.000){2}{\rule{0.250pt}{0.400pt}}
\multiput(324.00,519.93)(0.487,-0.477){7}{\rule{0.500pt}{0.115pt}}
\multiput(324.00,520.17)(3.962,-5.000){2}{\rule{0.250pt}{0.400pt}}
\multiput(329.00,514.93)(0.599,-0.477){7}{\rule{0.580pt}{0.115pt}}
\multiput(329.00,515.17)(4.796,-5.000){2}{\rule{0.290pt}{0.400pt}}
\multiput(335.00,509.93)(0.487,-0.477){7}{\rule{0.500pt}{0.115pt}}
\multiput(335.00,510.17)(3.962,-5.000){2}{\rule{0.250pt}{0.400pt}}
\multiput(340.00,504.93)(0.599,-0.477){7}{\rule{0.580pt}{0.115pt}}
\multiput(340.00,505.17)(4.796,-5.000){2}{\rule{0.290pt}{0.400pt}}
\multiput(346.00,499.94)(0.627,-0.468){5}{\rule{0.600pt}{0.113pt}}
\multiput(346.00,500.17)(3.755,-4.000){2}{\rule{0.300pt}{0.400pt}}
\multiput(351.00,495.93)(0.487,-0.477){7}{\rule{0.500pt}{0.115pt}}
\multiput(351.00,496.17)(3.962,-5.000){2}{\rule{0.250pt}{0.400pt}}
\multiput(356.00,490.93)(0.599,-0.477){7}{\rule{0.580pt}{0.115pt}}
\multiput(356.00,491.17)(4.796,-5.000){2}{\rule{0.290pt}{0.400pt}}
\multiput(362.00,485.94)(0.627,-0.468){5}{\rule{0.600pt}{0.113pt}}
\multiput(362.00,486.17)(3.755,-4.000){2}{\rule{0.300pt}{0.400pt}}
\multiput(367.00,481.94)(0.774,-0.468){5}{\rule{0.700pt}{0.113pt}}
\multiput(367.00,482.17)(4.547,-4.000){2}{\rule{0.350pt}{0.400pt}}
\multiput(373.00,477.94)(0.627,-0.468){5}{\rule{0.600pt}{0.113pt}}
\multiput(373.00,478.17)(3.755,-4.000){2}{\rule{0.300pt}{0.400pt}}
\multiput(378.00,473.93)(0.487,-0.477){7}{\rule{0.500pt}{0.115pt}}
\multiput(378.00,474.17)(3.962,-5.000){2}{\rule{0.250pt}{0.400pt}}
\multiput(383.00,468.94)(0.774,-0.468){5}{\rule{0.700pt}{0.113pt}}
\multiput(383.00,469.17)(4.547,-4.000){2}{\rule{0.350pt}{0.400pt}}
\multiput(389.00,464.94)(0.627,-0.468){5}{\rule{0.600pt}{0.113pt}}
\multiput(389.00,465.17)(3.755,-4.000){2}{\rule{0.300pt}{0.400pt}}
\multiput(394.00,460.95)(1.132,-0.447){3}{\rule{0.900pt}{0.108pt}}
\multiput(394.00,461.17)(4.132,-3.000){2}{\rule{0.450pt}{0.400pt}}
\multiput(400.00,457.94)(0.627,-0.468){5}{\rule{0.600pt}{0.113pt}}
\multiput(400.00,458.17)(3.755,-4.000){2}{\rule{0.300pt}{0.400pt}}
\multiput(405.00,453.94)(0.627,-0.468){5}{\rule{0.600pt}{0.113pt}}
\multiput(405.00,454.17)(3.755,-4.000){2}{\rule{0.300pt}{0.400pt}}
\multiput(410.00,449.95)(1.132,-0.447){3}{\rule{0.900pt}{0.108pt}}
\multiput(410.00,450.17)(4.132,-3.000){2}{\rule{0.450pt}{0.400pt}}
\multiput(416.00,446.94)(0.627,-0.468){5}{\rule{0.600pt}{0.113pt}}
\multiput(416.00,447.17)(3.755,-4.000){2}{\rule{0.300pt}{0.400pt}}
\multiput(421.00,442.95)(1.132,-0.447){3}{\rule{0.900pt}{0.108pt}}
\multiput(421.00,443.17)(4.132,-3.000){2}{\rule{0.450pt}{0.400pt}}
\multiput(427.00,439.95)(0.909,-0.447){3}{\rule{0.767pt}{0.108pt}}
\multiput(427.00,440.17)(3.409,-3.000){2}{\rule{0.383pt}{0.400pt}}
\multiput(432.00,436.95)(0.909,-0.447){3}{\rule{0.767pt}{0.108pt}}
\multiput(432.00,437.17)(3.409,-3.000){2}{\rule{0.383pt}{0.400pt}}
\multiput(437.00,433.95)(1.132,-0.447){3}{\rule{0.900pt}{0.108pt}}
\multiput(437.00,434.17)(4.132,-3.000){2}{\rule{0.450pt}{0.400pt}}
\multiput(443.00,430.95)(0.909,-0.447){3}{\rule{0.767pt}{0.108pt}}
\multiput(443.00,431.17)(3.409,-3.000){2}{\rule{0.383pt}{0.400pt}}
\multiput(448.00,427.95)(1.132,-0.447){3}{\rule{0.900pt}{0.108pt}}
\multiput(448.00,428.17)(4.132,-3.000){2}{\rule{0.450pt}{0.400pt}}
\multiput(454.00,424.95)(0.909,-0.447){3}{\rule{0.767pt}{0.108pt}}
\multiput(454.00,425.17)(3.409,-3.000){2}{\rule{0.383pt}{0.400pt}}
\multiput(459.00,421.95)(0.909,-0.447){3}{\rule{0.767pt}{0.108pt}}
\multiput(459.00,422.17)(3.409,-3.000){2}{\rule{0.383pt}{0.400pt}}
\put(464,418.17){\rule{1.300pt}{0.400pt}}
\multiput(464.00,419.17)(3.302,-2.000){2}{\rule{0.650pt}{0.400pt}}
\put(470,416.17){\rule{1.100pt}{0.400pt}}
\multiput(470.00,417.17)(2.717,-2.000){2}{\rule{0.550pt}{0.400pt}}
\multiput(475.00,414.95)(1.132,-0.447){3}{\rule{0.900pt}{0.108pt}}
\multiput(475.00,415.17)(4.132,-3.000){2}{\rule{0.450pt}{0.400pt}}
\put(481,411.17){\rule{1.100pt}{0.400pt}}
\multiput(481.00,412.17)(2.717,-2.000){2}{\rule{0.550pt}{0.400pt}}
\put(486,409.17){\rule{1.100pt}{0.400pt}}
\multiput(486.00,410.17)(2.717,-2.000){2}{\rule{0.550pt}{0.400pt}}
\put(491,407.17){\rule{1.300pt}{0.400pt}}
\multiput(491.00,408.17)(3.302,-2.000){2}{\rule{0.650pt}{0.400pt}}
\put(497,405.17){\rule{1.100pt}{0.400pt}}
\multiput(497.00,406.17)(2.717,-2.000){2}{\rule{0.550pt}{0.400pt}}
\put(502,403.17){\rule{1.300pt}{0.400pt}}
\multiput(502.00,404.17)(3.302,-2.000){2}{\rule{0.650pt}{0.400pt}}
\put(508,401.17){\rule{1.100pt}{0.400pt}}
\multiput(508.00,402.17)(2.717,-2.000){2}{\rule{0.550pt}{0.400pt}}
\put(513,399.67){\rule{1.204pt}{0.400pt}}
\multiput(513.00,400.17)(2.500,-1.000){2}{\rule{0.602pt}{0.400pt}}
\put(518,398.17){\rule{1.300pt}{0.400pt}}
\multiput(518.00,399.17)(3.302,-2.000){2}{\rule{0.650pt}{0.400pt}}
\put(524,396.67){\rule{1.204pt}{0.400pt}}
\multiput(524.00,397.17)(2.500,-1.000){2}{\rule{0.602pt}{0.400pt}}
\put(529,395.17){\rule{1.300pt}{0.400pt}}
\multiput(529.00,396.17)(3.302,-2.000){2}{\rule{0.650pt}{0.400pt}}
\put(535,393.67){\rule{1.204pt}{0.400pt}}
\multiput(535.00,394.17)(2.500,-1.000){2}{\rule{0.602pt}{0.400pt}}
\put(540,392.67){\rule{1.204pt}{0.400pt}}
\multiput(540.00,393.17)(2.500,-1.000){2}{\rule{0.602pt}{0.400pt}}
\put(545,391.67){\rule{1.445pt}{0.400pt}}
\multiput(545.00,392.17)(3.000,-1.000){2}{\rule{0.723pt}{0.400pt}}
\put(551,390.67){\rule{1.204pt}{0.400pt}}
\multiput(551.00,391.17)(2.500,-1.000){2}{\rule{0.602pt}{0.400pt}}
\put(556,389.67){\rule{1.445pt}{0.400pt}}
\multiput(556.00,390.17)(3.000,-1.000){2}{\rule{0.723pt}{0.400pt}}
\put(562,388.67){\rule{1.204pt}{0.400pt}}
\multiput(562.00,389.17)(2.500,-1.000){2}{\rule{0.602pt}{0.400pt}}
\put(567,387.67){\rule{1.204pt}{0.400pt}}
\multiput(567.00,388.17)(2.500,-1.000){2}{\rule{0.602pt}{0.400pt}}
\put(608.0,393.0){\rule[-0.200pt]{1.445pt}{0.400pt}}
\put(578,386.67){\rule{1.204pt}{0.400pt}}
\multiput(578.00,387.17)(2.500,-1.000){2}{\rule{0.602pt}{0.400pt}}
\put(572.0,388.0){\rule[-0.200pt]{1.445pt}{0.400pt}}
\put(583.0,387.0){\rule[-0.200pt]{1.445pt}{0.400pt}}
\put(589.0,387.0){\rule[-0.200pt]{1.204pt}{0.400pt}}
\put(594.0,387.0){\rule[-0.200pt]{1.204pt}{0.400pt}}
\put(599.0,387.0){\rule[-0.200pt]{1.445pt}{0.400pt}}
\put(605.0,387.0){\rule[-0.200pt]{1.204pt}{0.400pt}}
\put(610.0,387.0){\rule[-0.200pt]{1.445pt}{0.400pt}}
\put(616.0,387.0){\rule[-0.200pt]{1.204pt}{0.400pt}}
\put(626,386.67){\rule{1.445pt}{0.400pt}}
\multiput(626.00,386.17)(3.000,1.000){2}{\rule{0.723pt}{0.400pt}}
\put(621.0,387.0){\rule[-0.200pt]{1.204pt}{0.400pt}}
\put(637,387.67){\rule{1.445pt}{0.400pt}}
\multiput(637.00,387.17)(3.000,1.000){2}{\rule{0.723pt}{0.400pt}}
\put(643,388.67){\rule{1.204pt}{0.400pt}}
\multiput(643.00,388.17)(2.500,1.000){2}{\rule{0.602pt}{0.400pt}}
\put(632.0,388.0){\rule[-0.200pt]{1.204pt}{0.400pt}}
\put(653,389.67){\rule{1.445pt}{0.400pt}}
\multiput(653.00,389.17)(3.000,1.000){2}{\rule{0.723pt}{0.400pt}}
\put(659,390.67){\rule{1.204pt}{0.400pt}}
\multiput(659.00,390.17)(2.500,1.000){2}{\rule{0.602pt}{0.400pt}}
\put(664,391.67){\rule{1.445pt}{0.400pt}}
\multiput(664.00,391.17)(3.000,1.000){2}{\rule{0.723pt}{0.400pt}}
\put(670,393.17){\rule{1.100pt}{0.400pt}}
\multiput(670.00,392.17)(2.717,2.000){2}{\rule{0.550pt}{0.400pt}}
\put(675,394.67){\rule{1.204pt}{0.400pt}}
\multiput(675.00,394.17)(2.500,1.000){2}{\rule{0.602pt}{0.400pt}}
\put(680,395.67){\rule{1.445pt}{0.400pt}}
\multiput(680.00,395.17)(3.000,1.000){2}{\rule{0.723pt}{0.400pt}}
\put(686,397.17){\rule{1.100pt}{0.400pt}}
\multiput(686.00,396.17)(2.717,2.000){2}{\rule{0.550pt}{0.400pt}}
\put(691,398.67){\rule{1.445pt}{0.400pt}}
\multiput(691.00,398.17)(3.000,1.000){2}{\rule{0.723pt}{0.400pt}}
\put(697,400.17){\rule{1.100pt}{0.400pt}}
\multiput(697.00,399.17)(2.717,2.000){2}{\rule{0.550pt}{0.400pt}}
\put(702,402.17){\rule{1.100pt}{0.400pt}}
\multiput(702.00,401.17)(2.717,2.000){2}{\rule{0.550pt}{0.400pt}}
\put(707,404.17){\rule{1.300pt}{0.400pt}}
\multiput(707.00,403.17)(3.302,2.000){2}{\rule{0.650pt}{0.400pt}}
\put(713,406.17){\rule{1.100pt}{0.400pt}}
\multiput(713.00,405.17)(2.717,2.000){2}{\rule{0.550pt}{0.400pt}}
\put(718,408.17){\rule{1.100pt}{0.400pt}}
\multiput(718.00,407.17)(2.717,2.000){2}{\rule{0.550pt}{0.400pt}}
\put(723,410.17){\rule{1.300pt}{0.400pt}}
\multiput(723.00,409.17)(3.302,2.000){2}{\rule{0.650pt}{0.400pt}}
\put(729,412.17){\rule{1.100pt}{0.400pt}}
\multiput(729.00,411.17)(2.717,2.000){2}{\rule{0.550pt}{0.400pt}}
\multiput(734.00,414.61)(1.132,0.447){3}{\rule{0.900pt}{0.108pt}}
\multiput(734.00,413.17)(4.132,3.000){2}{\rule{0.450pt}{0.400pt}}
\put(740,417.17){\rule{1.100pt}{0.400pt}}
\multiput(740.00,416.17)(2.717,2.000){2}{\rule{0.550pt}{0.400pt}}
\multiput(745.00,419.61)(0.909,0.447){3}{\rule{0.767pt}{0.108pt}}
\multiput(745.00,418.17)(3.409,3.000){2}{\rule{0.383pt}{0.400pt}}
\multiput(750.00,422.61)(1.132,0.447){3}{\rule{0.900pt}{0.108pt}}
\multiput(750.00,421.17)(4.132,3.000){2}{\rule{0.450pt}{0.400pt}}
\put(756,425.17){\rule{1.100pt}{0.400pt}}
\multiput(756.00,424.17)(2.717,2.000){2}{\rule{0.550pt}{0.400pt}}
\multiput(761.00,427.61)(1.132,0.447){3}{\rule{0.900pt}{0.108pt}}
\multiput(761.00,426.17)(4.132,3.000){2}{\rule{0.450pt}{0.400pt}}
\multiput(767.00,430.61)(0.909,0.447){3}{\rule{0.767pt}{0.108pt}}
\multiput(767.00,429.17)(3.409,3.000){2}{\rule{0.383pt}{0.400pt}}
\multiput(772.00,433.61)(0.909,0.447){3}{\rule{0.767pt}{0.108pt}}
\multiput(772.00,432.17)(3.409,3.000){2}{\rule{0.383pt}{0.400pt}}
\multiput(777.00,436.60)(0.774,0.468){5}{\rule{0.700pt}{0.113pt}}
\multiput(777.00,435.17)(4.547,4.000){2}{\rule{0.350pt}{0.400pt}}
\multiput(783.00,440.61)(0.909,0.447){3}{\rule{0.767pt}{0.108pt}}
\multiput(783.00,439.17)(3.409,3.000){2}{\rule{0.383pt}{0.400pt}}
\multiput(788.00,443.61)(1.132,0.447){3}{\rule{0.900pt}{0.108pt}}
\multiput(788.00,442.17)(4.132,3.000){2}{\rule{0.450pt}{0.400pt}}
\multiput(794.00,446.60)(0.627,0.468){5}{\rule{0.600pt}{0.113pt}}
\multiput(794.00,445.17)(3.755,4.000){2}{\rule{0.300pt}{0.400pt}}
\multiput(799.00,450.61)(0.909,0.447){3}{\rule{0.767pt}{0.108pt}}
\multiput(799.00,449.17)(3.409,3.000){2}{\rule{0.383pt}{0.400pt}}
\multiput(804.00,453.60)(0.774,0.468){5}{\rule{0.700pt}{0.113pt}}
\multiput(804.00,452.17)(4.547,4.000){2}{\rule{0.350pt}{0.400pt}}
\multiput(810.00,457.60)(0.627,0.468){5}{\rule{0.600pt}{0.113pt}}
\multiput(810.00,456.17)(3.755,4.000){2}{\rule{0.300pt}{0.400pt}}
\multiput(315.00,572.93)(0.491,-0.482){9}{\rule{0.500pt}{0.116pt}}
\multiput(315.00,573.17)(4.962,-6.000){2}{\rule{0.250pt}{0.400pt}}
\multiput(321.59,565.59)(0.477,-0.599){7}{\rule{0.115pt}{0.580pt}}
\multiput(320.17,566.80)(5.000,-4.796){2}{\rule{0.400pt}{0.290pt}}
\multiput(326.59,559.59)(0.477,-0.599){7}{\rule{0.115pt}{0.580pt}}
\multiput(325.17,560.80)(5.000,-4.796){2}{\rule{0.400pt}{0.290pt}}
\multiput(331.00,554.93)(0.599,-0.477){7}{\rule{0.580pt}{0.115pt}}
\multiput(331.00,555.17)(4.796,-5.000){2}{\rule{0.290pt}{0.400pt}}
\multiput(337.59,548.59)(0.477,-0.599){7}{\rule{0.115pt}{0.580pt}}
\multiput(336.17,549.80)(5.000,-4.796){2}{\rule{0.400pt}{0.290pt}}
\multiput(342.00,543.93)(0.599,-0.477){7}{\rule{0.580pt}{0.115pt}}
\multiput(342.00,544.17)(4.796,-5.000){2}{\rule{0.290pt}{0.400pt}}
\multiput(348.00,538.93)(0.487,-0.477){7}{\rule{0.500pt}{0.115pt}}
\multiput(348.00,539.17)(3.962,-5.000){2}{\rule{0.250pt}{0.400pt}}
\multiput(353.59,532.59)(0.477,-0.599){7}{\rule{0.115pt}{0.580pt}}
\multiput(352.17,533.80)(5.000,-4.796){2}{\rule{0.400pt}{0.290pt}}
\multiput(358.00,527.93)(0.599,-0.477){7}{\rule{0.580pt}{0.115pt}}
\multiput(358.00,528.17)(4.796,-5.000){2}{\rule{0.290pt}{0.400pt}}
\multiput(364.00,522.93)(0.487,-0.477){7}{\rule{0.500pt}{0.115pt}}
\multiput(364.00,523.17)(3.962,-5.000){2}{\rule{0.250pt}{0.400pt}}
\multiput(369.00,517.93)(0.599,-0.477){7}{\rule{0.580pt}{0.115pt}}
\multiput(369.00,518.17)(4.796,-5.000){2}{\rule{0.290pt}{0.400pt}}
\multiput(375.00,512.94)(0.627,-0.468){5}{\rule{0.600pt}{0.113pt}}
\multiput(375.00,513.17)(3.755,-4.000){2}{\rule{0.300pt}{0.400pt}}
\multiput(380.00,508.93)(0.487,-0.477){7}{\rule{0.500pt}{0.115pt}}
\multiput(380.00,509.17)(3.962,-5.000){2}{\rule{0.250pt}{0.400pt}}
\multiput(385.00,503.93)(0.599,-0.477){7}{\rule{0.580pt}{0.115pt}}
\multiput(385.00,504.17)(4.796,-5.000){2}{\rule{0.290pt}{0.400pt}}
\multiput(391.00,498.94)(0.627,-0.468){5}{\rule{0.600pt}{0.113pt}}
\multiput(391.00,499.17)(3.755,-4.000){2}{\rule{0.300pt}{0.400pt}}
\multiput(396.00,494.93)(0.599,-0.477){7}{\rule{0.580pt}{0.115pt}}
\multiput(396.00,495.17)(4.796,-5.000){2}{\rule{0.290pt}{0.400pt}}
\multiput(402.00,489.94)(0.627,-0.468){5}{\rule{0.600pt}{0.113pt}}
\multiput(402.00,490.17)(3.755,-4.000){2}{\rule{0.300pt}{0.400pt}}
\multiput(407.00,485.94)(0.627,-0.468){5}{\rule{0.600pt}{0.113pt}}
\multiput(407.00,486.17)(3.755,-4.000){2}{\rule{0.300pt}{0.400pt}}
\multiput(412.00,481.94)(0.774,-0.468){5}{\rule{0.700pt}{0.113pt}}
\multiput(412.00,482.17)(4.547,-4.000){2}{\rule{0.350pt}{0.400pt}}
\multiput(418.00,477.94)(0.627,-0.468){5}{\rule{0.600pt}{0.113pt}}
\multiput(418.00,478.17)(3.755,-4.000){2}{\rule{0.300pt}{0.400pt}}
\multiput(423.00,473.94)(0.774,-0.468){5}{\rule{0.700pt}{0.113pt}}
\multiput(423.00,474.17)(4.547,-4.000){2}{\rule{0.350pt}{0.400pt}}
\multiput(429.00,469.94)(0.627,-0.468){5}{\rule{0.600pt}{0.113pt}}
\multiput(429.00,470.17)(3.755,-4.000){2}{\rule{0.300pt}{0.400pt}}
\multiput(434.00,465.94)(0.627,-0.468){5}{\rule{0.600pt}{0.113pt}}
\multiput(434.00,466.17)(3.755,-4.000){2}{\rule{0.300pt}{0.400pt}}
\multiput(439.00,461.95)(1.132,-0.447){3}{\rule{0.900pt}{0.108pt}}
\multiput(439.00,462.17)(4.132,-3.000){2}{\rule{0.450pt}{0.400pt}}
\multiput(445.00,458.94)(0.627,-0.468){5}{\rule{0.600pt}{0.113pt}}
\multiput(445.00,459.17)(3.755,-4.000){2}{\rule{0.300pt}{0.400pt}}
\multiput(450.00,454.95)(1.132,-0.447){3}{\rule{0.900pt}{0.108pt}}
\multiput(450.00,455.17)(4.132,-3.000){2}{\rule{0.450pt}{0.400pt}}
\multiput(456.00,451.94)(0.627,-0.468){5}{\rule{0.600pt}{0.113pt}}
\multiput(456.00,452.17)(3.755,-4.000){2}{\rule{0.300pt}{0.400pt}}
\multiput(461.00,447.95)(0.909,-0.447){3}{\rule{0.767pt}{0.108pt}}
\multiput(461.00,448.17)(3.409,-3.000){2}{\rule{0.383pt}{0.400pt}}
\multiput(466.00,444.95)(1.132,-0.447){3}{\rule{0.900pt}{0.108pt}}
\multiput(466.00,445.17)(4.132,-3.000){2}{\rule{0.450pt}{0.400pt}}
\multiput(472.00,441.95)(0.909,-0.447){3}{\rule{0.767pt}{0.108pt}}
\multiput(472.00,442.17)(3.409,-3.000){2}{\rule{0.383pt}{0.400pt}}
\multiput(477.00,438.95)(1.132,-0.447){3}{\rule{0.900pt}{0.108pt}}
\multiput(477.00,439.17)(4.132,-3.000){2}{\rule{0.450pt}{0.400pt}}
\multiput(483.00,435.95)(0.909,-0.447){3}{\rule{0.767pt}{0.108pt}}
\multiput(483.00,436.17)(3.409,-3.000){2}{\rule{0.383pt}{0.400pt}}
\multiput(488.00,432.95)(0.909,-0.447){3}{\rule{0.767pt}{0.108pt}}
\multiput(488.00,433.17)(3.409,-3.000){2}{\rule{0.383pt}{0.400pt}}
\put(493,429.17){\rule{1.300pt}{0.400pt}}
\multiput(493.00,430.17)(3.302,-2.000){2}{\rule{0.650pt}{0.400pt}}
\multiput(499.00,427.95)(0.909,-0.447){3}{\rule{0.767pt}{0.108pt}}
\multiput(499.00,428.17)(3.409,-3.000){2}{\rule{0.383pt}{0.400pt}}
\put(504,424.17){\rule{1.100pt}{0.400pt}}
\multiput(504.00,425.17)(2.717,-2.000){2}{\rule{0.550pt}{0.400pt}}
\multiput(509.00,422.95)(1.132,-0.447){3}{\rule{0.900pt}{0.108pt}}
\multiput(509.00,423.17)(4.132,-3.000){2}{\rule{0.450pt}{0.400pt}}
\put(515,419.17){\rule{1.100pt}{0.400pt}}
\multiput(515.00,420.17)(2.717,-2.000){2}{\rule{0.550pt}{0.400pt}}
\put(520,417.17){\rule{1.300pt}{0.400pt}}
\multiput(520.00,418.17)(3.302,-2.000){2}{\rule{0.650pt}{0.400pt}}
\put(526,415.17){\rule{1.100pt}{0.400pt}}
\multiput(526.00,416.17)(2.717,-2.000){2}{\rule{0.550pt}{0.400pt}}
\put(531,413.17){\rule{1.100pt}{0.400pt}}
\multiput(531.00,414.17)(2.717,-2.000){2}{\rule{0.550pt}{0.400pt}}
\put(536,411.17){\rule{1.300pt}{0.400pt}}
\multiput(536.00,412.17)(3.302,-2.000){2}{\rule{0.650pt}{0.400pt}}
\put(542,409.17){\rule{1.100pt}{0.400pt}}
\multiput(542.00,410.17)(2.717,-2.000){2}{\rule{0.550pt}{0.400pt}}
\put(547,407.67){\rule{1.445pt}{0.400pt}}
\multiput(547.00,408.17)(3.000,-1.000){2}{\rule{0.723pt}{0.400pt}}
\put(553,406.17){\rule{1.100pt}{0.400pt}}
\multiput(553.00,407.17)(2.717,-2.000){2}{\rule{0.550pt}{0.400pt}}
\put(558,404.67){\rule{1.204pt}{0.400pt}}
\multiput(558.00,405.17)(2.500,-1.000){2}{\rule{0.602pt}{0.400pt}}
\put(563,403.17){\rule{1.300pt}{0.400pt}}
\multiput(563.00,404.17)(3.302,-2.000){2}{\rule{0.650pt}{0.400pt}}
\put(569,401.67){\rule{1.204pt}{0.400pt}}
\multiput(569.00,402.17)(2.500,-1.000){2}{\rule{0.602pt}{0.400pt}}
\put(574,400.67){\rule{1.445pt}{0.400pt}}
\multiput(574.00,401.17)(3.000,-1.000){2}{\rule{0.723pt}{0.400pt}}
\put(580,399.67){\rule{1.204pt}{0.400pt}}
\multiput(580.00,400.17)(2.500,-1.000){2}{\rule{0.602pt}{0.400pt}}
\put(585,398.67){\rule{1.204pt}{0.400pt}}
\multiput(585.00,399.17)(2.500,-1.000){2}{\rule{0.602pt}{0.400pt}}
\put(590,397.67){\rule{1.445pt}{0.400pt}}
\multiput(590.00,398.17)(3.000,-1.000){2}{\rule{0.723pt}{0.400pt}}
\put(596,396.67){\rule{1.204pt}{0.400pt}}
\multiput(596.00,397.17)(2.500,-1.000){2}{\rule{0.602pt}{0.400pt}}
\put(648.0,390.0){\rule[-0.200pt]{1.204pt}{0.400pt}}
\put(607,395.67){\rule{1.204pt}{0.400pt}}
\multiput(607.00,396.17)(2.500,-1.000){2}{\rule{0.602pt}{0.400pt}}
\put(601.0,397.0){\rule[-0.200pt]{1.445pt}{0.400pt}}
\put(617,394.67){\rule{1.445pt}{0.400pt}}
\multiput(617.00,395.17)(3.000,-1.000){2}{\rule{0.723pt}{0.400pt}}
\put(612.0,396.0){\rule[-0.200pt]{1.204pt}{0.400pt}}
\put(623.0,395.0){\rule[-0.200pt]{1.204pt}{0.400pt}}
\put(628.0,395.0){\rule[-0.200pt]{1.445pt}{0.400pt}}
\put(634.0,395.0){\rule[-0.200pt]{1.204pt}{0.400pt}}
\put(639.0,395.0){\rule[-0.200pt]{1.204pt}{0.400pt}}
\put(644.0,395.0){\rule[-0.200pt]{1.445pt}{0.400pt}}
\put(650.0,395.0){\rule[-0.200pt]{1.204pt}{0.400pt}}
\put(661,394.67){\rule{1.204pt}{0.400pt}}
\multiput(661.00,394.17)(2.500,1.000){2}{\rule{0.602pt}{0.400pt}}
\put(655.0,395.0){\rule[-0.200pt]{1.445pt}{0.400pt}}
\put(671,395.67){\rule{1.445pt}{0.400pt}}
\multiput(671.00,395.17)(3.000,1.000){2}{\rule{0.723pt}{0.400pt}}
\put(677,396.67){\rule{1.204pt}{0.400pt}}
\multiput(677.00,396.17)(2.500,1.000){2}{\rule{0.602pt}{0.400pt}}
\put(682,397.67){\rule{1.445pt}{0.400pt}}
\multiput(682.00,397.17)(3.000,1.000){2}{\rule{0.723pt}{0.400pt}}
\put(666.0,396.0){\rule[-0.200pt]{1.204pt}{0.400pt}}
\put(693,399.17){\rule{1.100pt}{0.400pt}}
\multiput(693.00,398.17)(2.717,2.000){2}{\rule{0.550pt}{0.400pt}}
\put(698,400.67){\rule{1.445pt}{0.400pt}}
\multiput(698.00,400.17)(3.000,1.000){2}{\rule{0.723pt}{0.400pt}}
\put(704,401.67){\rule{1.204pt}{0.400pt}}
\multiput(704.00,401.17)(2.500,1.000){2}{\rule{0.602pt}{0.400pt}}
\put(709,402.67){\rule{1.445pt}{0.400pt}}
\multiput(709.00,402.17)(3.000,1.000){2}{\rule{0.723pt}{0.400pt}}
\put(715,404.17){\rule{1.100pt}{0.400pt}}
\multiput(715.00,403.17)(2.717,2.000){2}{\rule{0.550pt}{0.400pt}}
\put(720,405.67){\rule{1.204pt}{0.400pt}}
\multiput(720.00,405.17)(2.500,1.000){2}{\rule{0.602pt}{0.400pt}}
\put(725,407.17){\rule{1.300pt}{0.400pt}}
\multiput(725.00,406.17)(3.302,2.000){2}{\rule{0.650pt}{0.400pt}}
\put(731,408.67){\rule{1.204pt}{0.400pt}}
\multiput(731.00,408.17)(2.500,1.000){2}{\rule{0.602pt}{0.400pt}}
\put(736,410.17){\rule{1.300pt}{0.400pt}}
\multiput(736.00,409.17)(3.302,2.000){2}{\rule{0.650pt}{0.400pt}}
\put(742,412.17){\rule{1.100pt}{0.400pt}}
\multiput(742.00,411.17)(2.717,2.000){2}{\rule{0.550pt}{0.400pt}}
\put(747,414.17){\rule{1.100pt}{0.400pt}}
\multiput(747.00,413.17)(2.717,2.000){2}{\rule{0.550pt}{0.400pt}}
\put(752,416.17){\rule{1.300pt}{0.400pt}}
\multiput(752.00,415.17)(3.302,2.000){2}{\rule{0.650pt}{0.400pt}}
\put(758,418.17){\rule{1.100pt}{0.400pt}}
\multiput(758.00,417.17)(2.717,2.000){2}{\rule{0.550pt}{0.400pt}}
\multiput(763.00,420.61)(1.132,0.447){3}{\rule{0.900pt}{0.108pt}}
\multiput(763.00,419.17)(4.132,3.000){2}{\rule{0.450pt}{0.400pt}}
\put(769,423.17){\rule{1.100pt}{0.400pt}}
\multiput(769.00,422.17)(2.717,2.000){2}{\rule{0.550pt}{0.400pt}}
\multiput(774.00,425.61)(0.909,0.447){3}{\rule{0.767pt}{0.108pt}}
\multiput(774.00,424.17)(3.409,3.000){2}{\rule{0.383pt}{0.400pt}}
\put(779,428.17){\rule{1.300pt}{0.400pt}}
\multiput(779.00,427.17)(3.302,2.000){2}{\rule{0.650pt}{0.400pt}}
\multiput(785.00,430.61)(0.909,0.447){3}{\rule{0.767pt}{0.108pt}}
\multiput(785.00,429.17)(3.409,3.000){2}{\rule{0.383pt}{0.400pt}}
\multiput(790.00,433.61)(1.132,0.447){3}{\rule{0.900pt}{0.108pt}}
\multiput(790.00,432.17)(4.132,3.000){2}{\rule{0.450pt}{0.400pt}}
\multiput(796.00,436.61)(0.909,0.447){3}{\rule{0.767pt}{0.108pt}}
\multiput(796.00,435.17)(3.409,3.000){2}{\rule{0.383pt}{0.400pt}}
\multiput(801.00,439.61)(0.909,0.447){3}{\rule{0.767pt}{0.108pt}}
\multiput(801.00,438.17)(3.409,3.000){2}{\rule{0.383pt}{0.400pt}}
\multiput(806.00,442.61)(1.132,0.447){3}{\rule{0.900pt}{0.108pt}}
\multiput(806.00,441.17)(4.132,3.000){2}{\rule{0.450pt}{0.400pt}}
\multiput(812.00,445.61)(0.909,0.447){3}{\rule{0.767pt}{0.108pt}}
\multiput(812.00,444.17)(3.409,3.000){2}{\rule{0.383pt}{0.400pt}}
\multiput(817.00,448.61)(1.132,0.447){3}{\rule{0.900pt}{0.108pt}}
\multiput(817.00,447.17)(4.132,3.000){2}{\rule{0.450pt}{0.400pt}}
\multiput(823.00,451.61)(0.909,0.447){3}{\rule{0.767pt}{0.108pt}}
\multiput(823.00,450.17)(3.409,3.000){2}{\rule{0.383pt}{0.400pt}}
\multiput(828.00,454.60)(0.627,0.468){5}{\rule{0.600pt}{0.113pt}}
\multiput(828.00,453.17)(3.755,4.000){2}{\rule{0.300pt}{0.400pt}}
\multiput(833.00,458.60)(0.774,0.468){5}{\rule{0.700pt}{0.113pt}}
\multiput(833.00,457.17)(4.547,4.000){2}{\rule{0.350pt}{0.400pt}}
\multiput(839.00,462.61)(0.909,0.447){3}{\rule{0.767pt}{0.108pt}}
\multiput(839.00,461.17)(3.409,3.000){2}{\rule{0.383pt}{0.400pt}}
\multiput(844.00,465.60)(0.774,0.468){5}{\rule{0.700pt}{0.113pt}}
\multiput(844.00,464.17)(4.547,4.000){2}{\rule{0.350pt}{0.400pt}}
\multiput(349.00,592.93)(0.491,-0.482){9}{\rule{0.500pt}{0.116pt}}
\multiput(349.00,593.17)(4.962,-6.000){2}{\rule{0.250pt}{0.400pt}}
\multiput(355.59,585.59)(0.477,-0.599){7}{\rule{0.115pt}{0.580pt}}
\multiput(354.17,586.80)(5.000,-4.796){2}{\rule{0.400pt}{0.290pt}}
\multiput(360.00,580.93)(0.599,-0.477){7}{\rule{0.580pt}{0.115pt}}
\multiput(360.00,581.17)(4.796,-5.000){2}{\rule{0.290pt}{0.400pt}}
\multiput(366.59,574.59)(0.477,-0.599){7}{\rule{0.115pt}{0.580pt}}
\multiput(365.17,575.80)(5.000,-4.796){2}{\rule{0.400pt}{0.290pt}}
\multiput(371.59,568.59)(0.477,-0.599){7}{\rule{0.115pt}{0.580pt}}
\multiput(370.17,569.80)(5.000,-4.796){2}{\rule{0.400pt}{0.290pt}}
\multiput(376.00,563.93)(0.599,-0.477){7}{\rule{0.580pt}{0.115pt}}
\multiput(376.00,564.17)(4.796,-5.000){2}{\rule{0.290pt}{0.400pt}}
\multiput(382.00,558.93)(0.487,-0.477){7}{\rule{0.500pt}{0.115pt}}
\multiput(382.00,559.17)(3.962,-5.000){2}{\rule{0.250pt}{0.400pt}}
\multiput(387.00,553.93)(0.599,-0.477){7}{\rule{0.580pt}{0.115pt}}
\multiput(387.00,554.17)(4.796,-5.000){2}{\rule{0.290pt}{0.400pt}}
\multiput(393.59,547.59)(0.477,-0.599){7}{\rule{0.115pt}{0.580pt}}
\multiput(392.17,548.80)(5.000,-4.796){2}{\rule{0.400pt}{0.290pt}}
\multiput(398.00,542.93)(0.487,-0.477){7}{\rule{0.500pt}{0.115pt}}
\multiput(398.00,543.17)(3.962,-5.000){2}{\rule{0.250pt}{0.400pt}}
\multiput(403.00,537.94)(0.774,-0.468){5}{\rule{0.700pt}{0.113pt}}
\multiput(403.00,538.17)(4.547,-4.000){2}{\rule{0.350pt}{0.400pt}}
\multiput(409.00,533.93)(0.487,-0.477){7}{\rule{0.500pt}{0.115pt}}
\multiput(409.00,534.17)(3.962,-5.000){2}{\rule{0.250pt}{0.400pt}}
\multiput(414.00,528.93)(0.599,-0.477){7}{\rule{0.580pt}{0.115pt}}
\multiput(414.00,529.17)(4.796,-5.000){2}{\rule{0.290pt}{0.400pt}}
\multiput(420.00,523.93)(0.487,-0.477){7}{\rule{0.500pt}{0.115pt}}
\multiput(420.00,524.17)(3.962,-5.000){2}{\rule{0.250pt}{0.400pt}}
\multiput(425.00,518.94)(0.627,-0.468){5}{\rule{0.600pt}{0.113pt}}
\multiput(425.00,519.17)(3.755,-4.000){2}{\rule{0.300pt}{0.400pt}}
\multiput(430.00,514.93)(0.599,-0.477){7}{\rule{0.580pt}{0.115pt}}
\multiput(430.00,515.17)(4.796,-5.000){2}{\rule{0.290pt}{0.400pt}}
\multiput(436.00,509.94)(0.627,-0.468){5}{\rule{0.600pt}{0.113pt}}
\multiput(436.00,510.17)(3.755,-4.000){2}{\rule{0.300pt}{0.400pt}}
\multiput(441.00,505.94)(0.774,-0.468){5}{\rule{0.700pt}{0.113pt}}
\multiput(441.00,506.17)(4.547,-4.000){2}{\rule{0.350pt}{0.400pt}}
\multiput(447.00,501.94)(0.627,-0.468){5}{\rule{0.600pt}{0.113pt}}
\multiput(447.00,502.17)(3.755,-4.000){2}{\rule{0.300pt}{0.400pt}}
\multiput(452.00,497.94)(0.627,-0.468){5}{\rule{0.600pt}{0.113pt}}
\multiput(452.00,498.17)(3.755,-4.000){2}{\rule{0.300pt}{0.400pt}}
\multiput(457.00,493.94)(0.774,-0.468){5}{\rule{0.700pt}{0.113pt}}
\multiput(457.00,494.17)(4.547,-4.000){2}{\rule{0.350pt}{0.400pt}}
\multiput(463.00,489.94)(0.627,-0.468){5}{\rule{0.600pt}{0.113pt}}
\multiput(463.00,490.17)(3.755,-4.000){2}{\rule{0.300pt}{0.400pt}}
\multiput(468.00,485.94)(0.774,-0.468){5}{\rule{0.700pt}{0.113pt}}
\multiput(468.00,486.17)(4.547,-4.000){2}{\rule{0.350pt}{0.400pt}}
\multiput(474.00,481.95)(0.909,-0.447){3}{\rule{0.767pt}{0.108pt}}
\multiput(474.00,482.17)(3.409,-3.000){2}{\rule{0.383pt}{0.400pt}}
\multiput(479.00,478.94)(0.627,-0.468){5}{\rule{0.600pt}{0.113pt}}
\multiput(479.00,479.17)(3.755,-4.000){2}{\rule{0.300pt}{0.400pt}}
\multiput(484.00,474.95)(1.132,-0.447){3}{\rule{0.900pt}{0.108pt}}
\multiput(484.00,475.17)(4.132,-3.000){2}{\rule{0.450pt}{0.400pt}}
\multiput(490.00,471.94)(0.627,-0.468){5}{\rule{0.600pt}{0.113pt}}
\multiput(490.00,472.17)(3.755,-4.000){2}{\rule{0.300pt}{0.400pt}}
\multiput(495.00,467.95)(1.132,-0.447){3}{\rule{0.900pt}{0.108pt}}
\multiput(495.00,468.17)(4.132,-3.000){2}{\rule{0.450pt}{0.400pt}}
\multiput(501.00,464.95)(0.909,-0.447){3}{\rule{0.767pt}{0.108pt}}
\multiput(501.00,465.17)(3.409,-3.000){2}{\rule{0.383pt}{0.400pt}}
\multiput(506.00,461.95)(0.909,-0.447){3}{\rule{0.767pt}{0.108pt}}
\multiput(506.00,462.17)(3.409,-3.000){2}{\rule{0.383pt}{0.400pt}}
\multiput(511.00,458.95)(1.132,-0.447){3}{\rule{0.900pt}{0.108pt}}
\multiput(511.00,459.17)(4.132,-3.000){2}{\rule{0.450pt}{0.400pt}}
\multiput(517.00,455.95)(0.909,-0.447){3}{\rule{0.767pt}{0.108pt}}
\multiput(517.00,456.17)(3.409,-3.000){2}{\rule{0.383pt}{0.400pt}}
\multiput(522.00,452.95)(1.132,-0.447){3}{\rule{0.900pt}{0.108pt}}
\multiput(522.00,453.17)(4.132,-3.000){2}{\rule{0.450pt}{0.400pt}}
\put(528,449.17){\rule{1.100pt}{0.400pt}}
\multiput(528.00,450.17)(2.717,-2.000){2}{\rule{0.550pt}{0.400pt}}
\multiput(533.00,447.95)(0.909,-0.447){3}{\rule{0.767pt}{0.108pt}}
\multiput(533.00,448.17)(3.409,-3.000){2}{\rule{0.383pt}{0.400pt}}
\put(538,444.17){\rule{1.300pt}{0.400pt}}
\multiput(538.00,445.17)(3.302,-2.000){2}{\rule{0.650pt}{0.400pt}}
\put(544,442.17){\rule{1.100pt}{0.400pt}}
\multiput(544.00,443.17)(2.717,-2.000){2}{\rule{0.550pt}{0.400pt}}
\multiput(549.00,440.95)(1.132,-0.447){3}{\rule{0.900pt}{0.108pt}}
\multiput(549.00,441.17)(4.132,-3.000){2}{\rule{0.450pt}{0.400pt}}
\put(555,437.17){\rule{1.100pt}{0.400pt}}
\multiput(555.00,438.17)(2.717,-2.000){2}{\rule{0.550pt}{0.400pt}}
\put(560,435.17){\rule{1.100pt}{0.400pt}}
\multiput(560.00,436.17)(2.717,-2.000){2}{\rule{0.550pt}{0.400pt}}
\put(565,433.17){\rule{1.300pt}{0.400pt}}
\multiput(565.00,434.17)(3.302,-2.000){2}{\rule{0.650pt}{0.400pt}}
\put(571,431.17){\rule{1.100pt}{0.400pt}}
\multiput(571.00,432.17)(2.717,-2.000){2}{\rule{0.550pt}{0.400pt}}
\put(576,429.67){\rule{1.445pt}{0.400pt}}
\multiput(576.00,430.17)(3.000,-1.000){2}{\rule{0.723pt}{0.400pt}}
\put(582,428.17){\rule{1.100pt}{0.400pt}}
\multiput(582.00,429.17)(2.717,-2.000){2}{\rule{0.550pt}{0.400pt}}
\put(587,426.17){\rule{1.100pt}{0.400pt}}
\multiput(587.00,427.17)(2.717,-2.000){2}{\rule{0.550pt}{0.400pt}}
\put(592,424.67){\rule{1.445pt}{0.400pt}}
\multiput(592.00,425.17)(3.000,-1.000){2}{\rule{0.723pt}{0.400pt}}
\put(598,423.67){\rule{1.204pt}{0.400pt}}
\multiput(598.00,424.17)(2.500,-1.000){2}{\rule{0.602pt}{0.400pt}}
\put(603,422.17){\rule{1.300pt}{0.400pt}}
\multiput(603.00,423.17)(3.302,-2.000){2}{\rule{0.650pt}{0.400pt}}
\put(609,420.67){\rule{1.204pt}{0.400pt}}
\multiput(609.00,421.17)(2.500,-1.000){2}{\rule{0.602pt}{0.400pt}}
\put(614,419.67){\rule{1.204pt}{0.400pt}}
\multiput(614.00,420.17)(2.500,-1.000){2}{\rule{0.602pt}{0.400pt}}
\put(619,418.67){\rule{1.445pt}{0.400pt}}
\multiput(619.00,419.17)(3.000,-1.000){2}{\rule{0.723pt}{0.400pt}}
\put(625,417.67){\rule{1.204pt}{0.400pt}}
\multiput(625.00,418.17)(2.500,-1.000){2}{\rule{0.602pt}{0.400pt}}
\put(630,416.67){\rule{1.445pt}{0.400pt}}
\multiput(630.00,417.17)(3.000,-1.000){2}{\rule{0.723pt}{0.400pt}}
\put(688.0,399.0){\rule[-0.200pt]{1.204pt}{0.400pt}}
\put(641,415.67){\rule{1.204pt}{0.400pt}}
\multiput(641.00,416.17)(2.500,-1.000){2}{\rule{0.602pt}{0.400pt}}
\put(636.0,417.0){\rule[-0.200pt]{1.204pt}{0.400pt}}
\put(652,414.67){\rule{1.204pt}{0.400pt}}
\multiput(652.00,415.17)(2.500,-1.000){2}{\rule{0.602pt}{0.400pt}}
\put(646.0,416.0){\rule[-0.200pt]{1.445pt}{0.400pt}}
\put(657.0,415.0){\rule[-0.200pt]{1.445pt}{0.400pt}}
\put(663.0,415.0){\rule[-0.200pt]{1.204pt}{0.400pt}}
\put(668.0,415.0){\rule[-0.200pt]{1.204pt}{0.400pt}}
\put(673.0,415.0){\rule[-0.200pt]{1.445pt}{0.400pt}}
\put(679.0,415.0){\rule[-0.200pt]{1.204pt}{0.400pt}}
\put(690,414.67){\rule{1.204pt}{0.400pt}}
\multiput(690.00,414.17)(2.500,1.000){2}{\rule{0.602pt}{0.400pt}}
\put(684.0,415.0){\rule[-0.200pt]{1.445pt}{0.400pt}}
\put(700,415.67){\rule{1.445pt}{0.400pt}}
\multiput(700.00,415.17)(3.000,1.000){2}{\rule{0.723pt}{0.400pt}}
\put(695.0,416.0){\rule[-0.200pt]{1.204pt}{0.400pt}}
\put(711,416.67){\rule{1.204pt}{0.400pt}}
\multiput(711.00,416.17)(2.500,1.000){2}{\rule{0.602pt}{0.400pt}}
\put(716,417.67){\rule{1.445pt}{0.400pt}}
\multiput(716.00,417.17)(3.000,1.000){2}{\rule{0.723pt}{0.400pt}}
\put(722,418.67){\rule{1.204pt}{0.400pt}}
\multiput(722.00,418.17)(2.500,1.000){2}{\rule{0.602pt}{0.400pt}}
\put(727,419.67){\rule{1.445pt}{0.400pt}}
\multiput(727.00,419.17)(3.000,1.000){2}{\rule{0.723pt}{0.400pt}}
\put(733,420.67){\rule{1.204pt}{0.400pt}}
\multiput(733.00,420.17)(2.500,1.000){2}{\rule{0.602pt}{0.400pt}}
\put(738,421.67){\rule{1.204pt}{0.400pt}}
\multiput(738.00,421.17)(2.500,1.000){2}{\rule{0.602pt}{0.400pt}}
\put(743,422.67){\rule{1.445pt}{0.400pt}}
\multiput(743.00,422.17)(3.000,1.000){2}{\rule{0.723pt}{0.400pt}}
\put(749,424.17){\rule{1.100pt}{0.400pt}}
\multiput(749.00,423.17)(2.717,2.000){2}{\rule{0.550pt}{0.400pt}}
\put(754,425.67){\rule{1.445pt}{0.400pt}}
\multiput(754.00,425.17)(3.000,1.000){2}{\rule{0.723pt}{0.400pt}}
\put(760,427.17){\rule{1.100pt}{0.400pt}}
\multiput(760.00,426.17)(2.717,2.000){2}{\rule{0.550pt}{0.400pt}}
\put(765,429.17){\rule{1.100pt}{0.400pt}}
\multiput(765.00,428.17)(2.717,2.000){2}{\rule{0.550pt}{0.400pt}}
\put(770,430.67){\rule{1.445pt}{0.400pt}}
\multiput(770.00,430.17)(3.000,1.000){2}{\rule{0.723pt}{0.400pt}}
\put(776,432.17){\rule{1.100pt}{0.400pt}}
\multiput(776.00,431.17)(2.717,2.000){2}{\rule{0.550pt}{0.400pt}}
\put(781,434.17){\rule{1.300pt}{0.400pt}}
\multiput(781.00,433.17)(3.302,2.000){2}{\rule{0.650pt}{0.400pt}}
\put(787,436.17){\rule{1.100pt}{0.400pt}}
\multiput(787.00,435.17)(2.717,2.000){2}{\rule{0.550pt}{0.400pt}}
\multiput(792.00,438.61)(0.909,0.447){3}{\rule{0.767pt}{0.108pt}}
\multiput(792.00,437.17)(3.409,3.000){2}{\rule{0.383pt}{0.400pt}}
\put(797,441.17){\rule{1.300pt}{0.400pt}}
\multiput(797.00,440.17)(3.302,2.000){2}{\rule{0.650pt}{0.400pt}}
\put(803,443.17){\rule{1.100pt}{0.400pt}}
\multiput(803.00,442.17)(2.717,2.000){2}{\rule{0.550pt}{0.400pt}}
\multiput(808.00,445.61)(1.132,0.447){3}{\rule{0.900pt}{0.108pt}}
\multiput(808.00,444.17)(4.132,3.000){2}{\rule{0.450pt}{0.400pt}}
\put(814,448.17){\rule{1.100pt}{0.400pt}}
\multiput(814.00,447.17)(2.717,2.000){2}{\rule{0.550pt}{0.400pt}}
\multiput(819.00,450.61)(0.909,0.447){3}{\rule{0.767pt}{0.108pt}}
\multiput(819.00,449.17)(3.409,3.000){2}{\rule{0.383pt}{0.400pt}}
\multiput(824.00,453.61)(1.132,0.447){3}{\rule{0.900pt}{0.108pt}}
\multiput(824.00,452.17)(4.132,3.000){2}{\rule{0.450pt}{0.400pt}}
\multiput(830.00,456.61)(0.909,0.447){3}{\rule{0.767pt}{0.108pt}}
\multiput(830.00,455.17)(3.409,3.000){2}{\rule{0.383pt}{0.400pt}}
\multiput(835.00,459.61)(1.132,0.447){3}{\rule{0.900pt}{0.108pt}}
\multiput(835.00,458.17)(4.132,3.000){2}{\rule{0.450pt}{0.400pt}}
\multiput(841.00,462.61)(0.909,0.447){3}{\rule{0.767pt}{0.108pt}}
\multiput(841.00,461.17)(3.409,3.000){2}{\rule{0.383pt}{0.400pt}}
\multiput(846.00,465.61)(0.909,0.447){3}{\rule{0.767pt}{0.108pt}}
\multiput(846.00,464.17)(3.409,3.000){2}{\rule{0.383pt}{0.400pt}}
\multiput(851.00,468.61)(1.132,0.447){3}{\rule{0.900pt}{0.108pt}}
\multiput(851.00,467.17)(4.132,3.000){2}{\rule{0.450pt}{0.400pt}}
\multiput(857.00,471.60)(0.627,0.468){5}{\rule{0.600pt}{0.113pt}}
\multiput(857.00,470.17)(3.755,4.000){2}{\rule{0.300pt}{0.400pt}}
\multiput(862.00,475.61)(1.132,0.447){3}{\rule{0.900pt}{0.108pt}}
\multiput(862.00,474.17)(4.132,3.000){2}{\rule{0.450pt}{0.400pt}}
\multiput(868.00,478.60)(0.627,0.468){5}{\rule{0.600pt}{0.113pt}}
\multiput(868.00,477.17)(3.755,4.000){2}{\rule{0.300pt}{0.400pt}}
\multiput(873.00,482.61)(0.909,0.447){3}{\rule{0.767pt}{0.108pt}}
\multiput(873.00,481.17)(3.409,3.000){2}{\rule{0.383pt}{0.400pt}}
\multiput(878.00,485.60)(0.774,0.468){5}{\rule{0.700pt}{0.113pt}}
\multiput(878.00,484.17)(4.547,4.000){2}{\rule{0.350pt}{0.400pt}}
\multiput(384.59,623.59)(0.477,-0.599){7}{\rule{0.115pt}{0.580pt}}
\multiput(383.17,624.80)(5.000,-4.796){2}{\rule{0.400pt}{0.290pt}}
\multiput(389.00,618.93)(0.491,-0.482){9}{\rule{0.500pt}{0.116pt}}
\multiput(389.00,619.17)(4.962,-6.000){2}{\rule{0.250pt}{0.400pt}}
\multiput(395.00,612.93)(0.487,-0.477){7}{\rule{0.500pt}{0.115pt}}
\multiput(395.00,613.17)(3.962,-5.000){2}{\rule{0.250pt}{0.400pt}}
\multiput(400.59,606.59)(0.477,-0.599){7}{\rule{0.115pt}{0.580pt}}
\multiput(399.17,607.80)(5.000,-4.796){2}{\rule{0.400pt}{0.290pt}}
\multiput(405.00,601.93)(0.599,-0.477){7}{\rule{0.580pt}{0.115pt}}
\multiput(405.00,602.17)(4.796,-5.000){2}{\rule{0.290pt}{0.400pt}}
\multiput(411.59,595.59)(0.477,-0.599){7}{\rule{0.115pt}{0.580pt}}
\multiput(410.17,596.80)(5.000,-4.796){2}{\rule{0.400pt}{0.290pt}}
\multiput(416.00,590.93)(0.599,-0.477){7}{\rule{0.580pt}{0.115pt}}
\multiput(416.00,591.17)(4.796,-5.000){2}{\rule{0.290pt}{0.400pt}}
\multiput(422.00,585.93)(0.487,-0.477){7}{\rule{0.500pt}{0.115pt}}
\multiput(422.00,586.17)(3.962,-5.000){2}{\rule{0.250pt}{0.400pt}}
\multiput(427.00,580.93)(0.487,-0.477){7}{\rule{0.500pt}{0.115pt}}
\multiput(427.00,581.17)(3.962,-5.000){2}{\rule{0.250pt}{0.400pt}}
\multiput(432.00,575.93)(0.599,-0.477){7}{\rule{0.580pt}{0.115pt}}
\multiput(432.00,576.17)(4.796,-5.000){2}{\rule{0.290pt}{0.400pt}}
\multiput(438.00,570.93)(0.487,-0.477){7}{\rule{0.500pt}{0.115pt}}
\multiput(438.00,571.17)(3.962,-5.000){2}{\rule{0.250pt}{0.400pt}}
\multiput(443.00,565.93)(0.599,-0.477){7}{\rule{0.580pt}{0.115pt}}
\multiput(443.00,566.17)(4.796,-5.000){2}{\rule{0.290pt}{0.400pt}}
\multiput(449.00,560.93)(0.487,-0.477){7}{\rule{0.500pt}{0.115pt}}
\multiput(449.00,561.17)(3.962,-5.000){2}{\rule{0.250pt}{0.400pt}}
\multiput(454.00,555.93)(0.487,-0.477){7}{\rule{0.500pt}{0.115pt}}
\multiput(454.00,556.17)(3.962,-5.000){2}{\rule{0.250pt}{0.400pt}}
\multiput(459.00,550.94)(0.774,-0.468){5}{\rule{0.700pt}{0.113pt}}
\multiput(459.00,551.17)(4.547,-4.000){2}{\rule{0.350pt}{0.400pt}}
\multiput(465.00,546.94)(0.627,-0.468){5}{\rule{0.600pt}{0.113pt}}
\multiput(465.00,547.17)(3.755,-4.000){2}{\rule{0.300pt}{0.400pt}}
\multiput(470.00,542.93)(0.599,-0.477){7}{\rule{0.580pt}{0.115pt}}
\multiput(470.00,543.17)(4.796,-5.000){2}{\rule{0.290pt}{0.400pt}}
\multiput(476.00,537.94)(0.627,-0.468){5}{\rule{0.600pt}{0.113pt}}
\multiput(476.00,538.17)(3.755,-4.000){2}{\rule{0.300pt}{0.400pt}}
\multiput(481.00,533.94)(0.627,-0.468){5}{\rule{0.600pt}{0.113pt}}
\multiput(481.00,534.17)(3.755,-4.000){2}{\rule{0.300pt}{0.400pt}}
\multiput(486.00,529.94)(0.774,-0.468){5}{\rule{0.700pt}{0.113pt}}
\multiput(486.00,530.17)(4.547,-4.000){2}{\rule{0.350pt}{0.400pt}}
\multiput(492.00,525.94)(0.627,-0.468){5}{\rule{0.600pt}{0.113pt}}
\multiput(492.00,526.17)(3.755,-4.000){2}{\rule{0.300pt}{0.400pt}}
\multiput(497.00,521.94)(0.627,-0.468){5}{\rule{0.600pt}{0.113pt}}
\multiput(497.00,522.17)(3.755,-4.000){2}{\rule{0.300pt}{0.400pt}}
\multiput(502.00,517.94)(0.774,-0.468){5}{\rule{0.700pt}{0.113pt}}
\multiput(502.00,518.17)(4.547,-4.000){2}{\rule{0.350pt}{0.400pt}}
\multiput(508.00,513.95)(0.909,-0.447){3}{\rule{0.767pt}{0.108pt}}
\multiput(508.00,514.17)(3.409,-3.000){2}{\rule{0.383pt}{0.400pt}}
\multiput(513.00,510.94)(0.774,-0.468){5}{\rule{0.700pt}{0.113pt}}
\multiput(513.00,511.17)(4.547,-4.000){2}{\rule{0.350pt}{0.400pt}}
\multiput(519.00,506.95)(0.909,-0.447){3}{\rule{0.767pt}{0.108pt}}
\multiput(519.00,507.17)(3.409,-3.000){2}{\rule{0.383pt}{0.400pt}}
\multiput(524.00,503.94)(0.627,-0.468){5}{\rule{0.600pt}{0.113pt}}
\multiput(524.00,504.17)(3.755,-4.000){2}{\rule{0.300pt}{0.400pt}}
\multiput(529.00,499.95)(1.132,-0.447){3}{\rule{0.900pt}{0.108pt}}
\multiput(529.00,500.17)(4.132,-3.000){2}{\rule{0.450pt}{0.400pt}}
\multiput(535.00,496.95)(0.909,-0.447){3}{\rule{0.767pt}{0.108pt}}
\multiput(535.00,497.17)(3.409,-3.000){2}{\rule{0.383pt}{0.400pt}}
\multiput(540.00,493.95)(1.132,-0.447){3}{\rule{0.900pt}{0.108pt}}
\multiput(540.00,494.17)(4.132,-3.000){2}{\rule{0.450pt}{0.400pt}}
\multiput(546.00,490.95)(0.909,-0.447){3}{\rule{0.767pt}{0.108pt}}
\multiput(546.00,491.17)(3.409,-3.000){2}{\rule{0.383pt}{0.400pt}}
\multiput(551.00,487.95)(0.909,-0.447){3}{\rule{0.767pt}{0.108pt}}
\multiput(551.00,488.17)(3.409,-3.000){2}{\rule{0.383pt}{0.400pt}}
\put(556,484.17){\rule{1.300pt}{0.400pt}}
\multiput(556.00,485.17)(3.302,-2.000){2}{\rule{0.650pt}{0.400pt}}
\multiput(562.00,482.95)(0.909,-0.447){3}{\rule{0.767pt}{0.108pt}}
\multiput(562.00,483.17)(3.409,-3.000){2}{\rule{0.383pt}{0.400pt}}
\multiput(567.00,479.95)(1.132,-0.447){3}{\rule{0.900pt}{0.108pt}}
\multiput(567.00,480.17)(4.132,-3.000){2}{\rule{0.450pt}{0.400pt}}
\put(573,476.17){\rule{1.100pt}{0.400pt}}
\multiput(573.00,477.17)(2.717,-2.000){2}{\rule{0.550pt}{0.400pt}}
\put(578,474.17){\rule{1.100pt}{0.400pt}}
\multiput(578.00,475.17)(2.717,-2.000){2}{\rule{0.550pt}{0.400pt}}
\multiput(583.00,472.95)(1.132,-0.447){3}{\rule{0.900pt}{0.108pt}}
\multiput(583.00,473.17)(4.132,-3.000){2}{\rule{0.450pt}{0.400pt}}
\put(589,469.17){\rule{1.100pt}{0.400pt}}
\multiput(589.00,470.17)(2.717,-2.000){2}{\rule{0.550pt}{0.400pt}}
\put(594,467.17){\rule{1.300pt}{0.400pt}}
\multiput(594.00,468.17)(3.302,-2.000){2}{\rule{0.650pt}{0.400pt}}
\put(600,465.17){\rule{1.100pt}{0.400pt}}
\multiput(600.00,466.17)(2.717,-2.000){2}{\rule{0.550pt}{0.400pt}}
\put(605,463.17){\rule{1.100pt}{0.400pt}}
\multiput(605.00,464.17)(2.717,-2.000){2}{\rule{0.550pt}{0.400pt}}
\put(610,461.67){\rule{1.445pt}{0.400pt}}
\multiput(610.00,462.17)(3.000,-1.000){2}{\rule{0.723pt}{0.400pt}}
\put(616,460.17){\rule{1.100pt}{0.400pt}}
\multiput(616.00,461.17)(2.717,-2.000){2}{\rule{0.550pt}{0.400pt}}
\put(621,458.17){\rule{1.300pt}{0.400pt}}
\multiput(621.00,459.17)(3.302,-2.000){2}{\rule{0.650pt}{0.400pt}}
\put(627,456.67){\rule{1.204pt}{0.400pt}}
\multiput(627.00,457.17)(2.500,-1.000){2}{\rule{0.602pt}{0.400pt}}
\put(632,455.67){\rule{1.204pt}{0.400pt}}
\multiput(632.00,456.17)(2.500,-1.000){2}{\rule{0.602pt}{0.400pt}}
\put(637,454.17){\rule{1.300pt}{0.400pt}}
\multiput(637.00,455.17)(3.302,-2.000){2}{\rule{0.650pt}{0.400pt}}
\put(643,452.67){\rule{1.204pt}{0.400pt}}
\multiput(643.00,453.17)(2.500,-1.000){2}{\rule{0.602pt}{0.400pt}}
\put(648,451.67){\rule{1.445pt}{0.400pt}}
\multiput(648.00,452.17)(3.000,-1.000){2}{\rule{0.723pt}{0.400pt}}
\put(654,450.67){\rule{1.204pt}{0.400pt}}
\multiput(654.00,451.17)(2.500,-1.000){2}{\rule{0.602pt}{0.400pt}}
\put(659,449.67){\rule{1.204pt}{0.400pt}}
\multiput(659.00,450.17)(2.500,-1.000){2}{\rule{0.602pt}{0.400pt}}
\put(706.0,417.0){\rule[-0.200pt]{1.204pt}{0.400pt}}
\put(670,448.67){\rule{1.204pt}{0.400pt}}
\multiput(670.00,449.17)(2.500,-1.000){2}{\rule{0.602pt}{0.400pt}}
\put(675,447.67){\rule{1.445pt}{0.400pt}}
\multiput(675.00,448.17)(3.000,-1.000){2}{\rule{0.723pt}{0.400pt}}
\put(664.0,450.0){\rule[-0.200pt]{1.445pt}{0.400pt}}
\put(681.0,448.0){\rule[-0.200pt]{1.204pt}{0.400pt}}
\put(691,446.67){\rule{1.445pt}{0.400pt}}
\multiput(691.00,447.17)(3.000,-1.000){2}{\rule{0.723pt}{0.400pt}}
\put(686.0,448.0){\rule[-0.200pt]{1.204pt}{0.400pt}}
\put(697.0,447.0){\rule[-0.200pt]{1.204pt}{0.400pt}}
\put(702.0,447.0){\rule[-0.200pt]{1.445pt}{0.400pt}}
\put(708.0,447.0){\rule[-0.200pt]{1.204pt}{0.400pt}}
\put(713.0,447.0){\rule[-0.200pt]{1.204pt}{0.400pt}}
\put(724,446.67){\rule{1.204pt}{0.400pt}}
\multiput(724.00,446.17)(2.500,1.000){2}{\rule{0.602pt}{0.400pt}}
\put(718.0,447.0){\rule[-0.200pt]{1.445pt}{0.400pt}}
\put(735,447.67){\rule{1.204pt}{0.400pt}}
\multiput(735.00,447.17)(2.500,1.000){2}{\rule{0.602pt}{0.400pt}}
\put(729.0,448.0){\rule[-0.200pt]{1.445pt}{0.400pt}}
\put(745,448.67){\rule{1.445pt}{0.400pt}}
\multiput(745.00,448.17)(3.000,1.000){2}{\rule{0.723pt}{0.400pt}}
\put(751,449.67){\rule{1.204pt}{0.400pt}}
\multiput(751.00,449.17)(2.500,1.000){2}{\rule{0.602pt}{0.400pt}}
\put(756,450.67){\rule{1.445pt}{0.400pt}}
\multiput(756.00,450.17)(3.000,1.000){2}{\rule{0.723pt}{0.400pt}}
\put(762,451.67){\rule{1.204pt}{0.400pt}}
\multiput(762.00,451.17)(2.500,1.000){2}{\rule{0.602pt}{0.400pt}}
\put(767,452.67){\rule{1.204pt}{0.400pt}}
\multiput(767.00,452.17)(2.500,1.000){2}{\rule{0.602pt}{0.400pt}}
\put(772,453.67){\rule{1.445pt}{0.400pt}}
\multiput(772.00,453.17)(3.000,1.000){2}{\rule{0.723pt}{0.400pt}}
\put(778,454.67){\rule{1.204pt}{0.400pt}}
\multiput(778.00,454.17)(2.500,1.000){2}{\rule{0.602pt}{0.400pt}}
\put(783,456.17){\rule{1.300pt}{0.400pt}}
\multiput(783.00,455.17)(3.302,2.000){2}{\rule{0.650pt}{0.400pt}}
\put(789,457.67){\rule{1.204pt}{0.400pt}}
\multiput(789.00,457.17)(2.500,1.000){2}{\rule{0.602pt}{0.400pt}}
\put(794,459.17){\rule{1.100pt}{0.400pt}}
\multiput(794.00,458.17)(2.717,2.000){2}{\rule{0.550pt}{0.400pt}}
\put(799,461.17){\rule{1.300pt}{0.400pt}}
\multiput(799.00,460.17)(3.302,2.000){2}{\rule{0.650pt}{0.400pt}}
\put(805,462.67){\rule{1.204pt}{0.400pt}}
\multiput(805.00,462.17)(2.500,1.000){2}{\rule{0.602pt}{0.400pt}}
\put(810,464.17){\rule{1.300pt}{0.400pt}}
\multiput(810.00,463.17)(3.302,2.000){2}{\rule{0.650pt}{0.400pt}}
\put(816,466.17){\rule{1.100pt}{0.400pt}}
\multiput(816.00,465.17)(2.717,2.000){2}{\rule{0.550pt}{0.400pt}}
\put(821,468.17){\rule{1.100pt}{0.400pt}}
\multiput(821.00,467.17)(2.717,2.000){2}{\rule{0.550pt}{0.400pt}}
\multiput(826.00,470.61)(1.132,0.447){3}{\rule{0.900pt}{0.108pt}}
\multiput(826.00,469.17)(4.132,3.000){2}{\rule{0.450pt}{0.400pt}}
\put(832,473.17){\rule{1.100pt}{0.400pt}}
\multiput(832.00,472.17)(2.717,2.000){2}{\rule{0.550pt}{0.400pt}}
\put(837,475.17){\rule{1.300pt}{0.400pt}}
\multiput(837.00,474.17)(3.302,2.000){2}{\rule{0.650pt}{0.400pt}}
\multiput(843.00,477.61)(0.909,0.447){3}{\rule{0.767pt}{0.108pt}}
\multiput(843.00,476.17)(3.409,3.000){2}{\rule{0.383pt}{0.400pt}}
\put(848,480.17){\rule{1.100pt}{0.400pt}}
\multiput(848.00,479.17)(2.717,2.000){2}{\rule{0.550pt}{0.400pt}}
\multiput(853.00,482.61)(1.132,0.447){3}{\rule{0.900pt}{0.108pt}}
\multiput(853.00,481.17)(4.132,3.000){2}{\rule{0.450pt}{0.400pt}}
\multiput(859.00,485.61)(0.909,0.447){3}{\rule{0.767pt}{0.108pt}}
\multiput(859.00,484.17)(3.409,3.000){2}{\rule{0.383pt}{0.400pt}}
\multiput(864.00,488.61)(1.132,0.447){3}{\rule{0.900pt}{0.108pt}}
\multiput(864.00,487.17)(4.132,3.000){2}{\rule{0.450pt}{0.400pt}}
\multiput(870.00,491.61)(0.909,0.447){3}{\rule{0.767pt}{0.108pt}}
\multiput(870.00,490.17)(3.409,3.000){2}{\rule{0.383pt}{0.400pt}}
\multiput(875.00,494.61)(0.909,0.447){3}{\rule{0.767pt}{0.108pt}}
\multiput(875.00,493.17)(3.409,3.000){2}{\rule{0.383pt}{0.400pt}}
\multiput(880.00,497.61)(1.132,0.447){3}{\rule{0.900pt}{0.108pt}}
\multiput(880.00,496.17)(4.132,3.000){2}{\rule{0.450pt}{0.400pt}}
\multiput(886.00,500.61)(0.909,0.447){3}{\rule{0.767pt}{0.108pt}}
\multiput(886.00,499.17)(3.409,3.000){2}{\rule{0.383pt}{0.400pt}}
\multiput(891.00,503.60)(0.627,0.468){5}{\rule{0.600pt}{0.113pt}}
\multiput(891.00,502.17)(3.755,4.000){2}{\rule{0.300pt}{0.400pt}}
\multiput(896.00,507.61)(1.132,0.447){3}{\rule{0.900pt}{0.108pt}}
\multiput(896.00,506.17)(4.132,3.000){2}{\rule{0.450pt}{0.400pt}}
\multiput(902.00,510.60)(0.627,0.468){5}{\rule{0.600pt}{0.113pt}}
\multiput(902.00,509.17)(3.755,4.000){2}{\rule{0.300pt}{0.400pt}}
\multiput(907.00,514.60)(0.774,0.468){5}{\rule{0.700pt}{0.113pt}}
\multiput(907.00,513.17)(4.547,4.000){2}{\rule{0.350pt}{0.400pt}}
\multiput(913.00,518.61)(0.909,0.447){3}{\rule{0.767pt}{0.108pt}}
\multiput(913.00,517.17)(3.409,3.000){2}{\rule{0.383pt}{0.400pt}}
\multiput(418.59,667.59)(0.477,-0.599){7}{\rule{0.115pt}{0.580pt}}
\multiput(417.17,668.80)(5.000,-4.796){2}{\rule{0.400pt}{0.290pt}}
\multiput(423.00,662.93)(0.491,-0.482){9}{\rule{0.500pt}{0.116pt}}
\multiput(423.00,663.17)(4.962,-6.000){2}{\rule{0.250pt}{0.400pt}}
\multiput(429.00,656.93)(0.487,-0.477){7}{\rule{0.500pt}{0.115pt}}
\multiput(429.00,657.17)(3.962,-5.000){2}{\rule{0.250pt}{0.400pt}}
\multiput(434.00,651.93)(0.491,-0.482){9}{\rule{0.500pt}{0.116pt}}
\multiput(434.00,652.17)(4.962,-6.000){2}{\rule{0.250pt}{0.400pt}}
\multiput(440.00,645.93)(0.487,-0.477){7}{\rule{0.500pt}{0.115pt}}
\multiput(440.00,646.17)(3.962,-5.000){2}{\rule{0.250pt}{0.400pt}}
\multiput(445.59,639.59)(0.477,-0.599){7}{\rule{0.115pt}{0.580pt}}
\multiput(444.17,640.80)(5.000,-4.796){2}{\rule{0.400pt}{0.290pt}}
\multiput(450.00,634.93)(0.599,-0.477){7}{\rule{0.580pt}{0.115pt}}
\multiput(450.00,635.17)(4.796,-5.000){2}{\rule{0.290pt}{0.400pt}}
\multiput(456.00,629.93)(0.487,-0.477){7}{\rule{0.500pt}{0.115pt}}
\multiput(456.00,630.17)(3.962,-5.000){2}{\rule{0.250pt}{0.400pt}}
\multiput(461.00,624.93)(0.599,-0.477){7}{\rule{0.580pt}{0.115pt}}
\multiput(461.00,625.17)(4.796,-5.000){2}{\rule{0.290pt}{0.400pt}}
\multiput(467.00,619.93)(0.487,-0.477){7}{\rule{0.500pt}{0.115pt}}
\multiput(467.00,620.17)(3.962,-5.000){2}{\rule{0.250pt}{0.400pt}}
\multiput(472.00,614.93)(0.487,-0.477){7}{\rule{0.500pt}{0.115pt}}
\multiput(472.00,615.17)(3.962,-5.000){2}{\rule{0.250pt}{0.400pt}}
\multiput(477.00,609.93)(0.599,-0.477){7}{\rule{0.580pt}{0.115pt}}
\multiput(477.00,610.17)(4.796,-5.000){2}{\rule{0.290pt}{0.400pt}}
\multiput(483.00,604.93)(0.487,-0.477){7}{\rule{0.500pt}{0.115pt}}
\multiput(483.00,605.17)(3.962,-5.000){2}{\rule{0.250pt}{0.400pt}}
\multiput(488.00,599.94)(0.774,-0.468){5}{\rule{0.700pt}{0.113pt}}
\multiput(488.00,600.17)(4.547,-4.000){2}{\rule{0.350pt}{0.400pt}}
\multiput(494.00,595.93)(0.487,-0.477){7}{\rule{0.500pt}{0.115pt}}
\multiput(494.00,596.17)(3.962,-5.000){2}{\rule{0.250pt}{0.400pt}}
\multiput(499.00,590.94)(0.627,-0.468){5}{\rule{0.600pt}{0.113pt}}
\multiput(499.00,591.17)(3.755,-4.000){2}{\rule{0.300pt}{0.400pt}}
\multiput(504.00,586.93)(0.599,-0.477){7}{\rule{0.580pt}{0.115pt}}
\multiput(504.00,587.17)(4.796,-5.000){2}{\rule{0.290pt}{0.400pt}}
\multiput(510.00,581.94)(0.627,-0.468){5}{\rule{0.600pt}{0.113pt}}
\multiput(510.00,582.17)(3.755,-4.000){2}{\rule{0.300pt}{0.400pt}}
\multiput(515.00,577.94)(0.774,-0.468){5}{\rule{0.700pt}{0.113pt}}
\multiput(515.00,578.17)(4.547,-4.000){2}{\rule{0.350pt}{0.400pt}}
\multiput(521.00,573.94)(0.627,-0.468){5}{\rule{0.600pt}{0.113pt}}
\multiput(521.00,574.17)(3.755,-4.000){2}{\rule{0.300pt}{0.400pt}}
\multiput(526.00,569.94)(0.627,-0.468){5}{\rule{0.600pt}{0.113pt}}
\multiput(526.00,570.17)(3.755,-4.000){2}{\rule{0.300pt}{0.400pt}}
\multiput(531.00,565.94)(0.774,-0.468){5}{\rule{0.700pt}{0.113pt}}
\multiput(531.00,566.17)(4.547,-4.000){2}{\rule{0.350pt}{0.400pt}}
\multiput(537.00,561.94)(0.627,-0.468){5}{\rule{0.600pt}{0.113pt}}
\multiput(537.00,562.17)(3.755,-4.000){2}{\rule{0.300pt}{0.400pt}}
\multiput(542.00,557.95)(1.132,-0.447){3}{\rule{0.900pt}{0.108pt}}
\multiput(542.00,558.17)(4.132,-3.000){2}{\rule{0.450pt}{0.400pt}}
\multiput(548.00,554.94)(0.627,-0.468){5}{\rule{0.600pt}{0.113pt}}
\multiput(548.00,555.17)(3.755,-4.000){2}{\rule{0.300pt}{0.400pt}}
\multiput(553.00,550.95)(0.909,-0.447){3}{\rule{0.767pt}{0.108pt}}
\multiput(553.00,551.17)(3.409,-3.000){2}{\rule{0.383pt}{0.400pt}}
\multiput(558.00,547.95)(1.132,-0.447){3}{\rule{0.900pt}{0.108pt}}
\multiput(558.00,548.17)(4.132,-3.000){2}{\rule{0.450pt}{0.400pt}}
\multiput(564.00,544.94)(0.627,-0.468){5}{\rule{0.600pt}{0.113pt}}
\multiput(564.00,545.17)(3.755,-4.000){2}{\rule{0.300pt}{0.400pt}}
\multiput(569.00,540.95)(1.132,-0.447){3}{\rule{0.900pt}{0.108pt}}
\multiput(569.00,541.17)(4.132,-3.000){2}{\rule{0.450pt}{0.400pt}}
\multiput(575.00,537.95)(0.909,-0.447){3}{\rule{0.767pt}{0.108pt}}
\multiput(575.00,538.17)(3.409,-3.000){2}{\rule{0.383pt}{0.400pt}}
\multiput(580.00,534.95)(0.909,-0.447){3}{\rule{0.767pt}{0.108pt}}
\multiput(580.00,535.17)(3.409,-3.000){2}{\rule{0.383pt}{0.400pt}}
\multiput(585.00,531.95)(1.132,-0.447){3}{\rule{0.900pt}{0.108pt}}
\multiput(585.00,532.17)(4.132,-3.000){2}{\rule{0.450pt}{0.400pt}}
\put(591,528.17){\rule{1.100pt}{0.400pt}}
\multiput(591.00,529.17)(2.717,-2.000){2}{\rule{0.550pt}{0.400pt}}
\multiput(596.00,526.95)(1.132,-0.447){3}{\rule{0.900pt}{0.108pt}}
\multiput(596.00,527.17)(4.132,-3.000){2}{\rule{0.450pt}{0.400pt}}
\multiput(602.00,523.95)(0.909,-0.447){3}{\rule{0.767pt}{0.108pt}}
\multiput(602.00,524.17)(3.409,-3.000){2}{\rule{0.383pt}{0.400pt}}
\put(607,520.17){\rule{1.100pt}{0.400pt}}
\multiput(607.00,521.17)(2.717,-2.000){2}{\rule{0.550pt}{0.400pt}}
\put(612,518.17){\rule{1.300pt}{0.400pt}}
\multiput(612.00,519.17)(3.302,-2.000){2}{\rule{0.650pt}{0.400pt}}
\multiput(618.00,516.95)(0.909,-0.447){3}{\rule{0.767pt}{0.108pt}}
\multiput(618.00,517.17)(3.409,-3.000){2}{\rule{0.383pt}{0.400pt}}
\put(623,513.17){\rule{1.300pt}{0.400pt}}
\multiput(623.00,514.17)(3.302,-2.000){2}{\rule{0.650pt}{0.400pt}}
\put(629,511.17){\rule{1.100pt}{0.400pt}}
\multiput(629.00,512.17)(2.717,-2.000){2}{\rule{0.550pt}{0.400pt}}
\put(634,509.17){\rule{1.100pt}{0.400pt}}
\multiput(634.00,510.17)(2.717,-2.000){2}{\rule{0.550pt}{0.400pt}}
\put(639,507.17){\rule{1.300pt}{0.400pt}}
\multiput(639.00,508.17)(3.302,-2.000){2}{\rule{0.650pt}{0.400pt}}
\put(645,505.67){\rule{1.204pt}{0.400pt}}
\multiput(645.00,506.17)(2.500,-1.000){2}{\rule{0.602pt}{0.400pt}}
\put(650,504.17){\rule{1.300pt}{0.400pt}}
\multiput(650.00,505.17)(3.302,-2.000){2}{\rule{0.650pt}{0.400pt}}
\put(656,502.67){\rule{1.204pt}{0.400pt}}
\multiput(656.00,503.17)(2.500,-1.000){2}{\rule{0.602pt}{0.400pt}}
\put(661,501.17){\rule{1.100pt}{0.400pt}}
\multiput(661.00,502.17)(2.717,-2.000){2}{\rule{0.550pt}{0.400pt}}
\put(666,499.67){\rule{1.445pt}{0.400pt}}
\multiput(666.00,500.17)(3.000,-1.000){2}{\rule{0.723pt}{0.400pt}}
\put(672,498.17){\rule{1.100pt}{0.400pt}}
\multiput(672.00,499.17)(2.717,-2.000){2}{\rule{0.550pt}{0.400pt}}
\put(677,496.67){\rule{1.204pt}{0.400pt}}
\multiput(677.00,497.17)(2.500,-1.000){2}{\rule{0.602pt}{0.400pt}}
\put(682,495.67){\rule{1.445pt}{0.400pt}}
\multiput(682.00,496.17)(3.000,-1.000){2}{\rule{0.723pt}{0.400pt}}
\put(688,494.67){\rule{1.204pt}{0.400pt}}
\multiput(688.00,495.17)(2.500,-1.000){2}{\rule{0.602pt}{0.400pt}}
\put(693,493.67){\rule{1.445pt}{0.400pt}}
\multiput(693.00,494.17)(3.000,-1.000){2}{\rule{0.723pt}{0.400pt}}
\put(740.0,449.0){\rule[-0.200pt]{1.204pt}{0.400pt}}
\put(704,492.67){\rule{1.204pt}{0.400pt}}
\multiput(704.00,493.17)(2.500,-1.000){2}{\rule{0.602pt}{0.400pt}}
\put(709,491.67){\rule{1.445pt}{0.400pt}}
\multiput(709.00,492.17)(3.000,-1.000){2}{\rule{0.723pt}{0.400pt}}
\put(699.0,494.0){\rule[-0.200pt]{1.204pt}{0.400pt}}
\put(715.0,492.0){\rule[-0.200pt]{1.204pt}{0.400pt}}
\put(726,490.67){\rule{1.204pt}{0.400pt}}
\multiput(726.00,491.17)(2.500,-1.000){2}{\rule{0.602pt}{0.400pt}}
\put(720.0,492.0){\rule[-0.200pt]{1.445pt}{0.400pt}}
\put(731.0,491.0){\rule[-0.200pt]{1.204pt}{0.400pt}}
\put(736.0,491.0){\rule[-0.200pt]{1.445pt}{0.400pt}}
\put(742.0,491.0){\rule[-0.200pt]{1.204pt}{0.400pt}}
\put(747.0,491.0){\rule[-0.200pt]{1.445pt}{0.400pt}}
\put(758,490.67){\rule{1.204pt}{0.400pt}}
\multiput(758.00,490.17)(2.500,1.000){2}{\rule{0.602pt}{0.400pt}}
\put(753.0,491.0){\rule[-0.200pt]{1.204pt}{0.400pt}}
\put(769,491.67){\rule{1.204pt}{0.400pt}}
\multiput(769.00,491.17)(2.500,1.000){2}{\rule{0.602pt}{0.400pt}}
\put(763.0,492.0){\rule[-0.200pt]{1.445pt}{0.400pt}}
\put(780,492.67){\rule{1.204pt}{0.400pt}}
\multiput(780.00,492.17)(2.500,1.000){2}{\rule{0.602pt}{0.400pt}}
\put(785,493.67){\rule{1.204pt}{0.400pt}}
\multiput(785.00,493.17)(2.500,1.000){2}{\rule{0.602pt}{0.400pt}}
\put(790,494.67){\rule{1.445pt}{0.400pt}}
\multiput(790.00,494.17)(3.000,1.000){2}{\rule{0.723pt}{0.400pt}}
\put(796,495.67){\rule{1.204pt}{0.400pt}}
\multiput(796.00,495.17)(2.500,1.000){2}{\rule{0.602pt}{0.400pt}}
\put(801,496.67){\rule{1.445pt}{0.400pt}}
\multiput(801.00,496.17)(3.000,1.000){2}{\rule{0.723pt}{0.400pt}}
\put(807,497.67){\rule{1.204pt}{0.400pt}}
\multiput(807.00,497.17)(2.500,1.000){2}{\rule{0.602pt}{0.400pt}}
\put(812,498.67){\rule{1.204pt}{0.400pt}}
\multiput(812.00,498.17)(2.500,1.000){2}{\rule{0.602pt}{0.400pt}}
\put(817,500.17){\rule{1.300pt}{0.400pt}}
\multiput(817.00,499.17)(3.302,2.000){2}{\rule{0.650pt}{0.400pt}}
\put(823,501.67){\rule{1.204pt}{0.400pt}}
\multiput(823.00,501.17)(2.500,1.000){2}{\rule{0.602pt}{0.400pt}}
\put(828,503.17){\rule{1.300pt}{0.400pt}}
\multiput(828.00,502.17)(3.302,2.000){2}{\rule{0.650pt}{0.400pt}}
\put(834,505.17){\rule{1.100pt}{0.400pt}}
\multiput(834.00,504.17)(2.717,2.000){2}{\rule{0.550pt}{0.400pt}}
\put(839,506.67){\rule{1.204pt}{0.400pt}}
\multiput(839.00,506.17)(2.500,1.000){2}{\rule{0.602pt}{0.400pt}}
\put(844,508.17){\rule{1.300pt}{0.400pt}}
\multiput(844.00,507.17)(3.302,2.000){2}{\rule{0.650pt}{0.400pt}}
\put(850,510.17){\rule{1.100pt}{0.400pt}}
\multiput(850.00,509.17)(2.717,2.000){2}{\rule{0.550pt}{0.400pt}}
\multiput(855.00,512.61)(1.132,0.447){3}{\rule{0.900pt}{0.108pt}}
\multiput(855.00,511.17)(4.132,3.000){2}{\rule{0.450pt}{0.400pt}}
\put(861,515.17){\rule{1.100pt}{0.400pt}}
\multiput(861.00,514.17)(2.717,2.000){2}{\rule{0.550pt}{0.400pt}}
\put(866,517.17){\rule{1.100pt}{0.400pt}}
\multiput(866.00,516.17)(2.717,2.000){2}{\rule{0.550pt}{0.400pt}}
\put(871,519.17){\rule{1.300pt}{0.400pt}}
\multiput(871.00,518.17)(3.302,2.000){2}{\rule{0.650pt}{0.400pt}}
\multiput(877.00,521.61)(0.909,0.447){3}{\rule{0.767pt}{0.108pt}}
\multiput(877.00,520.17)(3.409,3.000){2}{\rule{0.383pt}{0.400pt}}
\put(882,524.17){\rule{1.300pt}{0.400pt}}
\multiput(882.00,523.17)(3.302,2.000){2}{\rule{0.650pt}{0.400pt}}
\multiput(888.00,526.61)(0.909,0.447){3}{\rule{0.767pt}{0.108pt}}
\multiput(888.00,525.17)(3.409,3.000){2}{\rule{0.383pt}{0.400pt}}
\multiput(893.00,529.61)(0.909,0.447){3}{\rule{0.767pt}{0.108pt}}
\multiput(893.00,528.17)(3.409,3.000){2}{\rule{0.383pt}{0.400pt}}
\multiput(898.00,532.61)(1.132,0.447){3}{\rule{0.900pt}{0.108pt}}
\multiput(898.00,531.17)(4.132,3.000){2}{\rule{0.450pt}{0.400pt}}
\multiput(904.00,535.61)(0.909,0.447){3}{\rule{0.767pt}{0.108pt}}
\multiput(904.00,534.17)(3.409,3.000){2}{\rule{0.383pt}{0.400pt}}
\multiput(909.00,538.61)(1.132,0.447){3}{\rule{0.900pt}{0.108pt}}
\multiput(909.00,537.17)(4.132,3.000){2}{\rule{0.450pt}{0.400pt}}
\multiput(915.00,541.61)(0.909,0.447){3}{\rule{0.767pt}{0.108pt}}
\multiput(915.00,540.17)(3.409,3.000){2}{\rule{0.383pt}{0.400pt}}
\multiput(920.00,544.61)(0.909,0.447){3}{\rule{0.767pt}{0.108pt}}
\multiput(920.00,543.17)(3.409,3.000){2}{\rule{0.383pt}{0.400pt}}
\multiput(925.00,547.60)(0.774,0.468){5}{\rule{0.700pt}{0.113pt}}
\multiput(925.00,546.17)(4.547,4.000){2}{\rule{0.350pt}{0.400pt}}
\multiput(931.00,551.61)(0.909,0.447){3}{\rule{0.767pt}{0.108pt}}
\multiput(931.00,550.17)(3.409,3.000){2}{\rule{0.383pt}{0.400pt}}
\multiput(936.00,554.60)(0.774,0.468){5}{\rule{0.700pt}{0.113pt}}
\multiput(936.00,553.17)(4.547,4.000){2}{\rule{0.350pt}{0.400pt}}
\multiput(942.00,558.60)(0.627,0.468){5}{\rule{0.600pt}{0.113pt}}
\multiput(942.00,557.17)(3.755,4.000){2}{\rule{0.300pt}{0.400pt}}
\multiput(947.00,562.61)(0.909,0.447){3}{\rule{0.767pt}{0.108pt}}
\multiput(947.00,561.17)(3.409,3.000){2}{\rule{0.383pt}{0.400pt}}
\multiput(452.00,724.93)(0.491,-0.482){9}{\rule{0.500pt}{0.116pt}}
\multiput(452.00,725.17)(4.962,-6.000){2}{\rule{0.250pt}{0.400pt}}
\multiput(458.59,717.59)(0.477,-0.599){7}{\rule{0.115pt}{0.580pt}}
\multiput(457.17,718.80)(5.000,-4.796){2}{\rule{0.400pt}{0.290pt}}
\multiput(463.00,712.93)(0.487,-0.477){7}{\rule{0.500pt}{0.115pt}}
\multiput(463.00,713.17)(3.962,-5.000){2}{\rule{0.250pt}{0.400pt}}
\multiput(468.00,707.93)(0.491,-0.482){9}{\rule{0.500pt}{0.116pt}}
\multiput(468.00,708.17)(4.962,-6.000){2}{\rule{0.250pt}{0.400pt}}
\multiput(474.00,701.93)(0.487,-0.477){7}{\rule{0.500pt}{0.115pt}}
\multiput(474.00,702.17)(3.962,-5.000){2}{\rule{0.250pt}{0.400pt}}
\multiput(479.00,696.93)(0.491,-0.482){9}{\rule{0.500pt}{0.116pt}}
\multiput(479.00,697.17)(4.962,-6.000){2}{\rule{0.250pt}{0.400pt}}
\multiput(485.00,690.93)(0.487,-0.477){7}{\rule{0.500pt}{0.115pt}}
\multiput(485.00,691.17)(3.962,-5.000){2}{\rule{0.250pt}{0.400pt}}
\multiput(490.00,685.93)(0.487,-0.477){7}{\rule{0.500pt}{0.115pt}}
\multiput(490.00,686.17)(3.962,-5.000){2}{\rule{0.250pt}{0.400pt}}
\multiput(495.00,680.93)(0.599,-0.477){7}{\rule{0.580pt}{0.115pt}}
\multiput(495.00,681.17)(4.796,-5.000){2}{\rule{0.290pt}{0.400pt}}
\multiput(501.00,675.93)(0.487,-0.477){7}{\rule{0.500pt}{0.115pt}}
\multiput(501.00,676.17)(3.962,-5.000){2}{\rule{0.250pt}{0.400pt}}
\multiput(506.00,670.93)(0.599,-0.477){7}{\rule{0.580pt}{0.115pt}}
\multiput(506.00,671.17)(4.796,-5.000){2}{\rule{0.290pt}{0.400pt}}
\multiput(512.00,665.93)(0.487,-0.477){7}{\rule{0.500pt}{0.115pt}}
\multiput(512.00,666.17)(3.962,-5.000){2}{\rule{0.250pt}{0.400pt}}
\multiput(517.00,660.93)(0.487,-0.477){7}{\rule{0.500pt}{0.115pt}}
\multiput(517.00,661.17)(3.962,-5.000){2}{\rule{0.250pt}{0.400pt}}
\multiput(522.00,655.94)(0.774,-0.468){5}{\rule{0.700pt}{0.113pt}}
\multiput(522.00,656.17)(4.547,-4.000){2}{\rule{0.350pt}{0.400pt}}
\multiput(528.00,651.93)(0.487,-0.477){7}{\rule{0.500pt}{0.115pt}}
\multiput(528.00,652.17)(3.962,-5.000){2}{\rule{0.250pt}{0.400pt}}
\multiput(533.00,646.94)(0.774,-0.468){5}{\rule{0.700pt}{0.113pt}}
\multiput(533.00,647.17)(4.547,-4.000){2}{\rule{0.350pt}{0.400pt}}
\multiput(539.00,642.93)(0.487,-0.477){7}{\rule{0.500pt}{0.115pt}}
\multiput(539.00,643.17)(3.962,-5.000){2}{\rule{0.250pt}{0.400pt}}
\multiput(544.00,637.94)(0.627,-0.468){5}{\rule{0.600pt}{0.113pt}}
\multiput(544.00,638.17)(3.755,-4.000){2}{\rule{0.300pt}{0.400pt}}
\multiput(549.00,633.94)(0.774,-0.468){5}{\rule{0.700pt}{0.113pt}}
\multiput(549.00,634.17)(4.547,-4.000){2}{\rule{0.350pt}{0.400pt}}
\multiput(555.00,629.94)(0.627,-0.468){5}{\rule{0.600pt}{0.113pt}}
\multiput(555.00,630.17)(3.755,-4.000){2}{\rule{0.300pt}{0.400pt}}
\multiput(560.00,625.94)(0.774,-0.468){5}{\rule{0.700pt}{0.113pt}}
\multiput(560.00,626.17)(4.547,-4.000){2}{\rule{0.350pt}{0.400pt}}
\multiput(566.00,621.94)(0.627,-0.468){5}{\rule{0.600pt}{0.113pt}}
\multiput(566.00,622.17)(3.755,-4.000){2}{\rule{0.300pt}{0.400pt}}
\multiput(571.00,617.94)(0.627,-0.468){5}{\rule{0.600pt}{0.113pt}}
\multiput(571.00,618.17)(3.755,-4.000){2}{\rule{0.300pt}{0.400pt}}
\multiput(576.00,613.95)(1.132,-0.447){3}{\rule{0.900pt}{0.108pt}}
\multiput(576.00,614.17)(4.132,-3.000){2}{\rule{0.450pt}{0.400pt}}
\multiput(582.00,610.94)(0.627,-0.468){5}{\rule{0.600pt}{0.113pt}}
\multiput(582.00,611.17)(3.755,-4.000){2}{\rule{0.300pt}{0.400pt}}
\multiput(587.00,606.95)(1.132,-0.447){3}{\rule{0.900pt}{0.108pt}}
\multiput(587.00,607.17)(4.132,-3.000){2}{\rule{0.450pt}{0.400pt}}
\multiput(593.00,603.95)(0.909,-0.447){3}{\rule{0.767pt}{0.108pt}}
\multiput(593.00,604.17)(3.409,-3.000){2}{\rule{0.383pt}{0.400pt}}
\multiput(598.00,600.94)(0.627,-0.468){5}{\rule{0.600pt}{0.113pt}}
\multiput(598.00,601.17)(3.755,-4.000){2}{\rule{0.300pt}{0.400pt}}
\multiput(603.00,596.95)(1.132,-0.447){3}{\rule{0.900pt}{0.108pt}}
\multiput(603.00,597.17)(4.132,-3.000){2}{\rule{0.450pt}{0.400pt}}
\multiput(609.00,593.95)(0.909,-0.447){3}{\rule{0.767pt}{0.108pt}}
\multiput(609.00,594.17)(3.409,-3.000){2}{\rule{0.383pt}{0.400pt}}
\multiput(614.00,590.95)(1.132,-0.447){3}{\rule{0.900pt}{0.108pt}}
\multiput(614.00,591.17)(4.132,-3.000){2}{\rule{0.450pt}{0.400pt}}
\multiput(620.00,587.95)(0.909,-0.447){3}{\rule{0.767pt}{0.108pt}}
\multiput(620.00,588.17)(3.409,-3.000){2}{\rule{0.383pt}{0.400pt}}
\put(625,584.17){\rule{1.100pt}{0.400pt}}
\multiput(625.00,585.17)(2.717,-2.000){2}{\rule{0.550pt}{0.400pt}}
\multiput(630.00,582.95)(1.132,-0.447){3}{\rule{0.900pt}{0.108pt}}
\multiput(630.00,583.17)(4.132,-3.000){2}{\rule{0.450pt}{0.400pt}}
\multiput(636.00,579.95)(0.909,-0.447){3}{\rule{0.767pt}{0.108pt}}
\multiput(636.00,580.17)(3.409,-3.000){2}{\rule{0.383pt}{0.400pt}}
\put(641,576.17){\rule{1.300pt}{0.400pt}}
\multiput(641.00,577.17)(3.302,-2.000){2}{\rule{0.650pt}{0.400pt}}
\put(647,574.17){\rule{1.100pt}{0.400pt}}
\multiput(647.00,575.17)(2.717,-2.000){2}{\rule{0.550pt}{0.400pt}}
\multiput(652.00,572.95)(0.909,-0.447){3}{\rule{0.767pt}{0.108pt}}
\multiput(652.00,573.17)(3.409,-3.000){2}{\rule{0.383pt}{0.400pt}}
\put(657,569.17){\rule{1.300pt}{0.400pt}}
\multiput(657.00,570.17)(3.302,-2.000){2}{\rule{0.650pt}{0.400pt}}
\put(663,567.17){\rule{1.100pt}{0.400pt}}
\multiput(663.00,568.17)(2.717,-2.000){2}{\rule{0.550pt}{0.400pt}}
\put(668,565.17){\rule{1.300pt}{0.400pt}}
\multiput(668.00,566.17)(3.302,-2.000){2}{\rule{0.650pt}{0.400pt}}
\put(674,563.17){\rule{1.100pt}{0.400pt}}
\multiput(674.00,564.17)(2.717,-2.000){2}{\rule{0.550pt}{0.400pt}}
\put(679,561.67){\rule{1.204pt}{0.400pt}}
\multiput(679.00,562.17)(2.500,-1.000){2}{\rule{0.602pt}{0.400pt}}
\put(684,560.17){\rule{1.300pt}{0.400pt}}
\multiput(684.00,561.17)(3.302,-2.000){2}{\rule{0.650pt}{0.400pt}}
\put(690,558.67){\rule{1.204pt}{0.400pt}}
\multiput(690.00,559.17)(2.500,-1.000){2}{\rule{0.602pt}{0.400pt}}
\put(695,557.17){\rule{1.300pt}{0.400pt}}
\multiput(695.00,558.17)(3.302,-2.000){2}{\rule{0.650pt}{0.400pt}}
\put(701,555.67){\rule{1.204pt}{0.400pt}}
\multiput(701.00,556.17)(2.500,-1.000){2}{\rule{0.602pt}{0.400pt}}
\put(706,554.17){\rule{1.100pt}{0.400pt}}
\multiput(706.00,555.17)(2.717,-2.000){2}{\rule{0.550pt}{0.400pt}}
\put(711,552.67){\rule{1.445pt}{0.400pt}}
\multiput(711.00,553.17)(3.000,-1.000){2}{\rule{0.723pt}{0.400pt}}
\put(717,551.67){\rule{1.204pt}{0.400pt}}
\multiput(717.00,552.17)(2.500,-1.000){2}{\rule{0.602pt}{0.400pt}}
\put(722,550.67){\rule{1.445pt}{0.400pt}}
\multiput(722.00,551.17)(3.000,-1.000){2}{\rule{0.723pt}{0.400pt}}
\put(728,549.67){\rule{1.204pt}{0.400pt}}
\multiput(728.00,550.17)(2.500,-1.000){2}{\rule{0.602pt}{0.400pt}}
\put(774.0,493.0){\rule[-0.200pt]{1.445pt}{0.400pt}}
\put(738,548.67){\rule{1.445pt}{0.400pt}}
\multiput(738.00,549.17)(3.000,-1.000){2}{\rule{0.723pt}{0.400pt}}
\put(744,547.67){\rule{1.204pt}{0.400pt}}
\multiput(744.00,548.17)(2.500,-1.000){2}{\rule{0.602pt}{0.400pt}}
\put(733.0,550.0){\rule[-0.200pt]{1.204pt}{0.400pt}}
\put(749.0,548.0){\rule[-0.200pt]{1.445pt}{0.400pt}}
\put(760,546.67){\rule{1.204pt}{0.400pt}}
\multiput(760.00,547.17)(2.500,-1.000){2}{\rule{0.602pt}{0.400pt}}
\put(755.0,548.0){\rule[-0.200pt]{1.204pt}{0.400pt}}
\put(765.0,547.0){\rule[-0.200pt]{1.445pt}{0.400pt}}
\put(771.0,547.0){\rule[-0.200pt]{1.204pt}{0.400pt}}
\put(776.0,547.0){\rule[-0.200pt]{1.445pt}{0.400pt}}
\put(782.0,547.0){\rule[-0.200pt]{1.204pt}{0.400pt}}
\put(792,546.67){\rule{1.445pt}{0.400pt}}
\multiput(792.00,546.17)(3.000,1.000){2}{\rule{0.723pt}{0.400pt}}
\put(787.0,547.0){\rule[-0.200pt]{1.204pt}{0.400pt}}
\put(803,547.67){\rule{1.445pt}{0.400pt}}
\multiput(803.00,547.17)(3.000,1.000){2}{\rule{0.723pt}{0.400pt}}
\put(798.0,548.0){\rule[-0.200pt]{1.204pt}{0.400pt}}
\put(814,548.67){\rule{1.204pt}{0.400pt}}
\multiput(814.00,548.17)(2.500,1.000){2}{\rule{0.602pt}{0.400pt}}
\put(819,549.67){\rule{1.445pt}{0.400pt}}
\multiput(819.00,549.17)(3.000,1.000){2}{\rule{0.723pt}{0.400pt}}
\put(825,550.67){\rule{1.204pt}{0.400pt}}
\multiput(825.00,550.17)(2.500,1.000){2}{\rule{0.602pt}{0.400pt}}
\put(830,551.67){\rule{1.445pt}{0.400pt}}
\multiput(830.00,551.17)(3.000,1.000){2}{\rule{0.723pt}{0.400pt}}
\put(836,552.67){\rule{1.204pt}{0.400pt}}
\multiput(836.00,552.17)(2.500,1.000){2}{\rule{0.602pt}{0.400pt}}
\put(841,553.67){\rule{1.204pt}{0.400pt}}
\multiput(841.00,553.17)(2.500,1.000){2}{\rule{0.602pt}{0.400pt}}
\put(846,554.67){\rule{1.445pt}{0.400pt}}
\multiput(846.00,554.17)(3.000,1.000){2}{\rule{0.723pt}{0.400pt}}
\put(852,556.17){\rule{1.100pt}{0.400pt}}
\multiput(852.00,555.17)(2.717,2.000){2}{\rule{0.550pt}{0.400pt}}
\put(857,557.67){\rule{1.445pt}{0.400pt}}
\multiput(857.00,557.17)(3.000,1.000){2}{\rule{0.723pt}{0.400pt}}
\put(863,559.17){\rule{1.100pt}{0.400pt}}
\multiput(863.00,558.17)(2.717,2.000){2}{\rule{0.550pt}{0.400pt}}
\put(868,561.17){\rule{1.100pt}{0.400pt}}
\multiput(868.00,560.17)(2.717,2.000){2}{\rule{0.550pt}{0.400pt}}
\put(873,562.67){\rule{1.445pt}{0.400pt}}
\multiput(873.00,562.17)(3.000,1.000){2}{\rule{0.723pt}{0.400pt}}
\put(879,564.17){\rule{1.100pt}{0.400pt}}
\multiput(879.00,563.17)(2.717,2.000){2}{\rule{0.550pt}{0.400pt}}
\put(884,566.17){\rule{1.100pt}{0.400pt}}
\multiput(884.00,565.17)(2.717,2.000){2}{\rule{0.550pt}{0.400pt}}
\multiput(889.00,568.61)(1.132,0.447){3}{\rule{0.900pt}{0.108pt}}
\multiput(889.00,567.17)(4.132,3.000){2}{\rule{0.450pt}{0.400pt}}
\put(895,571.17){\rule{1.100pt}{0.400pt}}
\multiput(895.00,570.17)(2.717,2.000){2}{\rule{0.550pt}{0.400pt}}
\put(900,573.17){\rule{1.300pt}{0.400pt}}
\multiput(900.00,572.17)(3.302,2.000){2}{\rule{0.650pt}{0.400pt}}
\put(906,575.17){\rule{1.100pt}{0.400pt}}
\multiput(906.00,574.17)(2.717,2.000){2}{\rule{0.550pt}{0.400pt}}
\multiput(911.00,577.61)(0.909,0.447){3}{\rule{0.767pt}{0.108pt}}
\multiput(911.00,576.17)(3.409,3.000){2}{\rule{0.383pt}{0.400pt}}
\put(916,580.17){\rule{1.300pt}{0.400pt}}
\multiput(916.00,579.17)(3.302,2.000){2}{\rule{0.650pt}{0.400pt}}
\multiput(922.00,582.61)(0.909,0.447){3}{\rule{0.767pt}{0.108pt}}
\multiput(922.00,581.17)(3.409,3.000){2}{\rule{0.383pt}{0.400pt}}
\multiput(927.00,585.61)(1.132,0.447){3}{\rule{0.900pt}{0.108pt}}
\multiput(927.00,584.17)(4.132,3.000){2}{\rule{0.450pt}{0.400pt}}
\multiput(933.00,588.61)(0.909,0.447){3}{\rule{0.767pt}{0.108pt}}
\multiput(933.00,587.17)(3.409,3.000){2}{\rule{0.383pt}{0.400pt}}
\multiput(938.00,591.61)(0.909,0.447){3}{\rule{0.767pt}{0.108pt}}
\multiput(938.00,590.17)(3.409,3.000){2}{\rule{0.383pt}{0.400pt}}
\multiput(943.00,594.61)(1.132,0.447){3}{\rule{0.900pt}{0.108pt}}
\multiput(943.00,593.17)(4.132,3.000){2}{\rule{0.450pt}{0.400pt}}
\multiput(949.00,597.61)(0.909,0.447){3}{\rule{0.767pt}{0.108pt}}
\multiput(949.00,596.17)(3.409,3.000){2}{\rule{0.383pt}{0.400pt}}
\multiput(954.00,600.61)(1.132,0.447){3}{\rule{0.900pt}{0.108pt}}
\multiput(954.00,599.17)(4.132,3.000){2}{\rule{0.450pt}{0.400pt}}
\multiput(960.00,603.60)(0.627,0.468){5}{\rule{0.600pt}{0.113pt}}
\multiput(960.00,602.17)(3.755,4.000){2}{\rule{0.300pt}{0.400pt}}
\multiput(965.00,607.61)(0.909,0.447){3}{\rule{0.767pt}{0.108pt}}
\multiput(965.00,606.17)(3.409,3.000){2}{\rule{0.383pt}{0.400pt}}
\multiput(970.00,610.60)(0.774,0.468){5}{\rule{0.700pt}{0.113pt}}
\multiput(970.00,609.17)(4.547,4.000){2}{\rule{0.350pt}{0.400pt}}
\multiput(976.00,614.60)(0.627,0.468){5}{\rule{0.600pt}{0.113pt}}
\multiput(976.00,613.17)(3.755,4.000){2}{\rule{0.300pt}{0.400pt}}
\multiput(981.00,618.61)(1.132,0.447){3}{\rule{0.900pt}{0.108pt}}
\multiput(981.00,617.17)(4.132,3.000){2}{\rule{0.450pt}{0.400pt}}
\multiput(487.59,791.59)(0.477,-0.599){7}{\rule{0.115pt}{0.580pt}}
\multiput(486.17,792.80)(5.000,-4.796){2}{\rule{0.400pt}{0.290pt}}
\multiput(492.59,785.59)(0.477,-0.599){7}{\rule{0.115pt}{0.580pt}}
\multiput(491.17,786.80)(5.000,-4.796){2}{\rule{0.400pt}{0.290pt}}
\multiput(497.00,780.93)(0.599,-0.477){7}{\rule{0.580pt}{0.115pt}}
\multiput(497.00,781.17)(4.796,-5.000){2}{\rule{0.290pt}{0.400pt}}
\multiput(503.59,774.59)(0.477,-0.599){7}{\rule{0.115pt}{0.580pt}}
\multiput(502.17,775.80)(5.000,-4.796){2}{\rule{0.400pt}{0.290pt}}
\multiput(508.00,769.93)(0.599,-0.477){7}{\rule{0.580pt}{0.115pt}}
\multiput(508.00,770.17)(4.796,-5.000){2}{\rule{0.290pt}{0.400pt}}
\multiput(514.59,763.59)(0.477,-0.599){7}{\rule{0.115pt}{0.580pt}}
\multiput(513.17,764.80)(5.000,-4.796){2}{\rule{0.400pt}{0.290pt}}
\multiput(519.00,758.93)(0.487,-0.477){7}{\rule{0.500pt}{0.115pt}}
\multiput(519.00,759.17)(3.962,-5.000){2}{\rule{0.250pt}{0.400pt}}
\multiput(524.00,753.93)(0.599,-0.477){7}{\rule{0.580pt}{0.115pt}}
\multiput(524.00,754.17)(4.796,-5.000){2}{\rule{0.290pt}{0.400pt}}
\multiput(530.00,748.93)(0.487,-0.477){7}{\rule{0.500pt}{0.115pt}}
\multiput(530.00,749.17)(3.962,-5.000){2}{\rule{0.250pt}{0.400pt}}
\multiput(535.00,743.93)(0.599,-0.477){7}{\rule{0.580pt}{0.115pt}}
\multiput(535.00,744.17)(4.796,-5.000){2}{\rule{0.290pt}{0.400pt}}
\multiput(541.00,738.93)(0.487,-0.477){7}{\rule{0.500pt}{0.115pt}}
\multiput(541.00,739.17)(3.962,-5.000){2}{\rule{0.250pt}{0.400pt}}
\multiput(546.00,733.93)(0.487,-0.477){7}{\rule{0.500pt}{0.115pt}}
\multiput(546.00,734.17)(3.962,-5.000){2}{\rule{0.250pt}{0.400pt}}
\multiput(551.00,728.93)(0.599,-0.477){7}{\rule{0.580pt}{0.115pt}}
\multiput(551.00,729.17)(4.796,-5.000){2}{\rule{0.290pt}{0.400pt}}
\multiput(557.00,723.93)(0.487,-0.477){7}{\rule{0.500pt}{0.115pt}}
\multiput(557.00,724.17)(3.962,-5.000){2}{\rule{0.250pt}{0.400pt}}
\multiput(562.00,718.94)(0.774,-0.468){5}{\rule{0.700pt}{0.113pt}}
\multiput(562.00,719.17)(4.547,-4.000){2}{\rule{0.350pt}{0.400pt}}
\multiput(568.00,714.94)(0.627,-0.468){5}{\rule{0.600pt}{0.113pt}}
\multiput(568.00,715.17)(3.755,-4.000){2}{\rule{0.300pt}{0.400pt}}
\multiput(573.00,710.93)(0.487,-0.477){7}{\rule{0.500pt}{0.115pt}}
\multiput(573.00,711.17)(3.962,-5.000){2}{\rule{0.250pt}{0.400pt}}
\multiput(578.00,705.94)(0.774,-0.468){5}{\rule{0.700pt}{0.113pt}}
\multiput(578.00,706.17)(4.547,-4.000){2}{\rule{0.350pt}{0.400pt}}
\multiput(584.00,701.94)(0.627,-0.468){5}{\rule{0.600pt}{0.113pt}}
\multiput(584.00,702.17)(3.755,-4.000){2}{\rule{0.300pt}{0.400pt}}
\multiput(589.00,697.94)(0.774,-0.468){5}{\rule{0.700pt}{0.113pt}}
\multiput(589.00,698.17)(4.547,-4.000){2}{\rule{0.350pt}{0.400pt}}
\multiput(595.00,693.94)(0.627,-0.468){5}{\rule{0.600pt}{0.113pt}}
\multiput(595.00,694.17)(3.755,-4.000){2}{\rule{0.300pt}{0.400pt}}
\multiput(600.00,689.94)(0.627,-0.468){5}{\rule{0.600pt}{0.113pt}}
\multiput(600.00,690.17)(3.755,-4.000){2}{\rule{0.300pt}{0.400pt}}
\multiput(605.00,685.94)(0.774,-0.468){5}{\rule{0.700pt}{0.113pt}}
\multiput(605.00,686.17)(4.547,-4.000){2}{\rule{0.350pt}{0.400pt}}
\multiput(611.00,681.95)(0.909,-0.447){3}{\rule{0.767pt}{0.108pt}}
\multiput(611.00,682.17)(3.409,-3.000){2}{\rule{0.383pt}{0.400pt}}
\multiput(616.00,678.94)(0.774,-0.468){5}{\rule{0.700pt}{0.113pt}}
\multiput(616.00,679.17)(4.547,-4.000){2}{\rule{0.350pt}{0.400pt}}
\multiput(622.00,674.95)(0.909,-0.447){3}{\rule{0.767pt}{0.108pt}}
\multiput(622.00,675.17)(3.409,-3.000){2}{\rule{0.383pt}{0.400pt}}
\multiput(627.00,671.94)(0.627,-0.468){5}{\rule{0.600pt}{0.113pt}}
\multiput(627.00,672.17)(3.755,-4.000){2}{\rule{0.300pt}{0.400pt}}
\multiput(632.00,667.95)(1.132,-0.447){3}{\rule{0.900pt}{0.108pt}}
\multiput(632.00,668.17)(4.132,-3.000){2}{\rule{0.450pt}{0.400pt}}
\multiput(638.00,664.95)(0.909,-0.447){3}{\rule{0.767pt}{0.108pt}}
\multiput(638.00,665.17)(3.409,-3.000){2}{\rule{0.383pt}{0.400pt}}
\multiput(643.00,661.95)(1.132,-0.447){3}{\rule{0.900pt}{0.108pt}}
\multiput(643.00,662.17)(4.132,-3.000){2}{\rule{0.450pt}{0.400pt}}
\multiput(649.00,658.95)(0.909,-0.447){3}{\rule{0.767pt}{0.108pt}}
\multiput(649.00,659.17)(3.409,-3.000){2}{\rule{0.383pt}{0.400pt}}
\multiput(654.00,655.95)(0.909,-0.447){3}{\rule{0.767pt}{0.108pt}}
\multiput(654.00,656.17)(3.409,-3.000){2}{\rule{0.383pt}{0.400pt}}
\put(659,652.17){\rule{1.300pt}{0.400pt}}
\multiput(659.00,653.17)(3.302,-2.000){2}{\rule{0.650pt}{0.400pt}}
\multiput(665.00,650.95)(0.909,-0.447){3}{\rule{0.767pt}{0.108pt}}
\multiput(665.00,651.17)(3.409,-3.000){2}{\rule{0.383pt}{0.400pt}}
\multiput(670.00,647.95)(0.909,-0.447){3}{\rule{0.767pt}{0.108pt}}
\multiput(670.00,648.17)(3.409,-3.000){2}{\rule{0.383pt}{0.400pt}}
\put(675,644.17){\rule{1.300pt}{0.400pt}}
\multiput(675.00,645.17)(3.302,-2.000){2}{\rule{0.650pt}{0.400pt}}
\put(681,642.17){\rule{1.100pt}{0.400pt}}
\multiput(681.00,643.17)(2.717,-2.000){2}{\rule{0.550pt}{0.400pt}}
\multiput(686.00,640.95)(1.132,-0.447){3}{\rule{0.900pt}{0.108pt}}
\multiput(686.00,641.17)(4.132,-3.000){2}{\rule{0.450pt}{0.400pt}}
\put(692,637.17){\rule{1.100pt}{0.400pt}}
\multiput(692.00,638.17)(2.717,-2.000){2}{\rule{0.550pt}{0.400pt}}
\put(697,635.17){\rule{1.100pt}{0.400pt}}
\multiput(697.00,636.17)(2.717,-2.000){2}{\rule{0.550pt}{0.400pt}}
\put(702,633.17){\rule{1.300pt}{0.400pt}}
\multiput(702.00,634.17)(3.302,-2.000){2}{\rule{0.650pt}{0.400pt}}
\put(708,631.17){\rule{1.100pt}{0.400pt}}
\multiput(708.00,632.17)(2.717,-2.000){2}{\rule{0.550pt}{0.400pt}}
\put(713,629.67){\rule{1.445pt}{0.400pt}}
\multiput(713.00,630.17)(3.000,-1.000){2}{\rule{0.723pt}{0.400pt}}
\put(719,628.17){\rule{1.100pt}{0.400pt}}
\multiput(719.00,629.17)(2.717,-2.000){2}{\rule{0.550pt}{0.400pt}}
\put(724,626.17){\rule{1.100pt}{0.400pt}}
\multiput(724.00,627.17)(2.717,-2.000){2}{\rule{0.550pt}{0.400pt}}
\put(729,624.67){\rule{1.445pt}{0.400pt}}
\multiput(729.00,625.17)(3.000,-1.000){2}{\rule{0.723pt}{0.400pt}}
\put(735,623.67){\rule{1.204pt}{0.400pt}}
\multiput(735.00,624.17)(2.500,-1.000){2}{\rule{0.602pt}{0.400pt}}
\put(740,622.17){\rule{1.300pt}{0.400pt}}
\multiput(740.00,623.17)(3.302,-2.000){2}{\rule{0.650pt}{0.400pt}}
\put(746,620.67){\rule{1.204pt}{0.400pt}}
\multiput(746.00,621.17)(2.500,-1.000){2}{\rule{0.602pt}{0.400pt}}
\put(751,619.67){\rule{1.204pt}{0.400pt}}
\multiput(751.00,620.17)(2.500,-1.000){2}{\rule{0.602pt}{0.400pt}}
\put(756,618.67){\rule{1.445pt}{0.400pt}}
\multiput(756.00,619.17)(3.000,-1.000){2}{\rule{0.723pt}{0.400pt}}
\put(762,617.67){\rule{1.204pt}{0.400pt}}
\multiput(762.00,618.17)(2.500,-1.000){2}{\rule{0.602pt}{0.400pt}}
\put(809.0,549.0){\rule[-0.200pt]{1.204pt}{0.400pt}}
\put(773,616.67){\rule{1.204pt}{0.400pt}}
\multiput(773.00,617.17)(2.500,-1.000){2}{\rule{0.602pt}{0.400pt}}
\put(778,615.67){\rule{1.204pt}{0.400pt}}
\multiput(778.00,616.17)(2.500,-1.000){2}{\rule{0.602pt}{0.400pt}}
\put(767.0,618.0){\rule[-0.200pt]{1.445pt}{0.400pt}}
\put(783.0,616.0){\rule[-0.200pt]{1.445pt}{0.400pt}}
\put(794,614.67){\rule{1.445pt}{0.400pt}}
\multiput(794.00,615.17)(3.000,-1.000){2}{\rule{0.723pt}{0.400pt}}
\put(789.0,616.0){\rule[-0.200pt]{1.204pt}{0.400pt}}
\put(800.0,615.0){\rule[-0.200pt]{1.204pt}{0.400pt}}
\put(805.0,615.0){\rule[-0.200pt]{1.204pt}{0.400pt}}
\put(810.0,615.0){\rule[-0.200pt]{1.445pt}{0.400pt}}
\put(816.0,615.0){\rule[-0.200pt]{1.204pt}{0.400pt}}
\put(827,614.67){\rule{1.204pt}{0.400pt}}
\multiput(827.00,614.17)(2.500,1.000){2}{\rule{0.602pt}{0.400pt}}
\put(821.0,615.0){\rule[-0.200pt]{1.445pt}{0.400pt}}
\put(837,615.67){\rule{1.445pt}{0.400pt}}
\multiput(837.00,615.17)(3.000,1.000){2}{\rule{0.723pt}{0.400pt}}
\put(832.0,616.0){\rule[-0.200pt]{1.204pt}{0.400pt}}
\put(848,616.67){\rule{1.445pt}{0.400pt}}
\multiput(848.00,616.17)(3.000,1.000){2}{\rule{0.723pt}{0.400pt}}
\put(854,617.67){\rule{1.204pt}{0.400pt}}
\multiput(854.00,617.17)(2.500,1.000){2}{\rule{0.602pt}{0.400pt}}
\put(859,618.67){\rule{1.204pt}{0.400pt}}
\multiput(859.00,618.17)(2.500,1.000){2}{\rule{0.602pt}{0.400pt}}
\put(864,619.67){\rule{1.445pt}{0.400pt}}
\multiput(864.00,619.17)(3.000,1.000){2}{\rule{0.723pt}{0.400pt}}
\put(870,620.67){\rule{1.204pt}{0.400pt}}
\multiput(870.00,620.17)(2.500,1.000){2}{\rule{0.602pt}{0.400pt}}
\put(875,621.67){\rule{1.445pt}{0.400pt}}
\multiput(875.00,621.17)(3.000,1.000){2}{\rule{0.723pt}{0.400pt}}
\put(881,622.67){\rule{1.204pt}{0.400pt}}
\multiput(881.00,622.17)(2.500,1.000){2}{\rule{0.602pt}{0.400pt}}
\put(886,624.17){\rule{1.100pt}{0.400pt}}
\multiput(886.00,623.17)(2.717,2.000){2}{\rule{0.550pt}{0.400pt}}
\put(891,625.67){\rule{1.445pt}{0.400pt}}
\multiput(891.00,625.17)(3.000,1.000){2}{\rule{0.723pt}{0.400pt}}
\put(897,627.17){\rule{1.100pt}{0.400pt}}
\multiput(897.00,626.17)(2.717,2.000){2}{\rule{0.550pt}{0.400pt}}
\put(902,629.17){\rule{1.300pt}{0.400pt}}
\multiput(902.00,628.17)(3.302,2.000){2}{\rule{0.650pt}{0.400pt}}
\put(908,630.67){\rule{1.204pt}{0.400pt}}
\multiput(908.00,630.17)(2.500,1.000){2}{\rule{0.602pt}{0.400pt}}
\put(913,632.17){\rule{1.100pt}{0.400pt}}
\multiput(913.00,631.17)(2.717,2.000){2}{\rule{0.550pt}{0.400pt}}
\put(918,634.17){\rule{1.300pt}{0.400pt}}
\multiput(918.00,633.17)(3.302,2.000){2}{\rule{0.650pt}{0.400pt}}
\put(924,636.17){\rule{1.100pt}{0.400pt}}
\multiput(924.00,635.17)(2.717,2.000){2}{\rule{0.550pt}{0.400pt}}
\multiput(929.00,638.61)(1.132,0.447){3}{\rule{0.900pt}{0.108pt}}
\multiput(929.00,637.17)(4.132,3.000){2}{\rule{0.450pt}{0.400pt}}
\put(935,641.17){\rule{1.100pt}{0.400pt}}
\multiput(935.00,640.17)(2.717,2.000){2}{\rule{0.550pt}{0.400pt}}
\put(940,643.17){\rule{1.100pt}{0.400pt}}
\multiput(940.00,642.17)(2.717,2.000){2}{\rule{0.550pt}{0.400pt}}
\multiput(945.00,645.61)(1.132,0.447){3}{\rule{0.900pt}{0.108pt}}
\multiput(945.00,644.17)(4.132,3.000){2}{\rule{0.450pt}{0.400pt}}
\put(951,648.17){\rule{1.100pt}{0.400pt}}
\multiput(951.00,647.17)(2.717,2.000){2}{\rule{0.550pt}{0.400pt}}
\multiput(956.00,650.61)(1.132,0.447){3}{\rule{0.900pt}{0.108pt}}
\multiput(956.00,649.17)(4.132,3.000){2}{\rule{0.450pt}{0.400pt}}
\multiput(962.00,653.61)(0.909,0.447){3}{\rule{0.767pt}{0.108pt}}
\multiput(962.00,652.17)(3.409,3.000){2}{\rule{0.383pt}{0.400pt}}
\multiput(967.00,656.61)(0.909,0.447){3}{\rule{0.767pt}{0.108pt}}
\multiput(967.00,655.17)(3.409,3.000){2}{\rule{0.383pt}{0.400pt}}
\multiput(972.00,659.61)(1.132,0.447){3}{\rule{0.900pt}{0.108pt}}
\multiput(972.00,658.17)(4.132,3.000){2}{\rule{0.450pt}{0.400pt}}
\multiput(978.00,662.61)(0.909,0.447){3}{\rule{0.767pt}{0.108pt}}
\multiput(978.00,661.17)(3.409,3.000){2}{\rule{0.383pt}{0.400pt}}
\multiput(983.00,665.61)(1.132,0.447){3}{\rule{0.900pt}{0.108pt}}
\multiput(983.00,664.17)(4.132,3.000){2}{\rule{0.450pt}{0.400pt}}
\multiput(989.00,668.61)(0.909,0.447){3}{\rule{0.767pt}{0.108pt}}
\multiput(989.00,667.17)(3.409,3.000){2}{\rule{0.383pt}{0.400pt}}
\multiput(994.00,671.60)(0.627,0.468){5}{\rule{0.600pt}{0.113pt}}
\multiput(994.00,670.17)(3.755,4.000){2}{\rule{0.300pt}{0.400pt}}
\multiput(999.00,675.61)(1.132,0.447){3}{\rule{0.900pt}{0.108pt}}
\multiput(999.00,674.17)(4.132,3.000){2}{\rule{0.450pt}{0.400pt}}
\multiput(1005.00,678.60)(0.627,0.468){5}{\rule{0.600pt}{0.113pt}}
\multiput(1005.00,677.17)(3.755,4.000){2}{\rule{0.300pt}{0.400pt}}
\multiput(1010.00,682.61)(1.132,0.447){3}{\rule{0.900pt}{0.108pt}}
\multiput(1010.00,681.17)(4.132,3.000){2}{\rule{0.450pt}{0.400pt}}
\multiput(1016.00,685.60)(0.627,0.468){5}{\rule{0.600pt}{0.113pt}}
\multiput(1016.00,684.17)(3.755,4.000){2}{\rule{0.300pt}{0.400pt}}
\multiput(178.00,611.95)(0.462,-0.447){3}{\rule{0.500pt}{0.108pt}}
\multiput(178.00,612.17)(1.962,-3.000){2}{\rule{0.250pt}{0.400pt}}
\multiput(181.00,608.95)(0.462,-0.447){3}{\rule{0.500pt}{0.108pt}}
\multiput(181.00,609.17)(1.962,-3.000){2}{\rule{0.250pt}{0.400pt}}
\multiput(184.00,605.95)(0.462,-0.447){3}{\rule{0.500pt}{0.108pt}}
\multiput(184.00,606.17)(1.962,-3.000){2}{\rule{0.250pt}{0.400pt}}
\multiput(187.00,602.95)(0.685,-0.447){3}{\rule{0.633pt}{0.108pt}}
\multiput(187.00,603.17)(2.685,-3.000){2}{\rule{0.317pt}{0.400pt}}
\put(191,599.17){\rule{0.700pt}{0.400pt}}
\multiput(191.00,600.17)(1.547,-2.000){2}{\rule{0.350pt}{0.400pt}}
\multiput(194.00,597.95)(0.462,-0.447){3}{\rule{0.500pt}{0.108pt}}
\multiput(194.00,598.17)(1.962,-3.000){2}{\rule{0.250pt}{0.400pt}}
\put(197,594.17){\rule{0.700pt}{0.400pt}}
\multiput(197.00,595.17)(1.547,-2.000){2}{\rule{0.350pt}{0.400pt}}
\multiput(200.00,592.95)(0.462,-0.447){3}{\rule{0.500pt}{0.108pt}}
\multiput(200.00,593.17)(1.962,-3.000){2}{\rule{0.250pt}{0.400pt}}
\put(203,589.17){\rule{0.700pt}{0.400pt}}
\multiput(203.00,590.17)(1.547,-2.000){2}{\rule{0.350pt}{0.400pt}}
\put(206,587.17){\rule{0.700pt}{0.400pt}}
\multiput(206.00,588.17)(1.547,-2.000){2}{\rule{0.350pt}{0.400pt}}
\put(209,585.17){\rule{0.700pt}{0.400pt}}
\multiput(209.00,586.17)(1.547,-2.000){2}{\rule{0.350pt}{0.400pt}}
\put(212,583.17){\rule{0.700pt}{0.400pt}}
\multiput(212.00,584.17)(1.547,-2.000){2}{\rule{0.350pt}{0.400pt}}
\put(215,581.17){\rule{0.900pt}{0.400pt}}
\multiput(215.00,582.17)(2.132,-2.000){2}{\rule{0.450pt}{0.400pt}}
\put(219,579.67){\rule{0.723pt}{0.400pt}}
\multiput(219.00,580.17)(1.500,-1.000){2}{\rule{0.361pt}{0.400pt}}
\put(222,578.17){\rule{0.700pt}{0.400pt}}
\multiput(222.00,579.17)(1.547,-2.000){2}{\rule{0.350pt}{0.400pt}}
\put(225,576.17){\rule{0.700pt}{0.400pt}}
\multiput(225.00,577.17)(1.547,-2.000){2}{\rule{0.350pt}{0.400pt}}
\put(228,574.67){\rule{0.723pt}{0.400pt}}
\multiput(228.00,575.17)(1.500,-1.000){2}{\rule{0.361pt}{0.400pt}}
\put(231,573.67){\rule{0.723pt}{0.400pt}}
\multiput(231.00,574.17)(1.500,-1.000){2}{\rule{0.361pt}{0.400pt}}
\put(234,572.17){\rule{0.700pt}{0.400pt}}
\multiput(234.00,573.17)(1.547,-2.000){2}{\rule{0.350pt}{0.400pt}}
\put(237,570.67){\rule{0.723pt}{0.400pt}}
\multiput(237.00,571.17)(1.500,-1.000){2}{\rule{0.361pt}{0.400pt}}
\put(240,569.67){\rule{0.964pt}{0.400pt}}
\multiput(240.00,570.17)(2.000,-1.000){2}{\rule{0.482pt}{0.400pt}}
\put(244,568.67){\rule{0.723pt}{0.400pt}}
\multiput(244.00,569.17)(1.500,-1.000){2}{\rule{0.361pt}{0.400pt}}
\put(247,567.67){\rule{0.723pt}{0.400pt}}
\multiput(247.00,568.17)(1.500,-1.000){2}{\rule{0.361pt}{0.400pt}}
\put(843.0,617.0){\rule[-0.200pt]{1.204pt}{0.400pt}}
\put(253,566.67){\rule{0.723pt}{0.400pt}}
\multiput(253.00,567.17)(1.500,-1.000){2}{\rule{0.361pt}{0.400pt}}
\put(250.0,568.0){\rule[-0.200pt]{0.723pt}{0.400pt}}
\put(259,565.67){\rule{0.723pt}{0.400pt}}
\multiput(259.00,566.17)(1.500,-1.000){2}{\rule{0.361pt}{0.400pt}}
\put(256.0,567.0){\rule[-0.200pt]{0.723pt}{0.400pt}}
\put(262.0,566.0){\rule[-0.200pt]{0.723pt}{0.400pt}}
\put(268,564.67){\rule{0.964pt}{0.400pt}}
\multiput(268.00,565.17)(2.000,-1.000){2}{\rule{0.482pt}{0.400pt}}
\put(265.0,566.0){\rule[-0.200pt]{0.723pt}{0.400pt}}
\put(272.0,565.0){\rule[-0.200pt]{0.723pt}{0.400pt}}
\put(278,564.67){\rule{0.723pt}{0.400pt}}
\multiput(278.00,564.17)(1.500,1.000){2}{\rule{0.361pt}{0.400pt}}
\put(275.0,565.0){\rule[-0.200pt]{0.723pt}{0.400pt}}
\put(281.0,566.0){\rule[-0.200pt]{0.723pt}{0.400pt}}
\put(287,565.67){\rule{0.723pt}{0.400pt}}
\multiput(287.00,565.17)(1.500,1.000){2}{\rule{0.361pt}{0.400pt}}
\put(284.0,566.0){\rule[-0.200pt]{0.723pt}{0.400pt}}
\put(293,566.67){\rule{0.723pt}{0.400pt}}
\multiput(293.00,566.17)(1.500,1.000){2}{\rule{0.361pt}{0.400pt}}
\put(296,567.67){\rule{0.964pt}{0.400pt}}
\multiput(296.00,567.17)(2.000,1.000){2}{\rule{0.482pt}{0.400pt}}
\put(290.0,567.0){\rule[-0.200pt]{0.723pt}{0.400pt}}
\put(303,568.67){\rule{0.723pt}{0.400pt}}
\multiput(303.00,568.17)(1.500,1.000){2}{\rule{0.361pt}{0.400pt}}
\put(306,569.67){\rule{0.723pt}{0.400pt}}
\multiput(306.00,569.17)(1.500,1.000){2}{\rule{0.361pt}{0.400pt}}
\put(309,571.17){\rule{0.700pt}{0.400pt}}
\multiput(309.00,570.17)(1.547,2.000){2}{\rule{0.350pt}{0.400pt}}
\put(312,572.67){\rule{0.723pt}{0.400pt}}
\multiput(312.00,572.17)(1.500,1.000){2}{\rule{0.361pt}{0.400pt}}
\put(315,573.67){\rule{0.723pt}{0.400pt}}
\multiput(315.00,573.17)(1.500,1.000){2}{\rule{0.361pt}{0.400pt}}
\put(318,575.17){\rule{0.700pt}{0.400pt}}
\multiput(318.00,574.17)(1.547,2.000){2}{\rule{0.350pt}{0.400pt}}
\put(321,576.67){\rule{0.964pt}{0.400pt}}
\multiput(321.00,576.17)(2.000,1.000){2}{\rule{0.482pt}{0.400pt}}
\put(325,578.17){\rule{0.700pt}{0.400pt}}
\multiput(325.00,577.17)(1.547,2.000){2}{\rule{0.350pt}{0.400pt}}
\put(328,579.67){\rule{0.723pt}{0.400pt}}
\multiput(328.00,579.17)(1.500,1.000){2}{\rule{0.361pt}{0.400pt}}
\put(331,581.17){\rule{0.700pt}{0.400pt}}
\multiput(331.00,580.17)(1.547,2.000){2}{\rule{0.350pt}{0.400pt}}
\put(334,583.17){\rule{0.700pt}{0.400pt}}
\multiput(334.00,582.17)(1.547,2.000){2}{\rule{0.350pt}{0.400pt}}
\put(337,585.17){\rule{0.700pt}{0.400pt}}
\multiput(337.00,584.17)(1.547,2.000){2}{\rule{0.350pt}{0.400pt}}
\put(340,587.17){\rule{0.700pt}{0.400pt}}
\multiput(340.00,586.17)(1.547,2.000){2}{\rule{0.350pt}{0.400pt}}
\multiput(343.00,589.61)(0.462,0.447){3}{\rule{0.500pt}{0.108pt}}
\multiput(343.00,588.17)(1.962,3.000){2}{\rule{0.250pt}{0.400pt}}
\put(346,592.17){\rule{0.700pt}{0.400pt}}
\multiput(346.00,591.17)(1.547,2.000){2}{\rule{0.350pt}{0.400pt}}
\put(349,594.17){\rule{0.900pt}{0.400pt}}
\multiput(349.00,593.17)(2.132,2.000){2}{\rule{0.450pt}{0.400pt}}
\multiput(353.00,596.61)(0.462,0.447){3}{\rule{0.500pt}{0.108pt}}
\multiput(353.00,595.17)(1.962,3.000){2}{\rule{0.250pt}{0.400pt}}
\put(356,599.17){\rule{0.700pt}{0.400pt}}
\multiput(356.00,598.17)(1.547,2.000){2}{\rule{0.350pt}{0.400pt}}
\multiput(359.00,601.61)(0.462,0.447){3}{\rule{0.500pt}{0.108pt}}
\multiput(359.00,600.17)(1.962,3.000){2}{\rule{0.250pt}{0.400pt}}
\multiput(362.00,604.61)(0.462,0.447){3}{\rule{0.500pt}{0.108pt}}
\multiput(362.00,603.17)(1.962,3.000){2}{\rule{0.250pt}{0.400pt}}
\multiput(365.00,607.61)(0.462,0.447){3}{\rule{0.500pt}{0.108pt}}
\multiput(365.00,606.17)(1.962,3.000){2}{\rule{0.250pt}{0.400pt}}
\multiput(368.00,610.61)(0.462,0.447){3}{\rule{0.500pt}{0.108pt}}
\multiput(368.00,609.17)(1.962,3.000){2}{\rule{0.250pt}{0.400pt}}
\multiput(371.00,613.61)(0.462,0.447){3}{\rule{0.500pt}{0.108pt}}
\multiput(371.00,612.17)(1.962,3.000){2}{\rule{0.250pt}{0.400pt}}
\multiput(374.00,616.61)(0.685,0.447){3}{\rule{0.633pt}{0.108pt}}
\multiput(374.00,615.17)(2.685,3.000){2}{\rule{0.317pt}{0.400pt}}
\multiput(378.61,619.00)(0.447,0.685){3}{\rule{0.108pt}{0.633pt}}
\multiput(377.17,619.00)(3.000,2.685){2}{\rule{0.400pt}{0.317pt}}
\multiput(381.00,623.61)(0.462,0.447){3}{\rule{0.500pt}{0.108pt}}
\multiput(381.00,622.17)(1.962,3.000){2}{\rule{0.250pt}{0.400pt}}
\multiput(384.61,626.00)(0.447,0.685){3}{\rule{0.108pt}{0.633pt}}
\multiput(383.17,626.00)(3.000,2.685){2}{\rule{0.400pt}{0.317pt}}
\multiput(387.00,630.61)(0.462,0.447){3}{\rule{0.500pt}{0.108pt}}
\multiput(387.00,629.17)(1.962,3.000){2}{\rule{0.250pt}{0.400pt}}
\multiput(390.61,633.00)(0.447,0.685){3}{\rule{0.108pt}{0.633pt}}
\multiput(389.17,633.00)(3.000,2.685){2}{\rule{0.400pt}{0.317pt}}
\multiput(393.61,637.00)(0.447,0.685){3}{\rule{0.108pt}{0.633pt}}
\multiput(392.17,637.00)(3.000,2.685){2}{\rule{0.400pt}{0.317pt}}
\multiput(396.61,641.00)(0.447,0.685){3}{\rule{0.108pt}{0.633pt}}
\multiput(395.17,641.00)(3.000,2.685){2}{\rule{0.400pt}{0.317pt}}
\multiput(399.61,645.00)(0.447,0.685){3}{\rule{0.108pt}{0.633pt}}
\multiput(398.17,645.00)(3.000,2.685){2}{\rule{0.400pt}{0.317pt}}
\multiput(402.00,649.60)(0.481,0.468){5}{\rule{0.500pt}{0.113pt}}
\multiput(402.00,648.17)(2.962,4.000){2}{\rule{0.250pt}{0.400pt}}
\multiput(406.61,653.00)(0.447,0.685){3}{\rule{0.108pt}{0.633pt}}
\multiput(405.17,653.00)(3.000,2.685){2}{\rule{0.400pt}{0.317pt}}
\multiput(409.61,657.00)(0.447,0.685){3}{\rule{0.108pt}{0.633pt}}
\multiput(408.17,657.00)(3.000,2.685){2}{\rule{0.400pt}{0.317pt}}
\multiput(412.61,661.00)(0.447,0.909){3}{\rule{0.108pt}{0.767pt}}
\multiput(411.17,661.00)(3.000,3.409){2}{\rule{0.400pt}{0.383pt}}
\multiput(415.61,666.00)(0.447,0.685){3}{\rule{0.108pt}{0.633pt}}
\multiput(414.17,666.00)(3.000,2.685){2}{\rule{0.400pt}{0.317pt}}
\multiput(418.61,670.00)(0.447,0.909){3}{\rule{0.108pt}{0.767pt}}
\multiput(417.17,670.00)(3.000,3.409){2}{\rule{0.400pt}{0.383pt}}
\multiput(421.61,675.00)(0.447,0.685){3}{\rule{0.108pt}{0.633pt}}
\multiput(420.17,675.00)(3.000,2.685){2}{\rule{0.400pt}{0.317pt}}
\multiput(424.61,679.00)(0.447,0.909){3}{\rule{0.108pt}{0.767pt}}
\multiput(423.17,679.00)(3.000,3.409){2}{\rule{0.400pt}{0.383pt}}
\multiput(427.61,684.00)(0.447,0.909){3}{\rule{0.108pt}{0.767pt}}
\multiput(426.17,684.00)(3.000,3.409){2}{\rule{0.400pt}{0.383pt}}
\multiput(430.60,689.00)(0.468,0.627){5}{\rule{0.113pt}{0.600pt}}
\multiput(429.17,689.00)(4.000,3.755){2}{\rule{0.400pt}{0.300pt}}
\multiput(434.61,694.00)(0.447,0.909){3}{\rule{0.108pt}{0.767pt}}
\multiput(433.17,694.00)(3.000,3.409){2}{\rule{0.400pt}{0.383pt}}
\multiput(437.61,699.00)(0.447,0.909){3}{\rule{0.108pt}{0.767pt}}
\multiput(436.17,699.00)(3.000,3.409){2}{\rule{0.400pt}{0.383pt}}
\multiput(440.61,704.00)(0.447,1.132){3}{\rule{0.108pt}{0.900pt}}
\multiput(439.17,704.00)(3.000,4.132){2}{\rule{0.400pt}{0.450pt}}
\multiput(443.61,710.00)(0.447,0.909){3}{\rule{0.108pt}{0.767pt}}
\multiput(442.17,710.00)(3.000,3.409){2}{\rule{0.400pt}{0.383pt}}
\multiput(446.61,715.00)(0.447,0.909){3}{\rule{0.108pt}{0.767pt}}
\multiput(445.17,715.00)(3.000,3.409){2}{\rule{0.400pt}{0.383pt}}
\multiput(449.61,720.00)(0.447,1.132){3}{\rule{0.108pt}{0.900pt}}
\multiput(448.17,720.00)(3.000,4.132){2}{\rule{0.400pt}{0.450pt}}
\multiput(452.61,726.00)(0.447,1.132){3}{\rule{0.108pt}{0.900pt}}
\multiput(451.17,726.00)(3.000,4.132){2}{\rule{0.400pt}{0.450pt}}
\multiput(455.60,732.00)(0.468,0.774){5}{\rule{0.113pt}{0.700pt}}
\multiput(454.17,732.00)(4.000,4.547){2}{\rule{0.400pt}{0.350pt}}
\multiput(459.61,738.00)(0.447,0.909){3}{\rule{0.108pt}{0.767pt}}
\multiput(458.17,738.00)(3.000,3.409){2}{\rule{0.400pt}{0.383pt}}
\multiput(462.61,743.00)(0.447,1.132){3}{\rule{0.108pt}{0.900pt}}
\multiput(461.17,743.00)(3.000,4.132){2}{\rule{0.400pt}{0.450pt}}
\multiput(465.61,749.00)(0.447,1.132){3}{\rule{0.108pt}{0.900pt}}
\multiput(464.17,749.00)(3.000,4.132){2}{\rule{0.400pt}{0.450pt}}
\multiput(468.61,755.00)(0.447,1.355){3}{\rule{0.108pt}{1.033pt}}
\multiput(467.17,755.00)(3.000,4.855){2}{\rule{0.400pt}{0.517pt}}
\multiput(471.61,762.00)(0.447,1.132){3}{\rule{0.108pt}{0.900pt}}
\multiput(470.17,762.00)(3.000,4.132){2}{\rule{0.400pt}{0.450pt}}
\multiput(474.61,768.00)(0.447,1.132){3}{\rule{0.108pt}{0.900pt}}
\multiput(473.17,768.00)(3.000,4.132){2}{\rule{0.400pt}{0.450pt}}
\multiput(477.61,774.00)(0.447,1.355){3}{\rule{0.108pt}{1.033pt}}
\multiput(476.17,774.00)(3.000,4.855){2}{\rule{0.400pt}{0.517pt}}
\multiput(480.61,781.00)(0.447,1.132){3}{\rule{0.108pt}{0.900pt}}
\multiput(479.17,781.00)(3.000,4.132){2}{\rule{0.400pt}{0.450pt}}
\multiput(483.60,787.00)(0.468,0.920){5}{\rule{0.113pt}{0.800pt}}
\multiput(482.17,787.00)(4.000,5.340){2}{\rule{0.400pt}{0.400pt}}
\multiput(237.00,551.95)(0.685,-0.447){3}{\rule{0.633pt}{0.108pt}}
\multiput(237.00,552.17)(2.685,-3.000){2}{\rule{0.317pt}{0.400pt}}
\multiput(241.00,548.95)(0.462,-0.447){3}{\rule{0.500pt}{0.108pt}}
\multiput(241.00,549.17)(1.962,-3.000){2}{\rule{0.250pt}{0.400pt}}
\multiput(244.00,545.95)(0.462,-0.447){3}{\rule{0.500pt}{0.108pt}}
\multiput(244.00,546.17)(1.962,-3.000){2}{\rule{0.250pt}{0.400pt}}
\put(247,542.17){\rule{0.700pt}{0.400pt}}
\multiput(247.00,543.17)(1.547,-2.000){2}{\rule{0.350pt}{0.400pt}}
\multiput(250.00,540.95)(0.462,-0.447){3}{\rule{0.500pt}{0.108pt}}
\multiput(250.00,541.17)(1.962,-3.000){2}{\rule{0.250pt}{0.400pt}}
\put(253,537.17){\rule{0.700pt}{0.400pt}}
\multiput(253.00,538.17)(1.547,-2.000){2}{\rule{0.350pt}{0.400pt}}
\multiput(256.00,535.95)(0.462,-0.447){3}{\rule{0.500pt}{0.108pt}}
\multiput(256.00,536.17)(1.962,-3.000){2}{\rule{0.250pt}{0.400pt}}
\put(259,532.17){\rule{0.700pt}{0.400pt}}
\multiput(259.00,533.17)(1.547,-2.000){2}{\rule{0.350pt}{0.400pt}}
\put(262,530.17){\rule{0.700pt}{0.400pt}}
\multiput(262.00,531.17)(1.547,-2.000){2}{\rule{0.350pt}{0.400pt}}
\put(265,528.17){\rule{0.900pt}{0.400pt}}
\multiput(265.00,529.17)(2.132,-2.000){2}{\rule{0.450pt}{0.400pt}}
\put(269,526.17){\rule{0.700pt}{0.400pt}}
\multiput(269.00,527.17)(1.547,-2.000){2}{\rule{0.350pt}{0.400pt}}
\put(272,524.17){\rule{0.700pt}{0.400pt}}
\multiput(272.00,525.17)(1.547,-2.000){2}{\rule{0.350pt}{0.400pt}}
\put(275,522.17){\rule{0.700pt}{0.400pt}}
\multiput(275.00,523.17)(1.547,-2.000){2}{\rule{0.350pt}{0.400pt}}
\put(278,520.17){\rule{0.700pt}{0.400pt}}
\multiput(278.00,521.17)(1.547,-2.000){2}{\rule{0.350pt}{0.400pt}}
\put(281,518.17){\rule{0.700pt}{0.400pt}}
\multiput(281.00,519.17)(1.547,-2.000){2}{\rule{0.350pt}{0.400pt}}
\put(284,516.67){\rule{0.723pt}{0.400pt}}
\multiput(284.00,517.17)(1.500,-1.000){2}{\rule{0.361pt}{0.400pt}}
\put(287,515.67){\rule{0.723pt}{0.400pt}}
\multiput(287.00,516.17)(1.500,-1.000){2}{\rule{0.361pt}{0.400pt}}
\put(290,514.17){\rule{0.900pt}{0.400pt}}
\multiput(290.00,515.17)(2.132,-2.000){2}{\rule{0.450pt}{0.400pt}}
\put(294,512.67){\rule{0.723pt}{0.400pt}}
\multiput(294.00,513.17)(1.500,-1.000){2}{\rule{0.361pt}{0.400pt}}
\put(297,511.67){\rule{0.723pt}{0.400pt}}
\multiput(297.00,512.17)(1.500,-1.000){2}{\rule{0.361pt}{0.400pt}}
\put(300,510.67){\rule{0.723pt}{0.400pt}}
\multiput(300.00,511.17)(1.500,-1.000){2}{\rule{0.361pt}{0.400pt}}
\put(303,509.67){\rule{0.723pt}{0.400pt}}
\multiput(303.00,510.17)(1.500,-1.000){2}{\rule{0.361pt}{0.400pt}}
\put(306,508.67){\rule{0.723pt}{0.400pt}}
\multiput(306.00,509.17)(1.500,-1.000){2}{\rule{0.361pt}{0.400pt}}
\put(309,507.67){\rule{0.723pt}{0.400pt}}
\multiput(309.00,508.17)(1.500,-1.000){2}{\rule{0.361pt}{0.400pt}}
\put(300.0,569.0){\rule[-0.200pt]{0.723pt}{0.400pt}}
\put(315,506.67){\rule{0.723pt}{0.400pt}}
\multiput(315.00,507.17)(1.500,-1.000){2}{\rule{0.361pt}{0.400pt}}
\put(312.0,508.0){\rule[-0.200pt]{0.723pt}{0.400pt}}
\put(322,505.67){\rule{0.723pt}{0.400pt}}
\multiput(322.00,506.17)(1.500,-1.000){2}{\rule{0.361pt}{0.400pt}}
\put(318.0,507.0){\rule[-0.200pt]{0.964pt}{0.400pt}}
\put(325.0,506.0){\rule[-0.200pt]{0.723pt}{0.400pt}}
\put(328.0,506.0){\rule[-0.200pt]{0.723pt}{0.400pt}}
\put(331.0,506.0){\rule[-0.200pt]{0.723pt}{0.400pt}}
\put(334.0,506.0){\rule[-0.200pt]{0.723pt}{0.400pt}}
\put(337.0,506.0){\rule[-0.200pt]{0.723pt}{0.400pt}}
\put(343,505.67){\rule{0.964pt}{0.400pt}}
\multiput(343.00,505.17)(2.000,1.000){2}{\rule{0.482pt}{0.400pt}}
\put(340.0,506.0){\rule[-0.200pt]{0.723pt}{0.400pt}}
\put(350,506.67){\rule{0.723pt}{0.400pt}}
\multiput(350.00,506.17)(1.500,1.000){2}{\rule{0.361pt}{0.400pt}}
\put(347.0,507.0){\rule[-0.200pt]{0.723pt}{0.400pt}}
\put(356,507.67){\rule{0.723pt}{0.400pt}}
\multiput(356.00,507.17)(1.500,1.000){2}{\rule{0.361pt}{0.400pt}}
\put(359,508.67){\rule{0.723pt}{0.400pt}}
\multiput(359.00,508.17)(1.500,1.000){2}{\rule{0.361pt}{0.400pt}}
\put(362,509.67){\rule{0.723pt}{0.400pt}}
\multiput(362.00,509.17)(1.500,1.000){2}{\rule{0.361pt}{0.400pt}}
\put(365,510.67){\rule{0.723pt}{0.400pt}}
\multiput(365.00,510.17)(1.500,1.000){2}{\rule{0.361pt}{0.400pt}}
\put(368,511.67){\rule{0.723pt}{0.400pt}}
\multiput(368.00,511.17)(1.500,1.000){2}{\rule{0.361pt}{0.400pt}}
\put(371,512.67){\rule{0.964pt}{0.400pt}}
\multiput(371.00,512.17)(2.000,1.000){2}{\rule{0.482pt}{0.400pt}}
\put(375,514.17){\rule{0.700pt}{0.400pt}}
\multiput(375.00,513.17)(1.547,2.000){2}{\rule{0.350pt}{0.400pt}}
\put(378,515.67){\rule{0.723pt}{0.400pt}}
\multiput(378.00,515.17)(1.500,1.000){2}{\rule{0.361pt}{0.400pt}}
\put(381,517.17){\rule{0.700pt}{0.400pt}}
\multiput(381.00,516.17)(1.547,2.000){2}{\rule{0.350pt}{0.400pt}}
\put(384,518.67){\rule{0.723pt}{0.400pt}}
\multiput(384.00,518.17)(1.500,1.000){2}{\rule{0.361pt}{0.400pt}}
\put(387,520.17){\rule{0.700pt}{0.400pt}}
\multiput(387.00,519.17)(1.547,2.000){2}{\rule{0.350pt}{0.400pt}}
\put(390,522.17){\rule{0.700pt}{0.400pt}}
\multiput(390.00,521.17)(1.547,2.000){2}{\rule{0.350pt}{0.400pt}}
\put(393,524.17){\rule{0.700pt}{0.400pt}}
\multiput(393.00,523.17)(1.547,2.000){2}{\rule{0.350pt}{0.400pt}}
\put(396,526.17){\rule{0.700pt}{0.400pt}}
\multiput(396.00,525.17)(1.547,2.000){2}{\rule{0.350pt}{0.400pt}}
\put(399,528.17){\rule{0.900pt}{0.400pt}}
\multiput(399.00,527.17)(2.132,2.000){2}{\rule{0.450pt}{0.400pt}}
\put(403,530.17){\rule{0.700pt}{0.400pt}}
\multiput(403.00,529.17)(1.547,2.000){2}{\rule{0.350pt}{0.400pt}}
\multiput(406.00,532.61)(0.462,0.447){3}{\rule{0.500pt}{0.108pt}}
\multiput(406.00,531.17)(1.962,3.000){2}{\rule{0.250pt}{0.400pt}}
\put(409,535.17){\rule{0.700pt}{0.400pt}}
\multiput(409.00,534.17)(1.547,2.000){2}{\rule{0.350pt}{0.400pt}}
\put(412,537.17){\rule{0.700pt}{0.400pt}}
\multiput(412.00,536.17)(1.547,2.000){2}{\rule{0.350pt}{0.400pt}}
\multiput(415.00,539.61)(0.462,0.447){3}{\rule{0.500pt}{0.108pt}}
\multiput(415.00,538.17)(1.962,3.000){2}{\rule{0.250pt}{0.400pt}}
\multiput(418.00,542.61)(0.462,0.447){3}{\rule{0.500pt}{0.108pt}}
\multiput(418.00,541.17)(1.962,3.000){2}{\rule{0.250pt}{0.400pt}}
\multiput(421.00,545.61)(0.462,0.447){3}{\rule{0.500pt}{0.108pt}}
\multiput(421.00,544.17)(1.962,3.000){2}{\rule{0.250pt}{0.400pt}}
\multiput(424.00,548.61)(0.685,0.447){3}{\rule{0.633pt}{0.108pt}}
\multiput(424.00,547.17)(2.685,3.000){2}{\rule{0.317pt}{0.400pt}}
\multiput(428.00,551.61)(0.462,0.447){3}{\rule{0.500pt}{0.108pt}}
\multiput(428.00,550.17)(1.962,3.000){2}{\rule{0.250pt}{0.400pt}}
\multiput(431.00,554.61)(0.462,0.447){3}{\rule{0.500pt}{0.108pt}}
\multiput(431.00,553.17)(1.962,3.000){2}{\rule{0.250pt}{0.400pt}}
\multiput(434.00,557.61)(0.462,0.447){3}{\rule{0.500pt}{0.108pt}}
\multiput(434.00,556.17)(1.962,3.000){2}{\rule{0.250pt}{0.400pt}}
\multiput(437.00,560.61)(0.462,0.447){3}{\rule{0.500pt}{0.108pt}}
\multiput(437.00,559.17)(1.962,3.000){2}{\rule{0.250pt}{0.400pt}}
\multiput(440.61,563.00)(0.447,0.685){3}{\rule{0.108pt}{0.633pt}}
\multiput(439.17,563.00)(3.000,2.685){2}{\rule{0.400pt}{0.317pt}}
\multiput(443.00,567.61)(0.462,0.447){3}{\rule{0.500pt}{0.108pt}}
\multiput(443.00,566.17)(1.962,3.000){2}{\rule{0.250pt}{0.400pt}}
\multiput(446.61,570.00)(0.447,0.685){3}{\rule{0.108pt}{0.633pt}}
\multiput(445.17,570.00)(3.000,2.685){2}{\rule{0.400pt}{0.317pt}}
\multiput(449.00,574.61)(0.462,0.447){3}{\rule{0.500pt}{0.108pt}}
\multiput(449.00,573.17)(1.962,3.000){2}{\rule{0.250pt}{0.400pt}}
\multiput(452.00,577.60)(0.481,0.468){5}{\rule{0.500pt}{0.113pt}}
\multiput(452.00,576.17)(2.962,4.000){2}{\rule{0.250pt}{0.400pt}}
\multiput(456.61,581.00)(0.447,0.685){3}{\rule{0.108pt}{0.633pt}}
\multiput(455.17,581.00)(3.000,2.685){2}{\rule{0.400pt}{0.317pt}}
\multiput(459.61,585.00)(0.447,0.685){3}{\rule{0.108pt}{0.633pt}}
\multiput(458.17,585.00)(3.000,2.685){2}{\rule{0.400pt}{0.317pt}}
\multiput(462.61,589.00)(0.447,0.685){3}{\rule{0.108pt}{0.633pt}}
\multiput(461.17,589.00)(3.000,2.685){2}{\rule{0.400pt}{0.317pt}}
\multiput(465.61,593.00)(0.447,0.685){3}{\rule{0.108pt}{0.633pt}}
\multiput(464.17,593.00)(3.000,2.685){2}{\rule{0.400pt}{0.317pt}}
\multiput(468.61,597.00)(0.447,0.909){3}{\rule{0.108pt}{0.767pt}}
\multiput(467.17,597.00)(3.000,3.409){2}{\rule{0.400pt}{0.383pt}}
\multiput(471.61,602.00)(0.447,0.685){3}{\rule{0.108pt}{0.633pt}}
\multiput(470.17,602.00)(3.000,2.685){2}{\rule{0.400pt}{0.317pt}}
\multiput(474.61,606.00)(0.447,0.909){3}{\rule{0.108pt}{0.767pt}}
\multiput(473.17,606.00)(3.000,3.409){2}{\rule{0.400pt}{0.383pt}}
\multiput(477.00,611.60)(0.481,0.468){5}{\rule{0.500pt}{0.113pt}}
\multiput(477.00,610.17)(2.962,4.000){2}{\rule{0.250pt}{0.400pt}}
\multiput(481.61,615.00)(0.447,0.909){3}{\rule{0.108pt}{0.767pt}}
\multiput(480.17,615.00)(3.000,3.409){2}{\rule{0.400pt}{0.383pt}}
\multiput(484.61,620.00)(0.447,0.909){3}{\rule{0.108pt}{0.767pt}}
\multiput(483.17,620.00)(3.000,3.409){2}{\rule{0.400pt}{0.383pt}}
\multiput(487.61,625.00)(0.447,0.909){3}{\rule{0.108pt}{0.767pt}}
\multiput(486.17,625.00)(3.000,3.409){2}{\rule{0.400pt}{0.383pt}}
\multiput(490.61,630.00)(0.447,0.909){3}{\rule{0.108pt}{0.767pt}}
\multiput(489.17,630.00)(3.000,3.409){2}{\rule{0.400pt}{0.383pt}}
\multiput(493.61,635.00)(0.447,0.909){3}{\rule{0.108pt}{0.767pt}}
\multiput(492.17,635.00)(3.000,3.409){2}{\rule{0.400pt}{0.383pt}}
\multiput(496.61,640.00)(0.447,0.909){3}{\rule{0.108pt}{0.767pt}}
\multiput(495.17,640.00)(3.000,3.409){2}{\rule{0.400pt}{0.383pt}}
\multiput(499.61,645.00)(0.447,0.909){3}{\rule{0.108pt}{0.767pt}}
\multiput(498.17,645.00)(3.000,3.409){2}{\rule{0.400pt}{0.383pt}}
\multiput(502.61,650.00)(0.447,1.132){3}{\rule{0.108pt}{0.900pt}}
\multiput(501.17,650.00)(3.000,4.132){2}{\rule{0.400pt}{0.450pt}}
\multiput(505.60,656.00)(0.468,0.627){5}{\rule{0.113pt}{0.600pt}}
\multiput(504.17,656.00)(4.000,3.755){2}{\rule{0.400pt}{0.300pt}}
\multiput(509.61,661.00)(0.447,1.132){3}{\rule{0.108pt}{0.900pt}}
\multiput(508.17,661.00)(3.000,4.132){2}{\rule{0.400pt}{0.450pt}}
\multiput(512.61,667.00)(0.447,0.909){3}{\rule{0.108pt}{0.767pt}}
\multiput(511.17,667.00)(3.000,3.409){2}{\rule{0.400pt}{0.383pt}}
\multiput(515.61,672.00)(0.447,1.132){3}{\rule{0.108pt}{0.900pt}}
\multiput(514.17,672.00)(3.000,4.132){2}{\rule{0.400pt}{0.450pt}}
\multiput(518.61,678.00)(0.447,1.132){3}{\rule{0.108pt}{0.900pt}}
\multiput(517.17,678.00)(3.000,4.132){2}{\rule{0.400pt}{0.450pt}}
\multiput(521.61,684.00)(0.447,1.132){3}{\rule{0.108pt}{0.900pt}}
\multiput(520.17,684.00)(3.000,4.132){2}{\rule{0.400pt}{0.450pt}}
\multiput(524.61,690.00)(0.447,1.132){3}{\rule{0.108pt}{0.900pt}}
\multiput(523.17,690.00)(3.000,4.132){2}{\rule{0.400pt}{0.450pt}}
\multiput(527.61,696.00)(0.447,1.132){3}{\rule{0.108pt}{0.900pt}}
\multiput(526.17,696.00)(3.000,4.132){2}{\rule{0.400pt}{0.450pt}}
\multiput(530.61,702.00)(0.447,1.355){3}{\rule{0.108pt}{1.033pt}}
\multiput(529.17,702.00)(3.000,4.855){2}{\rule{0.400pt}{0.517pt}}
\multiput(533.60,709.00)(0.468,0.774){5}{\rule{0.113pt}{0.700pt}}
\multiput(532.17,709.00)(4.000,4.547){2}{\rule{0.400pt}{0.350pt}}
\multiput(537.61,715.00)(0.447,1.132){3}{\rule{0.108pt}{0.900pt}}
\multiput(536.17,715.00)(3.000,4.132){2}{\rule{0.400pt}{0.450pt}}
\multiput(540.61,721.00)(0.447,1.355){3}{\rule{0.108pt}{1.033pt}}
\multiput(539.17,721.00)(3.000,4.855){2}{\rule{0.400pt}{0.517pt}}
\multiput(543.61,728.00)(0.447,1.355){3}{\rule{0.108pt}{1.033pt}}
\multiput(542.17,728.00)(3.000,4.855){2}{\rule{0.400pt}{0.517pt}}
\multiput(297.00,504.95)(0.462,-0.447){3}{\rule{0.500pt}{0.108pt}}
\multiput(297.00,505.17)(1.962,-3.000){2}{\rule{0.250pt}{0.400pt}}
\multiput(300.00,501.95)(0.462,-0.447){3}{\rule{0.500pt}{0.108pt}}
\multiput(300.00,502.17)(1.962,-3.000){2}{\rule{0.250pt}{0.400pt}}
\multiput(303.00,498.95)(0.462,-0.447){3}{\rule{0.500pt}{0.108pt}}
\multiput(303.00,499.17)(1.962,-3.000){2}{\rule{0.250pt}{0.400pt}}
\multiput(306.00,495.95)(0.462,-0.447){3}{\rule{0.500pt}{0.108pt}}
\multiput(306.00,496.17)(1.962,-3.000){2}{\rule{0.250pt}{0.400pt}}
\put(309,492.17){\rule{0.700pt}{0.400pt}}
\multiput(309.00,493.17)(1.547,-2.000){2}{\rule{0.350pt}{0.400pt}}
\multiput(312.00,490.95)(0.685,-0.447){3}{\rule{0.633pt}{0.108pt}}
\multiput(312.00,491.17)(2.685,-3.000){2}{\rule{0.317pt}{0.400pt}}
\put(316,487.17){\rule{0.700pt}{0.400pt}}
\multiput(316.00,488.17)(1.547,-2.000){2}{\rule{0.350pt}{0.400pt}}
\multiput(319.00,485.95)(0.462,-0.447){3}{\rule{0.500pt}{0.108pt}}
\multiput(319.00,486.17)(1.962,-3.000){2}{\rule{0.250pt}{0.400pt}}
\put(322,482.17){\rule{0.700pt}{0.400pt}}
\multiput(322.00,483.17)(1.547,-2.000){2}{\rule{0.350pt}{0.400pt}}
\put(325,480.17){\rule{0.700pt}{0.400pt}}
\multiput(325.00,481.17)(1.547,-2.000){2}{\rule{0.350pt}{0.400pt}}
\put(328,478.17){\rule{0.700pt}{0.400pt}}
\multiput(328.00,479.17)(1.547,-2.000){2}{\rule{0.350pt}{0.400pt}}
\put(331,476.17){\rule{0.700pt}{0.400pt}}
\multiput(331.00,477.17)(1.547,-2.000){2}{\rule{0.350pt}{0.400pt}}
\put(334,474.17){\rule{0.700pt}{0.400pt}}
\multiput(334.00,475.17)(1.547,-2.000){2}{\rule{0.350pt}{0.400pt}}
\put(337,472.67){\rule{0.723pt}{0.400pt}}
\multiput(337.00,473.17)(1.500,-1.000){2}{\rule{0.361pt}{0.400pt}}
\put(340,471.17){\rule{0.900pt}{0.400pt}}
\multiput(340.00,472.17)(2.132,-2.000){2}{\rule{0.450pt}{0.400pt}}
\put(344,469.17){\rule{0.700pt}{0.400pt}}
\multiput(344.00,470.17)(1.547,-2.000){2}{\rule{0.350pt}{0.400pt}}
\put(347,467.67){\rule{0.723pt}{0.400pt}}
\multiput(347.00,468.17)(1.500,-1.000){2}{\rule{0.361pt}{0.400pt}}
\put(350,466.67){\rule{0.723pt}{0.400pt}}
\multiput(350.00,467.17)(1.500,-1.000){2}{\rule{0.361pt}{0.400pt}}
\put(353,465.17){\rule{0.700pt}{0.400pt}}
\multiput(353.00,466.17)(1.547,-2.000){2}{\rule{0.350pt}{0.400pt}}
\put(356,463.67){\rule{0.723pt}{0.400pt}}
\multiput(356.00,464.17)(1.500,-1.000){2}{\rule{0.361pt}{0.400pt}}
\put(359,462.67){\rule{0.723pt}{0.400pt}}
\multiput(359.00,463.17)(1.500,-1.000){2}{\rule{0.361pt}{0.400pt}}
\put(362,461.67){\rule{0.723pt}{0.400pt}}
\multiput(362.00,462.17)(1.500,-1.000){2}{\rule{0.361pt}{0.400pt}}
\put(353.0,508.0){\rule[-0.200pt]{0.723pt}{0.400pt}}
\put(368,460.67){\rule{0.964pt}{0.400pt}}
\multiput(368.00,461.17)(2.000,-1.000){2}{\rule{0.482pt}{0.400pt}}
\put(372,459.67){\rule{0.723pt}{0.400pt}}
\multiput(372.00,460.17)(1.500,-1.000){2}{\rule{0.361pt}{0.400pt}}
\put(365.0,462.0){\rule[-0.200pt]{0.723pt}{0.400pt}}
\put(378,458.67){\rule{0.723pt}{0.400pt}}
\multiput(378.00,459.17)(1.500,-1.000){2}{\rule{0.361pt}{0.400pt}}
\put(375.0,460.0){\rule[-0.200pt]{0.723pt}{0.400pt}}
\put(381.0,459.0){\rule[-0.200pt]{0.723pt}{0.400pt}}
\put(387,457.67){\rule{0.723pt}{0.400pt}}
\multiput(387.00,458.17)(1.500,-1.000){2}{\rule{0.361pt}{0.400pt}}
\put(384.0,459.0){\rule[-0.200pt]{0.723pt}{0.400pt}}
\put(390.0,458.0){\rule[-0.200pt]{0.723pt}{0.400pt}}
\put(397,457.67){\rule{0.723pt}{0.400pt}}
\multiput(397.00,457.17)(1.500,1.000){2}{\rule{0.361pt}{0.400pt}}
\put(393.0,458.0){\rule[-0.200pt]{0.964pt}{0.400pt}}
\put(400.0,459.0){\rule[-0.200pt]{0.723pt}{0.400pt}}
\put(406,458.67){\rule{0.723pt}{0.400pt}}
\multiput(406.00,458.17)(1.500,1.000){2}{\rule{0.361pt}{0.400pt}}
\put(403.0,459.0){\rule[-0.200pt]{0.723pt}{0.400pt}}
\put(412,459.67){\rule{0.723pt}{0.400pt}}
\multiput(412.00,459.17)(1.500,1.000){2}{\rule{0.361pt}{0.400pt}}
\put(415,460.67){\rule{0.723pt}{0.400pt}}
\multiput(415.00,460.17)(1.500,1.000){2}{\rule{0.361pt}{0.400pt}}
\put(409.0,460.0){\rule[-0.200pt]{0.723pt}{0.400pt}}
\put(421,461.67){\rule{0.964pt}{0.400pt}}
\multiput(421.00,461.17)(2.000,1.000){2}{\rule{0.482pt}{0.400pt}}
\put(425,462.67){\rule{0.723pt}{0.400pt}}
\multiput(425.00,462.17)(1.500,1.000){2}{\rule{0.361pt}{0.400pt}}
\put(428,464.17){\rule{0.700pt}{0.400pt}}
\multiput(428.00,463.17)(1.547,2.000){2}{\rule{0.350pt}{0.400pt}}
\put(431,465.67){\rule{0.723pt}{0.400pt}}
\multiput(431.00,465.17)(1.500,1.000){2}{\rule{0.361pt}{0.400pt}}
\put(434,466.67){\rule{0.723pt}{0.400pt}}
\multiput(434.00,466.17)(1.500,1.000){2}{\rule{0.361pt}{0.400pt}}
\put(437,468.17){\rule{0.700pt}{0.400pt}}
\multiput(437.00,467.17)(1.547,2.000){2}{\rule{0.350pt}{0.400pt}}
\put(440,469.67){\rule{0.723pt}{0.400pt}}
\multiput(440.00,469.17)(1.500,1.000){2}{\rule{0.361pt}{0.400pt}}
\put(443,471.17){\rule{0.700pt}{0.400pt}}
\multiput(443.00,470.17)(1.547,2.000){2}{\rule{0.350pt}{0.400pt}}
\put(446,473.17){\rule{0.900pt}{0.400pt}}
\multiput(446.00,472.17)(2.132,2.000){2}{\rule{0.450pt}{0.400pt}}
\put(450,474.67){\rule{0.723pt}{0.400pt}}
\multiput(450.00,474.17)(1.500,1.000){2}{\rule{0.361pt}{0.400pt}}
\put(453,476.17){\rule{0.700pt}{0.400pt}}
\multiput(453.00,475.17)(1.547,2.000){2}{\rule{0.350pt}{0.400pt}}
\put(456,478.17){\rule{0.700pt}{0.400pt}}
\multiput(456.00,477.17)(1.547,2.000){2}{\rule{0.350pt}{0.400pt}}
\put(459,480.17){\rule{0.700pt}{0.400pt}}
\multiput(459.00,479.17)(1.547,2.000){2}{\rule{0.350pt}{0.400pt}}
\multiput(462.00,482.61)(0.462,0.447){3}{\rule{0.500pt}{0.108pt}}
\multiput(462.00,481.17)(1.962,3.000){2}{\rule{0.250pt}{0.400pt}}
\put(465,485.17){\rule{0.700pt}{0.400pt}}
\multiput(465.00,484.17)(1.547,2.000){2}{\rule{0.350pt}{0.400pt}}
\put(468,487.17){\rule{0.700pt}{0.400pt}}
\multiput(468.00,486.17)(1.547,2.000){2}{\rule{0.350pt}{0.400pt}}
\multiput(471.00,489.61)(0.462,0.447){3}{\rule{0.500pt}{0.108pt}}
\multiput(471.00,488.17)(1.962,3.000){2}{\rule{0.250pt}{0.400pt}}
\multiput(474.00,492.61)(0.685,0.447){3}{\rule{0.633pt}{0.108pt}}
\multiput(474.00,491.17)(2.685,3.000){2}{\rule{0.317pt}{0.400pt}}
\put(478,495.17){\rule{0.700pt}{0.400pt}}
\multiput(478.00,494.17)(1.547,2.000){2}{\rule{0.350pt}{0.400pt}}
\multiput(481.00,497.61)(0.462,0.447){3}{\rule{0.500pt}{0.108pt}}
\multiput(481.00,496.17)(1.962,3.000){2}{\rule{0.250pt}{0.400pt}}
\multiput(484.00,500.61)(0.462,0.447){3}{\rule{0.500pt}{0.108pt}}
\multiput(484.00,499.17)(1.962,3.000){2}{\rule{0.250pt}{0.400pt}}
\multiput(487.00,503.61)(0.462,0.447){3}{\rule{0.500pt}{0.108pt}}
\multiput(487.00,502.17)(1.962,3.000){2}{\rule{0.250pt}{0.400pt}}
\multiput(490.00,506.61)(0.462,0.447){3}{\rule{0.500pt}{0.108pt}}
\multiput(490.00,505.17)(1.962,3.000){2}{\rule{0.250pt}{0.400pt}}
\multiput(493.00,509.61)(0.462,0.447){3}{\rule{0.500pt}{0.108pt}}
\multiput(493.00,508.17)(1.962,3.000){2}{\rule{0.250pt}{0.400pt}}
\multiput(496.61,512.00)(0.447,0.685){3}{\rule{0.108pt}{0.633pt}}
\multiput(495.17,512.00)(3.000,2.685){2}{\rule{0.400pt}{0.317pt}}
\multiput(499.00,516.61)(0.462,0.447){3}{\rule{0.500pt}{0.108pt}}
\multiput(499.00,515.17)(1.962,3.000){2}{\rule{0.250pt}{0.400pt}}
\multiput(502.00,519.60)(0.481,0.468){5}{\rule{0.500pt}{0.113pt}}
\multiput(502.00,518.17)(2.962,4.000){2}{\rule{0.250pt}{0.400pt}}
\multiput(506.00,523.61)(0.462,0.447){3}{\rule{0.500pt}{0.108pt}}
\multiput(506.00,522.17)(1.962,3.000){2}{\rule{0.250pt}{0.400pt}}
\multiput(509.61,526.00)(0.447,0.685){3}{\rule{0.108pt}{0.633pt}}
\multiput(508.17,526.00)(3.000,2.685){2}{\rule{0.400pt}{0.317pt}}
\multiput(512.61,530.00)(0.447,0.685){3}{\rule{0.108pt}{0.633pt}}
\multiput(511.17,530.00)(3.000,2.685){2}{\rule{0.400pt}{0.317pt}}
\multiput(515.61,534.00)(0.447,0.685){3}{\rule{0.108pt}{0.633pt}}
\multiput(514.17,534.00)(3.000,2.685){2}{\rule{0.400pt}{0.317pt}}
\multiput(518.61,538.00)(0.447,0.685){3}{\rule{0.108pt}{0.633pt}}
\multiput(517.17,538.00)(3.000,2.685){2}{\rule{0.400pt}{0.317pt}}
\multiput(521.61,542.00)(0.447,0.685){3}{\rule{0.108pt}{0.633pt}}
\multiput(520.17,542.00)(3.000,2.685){2}{\rule{0.400pt}{0.317pt}}
\multiput(524.61,546.00)(0.447,0.685){3}{\rule{0.108pt}{0.633pt}}
\multiput(523.17,546.00)(3.000,2.685){2}{\rule{0.400pt}{0.317pt}}
\multiput(527.00,550.60)(0.481,0.468){5}{\rule{0.500pt}{0.113pt}}
\multiput(527.00,549.17)(2.962,4.000){2}{\rule{0.250pt}{0.400pt}}
\multiput(531.61,554.00)(0.447,0.909){3}{\rule{0.108pt}{0.767pt}}
\multiput(530.17,554.00)(3.000,3.409){2}{\rule{0.400pt}{0.383pt}}
\multiput(534.61,559.00)(0.447,0.685){3}{\rule{0.108pt}{0.633pt}}
\multiput(533.17,559.00)(3.000,2.685){2}{\rule{0.400pt}{0.317pt}}
\multiput(537.61,563.00)(0.447,0.909){3}{\rule{0.108pt}{0.767pt}}
\multiput(536.17,563.00)(3.000,3.409){2}{\rule{0.400pt}{0.383pt}}
\multiput(540.61,568.00)(0.447,0.685){3}{\rule{0.108pt}{0.633pt}}
\multiput(539.17,568.00)(3.000,2.685){2}{\rule{0.400pt}{0.317pt}}
\multiput(543.61,572.00)(0.447,0.909){3}{\rule{0.108pt}{0.767pt}}
\multiput(542.17,572.00)(3.000,3.409){2}{\rule{0.400pt}{0.383pt}}
\multiput(546.61,577.00)(0.447,0.909){3}{\rule{0.108pt}{0.767pt}}
\multiput(545.17,577.00)(3.000,3.409){2}{\rule{0.400pt}{0.383pt}}
\multiput(549.61,582.00)(0.447,0.909){3}{\rule{0.108pt}{0.767pt}}
\multiput(548.17,582.00)(3.000,3.409){2}{\rule{0.400pt}{0.383pt}}
\multiput(552.61,587.00)(0.447,0.909){3}{\rule{0.108pt}{0.767pt}}
\multiput(551.17,587.00)(3.000,3.409){2}{\rule{0.400pt}{0.383pt}}
\multiput(555.60,592.00)(0.468,0.627){5}{\rule{0.113pt}{0.600pt}}
\multiput(554.17,592.00)(4.000,3.755){2}{\rule{0.400pt}{0.300pt}}
\multiput(559.61,597.00)(0.447,1.132){3}{\rule{0.108pt}{0.900pt}}
\multiput(558.17,597.00)(3.000,4.132){2}{\rule{0.400pt}{0.450pt}}
\multiput(562.61,603.00)(0.447,0.909){3}{\rule{0.108pt}{0.767pt}}
\multiput(561.17,603.00)(3.000,3.409){2}{\rule{0.400pt}{0.383pt}}
\multiput(565.61,608.00)(0.447,1.132){3}{\rule{0.108pt}{0.900pt}}
\multiput(564.17,608.00)(3.000,4.132){2}{\rule{0.400pt}{0.450pt}}
\multiput(568.61,614.00)(0.447,0.909){3}{\rule{0.108pt}{0.767pt}}
\multiput(567.17,614.00)(3.000,3.409){2}{\rule{0.400pt}{0.383pt}}
\multiput(571.61,619.00)(0.447,1.132){3}{\rule{0.108pt}{0.900pt}}
\multiput(570.17,619.00)(3.000,4.132){2}{\rule{0.400pt}{0.450pt}}
\multiput(574.61,625.00)(0.447,1.132){3}{\rule{0.108pt}{0.900pt}}
\multiput(573.17,625.00)(3.000,4.132){2}{\rule{0.400pt}{0.450pt}}
\multiput(577.61,631.00)(0.447,1.132){3}{\rule{0.108pt}{0.900pt}}
\multiput(576.17,631.00)(3.000,4.132){2}{\rule{0.400pt}{0.450pt}}
\multiput(580.60,637.00)(0.468,0.774){5}{\rule{0.113pt}{0.700pt}}
\multiput(579.17,637.00)(4.000,4.547){2}{\rule{0.400pt}{0.350pt}}
\multiput(584.61,643.00)(0.447,1.132){3}{\rule{0.108pt}{0.900pt}}
\multiput(583.17,643.00)(3.000,4.132){2}{\rule{0.400pt}{0.450pt}}
\multiput(587.61,649.00)(0.447,1.132){3}{\rule{0.108pt}{0.900pt}}
\multiput(586.17,649.00)(3.000,4.132){2}{\rule{0.400pt}{0.450pt}}
\multiput(590.61,655.00)(0.447,1.132){3}{\rule{0.108pt}{0.900pt}}
\multiput(589.17,655.00)(3.000,4.132){2}{\rule{0.400pt}{0.450pt}}
\multiput(593.61,661.00)(0.447,1.132){3}{\rule{0.108pt}{0.900pt}}
\multiput(592.17,661.00)(3.000,4.132){2}{\rule{0.400pt}{0.450pt}}
\multiput(596.61,667.00)(0.447,1.355){3}{\rule{0.108pt}{1.033pt}}
\multiput(595.17,667.00)(3.000,4.855){2}{\rule{0.400pt}{0.517pt}}
\multiput(599.61,674.00)(0.447,1.132){3}{\rule{0.108pt}{0.900pt}}
\multiput(598.17,674.00)(3.000,4.132){2}{\rule{0.400pt}{0.450pt}}
\multiput(602.61,680.00)(0.447,1.355){3}{\rule{0.108pt}{1.033pt}}
\multiput(601.17,680.00)(3.000,4.855){2}{\rule{0.400pt}{0.517pt}}
\multiput(356.00,468.95)(0.462,-0.447){3}{\rule{0.500pt}{0.108pt}}
\multiput(356.00,469.17)(1.962,-3.000){2}{\rule{0.250pt}{0.400pt}}
\multiput(359.00,465.95)(0.462,-0.447){3}{\rule{0.500pt}{0.108pt}}
\multiput(359.00,466.17)(1.962,-3.000){2}{\rule{0.250pt}{0.400pt}}
\multiput(362.00,462.95)(0.685,-0.447){3}{\rule{0.633pt}{0.108pt}}
\multiput(362.00,463.17)(2.685,-3.000){2}{\rule{0.317pt}{0.400pt}}
\put(366,459.17){\rule{0.700pt}{0.400pt}}
\multiput(366.00,460.17)(1.547,-2.000){2}{\rule{0.350pt}{0.400pt}}
\multiput(369.00,457.95)(0.462,-0.447){3}{\rule{0.500pt}{0.108pt}}
\multiput(369.00,458.17)(1.962,-3.000){2}{\rule{0.250pt}{0.400pt}}
\put(372,454.17){\rule{0.700pt}{0.400pt}}
\multiput(372.00,455.17)(1.547,-2.000){2}{\rule{0.350pt}{0.400pt}}
\multiput(375.00,452.95)(0.462,-0.447){3}{\rule{0.500pt}{0.108pt}}
\multiput(375.00,453.17)(1.962,-3.000){2}{\rule{0.250pt}{0.400pt}}
\put(378,449.17){\rule{0.700pt}{0.400pt}}
\multiput(378.00,450.17)(1.547,-2.000){2}{\rule{0.350pt}{0.400pt}}
\put(381,447.17){\rule{0.700pt}{0.400pt}}
\multiput(381.00,448.17)(1.547,-2.000){2}{\rule{0.350pt}{0.400pt}}
\put(384,445.17){\rule{0.700pt}{0.400pt}}
\multiput(384.00,446.17)(1.547,-2.000){2}{\rule{0.350pt}{0.400pt}}
\put(387,443.17){\rule{0.700pt}{0.400pt}}
\multiput(387.00,444.17)(1.547,-2.000){2}{\rule{0.350pt}{0.400pt}}
\put(390,441.17){\rule{0.900pt}{0.400pt}}
\multiput(390.00,442.17)(2.132,-2.000){2}{\rule{0.450pt}{0.400pt}}
\put(394,439.17){\rule{0.700pt}{0.400pt}}
\multiput(394.00,440.17)(1.547,-2.000){2}{\rule{0.350pt}{0.400pt}}
\put(397,437.17){\rule{0.700pt}{0.400pt}}
\multiput(397.00,438.17)(1.547,-2.000){2}{\rule{0.350pt}{0.400pt}}
\put(400,435.17){\rule{0.700pt}{0.400pt}}
\multiput(400.00,436.17)(1.547,-2.000){2}{\rule{0.350pt}{0.400pt}}
\put(403,433.67){\rule{0.723pt}{0.400pt}}
\multiput(403.00,434.17)(1.500,-1.000){2}{\rule{0.361pt}{0.400pt}}
\put(406,432.67){\rule{0.723pt}{0.400pt}}
\multiput(406.00,433.17)(1.500,-1.000){2}{\rule{0.361pt}{0.400pt}}
\put(409,431.17){\rule{0.700pt}{0.400pt}}
\multiput(409.00,432.17)(1.547,-2.000){2}{\rule{0.350pt}{0.400pt}}
\put(412,429.67){\rule{0.723pt}{0.400pt}}
\multiput(412.00,430.17)(1.500,-1.000){2}{\rule{0.361pt}{0.400pt}}
\put(415,428.67){\rule{0.964pt}{0.400pt}}
\multiput(415.00,429.17)(2.000,-1.000){2}{\rule{0.482pt}{0.400pt}}
\put(419,427.67){\rule{0.723pt}{0.400pt}}
\multiput(419.00,428.17)(1.500,-1.000){2}{\rule{0.361pt}{0.400pt}}
\put(422,426.67){\rule{0.723pt}{0.400pt}}
\multiput(422.00,427.17)(1.500,-1.000){2}{\rule{0.361pt}{0.400pt}}
\put(425,425.67){\rule{0.723pt}{0.400pt}}
\multiput(425.00,426.17)(1.500,-1.000){2}{\rule{0.361pt}{0.400pt}}
\put(428,424.67){\rule{0.723pt}{0.400pt}}
\multiput(428.00,425.17)(1.500,-1.000){2}{\rule{0.361pt}{0.400pt}}
\put(418.0,462.0){\rule[-0.200pt]{0.723pt}{0.400pt}}
\put(434,423.67){\rule{0.723pt}{0.400pt}}
\multiput(434.00,424.17)(1.500,-1.000){2}{\rule{0.361pt}{0.400pt}}
\put(431.0,425.0){\rule[-0.200pt]{0.723pt}{0.400pt}}
\put(440,422.67){\rule{0.723pt}{0.400pt}}
\multiput(440.00,423.17)(1.500,-1.000){2}{\rule{0.361pt}{0.400pt}}
\put(437.0,424.0){\rule[-0.200pt]{0.723pt}{0.400pt}}
\put(443.0,423.0){\rule[-0.200pt]{0.964pt}{0.400pt}}
\put(447.0,423.0){\rule[-0.200pt]{0.723pt}{0.400pt}}
\put(450.0,423.0){\rule[-0.200pt]{0.723pt}{0.400pt}}
\put(453.0,423.0){\rule[-0.200pt]{0.723pt}{0.400pt}}
\put(456.0,423.0){\rule[-0.200pt]{0.723pt}{0.400pt}}
\put(462,422.67){\rule{0.723pt}{0.400pt}}
\multiput(462.00,422.17)(1.500,1.000){2}{\rule{0.361pt}{0.400pt}}
\put(459.0,423.0){\rule[-0.200pt]{0.723pt}{0.400pt}}
\put(468,423.67){\rule{0.723pt}{0.400pt}}
\multiput(468.00,423.17)(1.500,1.000){2}{\rule{0.361pt}{0.400pt}}
\put(465.0,424.0){\rule[-0.200pt]{0.723pt}{0.400pt}}
\put(475,424.67){\rule{0.723pt}{0.400pt}}
\multiput(475.00,424.17)(1.500,1.000){2}{\rule{0.361pt}{0.400pt}}
\put(478,425.67){\rule{0.723pt}{0.400pt}}
\multiput(478.00,425.17)(1.500,1.000){2}{\rule{0.361pt}{0.400pt}}
\put(481,426.67){\rule{0.723pt}{0.400pt}}
\multiput(481.00,426.17)(1.500,1.000){2}{\rule{0.361pt}{0.400pt}}
\put(484,427.67){\rule{0.723pt}{0.400pt}}
\multiput(484.00,427.17)(1.500,1.000){2}{\rule{0.361pt}{0.400pt}}
\put(487,428.67){\rule{0.723pt}{0.400pt}}
\multiput(487.00,428.17)(1.500,1.000){2}{\rule{0.361pt}{0.400pt}}
\put(490,429.67){\rule{0.723pt}{0.400pt}}
\multiput(490.00,429.17)(1.500,1.000){2}{\rule{0.361pt}{0.400pt}}
\put(493,431.17){\rule{0.700pt}{0.400pt}}
\multiput(493.00,430.17)(1.547,2.000){2}{\rule{0.350pt}{0.400pt}}
\put(496,432.67){\rule{0.964pt}{0.400pt}}
\multiput(496.00,432.17)(2.000,1.000){2}{\rule{0.482pt}{0.400pt}}
\put(500,434.17){\rule{0.700pt}{0.400pt}}
\multiput(500.00,433.17)(1.547,2.000){2}{\rule{0.350pt}{0.400pt}}
\put(503,435.67){\rule{0.723pt}{0.400pt}}
\multiput(503.00,435.17)(1.500,1.000){2}{\rule{0.361pt}{0.400pt}}
\put(506,437.17){\rule{0.700pt}{0.400pt}}
\multiput(506.00,436.17)(1.547,2.000){2}{\rule{0.350pt}{0.400pt}}
\put(509,439.17){\rule{0.700pt}{0.400pt}}
\multiput(509.00,438.17)(1.547,2.000){2}{\rule{0.350pt}{0.400pt}}
\put(512,441.17){\rule{0.700pt}{0.400pt}}
\multiput(512.00,440.17)(1.547,2.000){2}{\rule{0.350pt}{0.400pt}}
\put(515,443.17){\rule{0.700pt}{0.400pt}}
\multiput(515.00,442.17)(1.547,2.000){2}{\rule{0.350pt}{0.400pt}}
\put(518,445.17){\rule{0.700pt}{0.400pt}}
\multiput(518.00,444.17)(1.547,2.000){2}{\rule{0.350pt}{0.400pt}}
\put(521,447.17){\rule{0.700pt}{0.400pt}}
\multiput(521.00,446.17)(1.547,2.000){2}{\rule{0.350pt}{0.400pt}}
\put(524,449.17){\rule{0.900pt}{0.400pt}}
\multiput(524.00,448.17)(2.132,2.000){2}{\rule{0.450pt}{0.400pt}}
\multiput(528.00,451.61)(0.462,0.447){3}{\rule{0.500pt}{0.108pt}}
\multiput(528.00,450.17)(1.962,3.000){2}{\rule{0.250pt}{0.400pt}}
\put(531,454.17){\rule{0.700pt}{0.400pt}}
\multiput(531.00,453.17)(1.547,2.000){2}{\rule{0.350pt}{0.400pt}}
\multiput(534.00,456.61)(0.462,0.447){3}{\rule{0.500pt}{0.108pt}}
\multiput(534.00,455.17)(1.962,3.000){2}{\rule{0.250pt}{0.400pt}}
\multiput(537.00,459.61)(0.462,0.447){3}{\rule{0.500pt}{0.108pt}}
\multiput(537.00,458.17)(1.962,3.000){2}{\rule{0.250pt}{0.400pt}}
\multiput(540.00,462.61)(0.462,0.447){3}{\rule{0.500pt}{0.108pt}}
\multiput(540.00,461.17)(1.962,3.000){2}{\rule{0.250pt}{0.400pt}}
\multiput(543.00,465.61)(0.462,0.447){3}{\rule{0.500pt}{0.108pt}}
\multiput(543.00,464.17)(1.962,3.000){2}{\rule{0.250pt}{0.400pt}}
\multiput(546.00,468.61)(0.462,0.447){3}{\rule{0.500pt}{0.108pt}}
\multiput(546.00,467.17)(1.962,3.000){2}{\rule{0.250pt}{0.400pt}}
\multiput(549.00,471.61)(0.685,0.447){3}{\rule{0.633pt}{0.108pt}}
\multiput(549.00,470.17)(2.685,3.000){2}{\rule{0.317pt}{0.400pt}}
\multiput(553.00,474.61)(0.462,0.447){3}{\rule{0.500pt}{0.108pt}}
\multiput(553.00,473.17)(1.962,3.000){2}{\rule{0.250pt}{0.400pt}}
\multiput(556.00,477.61)(0.462,0.447){3}{\rule{0.500pt}{0.108pt}}
\multiput(556.00,476.17)(1.962,3.000){2}{\rule{0.250pt}{0.400pt}}
\multiput(559.61,480.00)(0.447,0.685){3}{\rule{0.108pt}{0.633pt}}
\multiput(558.17,480.00)(3.000,2.685){2}{\rule{0.400pt}{0.317pt}}
\multiput(562.00,484.61)(0.462,0.447){3}{\rule{0.500pt}{0.108pt}}
\multiput(562.00,483.17)(1.962,3.000){2}{\rule{0.250pt}{0.400pt}}
\multiput(565.61,487.00)(0.447,0.685){3}{\rule{0.108pt}{0.633pt}}
\multiput(564.17,487.00)(3.000,2.685){2}{\rule{0.400pt}{0.317pt}}
\multiput(568.00,491.61)(0.462,0.447){3}{\rule{0.500pt}{0.108pt}}
\multiput(568.00,490.17)(1.962,3.000){2}{\rule{0.250pt}{0.400pt}}
\multiput(571.61,494.00)(0.447,0.685){3}{\rule{0.108pt}{0.633pt}}
\multiput(570.17,494.00)(3.000,2.685){2}{\rule{0.400pt}{0.317pt}}
\multiput(574.61,498.00)(0.447,0.685){3}{\rule{0.108pt}{0.633pt}}
\multiput(573.17,498.00)(3.000,2.685){2}{\rule{0.400pt}{0.317pt}}
\multiput(577.00,502.60)(0.481,0.468){5}{\rule{0.500pt}{0.113pt}}
\multiput(577.00,501.17)(2.962,4.000){2}{\rule{0.250pt}{0.400pt}}
\multiput(581.61,506.00)(0.447,0.685){3}{\rule{0.108pt}{0.633pt}}
\multiput(580.17,506.00)(3.000,2.685){2}{\rule{0.400pt}{0.317pt}}
\multiput(584.61,510.00)(0.447,0.685){3}{\rule{0.108pt}{0.633pt}}
\multiput(583.17,510.00)(3.000,2.685){2}{\rule{0.400pt}{0.317pt}}
\multiput(587.61,514.00)(0.447,0.909){3}{\rule{0.108pt}{0.767pt}}
\multiput(586.17,514.00)(3.000,3.409){2}{\rule{0.400pt}{0.383pt}}
\multiput(590.61,519.00)(0.447,0.685){3}{\rule{0.108pt}{0.633pt}}
\multiput(589.17,519.00)(3.000,2.685){2}{\rule{0.400pt}{0.317pt}}
\multiput(593.61,523.00)(0.447,0.909){3}{\rule{0.108pt}{0.767pt}}
\multiput(592.17,523.00)(3.000,3.409){2}{\rule{0.400pt}{0.383pt}}
\multiput(596.61,528.00)(0.447,0.685){3}{\rule{0.108pt}{0.633pt}}
\multiput(595.17,528.00)(3.000,2.685){2}{\rule{0.400pt}{0.317pt}}
\multiput(599.61,532.00)(0.447,0.909){3}{\rule{0.108pt}{0.767pt}}
\multiput(598.17,532.00)(3.000,3.409){2}{\rule{0.400pt}{0.383pt}}
\multiput(602.61,537.00)(0.447,0.909){3}{\rule{0.108pt}{0.767pt}}
\multiput(601.17,537.00)(3.000,3.409){2}{\rule{0.400pt}{0.383pt}}
\multiput(605.60,542.00)(0.468,0.627){5}{\rule{0.113pt}{0.600pt}}
\multiput(604.17,542.00)(4.000,3.755){2}{\rule{0.400pt}{0.300pt}}
\multiput(609.61,547.00)(0.447,0.909){3}{\rule{0.108pt}{0.767pt}}
\multiput(608.17,547.00)(3.000,3.409){2}{\rule{0.400pt}{0.383pt}}
\multiput(612.61,552.00)(0.447,0.909){3}{\rule{0.108pt}{0.767pt}}
\multiput(611.17,552.00)(3.000,3.409){2}{\rule{0.400pt}{0.383pt}}
\multiput(615.61,557.00)(0.447,0.909){3}{\rule{0.108pt}{0.767pt}}
\multiput(614.17,557.00)(3.000,3.409){2}{\rule{0.400pt}{0.383pt}}
\multiput(618.61,562.00)(0.447,0.909){3}{\rule{0.108pt}{0.767pt}}
\multiput(617.17,562.00)(3.000,3.409){2}{\rule{0.400pt}{0.383pt}}
\multiput(621.61,567.00)(0.447,1.132){3}{\rule{0.108pt}{0.900pt}}
\multiput(620.17,567.00)(3.000,4.132){2}{\rule{0.400pt}{0.450pt}}
\multiput(624.61,573.00)(0.447,0.909){3}{\rule{0.108pt}{0.767pt}}
\multiput(623.17,573.00)(3.000,3.409){2}{\rule{0.400pt}{0.383pt}}
\multiput(627.61,578.00)(0.447,1.132){3}{\rule{0.108pt}{0.900pt}}
\multiput(626.17,578.00)(3.000,4.132){2}{\rule{0.400pt}{0.450pt}}
\multiput(630.60,584.00)(0.468,0.627){5}{\rule{0.113pt}{0.600pt}}
\multiput(629.17,584.00)(4.000,3.755){2}{\rule{0.400pt}{0.300pt}}
\multiput(634.61,589.00)(0.447,1.132){3}{\rule{0.108pt}{0.900pt}}
\multiput(633.17,589.00)(3.000,4.132){2}{\rule{0.400pt}{0.450pt}}
\multiput(637.61,595.00)(0.447,1.132){3}{\rule{0.108pt}{0.900pt}}
\multiput(636.17,595.00)(3.000,4.132){2}{\rule{0.400pt}{0.450pt}}
\multiput(640.61,601.00)(0.447,1.132){3}{\rule{0.108pt}{0.900pt}}
\multiput(639.17,601.00)(3.000,4.132){2}{\rule{0.400pt}{0.450pt}}
\multiput(643.61,607.00)(0.447,1.132){3}{\rule{0.108pt}{0.900pt}}
\multiput(642.17,607.00)(3.000,4.132){2}{\rule{0.400pt}{0.450pt}}
\multiput(646.61,613.00)(0.447,1.132){3}{\rule{0.108pt}{0.900pt}}
\multiput(645.17,613.00)(3.000,4.132){2}{\rule{0.400pt}{0.450pt}}
\multiput(649.61,619.00)(0.447,1.355){3}{\rule{0.108pt}{1.033pt}}
\multiput(648.17,619.00)(3.000,4.855){2}{\rule{0.400pt}{0.517pt}}
\multiput(652.61,626.00)(0.447,1.132){3}{\rule{0.108pt}{0.900pt}}
\multiput(651.17,626.00)(3.000,4.132){2}{\rule{0.400pt}{0.450pt}}
\multiput(655.61,632.00)(0.447,1.132){3}{\rule{0.108pt}{0.900pt}}
\multiput(654.17,632.00)(3.000,4.132){2}{\rule{0.400pt}{0.450pt}}
\multiput(658.60,638.00)(0.468,0.920){5}{\rule{0.113pt}{0.800pt}}
\multiput(657.17,638.00)(4.000,5.340){2}{\rule{0.400pt}{0.400pt}}
\multiput(662.61,645.00)(0.447,1.355){3}{\rule{0.108pt}{1.033pt}}
\multiput(661.17,645.00)(3.000,4.855){2}{\rule{0.400pt}{0.517pt}}
\multiput(416.00,445.95)(0.462,-0.447){3}{\rule{0.500pt}{0.108pt}}
\multiput(416.00,446.17)(1.962,-3.000){2}{\rule{0.250pt}{0.400pt}}
\multiput(419.00,442.95)(0.462,-0.447){3}{\rule{0.500pt}{0.108pt}}
\multiput(419.00,443.17)(1.962,-3.000){2}{\rule{0.250pt}{0.400pt}}
\multiput(422.00,439.95)(0.462,-0.447){3}{\rule{0.500pt}{0.108pt}}
\multiput(422.00,440.17)(1.962,-3.000){2}{\rule{0.250pt}{0.400pt}}
\multiput(425.00,436.95)(0.462,-0.447){3}{\rule{0.500pt}{0.108pt}}
\multiput(425.00,437.17)(1.962,-3.000){2}{\rule{0.250pt}{0.400pt}}
\put(428,433.17){\rule{0.700pt}{0.400pt}}
\multiput(428.00,434.17)(1.547,-2.000){2}{\rule{0.350pt}{0.400pt}}
\multiput(431.00,431.95)(0.462,-0.447){3}{\rule{0.500pt}{0.108pt}}
\multiput(431.00,432.17)(1.962,-3.000){2}{\rule{0.250pt}{0.400pt}}
\put(434,428.17){\rule{0.700pt}{0.400pt}}
\multiput(434.00,429.17)(1.547,-2.000){2}{\rule{0.350pt}{0.400pt}}
\multiput(437.00,426.95)(0.462,-0.447){3}{\rule{0.500pt}{0.108pt}}
\multiput(437.00,427.17)(1.962,-3.000){2}{\rule{0.250pt}{0.400pt}}
\put(440,423.17){\rule{0.900pt}{0.400pt}}
\multiput(440.00,424.17)(2.132,-2.000){2}{\rule{0.450pt}{0.400pt}}
\put(444,421.17){\rule{0.700pt}{0.400pt}}
\multiput(444.00,422.17)(1.547,-2.000){2}{\rule{0.350pt}{0.400pt}}
\put(447,419.17){\rule{0.700pt}{0.400pt}}
\multiput(447.00,420.17)(1.547,-2.000){2}{\rule{0.350pt}{0.400pt}}
\put(450,417.17){\rule{0.700pt}{0.400pt}}
\multiput(450.00,418.17)(1.547,-2.000){2}{\rule{0.350pt}{0.400pt}}
\put(453,415.17){\rule{0.700pt}{0.400pt}}
\multiput(453.00,416.17)(1.547,-2.000){2}{\rule{0.350pt}{0.400pt}}
\put(456,413.17){\rule{0.700pt}{0.400pt}}
\multiput(456.00,414.17)(1.547,-2.000){2}{\rule{0.350pt}{0.400pt}}
\put(459,411.67){\rule{0.723pt}{0.400pt}}
\multiput(459.00,412.17)(1.500,-1.000){2}{\rule{0.361pt}{0.400pt}}
\put(462,410.17){\rule{0.700pt}{0.400pt}}
\multiput(462.00,411.17)(1.547,-2.000){2}{\rule{0.350pt}{0.400pt}}
\put(465,408.67){\rule{0.964pt}{0.400pt}}
\multiput(465.00,409.17)(2.000,-1.000){2}{\rule{0.482pt}{0.400pt}}
\put(469,407.67){\rule{0.723pt}{0.400pt}}
\multiput(469.00,408.17)(1.500,-1.000){2}{\rule{0.361pt}{0.400pt}}
\put(472,406.17){\rule{0.700pt}{0.400pt}}
\multiput(472.00,407.17)(1.547,-2.000){2}{\rule{0.350pt}{0.400pt}}
\put(475,404.67){\rule{0.723pt}{0.400pt}}
\multiput(475.00,405.17)(1.500,-1.000){2}{\rule{0.361pt}{0.400pt}}
\put(478,403.67){\rule{0.723pt}{0.400pt}}
\multiput(478.00,404.17)(1.500,-1.000){2}{\rule{0.361pt}{0.400pt}}
\put(481,402.67){\rule{0.723pt}{0.400pt}}
\multiput(481.00,403.17)(1.500,-1.000){2}{\rule{0.361pt}{0.400pt}}
\put(484,401.67){\rule{0.723pt}{0.400pt}}
\multiput(484.00,402.17)(1.500,-1.000){2}{\rule{0.361pt}{0.400pt}}
\put(471.0,425.0){\rule[-0.200pt]{0.964pt}{0.400pt}}
\put(490,400.67){\rule{0.723pt}{0.400pt}}
\multiput(490.00,401.17)(1.500,-1.000){2}{\rule{0.361pt}{0.400pt}}
\put(487.0,402.0){\rule[-0.200pt]{0.723pt}{0.400pt}}
\put(497,399.67){\rule{0.723pt}{0.400pt}}
\multiput(497.00,400.17)(1.500,-1.000){2}{\rule{0.361pt}{0.400pt}}
\put(493.0,401.0){\rule[-0.200pt]{0.964pt}{0.400pt}}
\put(500.0,400.0){\rule[-0.200pt]{0.723pt}{0.400pt}}
\put(506,398.67){\rule{0.723pt}{0.400pt}}
\multiput(506.00,399.17)(1.500,-1.000){2}{\rule{0.361pt}{0.400pt}}
\put(503.0,400.0){\rule[-0.200pt]{0.723pt}{0.400pt}}
\put(509.0,399.0){\rule[-0.200pt]{0.723pt}{0.400pt}}
\put(515,398.67){\rule{0.723pt}{0.400pt}}
\multiput(515.00,398.17)(1.500,1.000){2}{\rule{0.361pt}{0.400pt}}
\put(512.0,399.0){\rule[-0.200pt]{0.723pt}{0.400pt}}
\put(518.0,400.0){\rule[-0.200pt]{0.964pt}{0.400pt}}
\put(525,399.67){\rule{0.723pt}{0.400pt}}
\multiput(525.00,399.17)(1.500,1.000){2}{\rule{0.361pt}{0.400pt}}
\put(522.0,400.0){\rule[-0.200pt]{0.723pt}{0.400pt}}
\put(531,400.67){\rule{0.723pt}{0.400pt}}
\multiput(531.00,400.17)(1.500,1.000){2}{\rule{0.361pt}{0.400pt}}
\put(534,401.67){\rule{0.723pt}{0.400pt}}
\multiput(534.00,401.17)(1.500,1.000){2}{\rule{0.361pt}{0.400pt}}
\put(528.0,401.0){\rule[-0.200pt]{0.723pt}{0.400pt}}
\put(540,402.67){\rule{0.723pt}{0.400pt}}
\multiput(540.00,402.17)(1.500,1.000){2}{\rule{0.361pt}{0.400pt}}
\put(543,403.67){\rule{0.723pt}{0.400pt}}
\multiput(543.00,403.17)(1.500,1.000){2}{\rule{0.361pt}{0.400pt}}
\put(546,405.17){\rule{0.900pt}{0.400pt}}
\multiput(546.00,404.17)(2.132,2.000){2}{\rule{0.450pt}{0.400pt}}
\put(550,406.67){\rule{0.723pt}{0.400pt}}
\multiput(550.00,406.17)(1.500,1.000){2}{\rule{0.361pt}{0.400pt}}
\put(553,407.67){\rule{0.723pt}{0.400pt}}
\multiput(553.00,407.17)(1.500,1.000){2}{\rule{0.361pt}{0.400pt}}
\put(556,409.17){\rule{0.700pt}{0.400pt}}
\multiput(556.00,408.17)(1.547,2.000){2}{\rule{0.350pt}{0.400pt}}
\put(559,410.67){\rule{0.723pt}{0.400pt}}
\multiput(559.00,410.17)(1.500,1.000){2}{\rule{0.361pt}{0.400pt}}
\put(562,412.17){\rule{0.700pt}{0.400pt}}
\multiput(562.00,411.17)(1.547,2.000){2}{\rule{0.350pt}{0.400pt}}
\put(565,413.67){\rule{0.723pt}{0.400pt}}
\multiput(565.00,413.17)(1.500,1.000){2}{\rule{0.361pt}{0.400pt}}
\put(568,415.17){\rule{0.700pt}{0.400pt}}
\multiput(568.00,414.17)(1.547,2.000){2}{\rule{0.350pt}{0.400pt}}
\put(571,417.17){\rule{0.700pt}{0.400pt}}
\multiput(571.00,416.17)(1.547,2.000){2}{\rule{0.350pt}{0.400pt}}
\put(574,419.17){\rule{0.900pt}{0.400pt}}
\multiput(574.00,418.17)(2.132,2.000){2}{\rule{0.450pt}{0.400pt}}
\put(578,421.17){\rule{0.700pt}{0.400pt}}
\multiput(578.00,420.17)(1.547,2.000){2}{\rule{0.350pt}{0.400pt}}
\multiput(581.00,423.61)(0.462,0.447){3}{\rule{0.500pt}{0.108pt}}
\multiput(581.00,422.17)(1.962,3.000){2}{\rule{0.250pt}{0.400pt}}
\put(584,426.17){\rule{0.700pt}{0.400pt}}
\multiput(584.00,425.17)(1.547,2.000){2}{\rule{0.350pt}{0.400pt}}
\put(587,428.17){\rule{0.700pt}{0.400pt}}
\multiput(587.00,427.17)(1.547,2.000){2}{\rule{0.350pt}{0.400pt}}
\multiput(590.00,430.61)(0.462,0.447){3}{\rule{0.500pt}{0.108pt}}
\multiput(590.00,429.17)(1.962,3.000){2}{\rule{0.250pt}{0.400pt}}
\put(593,433.17){\rule{0.700pt}{0.400pt}}
\multiput(593.00,432.17)(1.547,2.000){2}{\rule{0.350pt}{0.400pt}}
\multiput(596.00,435.61)(0.462,0.447){3}{\rule{0.500pt}{0.108pt}}
\multiput(596.00,434.17)(1.962,3.000){2}{\rule{0.250pt}{0.400pt}}
\multiput(599.00,438.61)(0.685,0.447){3}{\rule{0.633pt}{0.108pt}}
\multiput(599.00,437.17)(2.685,3.000){2}{\rule{0.317pt}{0.400pt}}
\multiput(603.00,441.61)(0.462,0.447){3}{\rule{0.500pt}{0.108pt}}
\multiput(603.00,440.17)(1.962,3.000){2}{\rule{0.250pt}{0.400pt}}
\multiput(606.00,444.61)(0.462,0.447){3}{\rule{0.500pt}{0.108pt}}
\multiput(606.00,443.17)(1.962,3.000){2}{\rule{0.250pt}{0.400pt}}
\multiput(609.00,447.61)(0.462,0.447){3}{\rule{0.500pt}{0.108pt}}
\multiput(609.00,446.17)(1.962,3.000){2}{\rule{0.250pt}{0.400pt}}
\multiput(612.00,450.61)(0.462,0.447){3}{\rule{0.500pt}{0.108pt}}
\multiput(612.00,449.17)(1.962,3.000){2}{\rule{0.250pt}{0.400pt}}
\multiput(615.61,453.00)(0.447,0.685){3}{\rule{0.108pt}{0.633pt}}
\multiput(614.17,453.00)(3.000,2.685){2}{\rule{0.400pt}{0.317pt}}
\multiput(618.00,457.61)(0.462,0.447){3}{\rule{0.500pt}{0.108pt}}
\multiput(618.00,456.17)(1.962,3.000){2}{\rule{0.250pt}{0.400pt}}
\multiput(621.61,460.00)(0.447,0.685){3}{\rule{0.108pt}{0.633pt}}
\multiput(620.17,460.00)(3.000,2.685){2}{\rule{0.400pt}{0.317pt}}
\multiput(624.00,464.61)(0.462,0.447){3}{\rule{0.500pt}{0.108pt}}
\multiput(624.00,463.17)(1.962,3.000){2}{\rule{0.250pt}{0.400pt}}
\multiput(627.00,467.60)(0.481,0.468){5}{\rule{0.500pt}{0.113pt}}
\multiput(627.00,466.17)(2.962,4.000){2}{\rule{0.250pt}{0.400pt}}
\multiput(631.61,471.00)(0.447,0.685){3}{\rule{0.108pt}{0.633pt}}
\multiput(630.17,471.00)(3.000,2.685){2}{\rule{0.400pt}{0.317pt}}
\multiput(634.61,475.00)(0.447,0.685){3}{\rule{0.108pt}{0.633pt}}
\multiput(633.17,475.00)(3.000,2.685){2}{\rule{0.400pt}{0.317pt}}
\multiput(637.61,479.00)(0.447,0.685){3}{\rule{0.108pt}{0.633pt}}
\multiput(636.17,479.00)(3.000,2.685){2}{\rule{0.400pt}{0.317pt}}
\multiput(640.61,483.00)(0.447,0.685){3}{\rule{0.108pt}{0.633pt}}
\multiput(639.17,483.00)(3.000,2.685){2}{\rule{0.400pt}{0.317pt}}
\multiput(643.61,487.00)(0.447,0.685){3}{\rule{0.108pt}{0.633pt}}
\multiput(642.17,487.00)(3.000,2.685){2}{\rule{0.400pt}{0.317pt}}
\multiput(646.61,491.00)(0.447,0.685){3}{\rule{0.108pt}{0.633pt}}
\multiput(645.17,491.00)(3.000,2.685){2}{\rule{0.400pt}{0.317pt}}
\multiput(649.61,495.00)(0.447,0.909){3}{\rule{0.108pt}{0.767pt}}
\multiput(648.17,495.00)(3.000,3.409){2}{\rule{0.400pt}{0.383pt}}
\multiput(652.00,500.60)(0.481,0.468){5}{\rule{0.500pt}{0.113pt}}
\multiput(652.00,499.17)(2.962,4.000){2}{\rule{0.250pt}{0.400pt}}
\multiput(656.61,504.00)(0.447,0.909){3}{\rule{0.108pt}{0.767pt}}
\multiput(655.17,504.00)(3.000,3.409){2}{\rule{0.400pt}{0.383pt}}
\multiput(659.61,509.00)(0.447,0.685){3}{\rule{0.108pt}{0.633pt}}
\multiput(658.17,509.00)(3.000,2.685){2}{\rule{0.400pt}{0.317pt}}
\multiput(662.61,513.00)(0.447,0.909){3}{\rule{0.108pt}{0.767pt}}
\multiput(661.17,513.00)(3.000,3.409){2}{\rule{0.400pt}{0.383pt}}
\multiput(665.61,518.00)(0.447,0.909){3}{\rule{0.108pt}{0.767pt}}
\multiput(664.17,518.00)(3.000,3.409){2}{\rule{0.400pt}{0.383pt}}
\multiput(668.61,523.00)(0.447,0.909){3}{\rule{0.108pt}{0.767pt}}
\multiput(667.17,523.00)(3.000,3.409){2}{\rule{0.400pt}{0.383pt}}
\multiput(671.61,528.00)(0.447,0.909){3}{\rule{0.108pt}{0.767pt}}
\multiput(670.17,528.00)(3.000,3.409){2}{\rule{0.400pt}{0.383pt}}
\multiput(674.61,533.00)(0.447,0.909){3}{\rule{0.108pt}{0.767pt}}
\multiput(673.17,533.00)(3.000,3.409){2}{\rule{0.400pt}{0.383pt}}
\multiput(677.61,538.00)(0.447,1.132){3}{\rule{0.108pt}{0.900pt}}
\multiput(676.17,538.00)(3.000,4.132){2}{\rule{0.400pt}{0.450pt}}
\multiput(680.60,544.00)(0.468,0.627){5}{\rule{0.113pt}{0.600pt}}
\multiput(679.17,544.00)(4.000,3.755){2}{\rule{0.400pt}{0.300pt}}
\multiput(684.61,549.00)(0.447,0.909){3}{\rule{0.108pt}{0.767pt}}
\multiput(683.17,549.00)(3.000,3.409){2}{\rule{0.400pt}{0.383pt}}
\multiput(687.61,554.00)(0.447,1.132){3}{\rule{0.108pt}{0.900pt}}
\multiput(686.17,554.00)(3.000,4.132){2}{\rule{0.400pt}{0.450pt}}
\multiput(690.61,560.00)(0.447,1.132){3}{\rule{0.108pt}{0.900pt}}
\multiput(689.17,560.00)(3.000,4.132){2}{\rule{0.400pt}{0.450pt}}
\multiput(693.61,566.00)(0.447,1.132){3}{\rule{0.108pt}{0.900pt}}
\multiput(692.17,566.00)(3.000,4.132){2}{\rule{0.400pt}{0.450pt}}
\multiput(696.61,572.00)(0.447,0.909){3}{\rule{0.108pt}{0.767pt}}
\multiput(695.17,572.00)(3.000,3.409){2}{\rule{0.400pt}{0.383pt}}
\multiput(699.61,577.00)(0.447,1.132){3}{\rule{0.108pt}{0.900pt}}
\multiput(698.17,577.00)(3.000,4.132){2}{\rule{0.400pt}{0.450pt}}
\multiput(702.61,583.00)(0.447,1.132){3}{\rule{0.108pt}{0.900pt}}
\multiput(701.17,583.00)(3.000,4.132){2}{\rule{0.400pt}{0.450pt}}
\multiput(705.61,589.00)(0.447,1.355){3}{\rule{0.108pt}{1.033pt}}
\multiput(704.17,589.00)(3.000,4.855){2}{\rule{0.400pt}{0.517pt}}
\multiput(708.60,596.00)(0.468,0.774){5}{\rule{0.113pt}{0.700pt}}
\multiput(707.17,596.00)(4.000,4.547){2}{\rule{0.400pt}{0.350pt}}
\multiput(712.61,602.00)(0.447,1.132){3}{\rule{0.108pt}{0.900pt}}
\multiput(711.17,602.00)(3.000,4.132){2}{\rule{0.400pt}{0.450pt}}
\multiput(715.61,608.00)(0.447,1.355){3}{\rule{0.108pt}{1.033pt}}
\multiput(714.17,608.00)(3.000,4.855){2}{\rule{0.400pt}{0.517pt}}
\multiput(718.61,615.00)(0.447,1.132){3}{\rule{0.108pt}{0.900pt}}
\multiput(717.17,615.00)(3.000,4.132){2}{\rule{0.400pt}{0.450pt}}
\multiput(721.61,621.00)(0.447,1.355){3}{\rule{0.108pt}{1.033pt}}
\multiput(720.17,621.00)(3.000,4.855){2}{\rule{0.400pt}{0.517pt}}
\multiput(475.00,433.95)(0.462,-0.447){3}{\rule{0.500pt}{0.108pt}}
\multiput(475.00,434.17)(1.962,-3.000){2}{\rule{0.250pt}{0.400pt}}
\multiput(478.00,430.95)(0.462,-0.447){3}{\rule{0.500pt}{0.108pt}}
\multiput(478.00,431.17)(1.962,-3.000){2}{\rule{0.250pt}{0.400pt}}
\multiput(481.00,427.95)(0.462,-0.447){3}{\rule{0.500pt}{0.108pt}}
\multiput(481.00,428.17)(1.962,-3.000){2}{\rule{0.250pt}{0.400pt}}
\put(484,424.17){\rule{0.700pt}{0.400pt}}
\multiput(484.00,425.17)(1.547,-2.000){2}{\rule{0.350pt}{0.400pt}}
\multiput(487.00,422.95)(0.685,-0.447){3}{\rule{0.633pt}{0.108pt}}
\multiput(487.00,423.17)(2.685,-3.000){2}{\rule{0.317pt}{0.400pt}}
\multiput(491.00,419.95)(0.462,-0.447){3}{\rule{0.500pt}{0.108pt}}
\multiput(491.00,420.17)(1.962,-3.000){2}{\rule{0.250pt}{0.400pt}}
\put(494,416.17){\rule{0.700pt}{0.400pt}}
\multiput(494.00,417.17)(1.547,-2.000){2}{\rule{0.350pt}{0.400pt}}
\put(497,414.17){\rule{0.700pt}{0.400pt}}
\multiput(497.00,415.17)(1.547,-2.000){2}{\rule{0.350pt}{0.400pt}}
\multiput(500.00,412.95)(0.462,-0.447){3}{\rule{0.500pt}{0.108pt}}
\multiput(500.00,413.17)(1.962,-3.000){2}{\rule{0.250pt}{0.400pt}}
\put(503,409.17){\rule{0.700pt}{0.400pt}}
\multiput(503.00,410.17)(1.547,-2.000){2}{\rule{0.350pt}{0.400pt}}
\put(506,407.17){\rule{0.700pt}{0.400pt}}
\multiput(506.00,408.17)(1.547,-2.000){2}{\rule{0.350pt}{0.400pt}}
\put(509,405.17){\rule{0.700pt}{0.400pt}}
\multiput(509.00,406.17)(1.547,-2.000){2}{\rule{0.350pt}{0.400pt}}
\put(512,403.67){\rule{0.723pt}{0.400pt}}
\multiput(512.00,404.17)(1.500,-1.000){2}{\rule{0.361pt}{0.400pt}}
\put(515,402.17){\rule{0.900pt}{0.400pt}}
\multiput(515.00,403.17)(2.132,-2.000){2}{\rule{0.450pt}{0.400pt}}
\put(519,400.17){\rule{0.700pt}{0.400pt}}
\multiput(519.00,401.17)(1.547,-2.000){2}{\rule{0.350pt}{0.400pt}}
\put(522,398.67){\rule{0.723pt}{0.400pt}}
\multiput(522.00,399.17)(1.500,-1.000){2}{\rule{0.361pt}{0.400pt}}
\put(525,397.17){\rule{0.700pt}{0.400pt}}
\multiput(525.00,398.17)(1.547,-2.000){2}{\rule{0.350pt}{0.400pt}}
\put(528,395.67){\rule{0.723pt}{0.400pt}}
\multiput(528.00,396.17)(1.500,-1.000){2}{\rule{0.361pt}{0.400pt}}
\put(531,394.67){\rule{0.723pt}{0.400pt}}
\multiput(531.00,395.17)(1.500,-1.000){2}{\rule{0.361pt}{0.400pt}}
\put(534,393.67){\rule{0.723pt}{0.400pt}}
\multiput(534.00,394.17)(1.500,-1.000){2}{\rule{0.361pt}{0.400pt}}
\put(537,392.67){\rule{0.723pt}{0.400pt}}
\multiput(537.00,393.17)(1.500,-1.000){2}{\rule{0.361pt}{0.400pt}}
\put(540,391.67){\rule{0.723pt}{0.400pt}}
\multiput(540.00,392.17)(1.500,-1.000){2}{\rule{0.361pt}{0.400pt}}
\put(543,390.67){\rule{0.964pt}{0.400pt}}
\multiput(543.00,391.17)(2.000,-1.000){2}{\rule{0.482pt}{0.400pt}}
\put(547,389.67){\rule{0.723pt}{0.400pt}}
\multiput(547.00,390.17)(1.500,-1.000){2}{\rule{0.361pt}{0.400pt}}
\put(550,388.67){\rule{0.723pt}{0.400pt}}
\multiput(550.00,389.17)(1.500,-1.000){2}{\rule{0.361pt}{0.400pt}}
\put(537.0,403.0){\rule[-0.200pt]{0.723pt}{0.400pt}}
\put(556,387.67){\rule{0.723pt}{0.400pt}}
\multiput(556.00,388.17)(1.500,-1.000){2}{\rule{0.361pt}{0.400pt}}
\put(553.0,389.0){\rule[-0.200pt]{0.723pt}{0.400pt}}
\put(559.0,388.0){\rule[-0.200pt]{0.723pt}{0.400pt}}
\put(562.0,388.0){\rule[-0.200pt]{0.723pt}{0.400pt}}
\put(565.0,388.0){\rule[-0.200pt]{0.723pt}{0.400pt}}
\put(568.0,388.0){\rule[-0.200pt]{0.964pt}{0.400pt}}
\put(572.0,388.0){\rule[-0.200pt]{0.723pt}{0.400pt}}
\put(575.0,388.0){\rule[-0.200pt]{0.723pt}{0.400pt}}
\put(581,387.67){\rule{0.723pt}{0.400pt}}
\multiput(581.00,387.17)(1.500,1.000){2}{\rule{0.361pt}{0.400pt}}
\put(578.0,388.0){\rule[-0.200pt]{0.723pt}{0.400pt}}
\put(587,388.67){\rule{0.723pt}{0.400pt}}
\multiput(587.00,388.17)(1.500,1.000){2}{\rule{0.361pt}{0.400pt}}
\put(584.0,389.0){\rule[-0.200pt]{0.723pt}{0.400pt}}
\put(593,389.67){\rule{0.723pt}{0.400pt}}
\multiput(593.00,389.17)(1.500,1.000){2}{\rule{0.361pt}{0.400pt}}
\put(596,390.67){\rule{0.964pt}{0.400pt}}
\multiput(596.00,390.17)(2.000,1.000){2}{\rule{0.482pt}{0.400pt}}
\put(600,391.67){\rule{0.723pt}{0.400pt}}
\multiput(600.00,391.17)(1.500,1.000){2}{\rule{0.361pt}{0.400pt}}
\put(603,392.67){\rule{0.723pt}{0.400pt}}
\multiput(603.00,392.17)(1.500,1.000){2}{\rule{0.361pt}{0.400pt}}
\put(606,393.67){\rule{0.723pt}{0.400pt}}
\multiput(606.00,393.17)(1.500,1.000){2}{\rule{0.361pt}{0.400pt}}
\put(609,394.67){\rule{0.723pt}{0.400pt}}
\multiput(609.00,394.17)(1.500,1.000){2}{\rule{0.361pt}{0.400pt}}
\put(612,395.67){\rule{0.723pt}{0.400pt}}
\multiput(612.00,395.17)(1.500,1.000){2}{\rule{0.361pt}{0.400pt}}
\put(615,397.17){\rule{0.700pt}{0.400pt}}
\multiput(615.00,396.17)(1.547,2.000){2}{\rule{0.350pt}{0.400pt}}
\put(618,398.67){\rule{0.723pt}{0.400pt}}
\multiput(618.00,398.17)(1.500,1.000){2}{\rule{0.361pt}{0.400pt}}
\put(621,400.17){\rule{0.900pt}{0.400pt}}
\multiput(621.00,399.17)(2.132,2.000){2}{\rule{0.450pt}{0.400pt}}
\put(625,402.17){\rule{0.700pt}{0.400pt}}
\multiput(625.00,401.17)(1.547,2.000){2}{\rule{0.350pt}{0.400pt}}
\put(628,404.17){\rule{0.700pt}{0.400pt}}
\multiput(628.00,403.17)(1.547,2.000){2}{\rule{0.350pt}{0.400pt}}
\put(631,406.17){\rule{0.700pt}{0.400pt}}
\multiput(631.00,405.17)(1.547,2.000){2}{\rule{0.350pt}{0.400pt}}
\put(634,408.17){\rule{0.700pt}{0.400pt}}
\multiput(634.00,407.17)(1.547,2.000){2}{\rule{0.350pt}{0.400pt}}
\put(637,410.17){\rule{0.700pt}{0.400pt}}
\multiput(637.00,409.17)(1.547,2.000){2}{\rule{0.350pt}{0.400pt}}
\put(640,412.17){\rule{0.700pt}{0.400pt}}
\multiput(640.00,411.17)(1.547,2.000){2}{\rule{0.350pt}{0.400pt}}
\put(643,414.17){\rule{0.700pt}{0.400pt}}
\multiput(643.00,413.17)(1.547,2.000){2}{\rule{0.350pt}{0.400pt}}
\multiput(646.00,416.61)(0.462,0.447){3}{\rule{0.500pt}{0.108pt}}
\multiput(646.00,415.17)(1.962,3.000){2}{\rule{0.250pt}{0.400pt}}
\put(649,419.17){\rule{0.900pt}{0.400pt}}
\multiput(649.00,418.17)(2.132,2.000){2}{\rule{0.450pt}{0.400pt}}
\multiput(653.00,421.61)(0.462,0.447){3}{\rule{0.500pt}{0.108pt}}
\multiput(653.00,420.17)(1.962,3.000){2}{\rule{0.250pt}{0.400pt}}
\multiput(656.00,424.61)(0.462,0.447){3}{\rule{0.500pt}{0.108pt}}
\multiput(656.00,423.17)(1.962,3.000){2}{\rule{0.250pt}{0.400pt}}
\put(659,427.17){\rule{0.700pt}{0.400pt}}
\multiput(659.00,426.17)(1.547,2.000){2}{\rule{0.350pt}{0.400pt}}
\multiput(662.00,429.61)(0.462,0.447){3}{\rule{0.500pt}{0.108pt}}
\multiput(662.00,428.17)(1.962,3.000){2}{\rule{0.250pt}{0.400pt}}
\multiput(665.00,432.61)(0.462,0.447){3}{\rule{0.500pt}{0.108pt}}
\multiput(665.00,431.17)(1.962,3.000){2}{\rule{0.250pt}{0.400pt}}
\multiput(668.00,435.61)(0.462,0.447){3}{\rule{0.500pt}{0.108pt}}
\multiput(668.00,434.17)(1.962,3.000){2}{\rule{0.250pt}{0.400pt}}
\multiput(671.61,438.00)(0.447,0.685){3}{\rule{0.108pt}{0.633pt}}
\multiput(670.17,438.00)(3.000,2.685){2}{\rule{0.400pt}{0.317pt}}
\multiput(674.00,442.61)(0.462,0.447){3}{\rule{0.500pt}{0.108pt}}
\multiput(674.00,441.17)(1.962,3.000){2}{\rule{0.250pt}{0.400pt}}
\multiput(677.00,445.61)(0.685,0.447){3}{\rule{0.633pt}{0.108pt}}
\multiput(677.00,444.17)(2.685,3.000){2}{\rule{0.317pt}{0.400pt}}
\multiput(681.61,448.00)(0.447,0.685){3}{\rule{0.108pt}{0.633pt}}
\multiput(680.17,448.00)(3.000,2.685){2}{\rule{0.400pt}{0.317pt}}
\multiput(684.00,452.61)(0.462,0.447){3}{\rule{0.500pt}{0.108pt}}
\multiput(684.00,451.17)(1.962,3.000){2}{\rule{0.250pt}{0.400pt}}
\multiput(687.61,455.00)(0.447,0.685){3}{\rule{0.108pt}{0.633pt}}
\multiput(686.17,455.00)(3.000,2.685){2}{\rule{0.400pt}{0.317pt}}
\multiput(690.61,459.00)(0.447,0.685){3}{\rule{0.108pt}{0.633pt}}
\multiput(689.17,459.00)(3.000,2.685){2}{\rule{0.400pt}{0.317pt}}
\multiput(693.61,463.00)(0.447,0.685){3}{\rule{0.108pt}{0.633pt}}
\multiput(692.17,463.00)(3.000,2.685){2}{\rule{0.400pt}{0.317pt}}
\multiput(696.61,467.00)(0.447,0.685){3}{\rule{0.108pt}{0.633pt}}
\multiput(695.17,467.00)(3.000,2.685){2}{\rule{0.400pt}{0.317pt}}
\multiput(699.61,471.00)(0.447,0.685){3}{\rule{0.108pt}{0.633pt}}
\multiput(698.17,471.00)(3.000,2.685){2}{\rule{0.400pt}{0.317pt}}
\multiput(702.00,475.60)(0.481,0.468){5}{\rule{0.500pt}{0.113pt}}
\multiput(702.00,474.17)(2.962,4.000){2}{\rule{0.250pt}{0.400pt}}
\multiput(706.61,479.00)(0.447,0.909){3}{\rule{0.108pt}{0.767pt}}
\multiput(705.17,479.00)(3.000,3.409){2}{\rule{0.400pt}{0.383pt}}
\multiput(709.61,484.00)(0.447,0.685){3}{\rule{0.108pt}{0.633pt}}
\multiput(708.17,484.00)(3.000,2.685){2}{\rule{0.400pt}{0.317pt}}
\multiput(712.61,488.00)(0.447,0.685){3}{\rule{0.108pt}{0.633pt}}
\multiput(711.17,488.00)(3.000,2.685){2}{\rule{0.400pt}{0.317pt}}
\multiput(715.61,492.00)(0.447,0.909){3}{\rule{0.108pt}{0.767pt}}
\multiput(714.17,492.00)(3.000,3.409){2}{\rule{0.400pt}{0.383pt}}
\multiput(718.61,497.00)(0.447,0.909){3}{\rule{0.108pt}{0.767pt}}
\multiput(717.17,497.00)(3.000,3.409){2}{\rule{0.400pt}{0.383pt}}
\multiput(721.61,502.00)(0.447,0.909){3}{\rule{0.108pt}{0.767pt}}
\multiput(720.17,502.00)(3.000,3.409){2}{\rule{0.400pt}{0.383pt}}
\multiput(724.61,507.00)(0.447,0.685){3}{\rule{0.108pt}{0.633pt}}
\multiput(723.17,507.00)(3.000,2.685){2}{\rule{0.400pt}{0.317pt}}
\multiput(727.61,511.00)(0.447,0.909){3}{\rule{0.108pt}{0.767pt}}
\multiput(726.17,511.00)(3.000,3.409){2}{\rule{0.400pt}{0.383pt}}
\multiput(730.60,516.00)(0.468,0.627){5}{\rule{0.113pt}{0.600pt}}
\multiput(729.17,516.00)(4.000,3.755){2}{\rule{0.400pt}{0.300pt}}
\multiput(734.61,521.00)(0.447,1.132){3}{\rule{0.108pt}{0.900pt}}
\multiput(733.17,521.00)(3.000,4.132){2}{\rule{0.400pt}{0.450pt}}
\multiput(737.61,527.00)(0.447,0.909){3}{\rule{0.108pt}{0.767pt}}
\multiput(736.17,527.00)(3.000,3.409){2}{\rule{0.400pt}{0.383pt}}
\multiput(740.61,532.00)(0.447,0.909){3}{\rule{0.108pt}{0.767pt}}
\multiput(739.17,532.00)(3.000,3.409){2}{\rule{0.400pt}{0.383pt}}
\multiput(743.61,537.00)(0.447,1.132){3}{\rule{0.108pt}{0.900pt}}
\multiput(742.17,537.00)(3.000,4.132){2}{\rule{0.400pt}{0.450pt}}
\multiput(746.61,543.00)(0.447,0.909){3}{\rule{0.108pt}{0.767pt}}
\multiput(745.17,543.00)(3.000,3.409){2}{\rule{0.400pt}{0.383pt}}
\multiput(749.61,548.00)(0.447,1.132){3}{\rule{0.108pt}{0.900pt}}
\multiput(748.17,548.00)(3.000,4.132){2}{\rule{0.400pt}{0.450pt}}
\multiput(752.61,554.00)(0.447,1.132){3}{\rule{0.108pt}{0.900pt}}
\multiput(751.17,554.00)(3.000,4.132){2}{\rule{0.400pt}{0.450pt}}
\multiput(755.60,560.00)(0.468,0.774){5}{\rule{0.113pt}{0.700pt}}
\multiput(754.17,560.00)(4.000,4.547){2}{\rule{0.400pt}{0.350pt}}
\multiput(759.61,566.00)(0.447,1.132){3}{\rule{0.108pt}{0.900pt}}
\multiput(758.17,566.00)(3.000,4.132){2}{\rule{0.400pt}{0.450pt}}
\multiput(762.61,572.00)(0.447,1.132){3}{\rule{0.108pt}{0.900pt}}
\multiput(761.17,572.00)(3.000,4.132){2}{\rule{0.400pt}{0.450pt}}
\multiput(765.61,578.00)(0.447,1.132){3}{\rule{0.108pt}{0.900pt}}
\multiput(764.17,578.00)(3.000,4.132){2}{\rule{0.400pt}{0.450pt}}
\multiput(768.61,584.00)(0.447,1.132){3}{\rule{0.108pt}{0.900pt}}
\multiput(767.17,584.00)(3.000,4.132){2}{\rule{0.400pt}{0.450pt}}
\multiput(771.61,590.00)(0.447,1.355){3}{\rule{0.108pt}{1.033pt}}
\multiput(770.17,590.00)(3.000,4.855){2}{\rule{0.400pt}{0.517pt}}
\multiput(774.61,597.00)(0.447,1.132){3}{\rule{0.108pt}{0.900pt}}
\multiput(773.17,597.00)(3.000,4.132){2}{\rule{0.400pt}{0.450pt}}
\multiput(777.61,603.00)(0.447,1.355){3}{\rule{0.108pt}{1.033pt}}
\multiput(776.17,603.00)(3.000,4.855){2}{\rule{0.400pt}{0.517pt}}
\multiput(780.61,610.00)(0.447,1.132){3}{\rule{0.108pt}{0.900pt}}
\multiput(779.17,610.00)(3.000,4.132){2}{\rule{0.400pt}{0.450pt}}
\multiput(534.00,433.95)(0.462,-0.447){3}{\rule{0.500pt}{0.108pt}}
\multiput(534.00,434.17)(1.962,-3.000){2}{\rule{0.250pt}{0.400pt}}
\multiput(537.00,430.95)(0.685,-0.447){3}{\rule{0.633pt}{0.108pt}}
\multiput(537.00,431.17)(2.685,-3.000){2}{\rule{0.317pt}{0.400pt}}
\put(541,427.17){\rule{0.700pt}{0.400pt}}
\multiput(541.00,428.17)(1.547,-2.000){2}{\rule{0.350pt}{0.400pt}}
\multiput(544.00,425.95)(0.462,-0.447){3}{\rule{0.500pt}{0.108pt}}
\multiput(544.00,426.17)(1.962,-3.000){2}{\rule{0.250pt}{0.400pt}}
\multiput(547.00,422.95)(0.462,-0.447){3}{\rule{0.500pt}{0.108pt}}
\multiput(547.00,423.17)(1.962,-3.000){2}{\rule{0.250pt}{0.400pt}}
\put(550,419.17){\rule{0.700pt}{0.400pt}}
\multiput(550.00,420.17)(1.547,-2.000){2}{\rule{0.350pt}{0.400pt}}
\multiput(553.00,417.95)(0.462,-0.447){3}{\rule{0.500pt}{0.108pt}}
\multiput(553.00,418.17)(1.962,-3.000){2}{\rule{0.250pt}{0.400pt}}
\put(556,414.17){\rule{0.700pt}{0.400pt}}
\multiput(556.00,415.17)(1.547,-2.000){2}{\rule{0.350pt}{0.400pt}}
\put(559,412.17){\rule{0.700pt}{0.400pt}}
\multiput(559.00,413.17)(1.547,-2.000){2}{\rule{0.350pt}{0.400pt}}
\put(562,410.17){\rule{0.700pt}{0.400pt}}
\multiput(562.00,411.17)(1.547,-2.000){2}{\rule{0.350pt}{0.400pt}}
\put(565,408.17){\rule{0.900pt}{0.400pt}}
\multiput(565.00,409.17)(2.132,-2.000){2}{\rule{0.450pt}{0.400pt}}
\put(569,406.17){\rule{0.700pt}{0.400pt}}
\multiput(569.00,407.17)(1.547,-2.000){2}{\rule{0.350pt}{0.400pt}}
\put(572,404.17){\rule{0.700pt}{0.400pt}}
\multiput(572.00,405.17)(1.547,-2.000){2}{\rule{0.350pt}{0.400pt}}
\put(575,402.17){\rule{0.700pt}{0.400pt}}
\multiput(575.00,403.17)(1.547,-2.000){2}{\rule{0.350pt}{0.400pt}}
\put(578,400.67){\rule{0.723pt}{0.400pt}}
\multiput(578.00,401.17)(1.500,-1.000){2}{\rule{0.361pt}{0.400pt}}
\put(581,399.17){\rule{0.700pt}{0.400pt}}
\multiput(581.00,400.17)(1.547,-2.000){2}{\rule{0.350pt}{0.400pt}}
\put(584,397.67){\rule{0.723pt}{0.400pt}}
\multiput(584.00,398.17)(1.500,-1.000){2}{\rule{0.361pt}{0.400pt}}
\put(587,396.17){\rule{0.700pt}{0.400pt}}
\multiput(587.00,397.17)(1.547,-2.000){2}{\rule{0.350pt}{0.400pt}}
\put(590,394.67){\rule{0.964pt}{0.400pt}}
\multiput(590.00,395.17)(2.000,-1.000){2}{\rule{0.482pt}{0.400pt}}
\put(594,393.67){\rule{0.723pt}{0.400pt}}
\multiput(594.00,394.17)(1.500,-1.000){2}{\rule{0.361pt}{0.400pt}}
\put(597,392.67){\rule{0.723pt}{0.400pt}}
\multiput(597.00,393.17)(1.500,-1.000){2}{\rule{0.361pt}{0.400pt}}
\put(600,391.67){\rule{0.723pt}{0.400pt}}
\multiput(600.00,392.17)(1.500,-1.000){2}{\rule{0.361pt}{0.400pt}}
\put(603,390.67){\rule{0.723pt}{0.400pt}}
\multiput(603.00,391.17)(1.500,-1.000){2}{\rule{0.361pt}{0.400pt}}
\put(606,389.67){\rule{0.723pt}{0.400pt}}
\multiput(606.00,390.17)(1.500,-1.000){2}{\rule{0.361pt}{0.400pt}}
\put(590.0,390.0){\rule[-0.200pt]{0.723pt}{0.400pt}}
\put(612,388.67){\rule{0.723pt}{0.400pt}}
\multiput(612.00,389.17)(1.500,-1.000){2}{\rule{0.361pt}{0.400pt}}
\put(609.0,390.0){\rule[-0.200pt]{0.723pt}{0.400pt}}
\put(618,387.67){\rule{0.964pt}{0.400pt}}
\multiput(618.00,388.17)(2.000,-1.000){2}{\rule{0.482pt}{0.400pt}}
\put(615.0,389.0){\rule[-0.200pt]{0.723pt}{0.400pt}}
\put(622.0,388.0){\rule[-0.200pt]{0.723pt}{0.400pt}}
\put(625.0,388.0){\rule[-0.200pt]{0.723pt}{0.400pt}}
\put(628.0,388.0){\rule[-0.200pt]{0.723pt}{0.400pt}}
\put(631.0,388.0){\rule[-0.200pt]{0.723pt}{0.400pt}}
\put(634.0,388.0){\rule[-0.200pt]{0.723pt}{0.400pt}}
\put(640,387.67){\rule{0.723pt}{0.400pt}}
\multiput(640.00,387.17)(1.500,1.000){2}{\rule{0.361pt}{0.400pt}}
\put(637.0,388.0){\rule[-0.200pt]{0.723pt}{0.400pt}}
\put(646,388.67){\rule{0.964pt}{0.400pt}}
\multiput(646.00,388.17)(2.000,1.000){2}{\rule{0.482pt}{0.400pt}}
\put(643.0,389.0){\rule[-0.200pt]{0.723pt}{0.400pt}}
\put(653,389.67){\rule{0.723pt}{0.400pt}}
\multiput(653.00,389.17)(1.500,1.000){2}{\rule{0.361pt}{0.400pt}}
\put(656,390.67){\rule{0.723pt}{0.400pt}}
\multiput(656.00,390.17)(1.500,1.000){2}{\rule{0.361pt}{0.400pt}}
\put(659,391.67){\rule{0.723pt}{0.400pt}}
\multiput(659.00,391.17)(1.500,1.000){2}{\rule{0.361pt}{0.400pt}}
\put(662,392.67){\rule{0.723pt}{0.400pt}}
\multiput(662.00,392.17)(1.500,1.000){2}{\rule{0.361pt}{0.400pt}}
\put(665,393.67){\rule{0.723pt}{0.400pt}}
\multiput(665.00,393.17)(1.500,1.000){2}{\rule{0.361pt}{0.400pt}}
\put(668,394.67){\rule{0.723pt}{0.400pt}}
\multiput(668.00,394.17)(1.500,1.000){2}{\rule{0.361pt}{0.400pt}}
\put(671,396.17){\rule{0.900pt}{0.400pt}}
\multiput(671.00,395.17)(2.132,2.000){2}{\rule{0.450pt}{0.400pt}}
\put(675,397.67){\rule{0.723pt}{0.400pt}}
\multiput(675.00,397.17)(1.500,1.000){2}{\rule{0.361pt}{0.400pt}}
\put(678,399.17){\rule{0.700pt}{0.400pt}}
\multiput(678.00,398.17)(1.547,2.000){2}{\rule{0.350pt}{0.400pt}}
\put(681,400.67){\rule{0.723pt}{0.400pt}}
\multiput(681.00,400.17)(1.500,1.000){2}{\rule{0.361pt}{0.400pt}}
\put(684,402.17){\rule{0.700pt}{0.400pt}}
\multiput(684.00,401.17)(1.547,2.000){2}{\rule{0.350pt}{0.400pt}}
\put(687,404.17){\rule{0.700pt}{0.400pt}}
\multiput(687.00,403.17)(1.547,2.000){2}{\rule{0.350pt}{0.400pt}}
\put(690,406.17){\rule{0.700pt}{0.400pt}}
\multiput(690.00,405.17)(1.547,2.000){2}{\rule{0.350pt}{0.400pt}}
\put(693,408.17){\rule{0.700pt}{0.400pt}}
\multiput(693.00,407.17)(1.547,2.000){2}{\rule{0.350pt}{0.400pt}}
\put(696,410.17){\rule{0.700pt}{0.400pt}}
\multiput(696.00,409.17)(1.547,2.000){2}{\rule{0.350pt}{0.400pt}}
\put(699,412.17){\rule{0.900pt}{0.400pt}}
\multiput(699.00,411.17)(2.132,2.000){2}{\rule{0.450pt}{0.400pt}}
\multiput(703.00,414.61)(0.462,0.447){3}{\rule{0.500pt}{0.108pt}}
\multiput(703.00,413.17)(1.962,3.000){2}{\rule{0.250pt}{0.400pt}}
\put(706,417.17){\rule{0.700pt}{0.400pt}}
\multiput(706.00,416.17)(1.547,2.000){2}{\rule{0.350pt}{0.400pt}}
\multiput(709.00,419.61)(0.462,0.447){3}{\rule{0.500pt}{0.108pt}}
\multiput(709.00,418.17)(1.962,3.000){2}{\rule{0.250pt}{0.400pt}}
\put(712,422.17){\rule{0.700pt}{0.400pt}}
\multiput(712.00,421.17)(1.547,2.000){2}{\rule{0.350pt}{0.400pt}}
\multiput(715.00,424.61)(0.462,0.447){3}{\rule{0.500pt}{0.108pt}}
\multiput(715.00,423.17)(1.962,3.000){2}{\rule{0.250pt}{0.400pt}}
\multiput(718.00,427.61)(0.462,0.447){3}{\rule{0.500pt}{0.108pt}}
\multiput(718.00,426.17)(1.962,3.000){2}{\rule{0.250pt}{0.400pt}}
\multiput(721.00,430.61)(0.462,0.447){3}{\rule{0.500pt}{0.108pt}}
\multiput(721.00,429.17)(1.962,3.000){2}{\rule{0.250pt}{0.400pt}}
\multiput(724.00,433.61)(0.685,0.447){3}{\rule{0.633pt}{0.108pt}}
\multiput(724.00,432.17)(2.685,3.000){2}{\rule{0.317pt}{0.400pt}}
\multiput(728.00,436.61)(0.462,0.447){3}{\rule{0.500pt}{0.108pt}}
\multiput(728.00,435.17)(1.962,3.000){2}{\rule{0.250pt}{0.400pt}}
\multiput(731.00,439.61)(0.462,0.447){3}{\rule{0.500pt}{0.108pt}}
\multiput(731.00,438.17)(1.962,3.000){2}{\rule{0.250pt}{0.400pt}}
\multiput(734.00,442.61)(0.462,0.447){3}{\rule{0.500pt}{0.108pt}}
\multiput(734.00,441.17)(1.962,3.000){2}{\rule{0.250pt}{0.400pt}}
\multiput(737.61,445.00)(0.447,0.685){3}{\rule{0.108pt}{0.633pt}}
\multiput(736.17,445.00)(3.000,2.685){2}{\rule{0.400pt}{0.317pt}}
\multiput(740.00,449.61)(0.462,0.447){3}{\rule{0.500pt}{0.108pt}}
\multiput(740.00,448.17)(1.962,3.000){2}{\rule{0.250pt}{0.400pt}}
\multiput(743.61,452.00)(0.447,0.685){3}{\rule{0.108pt}{0.633pt}}
\multiput(742.17,452.00)(3.000,2.685){2}{\rule{0.400pt}{0.317pt}}
\multiput(746.61,456.00)(0.447,0.685){3}{\rule{0.108pt}{0.633pt}}
\multiput(745.17,456.00)(3.000,2.685){2}{\rule{0.400pt}{0.317pt}}
\multiput(749.00,460.61)(0.462,0.447){3}{\rule{0.500pt}{0.108pt}}
\multiput(749.00,459.17)(1.962,3.000){2}{\rule{0.250pt}{0.400pt}}
\multiput(752.00,463.60)(0.481,0.468){5}{\rule{0.500pt}{0.113pt}}
\multiput(752.00,462.17)(2.962,4.000){2}{\rule{0.250pt}{0.400pt}}
\multiput(756.61,467.00)(0.447,0.685){3}{\rule{0.108pt}{0.633pt}}
\multiput(755.17,467.00)(3.000,2.685){2}{\rule{0.400pt}{0.317pt}}
\multiput(759.61,471.00)(0.447,0.685){3}{\rule{0.108pt}{0.633pt}}
\multiput(758.17,471.00)(3.000,2.685){2}{\rule{0.400pt}{0.317pt}}
\multiput(762.61,475.00)(0.447,0.909){3}{\rule{0.108pt}{0.767pt}}
\multiput(761.17,475.00)(3.000,3.409){2}{\rule{0.400pt}{0.383pt}}
\multiput(765.61,480.00)(0.447,0.685){3}{\rule{0.108pt}{0.633pt}}
\multiput(764.17,480.00)(3.000,2.685){2}{\rule{0.400pt}{0.317pt}}
\multiput(768.61,484.00)(0.447,0.685){3}{\rule{0.108pt}{0.633pt}}
\multiput(767.17,484.00)(3.000,2.685){2}{\rule{0.400pt}{0.317pt}}
\multiput(771.61,488.00)(0.447,0.909){3}{\rule{0.108pt}{0.767pt}}
\multiput(770.17,488.00)(3.000,3.409){2}{\rule{0.400pt}{0.383pt}}
\multiput(774.61,493.00)(0.447,0.685){3}{\rule{0.108pt}{0.633pt}}
\multiput(773.17,493.00)(3.000,2.685){2}{\rule{0.400pt}{0.317pt}}
\multiput(777.61,497.00)(0.447,0.909){3}{\rule{0.108pt}{0.767pt}}
\multiput(776.17,497.00)(3.000,3.409){2}{\rule{0.400pt}{0.383pt}}
\multiput(780.60,502.00)(0.468,0.627){5}{\rule{0.113pt}{0.600pt}}
\multiput(779.17,502.00)(4.000,3.755){2}{\rule{0.400pt}{0.300pt}}
\multiput(784.61,507.00)(0.447,0.909){3}{\rule{0.108pt}{0.767pt}}
\multiput(783.17,507.00)(3.000,3.409){2}{\rule{0.400pt}{0.383pt}}
\multiput(787.61,512.00)(0.447,0.909){3}{\rule{0.108pt}{0.767pt}}
\multiput(786.17,512.00)(3.000,3.409){2}{\rule{0.400pt}{0.383pt}}
\multiput(790.61,517.00)(0.447,0.909){3}{\rule{0.108pt}{0.767pt}}
\multiput(789.17,517.00)(3.000,3.409){2}{\rule{0.400pt}{0.383pt}}
\multiput(793.61,522.00)(0.447,0.909){3}{\rule{0.108pt}{0.767pt}}
\multiput(792.17,522.00)(3.000,3.409){2}{\rule{0.400pt}{0.383pt}}
\multiput(796.61,527.00)(0.447,0.909){3}{\rule{0.108pt}{0.767pt}}
\multiput(795.17,527.00)(3.000,3.409){2}{\rule{0.400pt}{0.383pt}}
\multiput(799.61,532.00)(0.447,1.132){3}{\rule{0.108pt}{0.900pt}}
\multiput(798.17,532.00)(3.000,4.132){2}{\rule{0.400pt}{0.450pt}}
\multiput(802.61,538.00)(0.447,0.909){3}{\rule{0.108pt}{0.767pt}}
\multiput(801.17,538.00)(3.000,3.409){2}{\rule{0.400pt}{0.383pt}}
\multiput(805.60,543.00)(0.468,0.774){5}{\rule{0.113pt}{0.700pt}}
\multiput(804.17,543.00)(4.000,4.547){2}{\rule{0.400pt}{0.350pt}}
\multiput(809.61,549.00)(0.447,0.909){3}{\rule{0.108pt}{0.767pt}}
\multiput(808.17,549.00)(3.000,3.409){2}{\rule{0.400pt}{0.383pt}}
\multiput(812.61,554.00)(0.447,1.132){3}{\rule{0.108pt}{0.900pt}}
\multiput(811.17,554.00)(3.000,4.132){2}{\rule{0.400pt}{0.450pt}}
\multiput(815.61,560.00)(0.447,1.132){3}{\rule{0.108pt}{0.900pt}}
\multiput(814.17,560.00)(3.000,4.132){2}{\rule{0.400pt}{0.450pt}}
\multiput(818.61,566.00)(0.447,1.132){3}{\rule{0.108pt}{0.900pt}}
\multiput(817.17,566.00)(3.000,4.132){2}{\rule{0.400pt}{0.450pt}}
\multiput(821.61,572.00)(0.447,1.132){3}{\rule{0.108pt}{0.900pt}}
\multiput(820.17,572.00)(3.000,4.132){2}{\rule{0.400pt}{0.450pt}}
\multiput(824.61,578.00)(0.447,1.132){3}{\rule{0.108pt}{0.900pt}}
\multiput(823.17,578.00)(3.000,4.132){2}{\rule{0.400pt}{0.450pt}}
\multiput(827.61,584.00)(0.447,1.355){3}{\rule{0.108pt}{1.033pt}}
\multiput(826.17,584.00)(3.000,4.855){2}{\rule{0.400pt}{0.517pt}}
\multiput(830.61,591.00)(0.447,1.132){3}{\rule{0.108pt}{0.900pt}}
\multiput(829.17,591.00)(3.000,4.132){2}{\rule{0.400pt}{0.450pt}}
\multiput(833.60,597.00)(0.468,0.774){5}{\rule{0.113pt}{0.700pt}}
\multiput(832.17,597.00)(4.000,4.547){2}{\rule{0.400pt}{0.350pt}}
\multiput(837.61,603.00)(0.447,1.355){3}{\rule{0.108pt}{1.033pt}}
\multiput(836.17,603.00)(3.000,4.855){2}{\rule{0.400pt}{0.517pt}}
\multiput(840.61,610.00)(0.447,1.355){3}{\rule{0.108pt}{1.033pt}}
\multiput(839.17,610.00)(3.000,4.855){2}{\rule{0.400pt}{0.517pt}}
\multiput(594.00,445.95)(0.462,-0.447){3}{\rule{0.500pt}{0.108pt}}
\multiput(594.00,446.17)(1.962,-3.000){2}{\rule{0.250pt}{0.400pt}}
\put(597,442.17){\rule{0.700pt}{0.400pt}}
\multiput(597.00,443.17)(1.547,-2.000){2}{\rule{0.350pt}{0.400pt}}
\multiput(600.00,440.95)(0.462,-0.447){3}{\rule{0.500pt}{0.108pt}}
\multiput(600.00,441.17)(1.962,-3.000){2}{\rule{0.250pt}{0.400pt}}
\multiput(603.00,437.95)(0.462,-0.447){3}{\rule{0.500pt}{0.108pt}}
\multiput(603.00,438.17)(1.962,-3.000){2}{\rule{0.250pt}{0.400pt}}
\multiput(606.00,434.95)(0.462,-0.447){3}{\rule{0.500pt}{0.108pt}}
\multiput(606.00,435.17)(1.962,-3.000){2}{\rule{0.250pt}{0.400pt}}
\put(609,431.17){\rule{0.700pt}{0.400pt}}
\multiput(609.00,432.17)(1.547,-2.000){2}{\rule{0.350pt}{0.400pt}}
\put(612,429.17){\rule{0.700pt}{0.400pt}}
\multiput(612.00,430.17)(1.547,-2.000){2}{\rule{0.350pt}{0.400pt}}
\multiput(615.00,427.95)(0.685,-0.447){3}{\rule{0.633pt}{0.108pt}}
\multiput(615.00,428.17)(2.685,-3.000){2}{\rule{0.317pt}{0.400pt}}
\put(619,424.17){\rule{0.700pt}{0.400pt}}
\multiput(619.00,425.17)(1.547,-2.000){2}{\rule{0.350pt}{0.400pt}}
\put(622,422.17){\rule{0.700pt}{0.400pt}}
\multiput(622.00,423.17)(1.547,-2.000){2}{\rule{0.350pt}{0.400pt}}
\put(625,420.17){\rule{0.700pt}{0.400pt}}
\multiput(625.00,421.17)(1.547,-2.000){2}{\rule{0.350pt}{0.400pt}}
\put(628,418.17){\rule{0.700pt}{0.400pt}}
\multiput(628.00,419.17)(1.547,-2.000){2}{\rule{0.350pt}{0.400pt}}
\put(631,416.17){\rule{0.700pt}{0.400pt}}
\multiput(631.00,417.17)(1.547,-2.000){2}{\rule{0.350pt}{0.400pt}}
\put(634,414.17){\rule{0.700pt}{0.400pt}}
\multiput(634.00,415.17)(1.547,-2.000){2}{\rule{0.350pt}{0.400pt}}
\put(637,412.67){\rule{0.723pt}{0.400pt}}
\multiput(637.00,413.17)(1.500,-1.000){2}{\rule{0.361pt}{0.400pt}}
\put(640,411.17){\rule{0.900pt}{0.400pt}}
\multiput(640.00,412.17)(2.132,-2.000){2}{\rule{0.450pt}{0.400pt}}
\put(644,409.67){\rule{0.723pt}{0.400pt}}
\multiput(644.00,410.17)(1.500,-1.000){2}{\rule{0.361pt}{0.400pt}}
\put(647,408.67){\rule{0.723pt}{0.400pt}}
\multiput(647.00,409.17)(1.500,-1.000){2}{\rule{0.361pt}{0.400pt}}
\put(650,407.17){\rule{0.700pt}{0.400pt}}
\multiput(650.00,408.17)(1.547,-2.000){2}{\rule{0.350pt}{0.400pt}}
\put(653,405.67){\rule{0.723pt}{0.400pt}}
\multiput(653.00,406.17)(1.500,-1.000){2}{\rule{0.361pt}{0.400pt}}
\put(656,404.67){\rule{0.723pt}{0.400pt}}
\multiput(656.00,405.17)(1.500,-1.000){2}{\rule{0.361pt}{0.400pt}}
\put(659,403.67){\rule{0.723pt}{0.400pt}}
\multiput(659.00,404.17)(1.500,-1.000){2}{\rule{0.361pt}{0.400pt}}
\put(662,402.67){\rule{0.723pt}{0.400pt}}
\multiput(662.00,403.17)(1.500,-1.000){2}{\rule{0.361pt}{0.400pt}}
\put(650.0,390.0){\rule[-0.200pt]{0.723pt}{0.400pt}}
\put(668,401.67){\rule{0.964pt}{0.400pt}}
\multiput(668.00,402.17)(2.000,-1.000){2}{\rule{0.482pt}{0.400pt}}
\put(672,400.67){\rule{0.723pt}{0.400pt}}
\multiput(672.00,401.17)(1.500,-1.000){2}{\rule{0.361pt}{0.400pt}}
\put(665.0,403.0){\rule[-0.200pt]{0.723pt}{0.400pt}}
\put(675.0,401.0){\rule[-0.200pt]{0.723pt}{0.400pt}}
\put(681,399.67){\rule{0.723pt}{0.400pt}}
\multiput(681.00,400.17)(1.500,-1.000){2}{\rule{0.361pt}{0.400pt}}
\put(678.0,401.0){\rule[-0.200pt]{0.723pt}{0.400pt}}
\put(684.0,400.0){\rule[-0.200pt]{0.723pt}{0.400pt}}
\put(687.0,400.0){\rule[-0.200pt]{0.723pt}{0.400pt}}
\put(690.0,400.0){\rule[-0.200pt]{0.723pt}{0.400pt}}
\put(697,399.67){\rule{0.723pt}{0.400pt}}
\multiput(697.00,399.17)(1.500,1.000){2}{\rule{0.361pt}{0.400pt}}
\put(693.0,400.0){\rule[-0.200pt]{0.964pt}{0.400pt}}
\put(703,400.67){\rule{0.723pt}{0.400pt}}
\multiput(703.00,400.17)(1.500,1.000){2}{\rule{0.361pt}{0.400pt}}
\put(700.0,401.0){\rule[-0.200pt]{0.723pt}{0.400pt}}
\put(709,401.67){\rule{0.723pt}{0.400pt}}
\multiput(709.00,401.17)(1.500,1.000){2}{\rule{0.361pt}{0.400pt}}
\put(706.0,402.0){\rule[-0.200pt]{0.723pt}{0.400pt}}
\put(715,402.67){\rule{0.723pt}{0.400pt}}
\multiput(715.00,402.17)(1.500,1.000){2}{\rule{0.361pt}{0.400pt}}
\put(718,403.67){\rule{0.723pt}{0.400pt}}
\multiput(718.00,403.17)(1.500,1.000){2}{\rule{0.361pt}{0.400pt}}
\put(721,404.67){\rule{0.964pt}{0.400pt}}
\multiput(721.00,404.17)(2.000,1.000){2}{\rule{0.482pt}{0.400pt}}
\put(725,405.67){\rule{0.723pt}{0.400pt}}
\multiput(725.00,405.17)(1.500,1.000){2}{\rule{0.361pt}{0.400pt}}
\put(728,407.17){\rule{0.700pt}{0.400pt}}
\multiput(728.00,406.17)(1.547,2.000){2}{\rule{0.350pt}{0.400pt}}
\put(731,408.67){\rule{0.723pt}{0.400pt}}
\multiput(731.00,408.17)(1.500,1.000){2}{\rule{0.361pt}{0.400pt}}
\put(734,409.67){\rule{0.723pt}{0.400pt}}
\multiput(734.00,409.17)(1.500,1.000){2}{\rule{0.361pt}{0.400pt}}
\put(737,411.17){\rule{0.700pt}{0.400pt}}
\multiput(737.00,410.17)(1.547,2.000){2}{\rule{0.350pt}{0.400pt}}
\put(740,413.17){\rule{0.700pt}{0.400pt}}
\multiput(740.00,412.17)(1.547,2.000){2}{\rule{0.350pt}{0.400pt}}
\put(743,414.67){\rule{0.723pt}{0.400pt}}
\multiput(743.00,414.17)(1.500,1.000){2}{\rule{0.361pt}{0.400pt}}
\put(746,416.17){\rule{0.700pt}{0.400pt}}
\multiput(746.00,415.17)(1.547,2.000){2}{\rule{0.350pt}{0.400pt}}
\put(749,418.17){\rule{0.900pt}{0.400pt}}
\multiput(749.00,417.17)(2.132,2.000){2}{\rule{0.450pt}{0.400pt}}
\put(753,420.17){\rule{0.700pt}{0.400pt}}
\multiput(753.00,419.17)(1.547,2.000){2}{\rule{0.350pt}{0.400pt}}
\put(756,422.17){\rule{0.700pt}{0.400pt}}
\multiput(756.00,421.17)(1.547,2.000){2}{\rule{0.350pt}{0.400pt}}
\multiput(759.00,424.61)(0.462,0.447){3}{\rule{0.500pt}{0.108pt}}
\multiput(759.00,423.17)(1.962,3.000){2}{\rule{0.250pt}{0.400pt}}
\put(762,427.17){\rule{0.700pt}{0.400pt}}
\multiput(762.00,426.17)(1.547,2.000){2}{\rule{0.350pt}{0.400pt}}
\put(765,429.17){\rule{0.700pt}{0.400pt}}
\multiput(765.00,428.17)(1.547,2.000){2}{\rule{0.350pt}{0.400pt}}
\multiput(768.00,431.61)(0.462,0.447){3}{\rule{0.500pt}{0.108pt}}
\multiput(768.00,430.17)(1.962,3.000){2}{\rule{0.250pt}{0.400pt}}
\put(771,434.17){\rule{0.700pt}{0.400pt}}
\multiput(771.00,433.17)(1.547,2.000){2}{\rule{0.350pt}{0.400pt}}
\multiput(774.00,436.61)(0.685,0.447){3}{\rule{0.633pt}{0.108pt}}
\multiput(774.00,435.17)(2.685,3.000){2}{\rule{0.317pt}{0.400pt}}
\multiput(778.00,439.61)(0.462,0.447){3}{\rule{0.500pt}{0.108pt}}
\multiput(778.00,438.17)(1.962,3.000){2}{\rule{0.250pt}{0.400pt}}
\multiput(781.00,442.61)(0.462,0.447){3}{\rule{0.500pt}{0.108pt}}
\multiput(781.00,441.17)(1.962,3.000){2}{\rule{0.250pt}{0.400pt}}
\multiput(784.00,445.61)(0.462,0.447){3}{\rule{0.500pt}{0.108pt}}
\multiput(784.00,444.17)(1.962,3.000){2}{\rule{0.250pt}{0.400pt}}
\multiput(787.00,448.61)(0.462,0.447){3}{\rule{0.500pt}{0.108pt}}
\multiput(787.00,447.17)(1.962,3.000){2}{\rule{0.250pt}{0.400pt}}
\multiput(790.00,451.61)(0.462,0.447){3}{\rule{0.500pt}{0.108pt}}
\multiput(790.00,450.17)(1.962,3.000){2}{\rule{0.250pt}{0.400pt}}
\multiput(793.61,454.00)(0.447,0.685){3}{\rule{0.108pt}{0.633pt}}
\multiput(792.17,454.00)(3.000,2.685){2}{\rule{0.400pt}{0.317pt}}
\multiput(796.00,458.61)(0.462,0.447){3}{\rule{0.500pt}{0.108pt}}
\multiput(796.00,457.17)(1.962,3.000){2}{\rule{0.250pt}{0.400pt}}
\multiput(799.00,461.61)(0.462,0.447){3}{\rule{0.500pt}{0.108pt}}
\multiput(799.00,460.17)(1.962,3.000){2}{\rule{0.250pt}{0.400pt}}
\multiput(802.00,464.60)(0.481,0.468){5}{\rule{0.500pt}{0.113pt}}
\multiput(802.00,463.17)(2.962,4.000){2}{\rule{0.250pt}{0.400pt}}
\multiput(806.61,468.00)(0.447,0.685){3}{\rule{0.108pt}{0.633pt}}
\multiput(805.17,468.00)(3.000,2.685){2}{\rule{0.400pt}{0.317pt}}
\multiput(809.61,472.00)(0.447,0.685){3}{\rule{0.108pt}{0.633pt}}
\multiput(808.17,472.00)(3.000,2.685){2}{\rule{0.400pt}{0.317pt}}
\multiput(812.00,476.61)(0.462,0.447){3}{\rule{0.500pt}{0.108pt}}
\multiput(812.00,475.17)(1.962,3.000){2}{\rule{0.250pt}{0.400pt}}
\multiput(815.61,479.00)(0.447,0.685){3}{\rule{0.108pt}{0.633pt}}
\multiput(814.17,479.00)(3.000,2.685){2}{\rule{0.400pt}{0.317pt}}
\multiput(818.61,483.00)(0.447,0.909){3}{\rule{0.108pt}{0.767pt}}
\multiput(817.17,483.00)(3.000,3.409){2}{\rule{0.400pt}{0.383pt}}
\multiput(821.61,488.00)(0.447,0.685){3}{\rule{0.108pt}{0.633pt}}
\multiput(820.17,488.00)(3.000,2.685){2}{\rule{0.400pt}{0.317pt}}
\multiput(824.61,492.00)(0.447,0.685){3}{\rule{0.108pt}{0.633pt}}
\multiput(823.17,492.00)(3.000,2.685){2}{\rule{0.400pt}{0.317pt}}
\multiput(827.00,496.60)(0.481,0.468){5}{\rule{0.500pt}{0.113pt}}
\multiput(827.00,495.17)(2.962,4.000){2}{\rule{0.250pt}{0.400pt}}
\multiput(831.61,500.00)(0.447,0.909){3}{\rule{0.108pt}{0.767pt}}
\multiput(830.17,500.00)(3.000,3.409){2}{\rule{0.400pt}{0.383pt}}
\multiput(834.61,505.00)(0.447,0.909){3}{\rule{0.108pt}{0.767pt}}
\multiput(833.17,505.00)(3.000,3.409){2}{\rule{0.400pt}{0.383pt}}
\multiput(837.61,510.00)(0.447,0.685){3}{\rule{0.108pt}{0.633pt}}
\multiput(836.17,510.00)(3.000,2.685){2}{\rule{0.400pt}{0.317pt}}
\multiput(840.61,514.00)(0.447,0.909){3}{\rule{0.108pt}{0.767pt}}
\multiput(839.17,514.00)(3.000,3.409){2}{\rule{0.400pt}{0.383pt}}
\multiput(843.61,519.00)(0.447,0.909){3}{\rule{0.108pt}{0.767pt}}
\multiput(842.17,519.00)(3.000,3.409){2}{\rule{0.400pt}{0.383pt}}
\multiput(846.61,524.00)(0.447,0.909){3}{\rule{0.108pt}{0.767pt}}
\multiput(845.17,524.00)(3.000,3.409){2}{\rule{0.400pt}{0.383pt}}
\multiput(849.61,529.00)(0.447,0.909){3}{\rule{0.108pt}{0.767pt}}
\multiput(848.17,529.00)(3.000,3.409){2}{\rule{0.400pt}{0.383pt}}
\multiput(852.61,534.00)(0.447,0.909){3}{\rule{0.108pt}{0.767pt}}
\multiput(851.17,534.00)(3.000,3.409){2}{\rule{0.400pt}{0.383pt}}
\multiput(855.60,539.00)(0.468,0.774){5}{\rule{0.113pt}{0.700pt}}
\multiput(854.17,539.00)(4.000,4.547){2}{\rule{0.400pt}{0.350pt}}
\multiput(859.61,545.00)(0.447,0.909){3}{\rule{0.108pt}{0.767pt}}
\multiput(858.17,545.00)(3.000,3.409){2}{\rule{0.400pt}{0.383pt}}
\multiput(862.61,550.00)(0.447,0.909){3}{\rule{0.108pt}{0.767pt}}
\multiput(861.17,550.00)(3.000,3.409){2}{\rule{0.400pt}{0.383pt}}
\multiput(865.61,555.00)(0.447,1.132){3}{\rule{0.108pt}{0.900pt}}
\multiput(864.17,555.00)(3.000,4.132){2}{\rule{0.400pt}{0.450pt}}
\multiput(868.61,561.00)(0.447,1.132){3}{\rule{0.108pt}{0.900pt}}
\multiput(867.17,561.00)(3.000,4.132){2}{\rule{0.400pt}{0.450pt}}
\multiput(871.61,567.00)(0.447,0.909){3}{\rule{0.108pt}{0.767pt}}
\multiput(870.17,567.00)(3.000,3.409){2}{\rule{0.400pt}{0.383pt}}
\multiput(874.61,572.00)(0.447,1.132){3}{\rule{0.108pt}{0.900pt}}
\multiput(873.17,572.00)(3.000,4.132){2}{\rule{0.400pt}{0.450pt}}
\multiput(877.61,578.00)(0.447,1.132){3}{\rule{0.108pt}{0.900pt}}
\multiput(876.17,578.00)(3.000,4.132){2}{\rule{0.400pt}{0.450pt}}
\multiput(880.61,584.00)(0.447,1.132){3}{\rule{0.108pt}{0.900pt}}
\multiput(879.17,584.00)(3.000,4.132){2}{\rule{0.400pt}{0.450pt}}
\multiput(883.60,590.00)(0.468,0.920){5}{\rule{0.113pt}{0.800pt}}
\multiput(882.17,590.00)(4.000,5.340){2}{\rule{0.400pt}{0.400pt}}
\multiput(887.61,597.00)(0.447,1.132){3}{\rule{0.108pt}{0.900pt}}
\multiput(886.17,597.00)(3.000,4.132){2}{\rule{0.400pt}{0.450pt}}
\multiput(890.61,603.00)(0.447,1.132){3}{\rule{0.108pt}{0.900pt}}
\multiput(889.17,603.00)(3.000,4.132){2}{\rule{0.400pt}{0.450pt}}
\multiput(893.61,609.00)(0.447,1.355){3}{\rule{0.108pt}{1.033pt}}
\multiput(892.17,609.00)(3.000,4.855){2}{\rule{0.400pt}{0.517pt}}
\multiput(896.61,616.00)(0.447,1.132){3}{\rule{0.108pt}{0.900pt}}
\multiput(895.17,616.00)(3.000,4.132){2}{\rule{0.400pt}{0.450pt}}
\multiput(899.61,622.00)(0.447,1.355){3}{\rule{0.108pt}{1.033pt}}
\multiput(898.17,622.00)(3.000,4.855){2}{\rule{0.400pt}{0.517pt}}
\multiput(653.00,470.95)(0.462,-0.447){3}{\rule{0.500pt}{0.108pt}}
\multiput(653.00,471.17)(1.962,-3.000){2}{\rule{0.250pt}{0.400pt}}
\multiput(656.00,467.95)(0.462,-0.447){3}{\rule{0.500pt}{0.108pt}}
\multiput(656.00,468.17)(1.962,-3.000){2}{\rule{0.250pt}{0.400pt}}
\multiput(659.00,464.95)(0.462,-0.447){3}{\rule{0.500pt}{0.108pt}}
\multiput(659.00,465.17)(1.962,-3.000){2}{\rule{0.250pt}{0.400pt}}
\multiput(662.00,461.95)(0.685,-0.447){3}{\rule{0.633pt}{0.108pt}}
\multiput(662.00,462.17)(2.685,-3.000){2}{\rule{0.317pt}{0.400pt}}
\put(666,458.17){\rule{0.700pt}{0.400pt}}
\multiput(666.00,459.17)(1.547,-2.000){2}{\rule{0.350pt}{0.400pt}}
\multiput(669.00,456.95)(0.462,-0.447){3}{\rule{0.500pt}{0.108pt}}
\multiput(669.00,457.17)(1.962,-3.000){2}{\rule{0.250pt}{0.400pt}}
\put(672,453.17){\rule{0.700pt}{0.400pt}}
\multiput(672.00,454.17)(1.547,-2.000){2}{\rule{0.350pt}{0.400pt}}
\multiput(675.00,451.95)(0.462,-0.447){3}{\rule{0.500pt}{0.108pt}}
\multiput(675.00,452.17)(1.962,-3.000){2}{\rule{0.250pt}{0.400pt}}
\put(678,448.17){\rule{0.700pt}{0.400pt}}
\multiput(678.00,449.17)(1.547,-2.000){2}{\rule{0.350pt}{0.400pt}}
\put(681,446.17){\rule{0.700pt}{0.400pt}}
\multiput(681.00,447.17)(1.547,-2.000){2}{\rule{0.350pt}{0.400pt}}
\put(684,444.17){\rule{0.700pt}{0.400pt}}
\multiput(684.00,445.17)(1.547,-2.000){2}{\rule{0.350pt}{0.400pt}}
\put(687,442.17){\rule{0.700pt}{0.400pt}}
\multiput(687.00,443.17)(1.547,-2.000){2}{\rule{0.350pt}{0.400pt}}
\put(690,440.17){\rule{0.900pt}{0.400pt}}
\multiput(690.00,441.17)(2.132,-2.000){2}{\rule{0.450pt}{0.400pt}}
\put(694,438.67){\rule{0.723pt}{0.400pt}}
\multiput(694.00,439.17)(1.500,-1.000){2}{\rule{0.361pt}{0.400pt}}
\put(697,437.17){\rule{0.700pt}{0.400pt}}
\multiput(697.00,438.17)(1.547,-2.000){2}{\rule{0.350pt}{0.400pt}}
\put(700,435.17){\rule{0.700pt}{0.400pt}}
\multiput(700.00,436.17)(1.547,-2.000){2}{\rule{0.350pt}{0.400pt}}
\put(703,433.67){\rule{0.723pt}{0.400pt}}
\multiput(703.00,434.17)(1.500,-1.000){2}{\rule{0.361pt}{0.400pt}}
\put(706,432.67){\rule{0.723pt}{0.400pt}}
\multiput(706.00,433.17)(1.500,-1.000){2}{\rule{0.361pt}{0.400pt}}
\put(709,431.17){\rule{0.700pt}{0.400pt}}
\multiput(709.00,432.17)(1.547,-2.000){2}{\rule{0.350pt}{0.400pt}}
\put(712,429.67){\rule{0.723pt}{0.400pt}}
\multiput(712.00,430.17)(1.500,-1.000){2}{\rule{0.361pt}{0.400pt}}
\put(715,428.67){\rule{0.723pt}{0.400pt}}
\multiput(715.00,429.17)(1.500,-1.000){2}{\rule{0.361pt}{0.400pt}}
\put(718,427.67){\rule{0.964pt}{0.400pt}}
\multiput(718.00,428.17)(2.000,-1.000){2}{\rule{0.482pt}{0.400pt}}
\put(712.0,403.0){\rule[-0.200pt]{0.723pt}{0.400pt}}
\put(725,426.67){\rule{0.723pt}{0.400pt}}
\multiput(725.00,427.17)(1.500,-1.000){2}{\rule{0.361pt}{0.400pt}}
\put(728,425.67){\rule{0.723pt}{0.400pt}}
\multiput(728.00,426.17)(1.500,-1.000){2}{\rule{0.361pt}{0.400pt}}
\put(722.0,428.0){\rule[-0.200pt]{0.723pt}{0.400pt}}
\put(734,424.67){\rule{0.723pt}{0.400pt}}
\multiput(734.00,425.17)(1.500,-1.000){2}{\rule{0.361pt}{0.400pt}}
\put(731.0,426.0){\rule[-0.200pt]{0.723pt}{0.400pt}}
\put(737.0,425.0){\rule[-0.200pt]{0.723pt}{0.400pt}}
\put(743,423.67){\rule{0.964pt}{0.400pt}}
\multiput(743.00,424.17)(2.000,-1.000){2}{\rule{0.482pt}{0.400pt}}
\put(740.0,425.0){\rule[-0.200pt]{0.723pt}{0.400pt}}
\put(750,423.67){\rule{0.723pt}{0.400pt}}
\multiput(750.00,423.17)(1.500,1.000){2}{\rule{0.361pt}{0.400pt}}
\put(747.0,424.0){\rule[-0.200pt]{0.723pt}{0.400pt}}
\put(753.0,425.0){\rule[-0.200pt]{0.723pt}{0.400pt}}
\put(756.0,425.0){\rule[-0.200pt]{0.723pt}{0.400pt}}
\put(762,424.67){\rule{0.723pt}{0.400pt}}
\multiput(762.00,424.17)(1.500,1.000){2}{\rule{0.361pt}{0.400pt}}
\put(759.0,425.0){\rule[-0.200pt]{0.723pt}{0.400pt}}
\put(768,425.67){\rule{0.723pt}{0.400pt}}
\multiput(768.00,425.17)(1.500,1.000){2}{\rule{0.361pt}{0.400pt}}
\put(771,426.67){\rule{0.964pt}{0.400pt}}
\multiput(771.00,426.17)(2.000,1.000){2}{\rule{0.482pt}{0.400pt}}
\put(775,427.67){\rule{0.723pt}{0.400pt}}
\multiput(775.00,427.17)(1.500,1.000){2}{\rule{0.361pt}{0.400pt}}
\put(765.0,426.0){\rule[-0.200pt]{0.723pt}{0.400pt}}
\put(781,428.67){\rule{0.723pt}{0.400pt}}
\multiput(781.00,428.17)(1.500,1.000){2}{\rule{0.361pt}{0.400pt}}
\put(784,430.17){\rule{0.700pt}{0.400pt}}
\multiput(784.00,429.17)(1.547,2.000){2}{\rule{0.350pt}{0.400pt}}
\put(787,431.67){\rule{0.723pt}{0.400pt}}
\multiput(787.00,431.17)(1.500,1.000){2}{\rule{0.361pt}{0.400pt}}
\put(790,432.67){\rule{0.723pt}{0.400pt}}
\multiput(790.00,432.17)(1.500,1.000){2}{\rule{0.361pt}{0.400pt}}
\put(793,434.17){\rule{0.700pt}{0.400pt}}
\multiput(793.00,433.17)(1.547,2.000){2}{\rule{0.350pt}{0.400pt}}
\put(796,435.67){\rule{0.964pt}{0.400pt}}
\multiput(796.00,435.17)(2.000,1.000){2}{\rule{0.482pt}{0.400pt}}
\put(800,437.17){\rule{0.700pt}{0.400pt}}
\multiput(800.00,436.17)(1.547,2.000){2}{\rule{0.350pt}{0.400pt}}
\put(803,439.17){\rule{0.700pt}{0.400pt}}
\multiput(803.00,438.17)(1.547,2.000){2}{\rule{0.350pt}{0.400pt}}
\put(806,440.67){\rule{0.723pt}{0.400pt}}
\multiput(806.00,440.17)(1.500,1.000){2}{\rule{0.361pt}{0.400pt}}
\put(809,442.17){\rule{0.700pt}{0.400pt}}
\multiput(809.00,441.17)(1.547,2.000){2}{\rule{0.350pt}{0.400pt}}
\put(812,444.17){\rule{0.700pt}{0.400pt}}
\multiput(812.00,443.17)(1.547,2.000){2}{\rule{0.350pt}{0.400pt}}
\put(815,446.17){\rule{0.700pt}{0.400pt}}
\multiput(815.00,445.17)(1.547,2.000){2}{\rule{0.350pt}{0.400pt}}
\multiput(818.00,448.61)(0.462,0.447){3}{\rule{0.500pt}{0.108pt}}
\multiput(818.00,447.17)(1.962,3.000){2}{\rule{0.250pt}{0.400pt}}
\put(821,451.17){\rule{0.700pt}{0.400pt}}
\multiput(821.00,450.17)(1.547,2.000){2}{\rule{0.350pt}{0.400pt}}
\put(824,453.17){\rule{0.900pt}{0.400pt}}
\multiput(824.00,452.17)(2.132,2.000){2}{\rule{0.450pt}{0.400pt}}
\multiput(828.00,455.61)(0.462,0.447){3}{\rule{0.500pt}{0.108pt}}
\multiput(828.00,454.17)(1.962,3.000){2}{\rule{0.250pt}{0.400pt}}
\multiput(831.00,458.61)(0.462,0.447){3}{\rule{0.500pt}{0.108pt}}
\multiput(831.00,457.17)(1.962,3.000){2}{\rule{0.250pt}{0.400pt}}
\put(834,461.17){\rule{0.700pt}{0.400pt}}
\multiput(834.00,460.17)(1.547,2.000){2}{\rule{0.350pt}{0.400pt}}
\multiput(837.00,463.61)(0.462,0.447){3}{\rule{0.500pt}{0.108pt}}
\multiput(837.00,462.17)(1.962,3.000){2}{\rule{0.250pt}{0.400pt}}
\multiput(840.00,466.61)(0.462,0.447){3}{\rule{0.500pt}{0.108pt}}
\multiput(840.00,465.17)(1.962,3.000){2}{\rule{0.250pt}{0.400pt}}
\multiput(843.00,469.61)(0.462,0.447){3}{\rule{0.500pt}{0.108pt}}
\multiput(843.00,468.17)(1.962,3.000){2}{\rule{0.250pt}{0.400pt}}
\multiput(846.00,472.61)(0.462,0.447){3}{\rule{0.500pt}{0.108pt}}
\multiput(846.00,471.17)(1.962,3.000){2}{\rule{0.250pt}{0.400pt}}
\multiput(849.00,475.61)(0.462,0.447){3}{\rule{0.500pt}{0.108pt}}
\multiput(849.00,474.17)(1.962,3.000){2}{\rule{0.250pt}{0.400pt}}
\multiput(852.00,478.60)(0.481,0.468){5}{\rule{0.500pt}{0.113pt}}
\multiput(852.00,477.17)(2.962,4.000){2}{\rule{0.250pt}{0.400pt}}
\multiput(856.00,482.61)(0.462,0.447){3}{\rule{0.500pt}{0.108pt}}
\multiput(856.00,481.17)(1.962,3.000){2}{\rule{0.250pt}{0.400pt}}
\multiput(859.61,485.00)(0.447,0.685){3}{\rule{0.108pt}{0.633pt}}
\multiput(858.17,485.00)(3.000,2.685){2}{\rule{0.400pt}{0.317pt}}
\multiput(862.00,489.61)(0.462,0.447){3}{\rule{0.500pt}{0.108pt}}
\multiput(862.00,488.17)(1.962,3.000){2}{\rule{0.250pt}{0.400pt}}
\multiput(865.61,492.00)(0.447,0.685){3}{\rule{0.108pt}{0.633pt}}
\multiput(864.17,492.00)(3.000,2.685){2}{\rule{0.400pt}{0.317pt}}
\multiput(868.61,496.00)(0.447,0.685){3}{\rule{0.108pt}{0.633pt}}
\multiput(867.17,496.00)(3.000,2.685){2}{\rule{0.400pt}{0.317pt}}
\multiput(871.61,500.00)(0.447,0.685){3}{\rule{0.108pt}{0.633pt}}
\multiput(870.17,500.00)(3.000,2.685){2}{\rule{0.400pt}{0.317pt}}
\multiput(874.61,504.00)(0.447,0.685){3}{\rule{0.108pt}{0.633pt}}
\multiput(873.17,504.00)(3.000,2.685){2}{\rule{0.400pt}{0.317pt}}
\multiput(877.00,508.60)(0.481,0.468){5}{\rule{0.500pt}{0.113pt}}
\multiput(877.00,507.17)(2.962,4.000){2}{\rule{0.250pt}{0.400pt}}
\multiput(881.61,512.00)(0.447,0.685){3}{\rule{0.108pt}{0.633pt}}
\multiput(880.17,512.00)(3.000,2.685){2}{\rule{0.400pt}{0.317pt}}
\multiput(884.61,516.00)(0.447,0.685){3}{\rule{0.108pt}{0.633pt}}
\multiput(883.17,516.00)(3.000,2.685){2}{\rule{0.400pt}{0.317pt}}
\multiput(887.61,520.00)(0.447,0.909){3}{\rule{0.108pt}{0.767pt}}
\multiput(886.17,520.00)(3.000,3.409){2}{\rule{0.400pt}{0.383pt}}
\multiput(890.61,525.00)(0.447,0.685){3}{\rule{0.108pt}{0.633pt}}
\multiput(889.17,525.00)(3.000,2.685){2}{\rule{0.400pt}{0.317pt}}
\multiput(893.61,529.00)(0.447,0.909){3}{\rule{0.108pt}{0.767pt}}
\multiput(892.17,529.00)(3.000,3.409){2}{\rule{0.400pt}{0.383pt}}
\multiput(896.61,534.00)(0.447,0.685){3}{\rule{0.108pt}{0.633pt}}
\multiput(895.17,534.00)(3.000,2.685){2}{\rule{0.400pt}{0.317pt}}
\multiput(899.61,538.00)(0.447,0.909){3}{\rule{0.108pt}{0.767pt}}
\multiput(898.17,538.00)(3.000,3.409){2}{\rule{0.400pt}{0.383pt}}
\multiput(902.61,543.00)(0.447,0.909){3}{\rule{0.108pt}{0.767pt}}
\multiput(901.17,543.00)(3.000,3.409){2}{\rule{0.400pt}{0.383pt}}
\multiput(905.60,548.00)(0.468,0.627){5}{\rule{0.113pt}{0.600pt}}
\multiput(904.17,548.00)(4.000,3.755){2}{\rule{0.400pt}{0.300pt}}
\multiput(909.61,553.00)(0.447,0.909){3}{\rule{0.108pt}{0.767pt}}
\multiput(908.17,553.00)(3.000,3.409){2}{\rule{0.400pt}{0.383pt}}
\multiput(912.61,558.00)(0.447,0.909){3}{\rule{0.108pt}{0.767pt}}
\multiput(911.17,558.00)(3.000,3.409){2}{\rule{0.400pt}{0.383pt}}
\multiput(915.61,563.00)(0.447,1.132){3}{\rule{0.108pt}{0.900pt}}
\multiput(914.17,563.00)(3.000,4.132){2}{\rule{0.400pt}{0.450pt}}
\multiput(918.61,569.00)(0.447,0.909){3}{\rule{0.108pt}{0.767pt}}
\multiput(917.17,569.00)(3.000,3.409){2}{\rule{0.400pt}{0.383pt}}
\multiput(921.61,574.00)(0.447,1.132){3}{\rule{0.108pt}{0.900pt}}
\multiput(920.17,574.00)(3.000,4.132){2}{\rule{0.400pt}{0.450pt}}
\multiput(924.61,580.00)(0.447,0.909){3}{\rule{0.108pt}{0.767pt}}
\multiput(923.17,580.00)(3.000,3.409){2}{\rule{0.400pt}{0.383pt}}
\multiput(927.61,585.00)(0.447,1.132){3}{\rule{0.108pt}{0.900pt}}
\multiput(926.17,585.00)(3.000,4.132){2}{\rule{0.400pt}{0.450pt}}
\multiput(930.60,591.00)(0.468,0.774){5}{\rule{0.113pt}{0.700pt}}
\multiput(929.17,591.00)(4.000,4.547){2}{\rule{0.400pt}{0.350pt}}
\multiput(934.61,597.00)(0.447,1.132){3}{\rule{0.108pt}{0.900pt}}
\multiput(933.17,597.00)(3.000,4.132){2}{\rule{0.400pt}{0.450pt}}
\multiput(937.61,603.00)(0.447,1.132){3}{\rule{0.108pt}{0.900pt}}
\multiput(936.17,603.00)(3.000,4.132){2}{\rule{0.400pt}{0.450pt}}
\multiput(940.61,609.00)(0.447,1.132){3}{\rule{0.108pt}{0.900pt}}
\multiput(939.17,609.00)(3.000,4.132){2}{\rule{0.400pt}{0.450pt}}
\multiput(943.61,615.00)(0.447,1.132){3}{\rule{0.108pt}{0.900pt}}
\multiput(942.17,615.00)(3.000,4.132){2}{\rule{0.400pt}{0.450pt}}
\multiput(946.61,621.00)(0.447,1.132){3}{\rule{0.108pt}{0.900pt}}
\multiput(945.17,621.00)(3.000,4.132){2}{\rule{0.400pt}{0.450pt}}
\multiput(949.61,627.00)(0.447,1.132){3}{\rule{0.108pt}{0.900pt}}
\multiput(948.17,627.00)(3.000,4.132){2}{\rule{0.400pt}{0.450pt}}
\multiput(952.61,633.00)(0.447,1.355){3}{\rule{0.108pt}{1.033pt}}
\multiput(951.17,633.00)(3.000,4.855){2}{\rule{0.400pt}{0.517pt}}
\multiput(955.61,640.00)(0.447,1.132){3}{\rule{0.108pt}{0.900pt}}
\multiput(954.17,640.00)(3.000,4.132){2}{\rule{0.400pt}{0.450pt}}
\multiput(958.60,646.00)(0.468,0.920){5}{\rule{0.113pt}{0.800pt}}
\multiput(957.17,646.00)(4.000,5.340){2}{\rule{0.400pt}{0.400pt}}
\multiput(712.00,506.95)(0.685,-0.447){3}{\rule{0.633pt}{0.108pt}}
\multiput(712.00,507.17)(2.685,-3.000){2}{\rule{0.317pt}{0.400pt}}
\multiput(716.00,503.95)(0.462,-0.447){3}{\rule{0.500pt}{0.108pt}}
\multiput(716.00,504.17)(1.962,-3.000){2}{\rule{0.250pt}{0.400pt}}
\multiput(719.00,500.95)(0.462,-0.447){3}{\rule{0.500pt}{0.108pt}}
\multiput(719.00,501.17)(1.962,-3.000){2}{\rule{0.250pt}{0.400pt}}
\multiput(722.00,497.95)(0.462,-0.447){3}{\rule{0.500pt}{0.108pt}}
\multiput(722.00,498.17)(1.962,-3.000){2}{\rule{0.250pt}{0.400pt}}
\put(725,494.17){\rule{0.700pt}{0.400pt}}
\multiput(725.00,495.17)(1.547,-2.000){2}{\rule{0.350pt}{0.400pt}}
\multiput(728.00,492.95)(0.462,-0.447){3}{\rule{0.500pt}{0.108pt}}
\multiput(728.00,493.17)(1.962,-3.000){2}{\rule{0.250pt}{0.400pt}}
\put(731,489.17){\rule{0.700pt}{0.400pt}}
\multiput(731.00,490.17)(1.547,-2.000){2}{\rule{0.350pt}{0.400pt}}
\put(734,487.17){\rule{0.700pt}{0.400pt}}
\multiput(734.00,488.17)(1.547,-2.000){2}{\rule{0.350pt}{0.400pt}}
\multiput(737.00,485.95)(0.462,-0.447){3}{\rule{0.500pt}{0.108pt}}
\multiput(737.00,486.17)(1.962,-3.000){2}{\rule{0.250pt}{0.400pt}}
\put(740,482.17){\rule{0.900pt}{0.400pt}}
\multiput(740.00,483.17)(2.132,-2.000){2}{\rule{0.450pt}{0.400pt}}
\put(744,480.17){\rule{0.700pt}{0.400pt}}
\multiput(744.00,481.17)(1.547,-2.000){2}{\rule{0.350pt}{0.400pt}}
\put(747,478.17){\rule{0.700pt}{0.400pt}}
\multiput(747.00,479.17)(1.547,-2.000){2}{\rule{0.350pt}{0.400pt}}
\put(750,476.17){\rule{0.700pt}{0.400pt}}
\multiput(750.00,477.17)(1.547,-2.000){2}{\rule{0.350pt}{0.400pt}}
\put(753,474.67){\rule{0.723pt}{0.400pt}}
\multiput(753.00,475.17)(1.500,-1.000){2}{\rule{0.361pt}{0.400pt}}
\put(756,473.17){\rule{0.700pt}{0.400pt}}
\multiput(756.00,474.17)(1.547,-2.000){2}{\rule{0.350pt}{0.400pt}}
\put(759,471.67){\rule{0.723pt}{0.400pt}}
\multiput(759.00,472.17)(1.500,-1.000){2}{\rule{0.361pt}{0.400pt}}
\put(762,470.17){\rule{0.700pt}{0.400pt}}
\multiput(762.00,471.17)(1.547,-2.000){2}{\rule{0.350pt}{0.400pt}}
\put(765,468.67){\rule{0.964pt}{0.400pt}}
\multiput(765.00,469.17)(2.000,-1.000){2}{\rule{0.482pt}{0.400pt}}
\put(769,467.67){\rule{0.723pt}{0.400pt}}
\multiput(769.00,468.17)(1.500,-1.000){2}{\rule{0.361pt}{0.400pt}}
\put(772,466.17){\rule{0.700pt}{0.400pt}}
\multiput(772.00,467.17)(1.547,-2.000){2}{\rule{0.350pt}{0.400pt}}
\put(775,464.67){\rule{0.723pt}{0.400pt}}
\multiput(775.00,465.17)(1.500,-1.000){2}{\rule{0.361pt}{0.400pt}}
\put(778.0,429.0){\rule[-0.200pt]{0.723pt}{0.400pt}}
\put(781,463.67){\rule{0.723pt}{0.400pt}}
\multiput(781.00,464.17)(1.500,-1.000){2}{\rule{0.361pt}{0.400pt}}
\put(784,462.67){\rule{0.723pt}{0.400pt}}
\multiput(784.00,463.17)(1.500,-1.000){2}{\rule{0.361pt}{0.400pt}}
\put(787,461.67){\rule{0.723pt}{0.400pt}}
\multiput(787.00,462.17)(1.500,-1.000){2}{\rule{0.361pt}{0.400pt}}
\put(778.0,465.0){\rule[-0.200pt]{0.723pt}{0.400pt}}
\put(793,460.67){\rule{0.964pt}{0.400pt}}
\multiput(793.00,461.17)(2.000,-1.000){2}{\rule{0.482pt}{0.400pt}}
\put(790.0,462.0){\rule[-0.200pt]{0.723pt}{0.400pt}}
\put(797.0,461.0){\rule[-0.200pt]{0.723pt}{0.400pt}}
\put(800.0,461.0){\rule[-0.200pt]{0.723pt}{0.400pt}}
\put(803.0,461.0){\rule[-0.200pt]{0.723pt}{0.400pt}}
\put(806.0,461.0){\rule[-0.200pt]{0.723pt}{0.400pt}}
\put(809.0,461.0){\rule[-0.200pt]{0.723pt}{0.400pt}}
\put(812.0,461.0){\rule[-0.200pt]{0.723pt}{0.400pt}}
\put(815.0,461.0){\rule[-0.200pt]{0.723pt}{0.400pt}}
\put(821,460.67){\rule{0.964pt}{0.400pt}}
\multiput(821.00,460.17)(2.000,1.000){2}{\rule{0.482pt}{0.400pt}}
\put(818.0,461.0){\rule[-0.200pt]{0.723pt}{0.400pt}}
\put(828,461.67){\rule{0.723pt}{0.400pt}}
\multiput(828.00,461.17)(1.500,1.000){2}{\rule{0.361pt}{0.400pt}}
\put(831,462.67){\rule{0.723pt}{0.400pt}}
\multiput(831.00,462.17)(1.500,1.000){2}{\rule{0.361pt}{0.400pt}}
\put(834,463.67){\rule{0.723pt}{0.400pt}}
\multiput(834.00,463.17)(1.500,1.000){2}{\rule{0.361pt}{0.400pt}}
\put(837,464.67){\rule{0.723pt}{0.400pt}}
\multiput(837.00,464.17)(1.500,1.000){2}{\rule{0.361pt}{0.400pt}}
\put(840,465.67){\rule{0.723pt}{0.400pt}}
\multiput(840.00,465.17)(1.500,1.000){2}{\rule{0.361pt}{0.400pt}}
\put(843,466.67){\rule{0.723pt}{0.400pt}}
\multiput(843.00,466.17)(1.500,1.000){2}{\rule{0.361pt}{0.400pt}}
\put(846,467.67){\rule{0.964pt}{0.400pt}}
\multiput(846.00,467.17)(2.000,1.000){2}{\rule{0.482pt}{0.400pt}}
\put(850,468.67){\rule{0.723pt}{0.400pt}}
\multiput(850.00,468.17)(1.500,1.000){2}{\rule{0.361pt}{0.400pt}}
\put(853,470.17){\rule{0.700pt}{0.400pt}}
\multiput(853.00,469.17)(1.547,2.000){2}{\rule{0.350pt}{0.400pt}}
\put(856,471.67){\rule{0.723pt}{0.400pt}}
\multiput(856.00,471.17)(1.500,1.000){2}{\rule{0.361pt}{0.400pt}}
\put(859,473.17){\rule{0.700pt}{0.400pt}}
\multiput(859.00,472.17)(1.547,2.000){2}{\rule{0.350pt}{0.400pt}}
\put(862,475.17){\rule{0.700pt}{0.400pt}}
\multiput(862.00,474.17)(1.547,2.000){2}{\rule{0.350pt}{0.400pt}}
\put(865,477.17){\rule{0.700pt}{0.400pt}}
\multiput(865.00,476.17)(1.547,2.000){2}{\rule{0.350pt}{0.400pt}}
\put(868,478.67){\rule{0.723pt}{0.400pt}}
\multiput(868.00,478.17)(1.500,1.000){2}{\rule{0.361pt}{0.400pt}}
\put(871,480.17){\rule{0.700pt}{0.400pt}}
\multiput(871.00,479.17)(1.547,2.000){2}{\rule{0.350pt}{0.400pt}}
\multiput(874.00,482.61)(0.685,0.447){3}{\rule{0.633pt}{0.108pt}}
\multiput(874.00,481.17)(2.685,3.000){2}{\rule{0.317pt}{0.400pt}}
\put(878,485.17){\rule{0.700pt}{0.400pt}}
\multiput(878.00,484.17)(1.547,2.000){2}{\rule{0.350pt}{0.400pt}}
\put(881,487.17){\rule{0.700pt}{0.400pt}}
\multiput(881.00,486.17)(1.547,2.000){2}{\rule{0.350pt}{0.400pt}}
\multiput(884.00,489.61)(0.462,0.447){3}{\rule{0.500pt}{0.108pt}}
\multiput(884.00,488.17)(1.962,3.000){2}{\rule{0.250pt}{0.400pt}}
\put(887,492.17){\rule{0.700pt}{0.400pt}}
\multiput(887.00,491.17)(1.547,2.000){2}{\rule{0.350pt}{0.400pt}}
\multiput(890.00,494.61)(0.462,0.447){3}{\rule{0.500pt}{0.108pt}}
\multiput(890.00,493.17)(1.962,3.000){2}{\rule{0.250pt}{0.400pt}}
\put(893,497.17){\rule{0.700pt}{0.400pt}}
\multiput(893.00,496.17)(1.547,2.000){2}{\rule{0.350pt}{0.400pt}}
\multiput(896.00,499.61)(0.462,0.447){3}{\rule{0.500pt}{0.108pt}}
\multiput(896.00,498.17)(1.962,3.000){2}{\rule{0.250pt}{0.400pt}}
\multiput(899.00,502.61)(0.685,0.447){3}{\rule{0.633pt}{0.108pt}}
\multiput(899.00,501.17)(2.685,3.000){2}{\rule{0.317pt}{0.400pt}}
\multiput(903.00,505.61)(0.462,0.447){3}{\rule{0.500pt}{0.108pt}}
\multiput(903.00,504.17)(1.962,3.000){2}{\rule{0.250pt}{0.400pt}}
\multiput(906.00,508.61)(0.462,0.447){3}{\rule{0.500pt}{0.108pt}}
\multiput(906.00,507.17)(1.962,3.000){2}{\rule{0.250pt}{0.400pt}}
\multiput(909.61,511.00)(0.447,0.685){3}{\rule{0.108pt}{0.633pt}}
\multiput(908.17,511.00)(3.000,2.685){2}{\rule{0.400pt}{0.317pt}}
\multiput(912.00,515.61)(0.462,0.447){3}{\rule{0.500pt}{0.108pt}}
\multiput(912.00,514.17)(1.962,3.000){2}{\rule{0.250pt}{0.400pt}}
\multiput(915.00,518.61)(0.462,0.447){3}{\rule{0.500pt}{0.108pt}}
\multiput(915.00,517.17)(1.962,3.000){2}{\rule{0.250pt}{0.400pt}}
\multiput(918.61,521.00)(0.447,0.685){3}{\rule{0.108pt}{0.633pt}}
\multiput(917.17,521.00)(3.000,2.685){2}{\rule{0.400pt}{0.317pt}}
\multiput(921.00,525.61)(0.462,0.447){3}{\rule{0.500pt}{0.108pt}}
\multiput(921.00,524.17)(1.962,3.000){2}{\rule{0.250pt}{0.400pt}}
\multiput(924.61,528.00)(0.447,0.685){3}{\rule{0.108pt}{0.633pt}}
\multiput(923.17,528.00)(3.000,2.685){2}{\rule{0.400pt}{0.317pt}}
\multiput(927.00,532.60)(0.481,0.468){5}{\rule{0.500pt}{0.113pt}}
\multiput(927.00,531.17)(2.962,4.000){2}{\rule{0.250pt}{0.400pt}}
\multiput(931.61,536.00)(0.447,0.685){3}{\rule{0.108pt}{0.633pt}}
\multiput(930.17,536.00)(3.000,2.685){2}{\rule{0.400pt}{0.317pt}}
\multiput(934.61,540.00)(0.447,0.685){3}{\rule{0.108pt}{0.633pt}}
\multiput(933.17,540.00)(3.000,2.685){2}{\rule{0.400pt}{0.317pt}}
\multiput(937.61,544.00)(0.447,0.685){3}{\rule{0.108pt}{0.633pt}}
\multiput(936.17,544.00)(3.000,2.685){2}{\rule{0.400pt}{0.317pt}}
\multiput(940.61,548.00)(0.447,0.685){3}{\rule{0.108pt}{0.633pt}}
\multiput(939.17,548.00)(3.000,2.685){2}{\rule{0.400pt}{0.317pt}}
\multiput(943.61,552.00)(0.447,0.685){3}{\rule{0.108pt}{0.633pt}}
\multiput(942.17,552.00)(3.000,2.685){2}{\rule{0.400pt}{0.317pt}}
\multiput(946.61,556.00)(0.447,0.909){3}{\rule{0.108pt}{0.767pt}}
\multiput(945.17,556.00)(3.000,3.409){2}{\rule{0.400pt}{0.383pt}}
\multiput(949.61,561.00)(0.447,0.685){3}{\rule{0.108pt}{0.633pt}}
\multiput(948.17,561.00)(3.000,2.685){2}{\rule{0.400pt}{0.317pt}}
\multiput(952.61,565.00)(0.447,0.909){3}{\rule{0.108pt}{0.767pt}}
\multiput(951.17,565.00)(3.000,3.409){2}{\rule{0.400pt}{0.383pt}}
\multiput(955.60,570.00)(0.468,0.627){5}{\rule{0.113pt}{0.600pt}}
\multiput(954.17,570.00)(4.000,3.755){2}{\rule{0.400pt}{0.300pt}}
\multiput(959.61,575.00)(0.447,0.685){3}{\rule{0.108pt}{0.633pt}}
\multiput(958.17,575.00)(3.000,2.685){2}{\rule{0.400pt}{0.317pt}}
\multiput(962.61,579.00)(0.447,0.909){3}{\rule{0.108pt}{0.767pt}}
\multiput(961.17,579.00)(3.000,3.409){2}{\rule{0.400pt}{0.383pt}}
\multiput(965.61,584.00)(0.447,0.909){3}{\rule{0.108pt}{0.767pt}}
\multiput(964.17,584.00)(3.000,3.409){2}{\rule{0.400pt}{0.383pt}}
\multiput(968.61,589.00)(0.447,0.909){3}{\rule{0.108pt}{0.767pt}}
\multiput(967.17,589.00)(3.000,3.409){2}{\rule{0.400pt}{0.383pt}}
\multiput(971.61,594.00)(0.447,1.132){3}{\rule{0.108pt}{0.900pt}}
\multiput(970.17,594.00)(3.000,4.132){2}{\rule{0.400pt}{0.450pt}}
\multiput(974.61,600.00)(0.447,0.909){3}{\rule{0.108pt}{0.767pt}}
\multiput(973.17,600.00)(3.000,3.409){2}{\rule{0.400pt}{0.383pt}}
\multiput(977.61,605.00)(0.447,0.909){3}{\rule{0.108pt}{0.767pt}}
\multiput(976.17,605.00)(3.000,3.409){2}{\rule{0.400pt}{0.383pt}}
\multiput(980.60,610.00)(0.468,0.774){5}{\rule{0.113pt}{0.700pt}}
\multiput(979.17,610.00)(4.000,4.547){2}{\rule{0.400pt}{0.350pt}}
\multiput(984.61,616.00)(0.447,0.909){3}{\rule{0.108pt}{0.767pt}}
\multiput(983.17,616.00)(3.000,3.409){2}{\rule{0.400pt}{0.383pt}}
\multiput(987.61,621.00)(0.447,1.132){3}{\rule{0.108pt}{0.900pt}}
\multiput(986.17,621.00)(3.000,4.132){2}{\rule{0.400pt}{0.450pt}}
\multiput(990.61,627.00)(0.447,1.132){3}{\rule{0.108pt}{0.900pt}}
\multiput(989.17,627.00)(3.000,4.132){2}{\rule{0.400pt}{0.450pt}}
\multiput(993.61,633.00)(0.447,1.132){3}{\rule{0.108pt}{0.900pt}}
\multiput(992.17,633.00)(3.000,4.132){2}{\rule{0.400pt}{0.450pt}}
\multiput(996.61,639.00)(0.447,1.132){3}{\rule{0.108pt}{0.900pt}}
\multiput(995.17,639.00)(3.000,4.132){2}{\rule{0.400pt}{0.450pt}}
\multiput(999.61,645.00)(0.447,1.132){3}{\rule{0.108pt}{0.900pt}}
\multiput(998.17,645.00)(3.000,4.132){2}{\rule{0.400pt}{0.450pt}}
\multiput(1002.61,651.00)(0.447,1.132){3}{\rule{0.108pt}{0.900pt}}
\multiput(1001.17,651.00)(3.000,4.132){2}{\rule{0.400pt}{0.450pt}}
\multiput(1005.61,657.00)(0.447,1.132){3}{\rule{0.108pt}{0.900pt}}
\multiput(1004.17,657.00)(3.000,4.132){2}{\rule{0.400pt}{0.450pt}}
\multiput(1008.60,663.00)(0.468,0.920){5}{\rule{0.113pt}{0.800pt}}
\multiput(1007.17,663.00)(4.000,5.340){2}{\rule{0.400pt}{0.400pt}}
\multiput(1012.61,670.00)(0.447,1.132){3}{\rule{0.108pt}{0.900pt}}
\multiput(1011.17,670.00)(3.000,4.132){2}{\rule{0.400pt}{0.450pt}}
\multiput(1015.61,676.00)(0.447,1.355){3}{\rule{0.108pt}{1.033pt}}
\multiput(1014.17,676.00)(3.000,4.855){2}{\rule{0.400pt}{0.517pt}}
\multiput(1018.61,683.00)(0.447,1.132){3}{\rule{0.108pt}{0.900pt}}
\multiput(1017.17,683.00)(3.000,4.132){2}{\rule{0.400pt}{0.450pt}}
\put(825.0,462.0){\rule[-0.200pt]{0.723pt}{0.400pt}}
\multiput(1017.75,325.92)(-0.854,-0.500){359}{\rule{0.783pt}{0.120pt}}
\multiput(1019.38,326.17)(-307.375,-181.000){2}{\rule{0.391pt}{0.400pt}}
\multiput(179.00,249.92)(2.544,-0.499){207}{\rule{2.130pt}{0.120pt}}
\multiput(179.00,250.17)(528.578,-105.000){2}{\rule{1.065pt}{0.400pt}}
\put(712.0,146.0){\rule[-0.200pt]{0.400pt}{87.206pt}}
\end{picture}

\caption{Plot of the cities of the Basin function in 2-dimensions.}
\label{plot:basin2}
\end{figure}

\begin{lstlisting}[caption=Basin Function in 2-dimensions, label=basin_2d]
set xrange [-5:5]
set yrange [-5:5]
splot x*x + y*y
\end{lstlisting}


% Traveling Salesman Problem
\subsection{Traveling Salesman Problem}
words

Figure~\ref{plot:tsp1} provides a plot of the Berlin52 Traveling Salesman problem used through out the algorithm description in the Clever Algorithms. Listing~\ref{tsp1} provides the gnuplot script used to prepare the plot, where \texttt{berlin52.tsp} is a file that contains a listing of the coordinates of all cities, one city per line separated by white space.

Shows cities clumped...

\begin{figure}[htp]
% GNUPLOT: LaTeX picture
\setlength{\unitlength}{0.240900pt}
\ifx\plotpoint\undefined\newsavebox{\plotpoint}\fi
\sbox{\plotpoint}{\rule[-0.200pt]{0.400pt}{0.400pt}}%
\begin{picture}(1200,900)(0,0)
\sbox{\plotpoint}{\rule[-0.200pt]{0.400pt}{0.400pt}}%
\put(190.0,82.0){\rule[-0.200pt]{4.818pt}{0.400pt}}
\put(170,82){\makebox(0,0)[r]{ 0}}
\put(1130.0,82.0){\rule[-0.200pt]{4.818pt}{0.400pt}}
\put(190.0,212.0){\rule[-0.200pt]{4.818pt}{0.400pt}}
\put(170,212){\makebox(0,0)[r]{ 200}}
\put(1130.0,212.0){\rule[-0.200pt]{4.818pt}{0.400pt}}
\put(190.0,341.0){\rule[-0.200pt]{4.818pt}{0.400pt}}
\put(170,341){\makebox(0,0)[r]{ 400}}
\put(1130.0,341.0){\rule[-0.200pt]{4.818pt}{0.400pt}}
\put(190.0,471.0){\rule[-0.200pt]{4.818pt}{0.400pt}}
\put(170,471){\makebox(0,0)[r]{ 600}}
\put(1130.0,471.0){\rule[-0.200pt]{4.818pt}{0.400pt}}
\put(190.0,601.0){\rule[-0.200pt]{4.818pt}{0.400pt}}
\put(170,601){\makebox(0,0)[r]{ 800}}
\put(1130.0,601.0){\rule[-0.200pt]{4.818pt}{0.400pt}}
\put(190.0,730.0){\rule[-0.200pt]{4.818pt}{0.400pt}}
\put(170,730){\makebox(0,0)[r]{ 1000}}
\put(1130.0,730.0){\rule[-0.200pt]{4.818pt}{0.400pt}}
\put(190.0,860.0){\rule[-0.200pt]{4.818pt}{0.400pt}}
\put(170,860){\makebox(0,0)[r]{ 1200}}
\put(1130.0,860.0){\rule[-0.200pt]{4.818pt}{0.400pt}}
\put(190.0,82.0){\rule[-0.200pt]{0.400pt}{4.818pt}}
\put(190,41){\makebox(0,0){ 0}}
\put(190.0,840.0){\rule[-0.200pt]{0.400pt}{4.818pt}}
\put(297.0,82.0){\rule[-0.200pt]{0.400pt}{4.818pt}}
\put(297,41){\makebox(0,0){ 200}}
\put(297.0,840.0){\rule[-0.200pt]{0.400pt}{4.818pt}}
\put(403.0,82.0){\rule[-0.200pt]{0.400pt}{4.818pt}}
\put(403,41){\makebox(0,0){ 400}}
\put(403.0,840.0){\rule[-0.200pt]{0.400pt}{4.818pt}}
\put(510.0,82.0){\rule[-0.200pt]{0.400pt}{4.818pt}}
\put(510,41){\makebox(0,0){ 600}}
\put(510.0,840.0){\rule[-0.200pt]{0.400pt}{4.818pt}}
\put(617.0,82.0){\rule[-0.200pt]{0.400pt}{4.818pt}}
\put(617,41){\makebox(0,0){ 800}}
\put(617.0,840.0){\rule[-0.200pt]{0.400pt}{4.818pt}}
\put(723.0,82.0){\rule[-0.200pt]{0.400pt}{4.818pt}}
\put(723,41){\makebox(0,0){ 1000}}
\put(723.0,840.0){\rule[-0.200pt]{0.400pt}{4.818pt}}
\put(830.0,82.0){\rule[-0.200pt]{0.400pt}{4.818pt}}
\put(830,41){\makebox(0,0){ 1200}}
\put(830.0,840.0){\rule[-0.200pt]{0.400pt}{4.818pt}}
\put(937.0,82.0){\rule[-0.200pt]{0.400pt}{4.818pt}}
\put(937,41){\makebox(0,0){ 1400}}
\put(937.0,840.0){\rule[-0.200pt]{0.400pt}{4.818pt}}
\put(1043.0,82.0){\rule[-0.200pt]{0.400pt}{4.818pt}}
\put(1043,41){\makebox(0,0){ 1600}}
\put(1043.0,840.0){\rule[-0.200pt]{0.400pt}{4.818pt}}
\put(1150.0,82.0){\rule[-0.200pt]{0.400pt}{4.818pt}}
\put(1150,41){\makebox(0,0){ 1800}}
\put(1150.0,840.0){\rule[-0.200pt]{0.400pt}{4.818pt}}
\put(190.0,82.0){\rule[-0.200pt]{0.400pt}{187.420pt}}
\put(190.0,82.0){\rule[-0.200pt]{231.264pt}{0.400pt}}
\put(1150.0,82.0){\rule[-0.200pt]{0.400pt}{187.420pt}}
\put(190.0,860.0){\rule[-0.200pt]{231.264pt}{0.400pt}}
\put(491,455){\raisebox{-.8pt}{\makebox(0,0){$\Diamond$}}}
\put(203,202){\raisebox{-.8pt}{\makebox(0,0){$\Diamond$}}}
\put(374,568){\raisebox{-.8pt}{\makebox(0,0){$\Diamond$}}}
\put(694,526){\raisebox{-.8pt}{\makebox(0,0){$\Diamond$}}}
\put(641,507){\raisebox{-.8pt}{\makebox(0,0){$\Diamond$}}}
\put(659,510){\raisebox{-.8pt}{\makebox(0,0){$\Diamond$}}}
\put(203,231){\raisebox{-.8pt}{\makebox(0,0){$\Diamond$}}}
\put(470,730){\raisebox{-.8pt}{\makebox(0,0){$\Diamond$}}}
\put(499,844){\raisebox{-.8pt}{\makebox(0,0){$\Diamond$}}}
\put(537,815){\raisebox{-.8pt}{\makebox(0,0){$\Diamond$}}}
\put(1046,484){\raisebox{-.8pt}{\makebox(0,0){$\Diamond$}}}
\put(841,458){\raisebox{-.8pt}{\makebox(0,0){$\Diamond$}}}
\put(971,212){\raisebox{-.8pt}{\makebox(0,0){$\Diamond$}}}
\put(1006,85){\raisebox{-.8pt}{\makebox(0,0){$\Diamond$}}}
\put(641,523){\raisebox{-.8pt}{\makebox(0,0){$\Diamond$}}}
\put(577,322){\raisebox{-.8pt}{\makebox(0,0){$\Diamond$}}}
\put(267,513){\raisebox{-.8pt}{\makebox(0,0){$\Diamond$}}}
\put(411,494){\raisebox{-.8pt}{\makebox(0,0){$\Diamond$}}}
\put(462,649){\raisebox{-.8pt}{\makebox(0,0){$\Diamond$}}}
\put(489,319){\raisebox{-.8pt}{\makebox(0,0){$\Diamond$}}}
\put(350,383){\raisebox{-.8pt}{\makebox(0,0){$\Diamond$}}}
\put(467,461){\raisebox{-.8pt}{\makebox(0,0){$\Diamond$}}}
\put(446,351){\raisebox{-.8pt}{\makebox(0,0){$\Diamond$}}}
\put(635,487){\raisebox{-.8pt}{\makebox(0,0){$\Diamond$}}}
\put(710,458){\raisebox{-.8pt}{\makebox(0,0){$\Diamond$}}}
\put(838,241){\raisebox{-.8pt}{\makebox(0,0){$\Diamond$}}}
\put(894,286){\raisebox{-.8pt}{\makebox(0,0){$\Diamond$}}}
\put(857,341){\raisebox{-.8pt}{\makebox(0,0){$\Diamond$}}}
\put(542,199){\raisebox{-.8pt}{\makebox(0,0){$\Diamond$}}}
\put(409,244){\raisebox{-.8pt}{\makebox(0,0){$\Diamond$}}}
\put(414,442){\raisebox{-.8pt}{\makebox(0,0){$\Diamond$}}}
\put(497,513){\raisebox{-.8pt}{\makebox(0,0){$\Diamond$}}}
\put(803,834){\raisebox{-.8pt}{\makebox(0,0){$\Diamond$}}}
\put(563,458){\raisebox{-.8pt}{\makebox(0,0){$\Diamond$}}}
\put(555,468){\raisebox{-.8pt}{\makebox(0,0){$\Diamond$}}}
\put(555,477){\raisebox{-.8pt}{\makebox(0,0){$\Diamond$}}}
\put(601,477){\raisebox{-.8pt}{\makebox(0,0){$\Diamond$}}}
\put(614,500){\raisebox{-.8pt}{\makebox(0,0){$\Diamond$}}}
\put(574,494){\raisebox{-.8pt}{\makebox(0,0){$\Diamond$}}}
\put(595,503){\raisebox{-.8pt}{\makebox(0,0){$\Diamond$}}}
\put(443,704){\raisebox{-.8pt}{\makebox(0,0){$\Diamond$}}}
\put(241,251){\raisebox{-.8pt}{\makebox(0,0){$\Diamond$}}}
\put(657,678){\raisebox{-.8pt}{\makebox(0,0){$\Diamond$}}}
\put(563,406){\raisebox{-.8pt}{\makebox(0,0){$\Diamond$}}}
\put(486,610){\raisebox{-.8pt}{\makebox(0,0){$\Diamond$}}}
\put(633,396){\raisebox{-.8pt}{\makebox(0,0){$\Diamond$}}}
\put(814,124){\raisebox{-.8pt}{\makebox(0,0){$\Diamond$}}}
\put(633,477){\raisebox{-.8pt}{\makebox(0,0){$\Diamond$}}}
\put(513,487){\raisebox{-.8pt}{\makebox(0,0){$\Diamond$}}}
\put(507,315){\raisebox{-.8pt}{\makebox(0,0){$\Diamond$}}}
\put(905,552){\raisebox{-.8pt}{\makebox(0,0){$\Diamond$}}}
\put(1118,241){\raisebox{-.8pt}{\makebox(0,0){$\Diamond$}}}
\put(190.0,82.0){\rule[-0.200pt]{0.400pt}{187.420pt}}
\put(190.0,82.0){\rule[-0.200pt]{231.264pt}{0.400pt}}
\put(1150.0,82.0){\rule[-0.200pt]{0.400pt}{187.420pt}}
\put(190.0,860.0){\rule[-0.200pt]{231.264pt}{0.400pt}}
\end{picture}

\caption{Plot of the cities of the Berlin52 Traveling Salesman Problem.}
\label{plot:tsp1}
\end{figure}

\begin{lstlisting}[caption=Basin Function in 2-dimensions, label=tsp1]
plot "berlin52.tsp"
\end{lstlisting}

%
% Performance
%
\section{Performance}
easy, first thing

\subsection{Single Algorithm Run}
line graphs

TODO: give example with a GA

\subsection{Multiple Algorithm Run}
same algorithm, different random seed

compare using boxplots generally

TODO: give example with a GA, ES, EP, ...


%
% Candidate Solutions
%
\section{Candidate Solutions}
asdasd

problem specific

\subsection{Continuous Function Optimization}
todo
on the plot

\subsection{Traveling Salesman Problem}
todo

%
% Tools
%
\section{Visualization Methods}
asdasd

\subsection{Tools}
first step should be to use a specailized tool if available

\subsection{GNU Plot}
focused on plotting

great for performance graphics and surfaces

\subsection{R Project}
stats methods and visualization


\subsection{Libraries}
write code if you must, but use a lib

may, for example, some used in java, ruby etc

open source physics, jfreegraph, jung, etc


\subsection{DIY}
last resort, better to use a lib


%
% Conclusions
%
\section{Conclusions}
\label{sec:conclusions}
This report provided a description and examples of visualization as a form of exploration and weak algorithm testing. 

% bibliography
\bibliographystyle{plain}
\bibliography{../bibtex}

\end{document}
% EOF
