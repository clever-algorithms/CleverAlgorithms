% The Clever Algorithms Project: http://www.CleverAlgorithms.com
% (c) Copyright 2010 Jason Brownlee. Some Rights Reserved.
% This work is licensed under a Creative Commons Attribution-Noncommercial-Share Alike 2.5 Australia License.

% This is a chapter

\renewcommand{\bibsection}{\subsection{\bibname}}
\begin{bibunit}

\chapter{Swarm Algorithms}
\label{ch:swarm}
\index{Swarm Algorithms}
\index{Swarm Intelligence}
\index{Collective Intelligence}
\index{Ant Colony Optimization}
\index{Particle Swarm Optimization}

\section{Overview}
This chapter describes Swarm Algorithms.

\subsection{Swarm Intelligence}
Swarm intelligence is the study of computational systems inspired by the `collective intelligence'. Collective Intelligence emerges through the cooperation of large numbers of homogeneous agents in the environment. Examples include schools of fish, flocks of birds, and colonies of ants. Such intelligence is decentralized, self-organizing and distributed through out an environment. In nature such systems are commonly used to solve problems such as effective foraging for food, prey evading, or colony re-location. The information is typically stored throughout the participating homogeneous agents, or is stored or communicated in the environment itself such as through the use of pheromones in ants, dancing in bees, and proximity in fish and birds.

The paradigm consists of two dominant sub-fields 1) \emph{Ant Colony Optimization} that investigates probabilistic algorithms inspired by the stigmergy and foraging behavior of ants, and 2) \emph{Particle Swarm Optimization} that investigates probabilistic algorithms inspired by the flocking, schooling and herding. Like evolutionary computation, swarm intelligence `algorithms' or `strategies' are considered adaptive strategies and are typically applied to search and optimization domains.

% References
\subsection{References}
% classical
Seminal books on the field of Swarm Intelligence include ``\emph{Swarm Intelligence}'' by Kennedy, Eberhart and Shi \cite{Kennedy2001}, and ``\emph{Swarm Intelligence: From Natural to Artificial Systems}'' by Bonabeau, Dorigo, and Theraulaz \cite{Bonabeau1999}. Another excellent text book on the area is ``\emph{Fundamentals of Computational Swarm Intelligence}'' by Engelbrecht \cite{Engelbrecht2006}. The seminal book reference for the field of Ant Colony Optimization is ``\emph{Ant Colony Optimization}'' by Dorigo and St\"utzle \cite{Dorigo2004}.

%
% Extensions
%
\subsection{Extensions}
There are many other algorithms and classes of algorithm that were not described from the field of Swarm Intelligence, not limited to:

\begin{itemize}
	\item \textbf{Ant Algorithms}: such as Max-Min Ant Systems \cite{Stutzle2000} Rank-Based Ant Systems \cite{Bullnheimer1999}, Elitist Ant Systems \cite{Dorigo1996}, Hyper Cube Ant Colony Optimization \cite{Blum2001} Approximate Nondeterministic Tree-Search (ANTS) \cite{Maniezzo1999} and Multiple Ant Colony System \cite{Gambardella1999}.
	\item \textbf{Bee Algorithms}: such as Bee System and Bee Colony Optimization \cite{Lucic2001}, the Honey Bee Algorithm \cite{Tovey2004}, and Artificial Bee Colony Optimization \cite{Karaboga2005, Basturk2006}.
	\item \textbf{Other Social Insects}: algorithms inspired by other social insects besides ants and bees, such as the Firefly Algorithm \cite{Yang2008} and the Wasp Swarm Algorithm \cite{Pinto2007}.
	\item \textbf{Extensions to Particle Swarm}: such as Repulsive Particle Swarm Optimization \cite{Urfalioglu2004}.
	\item \textbf{Bacteria Algorithms}: such as the Bacteria Chemotaxis Algorithm \cite{Muller2002}.
\end{itemize}

\putbib
\end{bibunit}


\newpage\begin{bibunit}% The Clever Algorithms Project: http://www.CleverAlgorithms.com
% (c) Copyright 2010 Jason Brownlee. Some Rights Reserved. 
% This work is licensed under a Creative Commons Attribution-Noncommercial-Share Alike 2.5 Australia License.

% This is an algorithm description, see:
% Jason Brownlee. A Template for Standardized Algorithm Descriptions. Technical Report CA-TR-20100107-1, The Clever Algorithms Project http://www.CleverAlgorithms.com, January 2010.

% Name
% The algorithm name defines the canonical name used to refer to the technique, in addition to common aliases, abbreviations, and acronyms. The name is used in terms of the heading and sub-headings of an algorithm description.
\section{Particle Swarm Optimization} 
\label{sec:pso}
\index{Particle Swarm Optimization}

% other names
% What is the canonical name and common aliases for a technique?
% What are the common abbreviations and acronyms for a technique?
\emph{Particle Swarm Optimization, PSO.}

% Taxonomy: Lineage and locality
% The algorithm taxonomy defines where a techniques fits into the field, both the specific subfields of Computational Intelligence and Biologically Inspired Computation as well as the broader field of Artificial Intelligence. The taxonomy also provides a context for determining the relation- ships between algorithms. The taxonomy may be described in terms of a series of relationship statements or pictorially as a venn diagram or a graph with hierarchical structure.
\subsection{Taxonomy}
% To what fields of study does a technique belong?
Particle Swarm Optimization belongs to the field of Swarm Intelligence and Collective Intelligence and is a sub-field of Computational Intelligence.
% What are the closely related approaches to a technique?
Particle Swarm Optimization is related to other Swarm Intelligence algorithms such as Ant Colony Optimization and it is a baseline algorithm for many variations, too numerous to list.

% Inspiration: Motivating system
% The inspiration describes the specific system or process that provoked the inception of the algorithm. The inspiring system may non-exclusively be natural, biological, physical, or social. The description of the inspiring system may include relevant domain specific theory, observation, nomenclature, and most important must include those salient attributes of the system that are somehow abstractly or conceptually manifest in the technique. The inspiration is described textually with citations and may include diagrams to highlight features and relationships within the inspiring system.
% Optional
\subsection{Inspiration}
% What is the system or process that motivated the development of a technique?
Particle Swarm Optimization is inspired by the social foraging behavior of some animals such as flocking behavior of birds and the schooling behavior of fish.
% Which features of the motivating system are relevant to a technique?

% Metaphor: Explanation via analogy
% The metaphor is a description of the technique in the context of the inspiring system or a different suitable system. The features of the technique are made apparent through an analogous description of the features of the inspiring system. The explanation through analogy is not expected to be literal scientific truth, rather the method is used as an allegorical communication tool. The inspiring system is not explicitly described, this is the role of the ‘inspiration’ element, which represents a loose dependency for this element. The explanation is textual and uses the nomenclature of the metaphorical system.
% Optional
\subsection{Metaphor}
% What is the explanation of a technique in the context of the inspiring system?
Particles in the swarm fly through an environment following the fitter members of the swarm and generally biasing their movement toward historically good areas of their environment.
% What are the functionalities inferred for a technique from the analogous inspiring system?

% Strategy: Problem solving plan
% The strategy is an abstract description of the computational model. The strategy describes the information processing actions a technique shall take in order to achieve an objective. The strategy provides a logical separation between a computational realization (procedure) and a analogous system (metaphor). A given problem solving strategy may be realized as one of a number specific algorithms or problem solving systems. The strategy description is textual using information processing and algorithmic terminology.
\subsection{Strategy}
% What is the information processing objective of a technique?
The goal of the algorithm is to have all the particles locate the optima in a multi-dimensional hyper-volume.
% What is a techniques plan of action?
This is achieved by assigning initially random positions to all particles in the space and small initial random velocities. The algorithm is executed like a simulation, advancing the position of each particle in turn based on its velocity, the best known global position in the problem space and the best position known to a particle. The objective function is sampled after each position update. Over time, through a combination of exploration and exploitation of known good positions in the search space, the particles cluster or converge together around an optima, or several optima.

% Procedure: Abstract computation
% The algorithmic procedure summarizes the specifics of realizing a strategy as a systemized and parameterized computation. It outlines how the algorithm is organized in terms of the data structures and representations. The procedure may be described in terms of software engineering and computer science artifacts such as Pseudocode, design diagrams, and relevant mathematical equations.
\subsection{Procedure}
% What are the data structures and representations used in a technique?
The Particle Swarm Optimization algorithm is comprised of a collection of particles that move around the search space influenced by their own best past location and the best past location of the whole swarm or a close neighbor. Each iteration a particle's velocity is updated using:

\begin{align*}
	v_{i}(t+1) = v_{i}(t) + 
	&\big( c_1 \times rand() \times (p_{i}^{best} - p_{i}(t)) \big) +  \\
	&\big( c_2 \times rand() \times (p_{gbest} - p_{i}(t)) \big)
\end{align*}

where $v_{i}(t+1)$ is the new velocity for the $i^{th}$ particle, $c_1$ and $c_2$ are the weighting coefficients for the personal best and global best positions respectively, $p_{i}(t)$ is the $i^{th}$ particle's position at time $t$, $p_{i}^{best}$ is the $i^{th}$ particle's best known position, and $p_{gbest}$ is the best position known to the swarm. The $rand()$ function generate a uniformly random variable $\in [0,1]$. Variants on this update equation consider best positions within a particles local neighborhood at time $t$.

A particle's position is updated using:

\begin{equation}
	p_{i}(t+1) = p_{i}(t) + v_{i}(t)
\end{equation}

% What is the computational recipe for a technique?
Algorithm~\ref{alg:pso} provides a pseudocode listing of the Particle Swarm Optimization algorithm for minimizing a cost function. 

\begin{algorithm}[ht]
	\SetLine  

	% data
	\SetKwData{GlobalBest}{$P_{g\_best}$}
	\SetKwData{ProblemSize}{ProblemSize}
	\SetKwData{Population}{Population}
	\SetKwData{PopulationSize}{$Population_{size}$}
	\SetKwData{Particle}{$P$}
	\SetKwData{CurrentBest}{$P_{p\_best}$}
	\SetKwData{CurrentPosition}{$P_{position}$}
	\SetKwData{CurrentVelocity}{$P_{velocity}$}
	% functions
	\SetKwFunction{Cost}{Cost}
	\SetKwFunction{StopCondition}{StopCondition}
	\SetKwFunction{RandomPosition}{RandomPosition}
	\SetKwFunction{RandomVelocity}{RandomVelocity}	
	\SetKwFunction{UpdatePosition}{UpdatePosition}
	\SetKwFunction{UpdateVelocity}{UpdateVelocity}
	
	% I/O
	\KwIn{\ProblemSize, \PopulationSize}		
	\KwOut{\GlobalBest}
  % Algorithm

	% initialize	
	\Population $\leftarrow$ $\emptyset$\;
	\GlobalBest $\leftarrow$ $\emptyset$\;
	\For{$i=1$ $\KwTo$ \PopulationSize} {
		\CurrentVelocity $\leftarrow$ \RandomVelocity{}\;
		\CurrentPosition $\leftarrow$ \RandomPosition{\PopulationSize}\;
		\CurrentBest $\leftarrow$ \CurrentPosition\;
		\If{\Cost{\CurrentBest} $\leq$ \Cost{\GlobalBest}} {
			\GlobalBest $\leftarrow$ \CurrentBest\;
		}
	}	

	% loop
	\While{$\neg$\StopCondition{}} {
		\ForEach{\Particle $\in$ \Population} {
			\CurrentVelocity $\leftarrow$ \UpdateVelocity{\CurrentVelocity, \GlobalBest, \CurrentBest}\;
			\CurrentPosition $\leftarrow$ \UpdatePosition{\CurrentPosition, \CurrentVelocity}\;
			\If{\Cost{\CurrentPosition} $\leq$ \Cost{\CurrentBest}} {
				\CurrentBest $\leftarrow$ \CurrentPosition\;
				\If{\Cost{\CurrentBest} $\leq$ \Cost{\GlobalBest}} {
					\GlobalBest $\leftarrow$ \CurrentBest\;
				}
			}
		}		
	}
	\Return{\GlobalBest}\;
	% end
	\caption{Pseudocode for PSO.}
	\label{alg:pso}
\end{algorithm}

% Heuristics: Usage guidelines
% The heuristics element describe the commonsense, best practice, and demonstrated rules for applying and configuring a parameterized algorithm. The heuristics relate to the technical details of the techniques procedure and data structures for general classes of application (neither specific implementations not specific problem instances). The heuristics are described textually, such as a series of guidelines in a bullet-point structure.
\subsection{Heuristics}
% What are the suggested configurations for a technique?
% What are the guidelines for the application of a technique to a problem instance?
\begin{itemize}
	\item The number of particles should be low, around 20-40
	\item The speed a particle can move (maximum change in its position per iteration) should be bounded, such as to a percentage of the size of the domain.
	\item The learning factors (biases towards global and personal best positions) should be between 0 and 4, typically 2.
	\item A local bias (local neighborhood) factor can be introduced where neighbors are determined based on Euclidean distance between particle positions.
	\item Particles may leave the boundary of the problem space and may be penalized, be reflected back into the domain or biased to return back toward a position in the problem domain. Alternatively, a wrapping strategy may be used at the edge of the domain creating a loop, torrid or related geometrical structures at the chosen dimensionality.
	\item An inertia or momentum coefficient can be introduced to limit the change in velocity.
\end{itemize}

% Code Listing
% The code description provides a minimal but functional version of the technique implemented with a programming language. The code description must be able to be typed into an appropriate computer, compiled or interpreted as need be, and provide a working execution of the technique. The technique implementation also includes a minimal problem instance to which it is applied, and both the problem and algorithm implementations are complete enough to demonstrate the techniques procedure. The description is presented as a programming source code listing.
\subsection{Code Listing}
% How is a technique implemented as an executable program?
% How is a technique applied to a concrete problem instance?
Listing~\ref{pso} provides an example of the Particle Swarm Optimization algorithm implemented in the Ruby Programming Language. 
% problem
The demonstration problem is an instance of a continuous function optimization that seeks $\min f(x)$ where $f=\sum_{i=1}^n x_{i}^2$, $-5.0\leq x_i \leq 5.0$ and $n=3$. The optimal solution for this basin function is $(v_0,\ldots,v_{n-1})=0.0$.
% algorithm
The algorithm is a conservative version of Particle Swarm Optimization based on the seminal papers. The implementation limits the velocity at a pre-defined maximum, and bounds particles to the search space, reflecting their movement and velocity if the bounds of the space are exceeded. Particles are influenced by the best position found as well as their own personal best position. Natural extensions may consider limiting velocity with an inertia coefficient and including a neighborhood function for the particles.

% the listing
\lstinputlisting[firstline=7,language=ruby,caption=Particle Swarm Optimization in Ruby, label=pso]{../src/algorithms/swarm/pso.rb}

% References: Deeper understanding
% The references element description includes a listing of both primary sources of information about the technique as well as useful introductory sources for novices to gain a deeper understanding of the theory and application of the technique. The description consists of hand-selected reference material including books, peer reviewed conference papers, journal articles, and potentially websites. A bullet-pointed structure is suggested.
\subsection{References}
% What are the primary sources for a technique?
% What are the suggested reference sources for learning more about a technique?

% 
% Primary Sources
% 
\subsubsection{Primary Sources}
% seminal
Particle Swarm Optimization was described as a stochastic global optimization method for continuous functions in 1995 by Eberhart and Kennedy \cite{Eberhart1995, Kennedy1995}. This work was motivated as an optimization method loosely based on the flocking behavioral models of Reynolds \cite{Reynolds1987}.
% early
Early works included the introduction of inertia \cite{Shi1998} and early study of social topologies in the swarm by Kennedy \cite{Kennedy1999}. 

% 
% Learn More
% 
\subsubsection{Learn More}
% reviews
Poli, Kennedy, and Blackwell provide a modern overview of the field of PSO with detailed coverage of extensions to the baseline technique \cite{Poli2007}. Poli provides a meta-analysis of PSO publications that focus on the application the technique, providing a systematic breakdown on application areas \cite{Poli2008a}. 
% books
An excellent book on Swarm Intelligence in general with detailed coverage of Particle Swarm Optimization is ``Swarm Intelligence'' by Kennedy, Eberhart, and Shi \cite{Kennedy2001}.



\putbib\end{bibunit}
\newpage\begin{bibunit}% The Clever Algorithms Project: http://www.CleverAlgorithms.com
% (c) Copyright 2010 Jason Brownlee. Some Rights Reserved. 
% This work is licensed under a Creative Commons Attribution-Noncommercial-Share Alike 2.5 Australia License.

% This is an algorithm description, see:
% Jason Brownlee. A Template for Standardized Algorithm Descriptions. Technical Report CA-TR-20100107-1, The Clever Algorithms Project http://www.CleverAlgorithms.com, January 2010.

% Name
% The algorithm name defines the canonical name used to refer to the technique, in addition to common aliases, abbreviations, and acronyms. The name is used in terms of the heading and sub-headings of an algorithm description.
\section{Ant System} 
\label{sec:ant_system}
\index{Ant System}
\index{Ant Cycle}
\index{Ant Colony Optimization}

% other names
% What is the canonical name and common aliases for a technique?
% What are the common abbreviations and acronyms for a technique?
\emph{Ant System, AS, Ant Cycle.}

% Taxonomy: Lineage and locality
% The algorithm taxonomy defines where a techniques fits into the field, both the specific subfields of Computational Intelligence and Biologically Inspired Computation as well as the broader field of Artificial Intelligence. The taxonomy also provides a context for determining the relation- ships between algorithms. The taxonomy may be described in terms of a series of relationship statements or pictorially as a venn diagram or a graph with hierarchical structure.
\subsection{Taxonomy}
% To what fields of study does a technique belong?
The Ant System algorithm is an example of an Ant Colony Optimization method from the field of Swarm Intelligence, Metaheuristics and Computational Intelligence.
% What are the closely related approaches to a technique?
Ant System was originally the term used to refer to a range of Ant based algorithms, where the specific algorithm implementation was referred to as Ant Cycle. The so-called Ant Cycle algorithm is now canonically referred to as Ant System. The Ant System algorithm is the baseline Ant Colony Optimization method for popular extensions such as Elite Ant System, Rank-based Ant System, Max-Min Ant System, and Ant Colony System.

% Inspiration: Motivating system
% The inspiration describes the specific system or process that provoked the inception of the algorithm. The inspiring system may non-exclusively be natural, biological, physical, or social. The description of the inspiring system may include relevant domain specific theory, observation, nomenclature, and most important must include those salient attributes of the system that are somehow abstractly or conceptually manifest in the technique. The inspiration is described textually with citations and may include diagrams to highlight features and relationships within the inspiring system.
% Optional
\subsection{Inspiration}
% What is the system or process that motivated the development of a technique?
The Ant system algorithm is inspired by the foraging behavior of ants, specifically the pheromone communication between ants regarding a good path between the colony and a food source in an environment. This mechanism is called stigmergy.
% Which features of the motivating system are relevant to a technique?

% Metaphor: Explanation via analogy
% The metaphor is a description of the technique in the context of the inspiring system or a different suitable system. The features of the technique are made apparent through an analogous description of the features of the inspiring system. The explanation through analogy is not expected to be literal scientific truth, rather the method is used as an allegorical communication tool. The inspiring system is not explicitly described, this is the role of the ‘inspiration’ element, which represents a loose dependency for this element. The explanation is textual and uses the nomenclature of the metaphorical system.
% Optional
\subsection{Metaphor}
% What is the explanation of a technique in the context of the inspiring system?
% What are the functionalities inferred for a technique from the analogous inspiring system?
Ants initially wander randomly around their environment. Once food is located an ant will begin laying down pheromone in the environment. Numerous trips between the food and the colony are performed and if the same route is followed that leads to food then additional pheromone is laid down. Pheromone decays in the environment, so that older paths are less likely to be followed. Other ants may discover the same path to the food and in turn may follow it and also lay down pheromone. A positive feedback process routes more and more ants to productive paths that are in turn further refined through use.

% Strategy: Problem solving plan
% The strategy is an abstract description of the computational model. The strategy describes the information processing actions a technique shall take in order to achieve an objective. The strategy provides a logical separation between a computational realization (procedure) and a analogous system (metaphor). A given problem solving strategy may be realized as one of a number specific algorithms or problem solving systems. The strategy description is textual using information processing and algorithmic terminology.
\subsection{Strategy}
% What is the information processing objective of a technique?
The objective of the strategy is to exploit historic and heuristic information to construct candidate solutions and fold the information learned from constructing solutions into the history.
% What is a techniques plan of action?
Solutions are constructed one discrete piece at a time in a probabilistic step-wise manner. The probability of selecting a component is determined by the heuristic contribution of the component to the overall cost of the solution and the quality of solutions from which the component has historically known to have been included. History is updated proportional to the quality of candidate solutions and is uniformly decreased ensuring the most recent and useful information is retained.

% Procedure: Abstract computation
% The algorithmic procedure summarizes the specifics of realizing a strategy as a systemized and parameterized computation. It outlines how the algorithm is organized in terms of the data structures and representations. The procedure may be described in terms of software engineering and computer science artifacts such as Pseudocode, design diagrams, and relevant mathematical equations.
\subsection{Procedure}
% What is the computational recipe for a technique?
% What are the data structures and representations used in a technique?
Algorithm~\ref{alg:as} provides a pseudocode listing of the main Ant System algorithm for minimizing a cost function. 
The pheromone update process is described by a single equation that combines the contributions of all candidate solutions with a decay coefficient to determine the new pheromone value, as follows:

\begin{equation}
\tau_{i,j} \leftarrow (1-\rho) \times \tau_{i,j} + \sum_{k=1}^m \Delta_{i,j}^k
\end{equation}

where $\tau_{i,j}$ represents the pheromone for the component (graph edge) ($i,j$), $\rho$ is the decay factor, $m$ is the number of ants, and  $\sum_{k=1}^m \Delta_{i,j}^k $ is the sum of $\frac{1}{S_{cost}} $ (maximizing solution cost) for those solutions that include component $i,j$. The Pseudocode listing shows this equation as an equivalent as a two step process of decay followed by update for simplicity. 

The probabilistic step-wise construction of solution makes use of both history (pheromone) and problem-specific heuristic information to incrementally construction a solution piece-by-piece. Each component can only be selected if it has not already been chosen (for most combinatorial problems), and for those components that can  be selected from (given the current component $i$), their probability for selection is defined as:

\begin{equation}
P_{i,j} \leftarrow \frac{\tau_{i,j}^{\alpha} \times \eta_{i,j}^{\beta}}{\sum_{k=1}^c \tau_{i,k}^{\alpha} \times \eta_{i,k}^{\beta}}
\end{equation}

where $\eta_{i,j}$ is the maximizing contribution to the overall score of selecting the component (such as $\frac{1.0}{distance_{i,j}}$ for the Traveling Salesman Problem), $\alpha$ is the heuristic coefficient, $\tau_{i,j}$ is the pheromone value for the component, $\beta$ is the history coefficient, and $c$ is the set of usable components. 

\begin{algorithm}[ht]
	\SetLine  

	% data
	\SetKwData{Best}{$P_{best}$}
	\SetKwData{BestCost}{$Pbest_{cost}$}
	\SetKwData{ProblemSize}{ProblemSize}
	\SetKwData{DecayFactor}{$\rho$}
	\SetKwData{HistoryContribution}{$\alpha$}
	\SetKwData{HeuristicContribution}{$\beta$}
	\SetKwData{NumAnts}{$m$}
	\SetKwData{PopulationSize}{$Population_{size}$}
	\SetKwData{Pheromone}{Pheromone}
	\SetKwData{Candidates}{Candidates}
	\SetKwData{Solution}{$S_{i}$}
	\SetKwData{SolutionCost}{$Si_{cost}$}
	\SetKwData{HeuristicSolution}{$S_{h}$}
	\SetKwData{HeuristicSolutionCost}{$Sh_{cost}$}
	
	% functions
	\SetKwFunction{Cost}{Cost}
	\SetKwFunction{StopCondition}{StopCondition}
	\SetKwFunction{InitializePheromone}{InitializePheromone}
	\SetKwFunction{ProbabilisticStepwiseConstruction}{ProbabilisticStepwiseConstruction}
	\SetKwFunction{DecayPheromone}{DecayPheromone}
	\SetKwFunction{UpdatePheromone}{UpdatePheromone}
	\SetKwFunction{CreateHeuristicSolution}{CreateHeuristicSolution}
	
	% I/O
	\KwIn{\ProblemSize, \PopulationSize, \NumAnts, \DecayFactor, \HistoryContribution, \HeuristicContribution}		
	\KwOut{\Best}
  % Algorithm
	\Best $\leftarrow$ \CreateHeuristicSolution{\ProblemSize}\;
	\BestCost $\leftarrow$ \Cost{\HeuristicSolution}\;
	\Pheromone $\leftarrow$ \InitializePheromone{\BestCost}\;
	% loop
	\While{$\neg$\StopCondition{}} {
		\Candidates $\leftarrow$ $\emptyset$\;
		\For{$i=1$ $\KwTo$ \NumAnts} {
			\Solution $\leftarrow$ \ProbabilisticStepwiseConstruction{\Pheromone, \ProblemSize, \HistoryContribution, \HeuristicContribution}\;
			\SolutionCost $\leftarrow$ \Cost{\Solution}\;
			\If{\SolutionCost $\leq$ \BestCost} {
				\BestCost $\leftarrow$ \SolutionCost\;
				\Best $\leftarrow$ \Solution\;
			}
			\Candidates $\leftarrow$ \Solution\;
		}
		\DecayPheromone{\Pheromone, \DecayFactor}\;
		\ForEach{\Solution $\in$ \Candidates}{
			\UpdatePheromone{\Pheromone, \Solution, \SolutionCost}\;
		}
	}
	\Return{\Best}\;
	% end
	\caption{Pseudocode for Ant System.}
	\label{alg:as}
\end{algorithm}

% Heuristics: Usage guidelines
% The heuristics element describe the commonsense, best practice, and demonstrated rules for applying and configuring a parameterized algorithm. The heuristics relate to the technical details of the techniques procedure and data structures for general classes of application (neither specific implementations not specific problem instances). The heuristics are described textually, such as a series of guidelines in a bullet-point structure.
\subsection{Heuristics}
% What are the suggested configurations for a technique?
% What are the guidelines for the application of a technique to a problem instance?
\begin{itemize}
	\item The Ant Systems algorithm was designed for use with combinatorial problems such as the TSP, knapsack problem, quadratic assignment problems, graph coloring problems and many others.
	\item The history coefficient ($\alpha$) controls the amount of contribution history plays in a components probability of selection and is commonly set to 1.0.
	\item The heuristic coefficient ($\beta$) controls the amount of contribution problem-specific heuristic information plays in a components probability of selection and is commonly between 2 and 5, such as 2.5.
	\item The decay factor ($\rho$) controls the rate at which historic information is lost and is commonly set to 0.5.
	\item The total number of ants ($m$) is commonly set to the number of components in the problem, such as the number of cities in the TSP.
\end{itemize}

% Code Listing
% The code description provides a minimal but functional version of the technique implemented with a programming language. The code description must be able to be typed into an appropriate computer, compiled or interpreted as need be, and provide a working execution of the technique. The technique implementation also includes a minimal problem instance to which it is applied, and both the problem and algorithm implementations are complete enough to demonstrate the techniques procedure. The description is presented as a programming source code listing.
\subsection{Code Listing}
% How is a technique implemented as an executable program?
% How is a technique applied to a concrete problem instance?
Listing~\ref{ant_system} provides an example of the Ant System algorithm implemented in the Ruby Programming Language. 
% problem
The algorithm is applied to the Berlin52 instance of the Traveling Salesman Problem (TSP), taken from the TSPLIB. The problem seeks a permutation of the order to visit cities (called a tour) that minimized the total distance traveled. The optimal tour distance for Berlin52 instance is 7542 units.
% algorithm
Some extensions to the algorithm implementation for speed improvements may consider pre-calculating a distance matrix for all the cities in the problem, and pre-computing a probability matrix for choices during the probabilistic step-wise construction of tours. 

% the listing
\lstinputlisting[firstline=7,language=ruby,caption=Ant System in Ruby, label=ant_system]{../src/algorithms/swarm/ant_system.rb}

% References: Deeper understanding
% The references element description includes a listing of both primary sources of information about the technique as well as useful introductory sources for novices to gain a deeper understanding of the theory and application of the technique. The description consists of hand-selected reference material including books, peer reviewed conference papers, journal articles, and potentially websites. A bullet-pointed structure is suggested.
\subsection{References}
% What are the primary sources for a technique?
% What are the suggested reference sources for learning more about a technique?

% 
% Primary Sources
% 
\subsubsection{Primary Sources}
% seminal
The Ant System was described by Dorigo, Maniezzo, and Colorni in an early technical report as a class of algorithms and was applied to a number of standard combinatorial optimization algorithms \cite{Dorigo1991}. A series of technical reports at this time investigated the class of algorithms called Ant System and the specific implementation called Ant Cycle. This effort contributed to Dorigo's PhD thesis published in Italian \cite{Dorigo1992}.
% early
The seminal publication into the investigation of Ant System (with the implementation still referred to as Ant Cycle) was by Dorigo in 1996 \cite{Dorigo1996}.

% 
% Learn More
% 
\subsubsection{Learn More}
% books
The seminal book on Ant Colony Optimization in general with a detailed treatment of Ant system is ``Ant colony optimization'' by Dorigo and St\"utzle \cite{Dorigo2004}. An earlier book ``Swarm intelligence: from natural to artificial systems'' by Bonabeau, Dorigo, and Theraulaz also provides an introduction to Swarm Intelligence with a detailed treatment of Ant System \cite{Bonabeau1999}.



\putbib\end{bibunit}
\newpage\begin{bibunit}% The Clever Algorithms Project: http://www.CleverAlgorithms.com
% (c) Copyright 2010 Jason Brownlee. Some Rights Reserved. 
% This work is licensed under a Creative Commons Attribution-Noncommercial-Share Alike 2.5 Australia License.

% This is an algorithm description, see:
% Jason Brownlee. A Template for Standardized Algorithm Descriptions. Technical Report CA-TR-20100107-1, The Clever Algorithms Project http://www.CleverAlgorithms.com, January 2010.

% Name
% The algorithm name defines the canonical name used to refer to the technique, in addition to common aliases, abbreviations, and acronyms. The name is used in terms of the heading and sub-headings of an algorithm description.
\section{Ant Colony System} 
\label{sec:ant_colony_system}
\index{Ant Colony System}
\index{Ant-Q}
\index{Ant Colony Optimization}

% other names
% What is the canonical name and common aliases for a technique?
% What are the common abbreviations and acronyms for a technique?
\emph{Ant Colony System, ACS, Ant-Q.}

% Taxonomy: Lineage and locality
% The algorithm taxonomy defines where a techniques fits into the field, both the specific subfields of Computational Intelligence and Biologically Inspired Computation as well as the broader field of Artificial Intelligence. The taxonomy also provides a context for determining the relation- ships between algorithms. The taxonomy may be described in terms of a series of relationship statements or pictorially as a venn diagram or a graph with hierarchical structure.
\subsection{Taxonomy}
% To what fields of study does a technique belong?
The Ant Colony System algorithm is an example of an Ant Colony Optimization method from the field of Swarm Intelligence, Metaheuristics and Computational Intelligence.
% What are the closely related approaches to a technique?
Ant Colony System is an extension to the Ant System algorithm and is related to other Ant Colony Optimization methods such as Elite Ant System, and Rank-based Ant System.

% Inspiration: Motivating system
% The inspiration describes the specific system or process that provoked the inception of the algorithm. The inspiring system may non-exclusively be natural, biological, physical, or social. The description of the inspiring system may include relevant domain specific theory, observation, nomenclature, and most important must include those salient attributes of the system that are somehow abstractly or conceptually manifest in the technique. The inspiration is described textually with citations and may include diagrams to highlight features and relationships within the inspiring system.
% Optional
\subsection{Inspiration}
% What is the system or process that motivated the development of a technique?
The Ant Colony System algorithm is inspired by the foraging behavior of ants, specifically the pheromone communication between ants regarding a good path between the colony and a food source in an environment. This mechanism is called stigmergy.
% Which features of the motivating system are relevant to a technique?

% Metaphor: Explanation via analogy
% The metaphor is a description of the technique in the context of the inspiring system or a different suitable system. The features of the technique are made apparent through an analogous description of the features of the inspiring system. The explanation through analogy is not expected to be literal scientific truth, rather the method is used as an allegorical communication tool. The inspiring system is not explicitly described, this is the role of the ‘inspiration’ element, which represents a loose dependency for this element. The explanation is textual and uses the nomenclature of the metaphorical system.
% Optional
\subsection{Metaphor}
% What is the explanation of a technique in the context of the inspiring system?
% What are the functionalities inferred for a technique from the analogous inspiring system?
Ants initially wander randomly around their environment. Once food is located an ant will begin laying down pheromone in the environment. Numerous trips between the food and the colony are performed and if the same route is followed that leads to food then additional pheromone is laid down. Pheromone decays in the environment, so that older paths are less likely to be followed. Other ants may discover the same path to the food and in turn may follow it and also lay down pheromone. A positive feedback process routes more and more ants to productive paths that are in turn further refined through use.

% Strategy: Problem solving plan
% The strategy is an abstract description of the computational model. The strategy describes the information processing actions a technique shall take in order to achieve an objective. The strategy provides a logical separation between a computational realization (procedure) and a analogous system (metaphor). A given problem solving strategy may be realized as one of a number specific algorithms or problem solving systems. The strategy description is textual using information processing and algorithmic terminology.
\subsection{Strategy}
% What is the information processing objective of a technique?
The objective of the strategy is to exploit historic and heuristic information to construct candidate solutions and fold the information learned from constructing solutions into the history.
% What is a techniques plan of action?
Solutions are constructed one discrete piece at a time in a probabilistic step-wise manner. The probability of selecting a component is determined by the heuristic contribution of the component to the overall cost of the solution and the quality of solutions from which the component has historically known to have been included. History is updated proportional to the quality of the best known solution and is decreased proportional to the usage if discrete solution components.

% Procedure: Abstract computation
% The algorithmic procedure summarizes the specifics of realizing a strategy as a systemized and parameterized computation. It outlines how the algorithm is organized in terms of the data structures and representations. The procedure may be described in terms of software engineering and computer science artifacts such as Pseudocode, design diagrams, and relevant mathematical equations.
\subsection{Procedure}
% What is the computational recipe for a technique?
% What are the data structures and representations used in a technique?
Algorithm~\ref{alg:acs} provides a pseudocode listing of the main Ant Colony System algorithm for minimizing a cost function. 
The probabilistic step-wise construction of solution makes use of both history (pheromone) and problem-specific heuristic information to incrementally construct a solution piece-by-piece. Each component can only be selected if it has not already been chosen (for most combinatorial problems), and for those components that can  be selected from given the current component $i$, their probability for selection is defined as:

\begin{equation}
P_{i,j} \leftarrow \frac{\tau_{i,j}^{\alpha} \times \eta_{i,j}^{\beta}}{\sum_{k=1}^c \tau_{i,k}^{\alpha} \times \eta_{i,k}^{\beta}}
\end{equation}

where $\eta_{i,j}$ is the maximizing contribution to the overall score of selecting the component (such as $\frac{1.0}{distance_{i,j}}$ for the Traveling Salesman Problem), $\beta$ is the heuristic coefficient (commonly fixed at 1.0), $\tau_{i,j}$ is the pheromone value for the component, $\alpha$ is the history coefficient, and $c$ is the set of usable components. A greediness factor ($q0$) is used to influence when to use the above probabilistic component selection and when to greedily select the best possible component. 

A local pheromone update is performed for each solution that is constructed to dissuade following solutions to use the same components in the same order, as follows:

\begin{equation}
\tau_{i,j} \leftarrow (1-\sigma) \times \tau_{i,j} + \sigma \times \tau_{i,j}^{0}
\end{equation}

where $\tau_{i,j}$ represents the pheromone for the component (graph edge) ($i,j$), $\sigma$ is the local pheromone factor, and $\tau_{i,j}^{0}$ is the initial pheromone value.

At the end of each iteration, the pheromone is updated and decayed using the best candidate solution found thus far (or the best candidate solution found for the iteration), as follows:

\begin{equation}
\tau_{i,j} \leftarrow (1-\rho) \times \tau_{i,j} + \rho \times \Delta\tau{i,j}
\end{equation}

where $\tau_{i,j}$ represents the pheromone for the component (graph edge) ($i,j$), $\rho$ is the decay factor, and $\Delta\tau{i,j}$ is the maximizing solution cost for the best solution found so far if the component $ij$ is used in the globally best known solution, otherwise it is 0. 

\begin{algorithm}[ht]
	\SetLine  

	% data
	\SetKwData{Best}{$P_{best}$}
	\SetKwData{BestCost}{$Pbest_{cost}$}
	\SetKwData{ProblemSize}{ProblemSize}
	\SetKwData{DecayFactor}{$\rho$}
	\SetKwData{LocalPheromone}{$\sigma$}
	\SetKwData{HeuristicContribution}{$\beta$}
	\SetKwData{GreedinessFactor}{$q0$}
	\SetKwData{NumAnts}{$m$}
	\SetKwData{PopulationSize}{$Population_{size}$}
	\SetKwData{Pheromone}{Pheromone}
	\SetKwData{Candidates}{Candidates}
	\SetKwData{Solution}{$S_{i}$}
	\SetKwData{SolutionCost}{$Si_{cost}$}
	\SetKwData{HeuristicSolution}{$S_{h}$}
	\SetKwData{HeuristicSolutionCost}{$Sh_{cost}$}
	\SetKwData{InitialPheromone}{$Pheromone_{init}$}
	
	% functions
	\SetKwFunction{Cost}{Cost}
	\SetKwFunction{StopCondition}{StopCondition}
	\SetKwFunction{InitializePheromone}{InitializePheromone}
	\SetKwFunction{ConstructSolution}{ConstructSolution}
	\SetKwFunction{LocalUpdateAndDecayPheromone}{LocalUpdateAndDecayPheromone}
	\SetKwFunction{GlobalUpdateAndDecayPheromone}{GlobalUpdateAndDecayPheromone}
	\SetKwFunction{CreateHeuristicSolution}{CreateHeuristicSolution}
	
	% I/O
	\KwIn{\ProblemSize, \PopulationSize, \NumAnts, \DecayFactor, \HeuristicContribution, \LocalPheromone, \GreedinessFactor}		
	\KwOut{\Best}
  % Algorithm
	\Best $\leftarrow$ \CreateHeuristicSolution{\ProblemSize}\;
	\BestCost $\leftarrow$ \Cost{\HeuristicSolution}\;
	\InitialPheromone $\leftarrow$ $\frac{1.0}{\ProblemSize \times \BestCost}$\;
	\Pheromone $\leftarrow$ \InitializePheromone{\InitialPheromone}\;
	% loop
	\While{$\neg$\StopCondition{}} {		
		\For{$i=1$ $\KwTo$ \NumAnts} {
			\Solution $\leftarrow$ \ConstructSolution{\Pheromone, \ProblemSize, \HeuristicContribution, \GreedinessFactor}\;
			\SolutionCost $\leftarrow$ \Cost{\Solution}\;
			\If{\SolutionCost $\leq$ \BestCost} {
				\BestCost $\leftarrow$ \SolutionCost\;
				\Best $\leftarrow$ \Solution\;
			}
			\LocalUpdateAndDecayPheromone{\Pheromone, \Solution, \SolutionCost, \LocalPheromone}\;
		}
		\GlobalUpdateAndDecayPheromone{\Pheromone, \Best, \BestCost, \DecayFactor}\;
	}
	\Return{\Best}\;
	% end
	\caption{Pseudocode for Ant Colony System.}
	\label{alg:acs}
\end{algorithm}

% Heuristics: Usage guidelines
% The heuristics element describe the commonsense, best practice, and demonstrated rules for applying and configuring a parameterized algorithm. The heuristics relate to the technical details of the techniques procedure and data structures for general classes of application (neither specific implementations not specific problem instances). The heuristics are described textually, such as a series of guidelines in a bullet-point structure.
\subsection{Heuristics}
% What are the suggested configurations for a technique?
% What are the guidelines for the application of a technique to a problem instance?
\begin{itemize}
	\item The Ant Colony System algorithm was designed for use with combinatorial problems such as the TSP, knapsack problem, quadratic assignment problems, graph coloring problems and many others.
	\item The local pheromone (history) coefficient ($\sigma$) controls the amount of contribution history plays in a components probability of selection and is commonly set to 0.1.
	\item The heuristic coefficient ($\beta$) controls the amount of contribution problem-specific heuristic information plays in a components probability of selection and is commonly between 2 and 5, such as 2.5.
	\item The decay factor ($\rho$) controls the rate at which historic information is lost and is commonly set to 0.1.
	\item The greediness factor ($q0$) is commonly set to 0.9.
	\item The total number of ants ($m$) is commonly set low, such as 10.
\end{itemize}

% Code Listing
% The code description provides a minimal but functional version of the technique implemented with a programming language. The code description must be able to be typed into an appropriate computer, compiled or interpreted as need be, and provide a working execution of the technique. The technique implementation also includes a minimal problem instance to which it is applied, and both the problem and algorithm implementations are complete enough to demonstrate the techniques procedure. The description is presented as a programming source code listing.
\subsection{Code Listing}
% How is a technique implemented as an executable program?
% How is a technique applied to a concrete problem instance?
Listing~\ref{ant_colony_system} provides an example of the Ant Colony System algorithm implemented in the Ruby Programming Language. 
% problem
The algorithm is applied to the Berlin52 instance of the Traveling Salesman Problem (TSP), taken from the TSPLIB. The problem seeks a permutation of the order to visit cities (called a tour) that minimized the total distance traveled. The optimal tour distance for Berlin52 instance is 7542 units.
% algorithm
Some extensions to the algorithm implementation for speed improvements may consider pre-calculating a distance matrix for all the cities in the problem, and pre-computing a probability matrix for choices during the probabilistic step-wise construction of tours. 

% the listing
\lstinputlisting[firstline=7,language=ruby,caption=Ant Colony System in Ruby, label=ant_colony_system]{../src/algorithms/swarm/ant_colony_system.rb}

% References: Deeper understanding
% The references element description includes a listing of both primary sources of information about the technique as well as useful introductory sources for novices to gain a deeper understanding of the theory and application of the technique. The description consists of hand-selected reference material including books, peer reviewed conference papers, journal articles, and potentially websites. A bullet-pointed structure is suggested.
\subsection{References}
% What are the primary sources for a technique?
% What are the suggested reference sources for learning more about a technique?

% 
% Primary Sources
% 
\subsubsection{Primary Sources}
The algorithm was initially investigated by Dorigo and Gambardella under the name Ant-Q \cite{Dorigo1996a, Gambardella1995}.
% seminal
It was renamed Ant Colony System and further investigated first in a technical report by Dorigo and Gambardella \cite{Dorigo1997a}, and later published \cite{Dorigo1997}.

% 
% Learn More
% 
\subsubsection{Learn More}
% reviews
% books
The seminal book on Ant Colony Optimization in general with a detailed treatment of Ant Colony System is ``Ant colony optimization'' by Dorigo and St\"utzle \cite{Dorigo2004}. An earlier book ``Swarm intelligence: from natural to artificial systems'' by Bonabeau, Dorigo, and Theraulaz also provides an introduction to Swarm Intelligence with a detailed treatment of Ant Colony System \cite{Bonabeau1999}.


\putbib\end{bibunit}
\newpage\begin{bibunit}% The Clever Algorithms Project: http://www.CleverAlgorithms.com
% (c) Copyright 2010 Jason Brownlee. Some Rights Reserved. 
% This work is licensed under a Creative Commons Attribution-Noncommercial-Share Alike 2.5 Australia License.

% This is an algorithm description, see:
% Jason Brownlee. A Template for Standardized Algorithm Descriptions. Technical Report CA-TR-20100107-1, The Clever Algorithms Project http://www.CleverAlgorithms.com, January 2010.

% Name
% The algorithm name defines the canonical name used to refer to the technique, in addition to common aliases, abbreviations, and acronyms. The name is used in terms of the heading and sub-headings of an algorithm description.
\section{Bees Algorithm} 
\label{sec:bees_algorithm}
\index{Bees Algorithm}

% other names
% What is the canonical name and common aliases for a technique?
% What are the common abbreviations and acronyms for a technique?
\emph{Bees Algorithm, BA.}

% Taxonomy: Lineage and locality
% The algorithm taxonomy defines where a techniques fits into the field, both the specific subfields of Computational Intelligence and Biologically Inspired Computation as well as the broader field of Artificial Intelligence. The taxonomy also provides a context for determining the relation- ships between algorithms. The taxonomy may be described in terms of a series of relationship statements or pictorially as a venn diagram or a graph with hierarchical structure.
\subsection{Taxonomy}
% To what fields of study does a technique belong?
The Bees Algorithm beings to Bee Inspired Algorithms and the field of Swarm Intelligence, and more broadly the fields of Computational Intelligence and Metaheuristics.
% What are the closely related approaches to a technique?
The Bees Algorithm is related to other Bee Inspired Algorithms, such as Bee Colony Optimization, and other Swarm Intelligence algorithms such as Ant Colony Optimization and Particle Swarm Optimization.

% Inspiration: Motivating system
% The inspiration describes the specific system or process that provoked the inception of the algorithm. The inspiring system may non-exclusively be natural, biological, physical, or social. The description of the inspiring system may include relevant domain specific theory, observation, nomenclature, and most important must include those salient attributes of the system that are somehow abstractly or conceptually manifest in the technique. The inspiration is described textually with citations and may include diagrams to highlight features and relationships within the inspiring system.
% Optional
\subsection{Inspiration}
% What is the system or process that motivated the development of a technique?
The Bees Algorithm is inspired by the foraging behavior of honey bees.
% Which features of the motivating system are relevant to a technique?
Honey bees collect nectar from vast areas around their hive (more than 10 kilometers). Bee Colonies have been observed to send bees to collect nectar from flower patches relative to the amount of food available at each patch.
Bees communicate with each other at the hive via a waggle dance that informs other bees in the hive as to the direction, distance, and quality rating of food sources.

% Metaphor: Explanation via analogy
% The metaphor is a description of the technique in the context of the inspiring system or a different suitable system. The features of the technique are made apparent through an analogous description of the features of the inspiring system. The explanation through analogy is not expected to be literal scientific truth, rather the method is used as an allegorical communication tool. The inspiring system is not explicitly described, this is the role of the ‘inspiration’ element, which represents a loose dependency for this element. The explanation is textual and uses the nomenclature of the metaphorical system.
% Optional
\subsection{Metaphor}
% What is the explanation of a technique in the context of the inspiring system?
% What are the functionalities inferred for a technique from the analogous inspiring system?
Honey bees collect nectar from flower patches as a food source for the hive. The hive sends out scout's that locate patches of flowers, who then return to the hive and inform other bees about the fitness and location of a food source via a waggle dance. The scout returns to the flower patch with follower bees. A small number of scouts continue to search for new patches, while bees returning from flower patches continue to communicate the quality of the patch.

% Strategy: Problem solving plan
% The strategy is an abstract description of the computational model. The strategy describes the information processing actions a technique shall take in order to achieve an objective. The strategy provides a logical separation between a computational realization (procedure) and a analogous system (metaphor). A given problem solving strategy may be realized as one of a number specific algorithms or problem solving systems. The strategy description is textual using information processing and algorithmic terminology.
\subsection{Strategy}
% What is the information processing objective of a technique?
The information processing objective of the algorithm is to locate and explore good sites within a problem search space.
% What is a techniques plan of action?
Scouts are sent out to randomly sample the problem space and locate good sites. The good sites are exploited via the application of a local search, where a small number of good sites are explored more than the others. Good sites are continually exploited, although many scouts are sent out each iteration always in search of additional good sites.

% Procedure: Abstract computation
% The algorithmic procedure summarizes the specifics of realizing a strategy as a systemized and parameterized computation. It outlines how the algorithm is organized in terms of the data structures and representations. The procedure may be described in terms of software engineering and computer science artifacts such as Pseudocode, design diagrams, and relevant mathematical equations.
\subsection{Procedure}
% What is the computational recipe for a technique?
% What are the data structures and representations used in a technique?
Algorithm~\ref{alg:bees_algorithm} provides a pseudocode listing of the Bees Algorithm for minimizing a cost function. 

\begin{algorithm}[h!t]
	\SetLine

	% params
	\SetKwData{NumBees}{$Bees_{num}$}
	\SetKwData{NumSites}{$Sites_{num}$}
	\SetKwData{NumEliteSites}{$EliteSites_{num}$}
	\SetKwData{InitialPatchSize}{$PatchSize_{init}$}
	\SetKwData{NumEliteBees}{$EliteBees_{num}$}
	\SetKwData{NumOtherBees}{$OtherBees_{num}$}
	\SetKwData{ProblemSize}{$Problem_{size}$}
	% data
	\SetKwData{Neighborhood}{Neighborhood}
	\SetKwData{Population}{Population}
	\SetKwData{Best}{$Bee_{best}$}
	\SetKwData{NextGeneration}{NextGeneration}
	\SetKwData{BestSites}{$Sites_{best}$}
	\SetKwData{PatchDecreaseFactor}{$PatchDecrease_{factor}$}
	\SetKwData{Site}{$Site_{i}$}
	\SetKwData{NumRecruitedBees}{$RecruitedBees_{num}$}
	\SetKwData{PatchSize}{$Patch_{size}$}
	\SetKwData{RemainingBees}{$RemainingBees_{num}$}
	
	% functions
	\SetKwFunction{InitializePopulation}{InitializePopulation}  
	\SetKwFunction{StopCondition}{StopCondition} 
	\SetKwFunction{GetBestSolution}{GetBestSolution} 
	\SetKwFunction{SelectBestSites}{SelectBestSites}
	\SetKwFunction{CreateNeighborhoodBee}{CreateNeighborhoodBee}
	\SetKwFunction{CreateRandomBee}{CreateRandomBee}
	\SetKwFunction{EvaluatePopulation}{EvaluatePopulation}
  
	% I/O
	\KwIn{\ProblemSize, \NumBees, \NumSites, \NumEliteSites, \InitialPatchSize, \NumEliteBees, \NumOtherBees}		
	\KwOut{\Best}
  % Algorithm

	\Population $\leftarrow$ \InitializePopulation{\NumBees, \ProblemSize}\;

	\While{$\neg$\StopCondition{}} {
		\EvaluatePopulation{\Population}\;
		\Best $\leftarrow$ \GetBestSolution{\Population}\;
		\NextGeneration $\leftarrow \emptyset$\;		
		\PatchSize $\leftarrow$ ( \InitialPatchSize $\times$ \PatchDecreaseFactor )\;
		\BestSites $\leftarrow$ \SelectBestSites{\Population, \NumSites}\;
		
		\ForEach{\Site $\in$ \BestSites} {
				\NumRecruitedBees $\leftarrow$ $\emptyset$\;
				\eIf{$i <$ \NumEliteSites} {
					\NumRecruitedBees $\leftarrow$ \NumEliteBees\;
				}{
					\NumRecruitedBees $\leftarrow$ \NumOtherBees\;
				}
				\Neighborhood $\leftarrow$ $\emptyset$\;
				\For{$j$ \KwTo \NumRecruitedBees} {
					\Neighborhood $\leftarrow$ \CreateNeighborhoodBee{\Site, \PatchSize}\;
				}
				\NextGeneration $\leftarrow$ \GetBestSolution{\Neighborhood}\;
			}
		
		\RemainingBees $\leftarrow$ (\NumBees - \NumSites)\;
		\For{$j$ \KwTo \RemainingBees} {
			\NextGeneration $\leftarrow$ \CreateRandomBee{}\;
		}
		\Population $\leftarrow$ \NextGeneration\;
	}
	\Return{\Best}\;
	% end
	\caption{Pseudocode for the Bees Algorithm.}
	\label{alg:bees_algorithm}
\end{algorithm}

% Heuristics: Usage guidelines
% The heuristics element describe the commonsense, best practice, and demonstrated rules for applying and configuring a parameterized algorithm. The heuristics relate to the technical details of the techniques procedure and data structures for general classes of application (neither specific implementations not specific problem instances). The heuristics are described textually, such as a series of guidelines in a bullet-point structure.
\subsection{Heuristics}
% What are the suggested configurations for a technique?
% What are the guidelines for the application of a technique to a problem instance?
\begin{itemize}
	\item The Bees Algorithm was developed to be used with continuous and combinatorial function optimization problems.
	\item The $Patch_{size}$ variable is used as the neighborhood size. For example, in a continuous function optimization problem, each dimension of a site would be sampled as $x_i \pm (rand() \times Patch_{size})$.
	\item The $Patch_{size}$ variable is decreased each iteration, typically by a constant amount (such as 0.95).
	\item The number of elite sites ($EliteSites_{num}$) must be $<$ the number of sites ($Sites_{num}$), and the number of elite bees ($EliteBees_{num}$) is traditionally $<$ the number of other bees ($OtherBees_{num}$).
\end{itemize}

% Code Listing
% The code description provides a minimal but functional version of the technique implemented with a programming language. The code description must be able to be typed into an appropriate computer, compiled or interpreted as need be, and provide a working execution of the technique. The technique implementation also includes a minimal problem instance to which it is applied, and both the problem and algorithm implementations are complete enough to demonstrate the techniques procedure. The description is presented as a programming source code listing.
\subsection{Code Listing}
% How is a technique implemented as an executable program?
% How is a technique applied to a concrete problem instance?
Listing~\ref{bees_algorithm} provides an example of the Bees Algorithm implemented in the Ruby Programming Language. 
% problem
The demonstration problem is an instance of a continuous function optimization that seeks $\min f(x)$ where $f=\sum_{i=1}^n x_{i}^2$, $-5.0\leq x_i \leq 5.0$ and $n=3$. The optimal solution for this basin function is $(v_0,\ldots,v_{n-1})=0.0$.
% algorithm
The algorithm is an implementation of the Bees Algorithm as described in the seminal paper \cite{Pham2006}. A fixed patch size decrease factor of 0.95 was applied each iteration.

% the listing
\lstinputlisting[firstline=7,language=ruby,caption=Bees Algorithm in Ruby, label=bees_algorithm]{../src/algorithms/swarm/bees_algorithm.rb}

% References: Deeper understanding
% The references element description includes a listing of both primary sources of information about the technique as well as useful introductory sources for novices to gain a deeper understanding of the theory and application of the technique. The description consists of hand-selected reference material including books, peer reviewed conference papers, journal articles, and potentially websites. A bullet-pointed structure is suggested.
\subsection{References}
% What are the primary sources for a technique?
% What are the suggested reference sources for learning more about a technique?

% 
% Primary Sources
% 
\subsubsection{Primary Sources}
% seminal
The Bees Algorithm was proposed by Pham et al.\ in a technical report in 2005 \cite{Pham2005}, and later published \cite{Pham2006}. In this work, the algorithm was applied to standard instances of continuous function optimization problems.
% early

% 
% Learn More
% 
\subsubsection{Learn More}
% reviews
The majority of the work on the algorithm has concerned its application to various problem domains.
The following is a selection of popular application papers: the optimization of linear antenna arrays by Guney and Onay \cite{Guney2007}, the optimization of codebook vectors in the Learning Vector Quantization algorithm for classification by Pham et al.\ \cite{Pham2006a}, optimization of neural networks for classification by Pham et al.\ \cite{Pham2006b}, and the optimization of clustering methods by Pham et al.\ \cite{Pham2007}.
% books


\putbib\end{bibunit}
\newpage\begin{bibunit}% The Clever Algorithms Project: http://www.CleverAlgorithms.com
% (c) Copyright 2010 Jason Brownlee. Some Rights Reserved. 
% This work is licensed under a Creative Commons Attribution-Noncommercial-Share Alike 2.5 Australia License.

% This is an algorithm description, see:
% Jason Brownlee. A Template for Standardized Algorithm Descriptions. Technical Report CA-TR-20100107-1, The Clever Algorithms Project http://www.CleverAlgorithms.com, January 2010.

% Name
% The algorithm name defines the canonical name used to refer to the technique, in addition to common aliases, abbreviations, and acronyms. The name is used in terms of the heading and sub-headings of an algorithm description.
\section{Bacterial Foraging Optimization Algorithm} 
\label{sec:bfoa}
\index{Bacterial Foraging Optimization Algorithm}

% other names
% What is the canonical name and common aliases for a technique?
% What are the common abbreviations and acronyms for a technique?
\emph{Bacterial Foraging Optimization Algorithm, BFOA, Bacterial Foraging Optimization, BFO.}

% Taxonomy: Lineage and locality
% The algorithm taxonomy defines where a techniques fits into the field, both the specific subfields of Computational Intelligence and Biologically Inspired Computation as well as the broader field of Artificial Intelligence. The taxonomy also provides a context for determining the relation- ships between algorithms. The taxonomy may be described in terms of a series of relationship statements or pictorially as a venn diagram or a graph with hierarchical structure.
\subsection{Taxonomy}
% To what fields of study does a technique belong?
The Bacterial Foraging Optimization Algorithm belongs to the field of Bacteria Optimization Algorithms and Swarm Optimization, and more broadly to the fields of Computational Intelligence and Metaheuristics.
% What are the closely related approaches to a technique?
It is related to other Bacteria Optimization Algorithms such as the Bacteria Chemotaxis Algorithm \cite{Muller2002}, and other Swarm Intelligence algorithms such as Ant Colony Optimization and Particle Swarm Optimization.
There have been many extensions of the approach that attempt to hybridize the algorithm with other Computational Intelligence algorithms and Metaheuristics such as Particle Swarm Optimization, Genetic Algorithm, and Tabu Search.

% Inspiration: Motivating system
% The inspiration describes the specific system or process that provoked the inception of the algorithm. The inspiring system may non-exclusively be natural, biological, physical, or social. The description of the inspiring system may include relevant domain specific theory, observation, nomenclature, and most important must include those salient attributes of the system that are somehow abstractly or conceptually manifest in the technique. The inspiration is described textually with citations and may include diagrams to highlight features and relationships within the inspiring system.
% Optional
\subsection{Inspiration}
% What is the system or process that motivated the development of a technique?
The Bacterial Foraging Optimization Algorithm is inspired by the group foraging behavior of bacteria such as E.coli and M.xanthus.
% Which features of the motivating system are relevant to a technique?
Specifically, the BFOA is inspired by the chemotaxis behavior of bacteria that will perceive chemical gradients in the environment (such as nutrients) and move toward or away from specific signals.

% Metaphor: Explanation via analogy
% The metaphor is a description of the technique in the context of the inspiring system or a different suitable system. The features of the technique are made apparent through an analogous description of the features of the inspiring system. The explanation through analogy is not expected to be literal scientific truth, rather the method is used as an allegorical communication tool. The inspiring system is not explicitly described, this is the role of the ‘inspiration’ element, which represents a loose dependency for this element. The explanation is textual and uses the nomenclature of the metaphorical system.
% Optional
\subsection{Metaphor}
% What is the explanation of a technique in the context of the inspiring system?
% What are the functionalities inferred for a technique from the analogous inspiring system?
Bacteria perceive the direction to food based on the gradients of chemicals in their environment. Similarly, bacteria secrete attracting and repelling chemicals into the environment and can perceive each other in a similar way. Using locomotion mechanisms (such as flagella) bacteria can move around in their environment, sometimes moving chaotically (tumbling and spinning), and other times moving in a directed manner that may be referred to as swimming. Bacterial cells are treated like agents in an environment, using their perception of food and other cells as motivation to move, and stochastic tumbling and swimming like movement to re-locate. Depending on the cell-cell interactions, cells may swarm a food source, and/or may aggressively repel or ignore each other.

% Strategy: Problem solving plan
% The strategy is an abstract description of the computational model. The strategy describes the information processing actions a technique shall take in order to achieve an objective. The strategy provides a logical separation between a computational realization (procedure) and a analogous system (metaphor). A given problem solving strategy may be realized as one of a number specific algorithms or problem solving systems. The strategy description is textual using information processing and algorithmic terminology.
\subsection{Strategy}
% What is the information processing objective of a technique?
The information processing strategy of the algorithm is to allow cells to stochastically and collectively swarm toward optima.
% What is a techniques plan of action?
This is achieved through a series of three processes on a population of simulated cells: 1) `Chemotaxis' where the cost of cells is derated by the proximity to other cells and cells move along the manipulated cost surface one at a time (the majority of the work of the algorithm), 2) `Reproduction' where only those cells that performed well over their lifetime may contribute to the next generation, and 3) `Elimination-dispersal' where cells are discarded and new random samples are inserted with a low probability.

% Procedure: Abstract computation
% The algorithmic procedure summarizes the specifics of realizing a strategy as a systemized and parameterized computation. It outlines how the algorithm is organized in terms of the data structures and representations. The procedure may be described in terms of software engineering and computer science artifacts such as Pseudocode, design diagrams, and relevant mathematical equations.
\subsection{Procedure}
% What is the computational recipe for a technique?
% What are the data structures and representations used in a technique?
Algorithm~\ref{alg:bfoa} provides a pseudocode listing of the Bacterial Foraging Optimization Algorithm for minimizing a cost function. Algorithm~\ref{alg:chemotaxis} provides the pseudocode listing for the chemotaxis and swing behaviour of the BFOA algorithm.
A bacteria cost is derated by its interaction with other cells. This interaction function ($g()$) is calculated as follows: 

\begin{align*}
	g(cell_k) = 
	&\sum_{i=1}^S\bigg[-d_{attr}\times exp\bigg(-w_{attr}\times \sum_{m=1}^P (cell_m^k - other_m^i)^2 \bigg) \bigg] + \\
	&\sum_{i=1}^S\bigg[h_{repel}\times exp\bigg(-w_{repel}\times \sum_{m=1}^P cell_m^k - other_m^i)^2 \bigg) \bigg]
\end{align*}

where $cell_k$ is a given cell, $d_{attr}$ and $w_{attr}$ are attraction coefficients, $h_{repel}$ and $w_{repel}$ are repulsion coefficients, $S$ is the number of cells in the population, $P$ is the number of dimensions on a given cells position vector.

The remaining parameters of the algorithm are as follows $Cells_{num}$ is the number of cells maintained in the population, $N_{ed}$ is the number of elimination-dispersal steps, $N_{re}$ is the number of reproduction steps, $N_{c}$ is the number of chemotaxis steps, $N_{s}$ is the number of swim steps for a given cell, $Step_{size}$ is a random direction vector with the same number of dimensions as the problem space, and each value $\in [-1,1]$, and $P_{ed}$ is the probability of a cell being subjected to elimination and dispersal. 

\begin{algorithm}[ht]
	\SetLine

	% params
	\SetKwData{ProblemSize}{$Problem_{size}$}
	\SetKwData{NumCells}{$Cells_{num}$}
	\SetKwData{EliminationDispersalSteps}{$N_{ed}$}
	\SetKwData{ReproductionSteps}{$N_{re}$}
	\SetKwData{ChemotaxisSteps}{$N_{c}$}
	\SetKwData{SwimSteps}{$N_{s}$}
	\SetKwData{StepSize}{$Step_{size}$}
	\SetKwData{ProbElimination}{$P_{ed}$}
	
	\SetKwData{DA}{$d_{attract}$}
	\SetKwData{DW}{$w_{attract}$}
	\SetKwData{HR}{$h_{repellant}$}
	\SetKwData{WR}{$w_{repellant}$}
	
	% data
	\SetKwData{Population}{Population}
	\SetKwData{Cell}{Cell}
	\SetKwData{CellNewPosition}{$Cell'$}
	\SetKwData{CellFitness}{$Cell_{fitness}$}
	\SetKwData{CellNewPositionFitness}{${Cell'}_{fitness}$}
	\SetKwData{RandomStepDirection}{RandomStepDirection}
	\SetKwData{Best}{$Cell_{best}$}
	\SetKwData{CellHealth}{$Cell_{health}$}
	\SetKwData{Selected}{Selected}
	
	% functions
	\SetKwFunction{InitializePopulation}{InitializePopulation}  
	\SetKwFunction{Interaction}{Interaction}
	\SetKwFunction{CreateStep}{CreateStep}
	\SetKwFunction{TakeStep}{TakeStep}
	\SetKwFunction{Cost}{Cost}
	\SetKwFunction{SortByCellHealth}{SortByCellHealth}
	\SetKwFunction{SelectByCellHealth}{SelectByCellHealth}
	\SetKwFunction{Rand}{Rand}
	\SetKwFunction{CreateCellAtRandomLocation}{CreateCellAtRandomLocation}
  \SetKwFunction{ChemotaxisAndSwim}{ChemotaxisAndSwim}
	% I/O
	\KwIn{\ProblemSize, \NumCells, \EliminationDispersalSteps, \ReproductionSteps, \ChemotaxisSteps, \SwimSteps, \StepSize, \DA, \DW, \HR, \WR, \ProbElimination}		
	\KwOut{\Best}
  % Algorithm

	\Population $\leftarrow$ \InitializePopulation{\NumCells, \ProblemSize}\;
	\For{$l=0$ \KwTo \EliminationDispersalSteps}{
		\For{$k=0$ \KwTo \ReproductionSteps}{		
			\For{$j=0$ \KwTo \ChemotaxisSteps}{
				% perform chemotaxis and swim steps
				\ChemotaxisAndSwim{\Population, \ProblemSize, \NumCells, \SwimSteps, \StepSize, \DA, \DW, \HR, \WR}\;
				\ForEach{\Cell $\in$ \Population} {
					\If{\Cost{\Cell} $\leq$ \Cost{\Best}} {
						\Best $\leftarrow$ \Cell\;
					}
				}
			}			
			% perform reproduction
			\SortByCellHealth{\Population}\;
			\Selected $\leftarrow$ \SelectByCellHealth{\Population, $\frac{\NumCells}{2}$}\;
			\Population $\leftarrow$ \Selected\;
			\Population $\leftarrow$ \Selected\;
		}
		% perform elimination and dispersal
		\ForEach{\Cell $\in$ \Population} {
			\If{\Rand{} $\leq$ \ProbElimination} {
				\Cell $\leftarrow$ \CreateCellAtRandomLocation{}\;
			}
		}
	}
	\Return{\Best}\;
	% end
	\caption{Pseudocode for the BFOA.}
	\label{alg:bfoa}
\end{algorithm}



\begin{algorithm}[ht]
	\SetLine

	% params
	\SetKwData{ProblemSize}{$Problem_{size}$}
	\SetKwData{NumCells}{$Cells_{num}$}
	\SetKwData{EliminationDispersalSteps}{$N_{ed}$}
	\SetKwData{ReproductionSteps}{$N_{re}$}
	\SetKwData{ChemotaxisSteps}{$N_{c}$}
	\SetKwData{SwimSteps}{$N_{s}$}
	\SetKwData{StepSize}{$Step_{size}$}
	\SetKwData{ProbElimination}{$P_{ed}$}
	
	\SetKwData{DA}{$d_{attract}$}
	\SetKwData{DW}{$w_{attract}$}
	\SetKwData{HR}{$h_{repellant}$}
	\SetKwData{WR}{$w_{repellant}$}
	
	% data
	\SetKwData{Population}{Population}
	\SetKwData{Cell}{Cell}
	\SetKwData{CellNewPosition}{$Cell'$}
	\SetKwData{CellFitness}{$Cell_{fitness}$}
	\SetKwData{CellNewPositionFitness}{${Cell'}_{fitness}$}
	\SetKwData{RandomStepDirection}{RandomStepDirection}
	\SetKwData{Best}{$Cell_{best}$}
	\SetKwData{CellHealth}{$Cell_{health}$}
	\SetKwData{Selected}{Selected}
	
	% functions
	\SetKwFunction{InitializePopulation}{InitializePopulation}  
	\SetKwFunction{Interaction}{Interaction}
	\SetKwFunction{CreateStep}{CreateStep}
	\SetKwFunction{TakeStep}{TakeStep}
	\SetKwFunction{Cost}{Cost}
	\SetKwFunction{SortByCellHealth}{SortByCellHealth}
	\SetKwFunction{SelectByCellHealth}{SelectByCellHealth}
	\SetKwFunction{Rand}{Rand}
	\SetKwFunction{CreateCellAtRandomLocation}{CreateCellAtRandomLocation}
  
	% I/O
	\KwIn{\Population, \ProblemSize, \NumCells, \SwimSteps, \StepSize, \DA, \DW, \HR, \WR}		

	% perform chemotaxis
	\ForEach{\Cell $\in$ \Population} {
		\CellFitness $\leftarrow$ \Cost{\Cell} $+$ \Interaction{\Cell, \Population, \DA, \DW, \HR, \WR}\;
		\CellHealth $\leftarrow$ \CellFitness\;
		\CellNewPosition $\leftarrow$ $\emptyset$\;
		% swim
		\For{$i=0$ \KwTo \SwimSteps}{
			\RandomStepDirection $\leftarrow$ \CreateStep{\ProblemSize}\;
			\CellNewPosition $\leftarrow$ \TakeStep{\RandomStepDirection, \StepSize}\;
			\CellNewPositionFitness $\leftarrow$ \Cost{\CellNewPosition} + \Interaction{\CellNewPosition, \Population, \DA, \DW, \HR, \WR}\;
			\eIf {\CellNewPositionFitness $>$ \CellFitness} {
				$i \leftarrow$ \SwimSteps\;
			}{
				\Cell $\leftarrow$ \CellNewPosition\;
				\CellHealth $\leftarrow$ \CellHealth + \CellNewPositionFitness\;
			}						
		}
	}
	% end
	\caption{Pseudocode for the \texttt{Chemotaxis\-And\-Swim} function.}
	\label{alg:chemotaxis}
\end{algorithm}

% Heuristics: Usage guidelines
% The heuristics element describe the commonsense, best practice, and demonstrated rules for applying and configuring a parameterized algorithm. The heuristics relate to the technical details of the techniques procedure and data structures for general classes of application (neither specific implementations not specific problem instances). The heuristics are described textually, such as a series of guidelines in a bullet-point structure.
\subsection{Heuristics}
% What are the suggested configurations for a technique?
% What are the guidelines for the application of a technique to a problem instance?
\begin{itemize}
	\item The algorithm was designed for application to continuous function optimization problem domains.
	\item Given the loops in the algorithm, it can be configured numerous ways to elicit different search behavior. It is common to have a large number of chemotaxis iterations, and small numbers of the other iterations.
	\item The default coefficients for swarming behavior (cell-cell interactions) are as follows $d_{attract}=0.1$, $w_{attract}=0.2$, $h_{repellant}=d_{attract}$, and $w_{repellant}=10$.	
	\item The step size is commonly a small fraction of the search space, such as 0.1.
	\item During reproduction, typically half the population with a low health metric are discarded, and two copies of each member from the first (high-health) half of the population are retained.
	\item The probability of elimination and dispersal ($p_{ed}$) is commonly set quite large, such as 0.25.
\end{itemize}

% Code Listing
% The code description provides a minimal but functional version of the technique implemented with a programming language. The code description must be able to be typed into an appropriate computer, compiled or interpreted as need be, and provide a working execution of the technique. The technique implementation also includes a minimal problem instance to which it is applied, and both the problem and algorithm implementations are complete enough to demonstrate the techniques procedure. The description is presented as a programming source code listing.
\subsection{Code Listing}
% How is a technique implemented as an executable program?
% How is a technique applied to a concrete problem instance?
Listing~\ref{bfoa} provides an example of the Bacterial Foraging Optimization Algorithm implemented in the Ruby Programming Language. 
% problem
The demonstration problem is an instance of a continuous function optimization that seeks $\min f(x)$ where $f=\sum_{i=1}^n x_{i}^2$, $-5.0\leq x_i \leq 5.0$ and $n=2$. The optimal solution for this basin function is $(v_0,\ldots,v_{n-1})=0.0$.
% algorithm
The algorithm is an implementation based on the description on the seminal work \cite{Passino2002}. The parameters for cell-cell interactions (attraction and repulsion) were taken from the paper, and the various loop parameters were taken from the `Swarming Effects' example.

% the listing
\lstinputlisting[firstline=7,language=ruby,caption=Bacterial Foraging Optimization Algorithm in Ruby, label=bfoa]{../src/algorithms/swarm/bfoa.rb}

% References: Deeper understanding
% The references element description includes a listing of both primary sources of information about the technique as well as useful introductory sources for novices to gain a deeper understanding of the theory and application of the technique. The description consists of hand-selected reference material including books, peer reviewed conference papers, journal articles, and potentially websites. A bullet-pointed structure is suggested.
\subsection{References}
% What are the primary sources for a technique?
% What are the suggested reference sources for learning more about a technique?

% 
% Primary Sources
% 
\subsubsection{Primary Sources}
% seminal
Early work by Liu and Passino considered models of chemotaxis as optimization for both E.coli and M.xanthus which were applied to continuous function optimization \cite{Liu2002}.
This work was consolidated by Passino who presented the Bacterial Foraging Optimization Algorithm that included a detailed presentation of the algorithm, heuristics for configuration, and demonstration applications and behavior dynamics \cite{Passino2002}.

% 
% Learn More
% 
\subsubsection{Learn More}
% reviews
A detailed summary of social foraging and the BFOA is provided in the book by Passino \cite{Passino2005}.
Passino provides a follow-up review of the background models of chemotaxis as optimization and describes the equations of the  Bacterial Foraging Optimization Algorithm in detail in a Journal article \cite{Passino2010}.
Das et al.\ present the algorithm and its inspiration, and go on to provide an in depth analysis the dynamics of chemotaxis using simplified mathematical models \cite{Das2009}.
% books


\putbib\end{bibunit}
