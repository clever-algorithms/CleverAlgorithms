% background information required to understand the content of this book
\chapter{Clever Algorithms}
\label{chap:field}

\section{Overview}
What is this chapter all about?


\section{Taxonomy}
What are all the fields and subfields we need to know about?

\subsection{Artificial Intelligence}
Messy and Neat AI

\subsection{Computational Intelligence}
Fuzzy Logic, Artificial Neural Networks, and Evolutionary Computation

\subsection{Natural Computation}
Biologically Inspired Computation, Computation with Biology, Computational Biology 

\subsection{Metaheuristics}
Heuristics for driving heuristics

\subsection{Machine Learning}
Statistical methods for learning


\section{Algorithms}
What do we need to know about this general class of algorithms.

\subsection{Black Box Methods}
They make little or few assumptions about the problem domain

\subsection{Randomness}
The are stochastic processes.

\subsection{State Space}
The typically require the problem to be phrased as a search space which is traversed and sampled.
We care about the size of moves, the patters of sampling and re-sampling, the number of samples managed.

\subsection{Induction}
The typically learn using indiction (trial and error)



\section{Problems}
What types of computational problems are we solving with these algorithms?

\subsection{Function Optimization}
Generate a set of parameters (continuous) or something like a permutation (combinatorial).

\subsection{Function Approximation}
Generate a representation that produces outputs in the presence of inputs.



\section{Clever Algorithms}

\subsection{Algorithm Selection}
How were algorithms selected and included in this book?

\subsection{Adopted Taxonomy}
How are algorithms organized in this book and why?

\subsection{Tutorial Programming Language}
What language was selected for presenting algorithms and why?

\section{Summary}
what did we learn in this chapter?
