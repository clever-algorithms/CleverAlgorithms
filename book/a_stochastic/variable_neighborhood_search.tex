% The Clever Algorithms Project: http://www.CleverAlgorithms.com
% (c) Copyright 2010 Jason Brownlee. Some Rights Reserved. 
% This work is licensed under a Creative Commons Attribution-Noncommercial-Share Alike 2.5 Australia License.

% This is an algorithm description, see:
% Jason Brownlee. A Template for Standardized Algorithm Descriptions. Technical Report CA-TR-20100107-1, The Clever Algorithms Project http://www.CleverAlgorithms.com, January 2010.

% Name
% The algorithm name defines the canonical name used to refer to the technique, in addition to common aliases, abbreviations, and acronyms. The name is used in terms of the heading and sub-headings of an algorithm description.
\section{Variable Neighborhood Search} 
\label{sec:variable_neighborhood_search}
\index{Variable Neighborhood Search}

% other names
% What is the canonical name and common aliases for a technique?
% What are the common abbreviations and acronyms for a technique?
\emph{Variable Neighborhood Search, VNS.}

% Taxonomy: Lineage and locality
% The algorithm taxonomy defines where a techniques fits into the field, both the specific subfields of Computational Intelligence and Biologically Inspired Computation as well as the broader field of Artificial Intelligence. The taxonomy also provides a context for determining the relation- ships between algorithms. The taxonomy may be described in terms of a series of relationship statements or pictorially as a venn diagram or a graph with hierarchical structure.
\subsection{Taxonomy}
% To what fields of study does a technique belong?
% What are the closely related approaches to a technique?
% what is it
Variable Neighborhood Search is a Metaheuristic and a Global Optimization technique that manages a Local Search technique.
% related
It is related to the Iterative Local Search algorithm (Section~\ref{sec:iterated_local_search}).

% Strategy: Problem solving plan
% The strategy is an abstract description of the computational model. The strategy describes the information processing actions a technique shall take in order to achieve an objective. The strategy provides a logical separation between a computational realization (procedure) and a analogous system (metaphor). A given problem solving strategy may be realized as one of a number specific algorithms or problem solving systems. The strategy description is textual using information processing and algorithmic terminology.
\subsection{Strategy}
% What is the information processing objective of a technique?
The strategy for the Variable Neighborhood Search involves iterative exploration of larger and larger neighborhoods for a given local optima until an improvement is located after which time the search across expanding neighborhoods is repeated. 
% What is a techniques plan of action?
The strategy is motivated by three principles: 1) a local minimum for one neighborhood structure may not be a local minimum for a different neighborhood structure, 2) a global minimum is a local minimum for all possible neighborhood structures, and 3) local minima are relatively close to global minima for many problem classes.

% Procedure: Abstract computation
% The algorithmic procedure summarizes the specifics of realizing a strategy as a systemized and parameterized computation. It outlines how the algorithm is organized in terms of the data structures and representations. The procedure may be described in terms of software engineering and computer science artifacts such as Pseudocode, design diagrams, and relevant mathematical equations.
\subsection{Procedure}
% What is the computational recipe for a technique?
% What are the data structures and representations used in a technique?
Algorithm~\ref{alg:variable_neighborhood_search} provides a pseudocode listing of the Variable Neighborhood Search algorithm for minimizing a cost function.
% explaination
The Pseudocode shows that the systematic search of expanding neighborhoods for a local optimum is abandoned when a global improvement is achieved (shown with the \texttt{Break} jump).

\begin{algorithm}[htp]
	\SetLine
	% data
	\SetKwData{Neighborhoods}{Neighborhoods}
	\SetKwData{CurrentNeighborhood}{$Neighborhood_{i}$}
	\SetKwData{Neighborhood}{$Neighborhood_{curr}$}
	\SetKwData{Best}{$S_{best}$}
	\SetKwData{Candidate}{$S_{candidate}$}
	% functions
	\SetKwFunction{Break}{Break}
	\SetKwFunction{Cost}{Cost}
	\SetKwFunction{StopCondition}{StopCondition}
	\SetKwFunction{CalculateNeighborhood}{CalculateNeighborhood}
	\SetKwFunction{RandomSolution}{RandomSolution}
	\SetKwFunction{RandomSolutionInNeighborhood}{RandomSolutionInNeighborhood}
  	\SetKwFunction{LocalSearch}{LocalSearch}

	% I/O
	\KwIn{\Neighborhoods}
	\KwOut{\Best}
  	% Algorithm
	\Best $\leftarrow$ \RandomSolution{}\;
	\While{$\neg$ \StopCondition{}} {
		\ForEach{\CurrentNeighborhood $\in$ \Neighborhoods} {
			% determine neighbourhood
			\Neighborhood $\leftarrow$ \CalculateNeighborhood{\Best, \CurrentNeighborhood}\;
			% starting point
			\Candidate $\leftarrow$ \RandomSolutionInNeighborhood{\Neighborhood}\;
			% local search
			\Candidate $\leftarrow$ \LocalSearch{\Candidate}\;
			% keep best
			\If{\Cost{\Candidate} $<$ \Cost{\Best}} {
				\Best $\leftarrow$ \Candidate\;
				\Break\;
			}
		}
	}
	\Return{\Best}\;
	% caption
	\caption{Pseudocode for VNS.}
	\label{alg:variable_neighborhood_search}
\end{algorithm}

% Heuristics: Usage guidelines
% The heuristics element describe the commonsense, best practice, and demonstrated rules for applying and configuring a parameterized algorithm. The heuristics relate to the technical details of the techniques procedure and data structures for general classes of application (neither specific implementations not specific problem instances). The heuristics are described textually, such as a series of guidelines in a bullet-point structure.
\subsection{Heuristics}
% What are the suggested configurations for a technique?
% What are the guidelines for the application of a technique to a problem instance?
\begin{itemize} 
	\item Approximation methods (such as stochastic hill climbing) are suggested for use as the Local Search procedure for large problem instances in order to reduce the running time.
	\item Variable Neighborhood Search has been applied to a very wide array of combinatorial optimization problems as well as clustering and continuous function optimization problems.
	\item The embedded Local Search technique should be specialized to the problem type and instance to which the technique is being applied. 
	\item The Variable Neighborhood Descent (VND) can be embedded in the Variable Neighborhood Search as a the Local Search procedure and has been shown to be most effective.
\end{itemize}

% The code description provides a minimal but functional version of the technique implemented with a programming language. The code description must be able to be typed into an appropriate computer, compiled or interpreted as need be, and provide a working execution of the technique. The technique implementation also includes a minimal problem instance to which it is applied, and both the problem and algorithm implementations are complete enough to demonstrate the techniques procedure. The description is presented as a programming source code listing.
\subsection{Code Listing}
% How is a technique implemented as an executable program?
% How is a technique applied to a concrete problem instance?
Listing~\ref{variable_neighborhood_search} provides an example of the Variable Neighborhood Search algorithm implemented in the Ruby Programming Language. 
% problem
The algorithm is applied to the Berlin52 instance of the Traveling Salesman Problem (TSP), taken from the TSPLIB. The problem seeks a permutation of the order to visit cities (called a tour) that minimizes the total distance traveled. The optimal tour distance for Berlin52 instance is 7542 units.

% algorithm
The Variable Neighborhood Search uses a stochastic 2-opt procedure as the embedded local search. The procedure deletes two edges and reverses the sequence in-between the deleted edges, potentially removing `twists' in the tour. The neighborhood structure used in the search is the number of times the 2-opt procedure is performed on a permutation, between 1 and 20 times. The stopping condition for the local search procedure is a maximum number of iterations without improvement. The same stop condition is employed by the higher-order Variable Neighborhood Search procedure, although with a lower boundary on the number of non-improving iterations.

% the listing
\lstinputlisting[firstline=7,language=ruby,caption=Variable Neighborhood Search in Ruby, label=variable_neighborhood_search]{../src/algorithms/stochastic/variable_neighborhood_search.rb}

% References: Deeper understanding
% The references element description includes a listing of both primary sources of information about the technique as well as useful introductory sources for novices to gain a deeper understanding of the theory and application of the technique. The description consists of hand-selected reference material including books, peer reviewed conference papers, journal articles, and potentially websites. A bullet-pointed structure is suggested.
\subsection{References}
% What are the primary sources for a technique?
% What are the suggested reference sources for learning more about a technique?

% 
% Primary Sources
% 
\subsubsection{Primary Sources}
% initial
The seminal paper for describing Variable Neighborhood Search was by Mladenovic and Hansen in 1997 \cite{Mladenovic1997}, although an early abstract by Mladenovic is sometimes cited \cite{Mladenovic1995}.
% names
The approach is explained in terms of three different variations on the general theme. Variable Neighborhood Descent (VND) refers to the use of a Local Search procedure and the deterministic (as opposed to stochastic or probabilistic) change of neighborhood size. Reduced Variable Neighborhood Search (RVNS) involves performing a stochastic random search within a neighborhood and no refinement via a local search technique. Basic Variable Neighborhood Search is the canonical approach described by Mladenovic and Hansen in the seminal paper.

% 
% Learn More
% 
\subsubsection{Learn More}
There are a large number of papers published on Variable Neighborhood Search, its applications and variations.
% overview
Hansen and Mladenovic provide an overview of the approach that includes its recent history, extensions and a detailed review of the numerous areas of application \cite{Hansen2003}. 
% other
For some additional useful overviews of the technique, its principles, and applications, see \cite{Hansen1998, Hansen2001a, Hansen2002}.

% extensions
There are many extensions to Variable Neighborhood Search. Some popular examples include: Variable Neighborhood Decomposition Search (VNDS) that involves embedding a second heuristic or metaheuristic approach in VNS to replace the Local Search procedure \cite{Hansen2001}, Skewed Variable Neighborhood Search (SVNS) that encourages exploration of neighborhoods far away from discovered local optima, and Parallel Variable Neighborhood Search (PVNS) that either parallelizes the local search of a neighborhood or parallelizes the searching of the neighborhoods themselves.


