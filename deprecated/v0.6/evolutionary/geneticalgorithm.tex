\section{Genetic Algorithm}
\emph{Genetic Algorithm (GA), Canonical Genetic Algorithm (CGA), Simple Genetic Algorithm (SGA)}

\subsection{Inspiration}
The Genetic Algorithm is inspired by the modern evolutionary synthesis including the process of Darwinian evolution operating under the constraints of genetics on a population of individuals. Population members are defined by a genome with one or more chromosomes made up of one or more genes. This genotype is expressed as a phenotype, the fitness of which is assessed in a specific environment. A new population of individuals is created from existing population using the evolutionary processes of fitness proportionate selection and replication via genetic recombination with mutation. The process is repeated over a number of generations until the population adapts to the defined environment.

\subsection{Further Reading}

\subsubsection{Books}

\begin{itemize}
	\item Holland, John H (1975), Adaptation in Natural and Artificial Systems, University of Michigan Press, Ann Arbor
	\item Goldberg, David E (1989), Genetic Algorithms in Search, Optimization and Machine Learning, Kluwer Academic Publishers, Boston, MA.
	\item Eshelman, L. J. Genetic algorithms. Evolutionary Computation 1 Basic Algorithms and Operator. Institute of Physics Publishing, Bristol and Philadelphia, 2000, 64-80.
	\item Mitchell, Melanie, (1996), An Introduction to Genetic Algorithms, MIT Press, Cambridge, MA.
	\item Vose, Michael D (1999), The Simple Genetic Algorithm: Foundations and Theory, MIT Press, Cambridge, MA.
	\item De Jong, K.A. (1975) An analysis of the behavior of a class of genetic adaptive systems, Doctoral thesis, Dept. of Computer and Communication Sciences, University of Michigan, Ann Arbor.
\end{itemize}

\subsubsection{Papers}

\begin{itemize}
	\item Whitley, D. (1994). A genetic algorithm tutorial. Statistics and Computing 4, 65–85.
	\item David Beasley,  David R. Bull,  Ralph R. Martin (1993), An Overview of Genetic Algorithms: Part 1, Fundamentals, University Computing, 15(2) pages 58-69
	\item H. Muhlenbein. How genetic algorithms really work: Mutation and hillclimbing. In R. Männer and B. Manderick, editors, Proc. of the Second Conference on Parallel Problem Solving from Nature (PPSN II), pages 15-25, Amsterdam, 1992. North-Holland.
	\item Goldberg, D. E., Genetic and Evolutionary Algorithms Come of Age, Communications of the ACM, 37, 3, Mar (1994). 
	\item G. Harik, D. E. Goldberg, E. Cantú-paz, and B. L. Miller. The gambler's ruin problem, genetic algorithms, and the sizing of populations. In IEEE Conference on Evolutionary Computation (IEEE-CEC 1997), pages 7--12, 1997.
\end{itemize}