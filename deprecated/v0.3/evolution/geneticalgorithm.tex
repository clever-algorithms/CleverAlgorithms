\section{Genetic Algorithm}

\subsection{Inspiration}
The Genetic Algorithm (GA) is inspired by adaptation of fit between individuals and their environment. Within this broader concern, the GA is focused on the relationship between an organisms genotype (inheritable information that describes an organism) with its phenotype (the expression of the genotype on which natural selection operates).

\subsection{Strategy}
The strategy of the GA is to map the problem onto an environment and candidate solutions onto organisms. Solutions are defined with a sub-symbolic representation (chromosome(s) or genome) on which genetic operators like variation and recombination (mutation and crossover) are applied and a problem specific representation that the sub-symbolic representation is mapped to that is used to evaluate the solution (natural selection). 

The procedure for the GA involves a large number of iterations called generations where the population is evaluated, the fitter members of which are selected to contribute their genetic material in the creation of the subsequent generation with minor copying errors. The GA starts with a randomly initialized population of candidate solutions defined as binary strings. Each algorithm iteration, the population is exposed to a series of genetic operator functions that manipulate the population. The candidate solutions are first evaluated against a cost function. A subset of candidate solutions are selected proportional to their relative fitness and recombined to produce new candidate solutions that displace the current population. The creation of new child candidate solutions are random mixes the binary strings of two parents. Low frequency random copying errors are introduced during the recombination process providing a background level of diversity.

\subsection{Heuristics}
\begin{itemize}
	\item Binary strings are the standard genetic representation when implementing the algorithm and its genetic operators, although domain specific representations may be used and the operators modified to support domain specific constraints.
	\item The selection process must be balanced between random selection and greedy selection to biased towards fitter candidate solutions (exploitation), whilst promoting useful diversity into the population (exploration). 
	\item Recombination is intended to create new candidate solutions with the favorable attributes of both parents, and as such should attempt to preserve patterns (sub-sequences and/or patterns) of both parents.
	\item Mutation is a low frequency event determined probabilistically for each position in the genome (locus) where the optimal probability is proposed to be 1.0/L where L is the length of the binary string.
	\item Population sizes must be large enough to provide a sufficiently diverse pool of building blocks for the process to be effective, typically equal to the number bits in a candidate solution or some function thereof.
\end{itemize}

\subsection{Further Reading}
The following references provide an introduction in the genetic algorithm and how to configure it.

\begin{itemize}
	\item none
\end{itemize}


\subsection{Tutorial}
This tutorial covers how to implement a genetic algorithm that operates on candidate solutions with binary genomes applied to solve a real-valued function optimization problem.

\subsubsection{Problem}
The problem is a function optimization problem called F1 and involves finding a set of parameters for the function that minimize the evaluation of the function. The function is flexible, allowing the number of parameters for the function to be varied. An increase in parameters is intended to represent an increase in the size of the search space (geometrical increase in search space volume) and as such represent an increase in the difficulty of the problem.


\begin{lstlisting}[name=ga-problem, caption={F1 Problem}, label=ga-problem] 
class F1Problem
  attr_reader :dimensions, :min, :max
  
  def initialize(dimensions=2)
    @dimensions = dimensions
    @min, @max = -5.12, +5.12
  end
  
  def num_bits
    @dimensions * bits_per_param
  end
  
  def bits_per_param
    16
  end
  
  def evaluate(vector)
    vector.inject(0) {|sum, x| sum +  (x ** 2.0)}
  end  
  
  def is_optimal?(scoring)
    scoring == optimal_score
  end
  
  def optimal_score
    0.0
  end
  
  def maximizing?
    false
  end
end
\end{lstlisting}

\subsubsection{Solution}

\begin{lstlisting}[name=ga-solution] 
class BinarySolution
  attr_reader :bitstring, :objective_params 
  attr_accessor :fitness
  
  def initialize(bitstring=nil)
    @bitstring, @fitness = bitstring, 0.0/0.0
  end
  
  def self.random_solution(problem)
    BinarySolution.new(Array.new(problem.num_bits) {|i| (rand<0.5) ? "1" : "0"})
  end
  
  def evaluate(problem)
    @fitness = problem.cost(phenotype(problem)) if @fitness.nan?
  end
  
  def phenotype(problem)
    if @object_params.nil? 
      @object_params = Array.new(problem.dimensions) do |i| 
        s, e = (i * problem.bits_per_param), ((i+1) * problem.bits_per_param)
        BinarySolution.bcd(@bitstring[s...e], problem.min, problem.max)
      end
    end
    return @object_params
  end  
  
  def self.bcd(bitstring, min, max)
    sum = 0.0
    bitstring.each_with_index do |x, i|
      sum += ((x=='1') ? 1.0 : 0.0) * (2.0 ** i.to_f)
    end
    # rescale [0,2**L-1] to [min,max]
    return min + ((max-min) / ((2.0**bitstring.length.to_f) - 1.0)) * sum
  end
  
  def recombine(mrate, other=nil)
    cut = other.nil? ? @bitstring.length : rand(@bitstring.length - 2) + 1
    offspring = Array.new(@bitstring.length) do |i| 
      if (i<cut) 
        transcribe(@bitstring[i], mrate) 
      else
        transcribe(other.bitstring[i], mrate) 
      end
    end
    BinarySolution.new(offspring)
  end

  def transcribe(value, mrate)
    (rand < mrate) ? ((value == "1") ? "0" : "1" ) : value
  end
  
  def to_s
    "[#{@bitstring.collect{|x|x}}] (#{@fitness})"
  end
  
  def is_better?(problem, other)
    problem.maximizing? ? @fitness>other.fitness : @fitness<other.fitness
  end      
end
\end{lstlisting}



\subsubsection{Algorithm}

\begin{lstlisting}[name=ga-algorithm]
class GeneticAlgorithm
  attr_reader :config
  
  def initialize
    @config = {}
  end
  
  def configure(problem, max_gens)    
    @config[:mutation] = 1.0 / problem.num_bits.to_f
    @config[:crossover] = 0.98
    @config[:pop_size] = problem.num_bits * 2
    @config[:num_bouts] = 3
    @config[:max_gens] = max_gens
  end
  
  def evolve(problem)    
    pop = Array.new(@config[:pop_size]) do |i| 
      BinarySolution.random_solution(problem)
    end
    solution = pop[0]; gen = 0
    begin
      pop.each do |sol| 
        sol.evaluate(problem) 
        solution = sol if sol.is_better?(problem, solution)
      end
      evolve_population(pop, problem)
      gen += 1
      puts "#{gen}, score:#{solution.fitness}"
    end until gen>=@config[:max_gens] or solution.fitness==problem.optimal_score
    return solution
  end
  
  def evolve_population(pop, problem)
    selected = pop.collect {|sol| tournament_select(sol, pop, problem)}
    selected.each_with_index do |sol, i|
      if (rand < @config[:crossover]) 
        other = (i.modulo(2)==0) ? selected[i+1] : selected[i-1]
        pop[i] = sol.recombine(@config[:mutation], other)
      else
        pop[i] = sol.recombine(@config[:mutation])
      end
    end
  end
  
  def tournament_select(base, pop, problem)
    @config[:num_bouts].times do    
      other = pop[rand(pop.length)]
      base = other if other.is_better?(problem, base)
    end
    return base
  end
end
\end{lstlisting}


\subsubsection{Running It}

\begin{lstlisting}[name=ga-algorithm]
# run it
seed = Time.now.to_f and srand(seed)
puts "Random seed: #{seed}"
problem = F1Problem.new(5)
algorithm = GeneticAlgorithm.new
algorithm.configure(problem, 1000)
solution = algorithm.evolve(problem)
puts "Finished, solution:#{solution}, optimal:#{problem.optimal_score}"
\end{lstlisting}


\subsubsection{Extensions}