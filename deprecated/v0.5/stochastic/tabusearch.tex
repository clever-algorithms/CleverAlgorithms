\section{Tabu Search}
A problem with local search techniques is that after locating a local optima, they will continue to re-sample the same optima and neighboring solutions in the search space over and over again until a stopping condition is triggered. \textbf{Tabu Search} is an algorithm to address this problem by adding memory to a search algorithm, preventing or penalizing the algorithm for re-sampling (cycling) the same solutions and in turn encouraging further exploration of the search space.

The core search algorithm embedded within a tabu search generates candidate solutions in the neighborhood of a current working solution. A \emph{tabu list} is maintained that holds a set of recently visited neighboring solutions (or components thereof) that may not be visited again (they are taboo). Moves are held in the tabu list for a defined \emph{tabu tenure} that is typically fixed, proportional to the problem size, or generated randomly within a pre-defined range.

The tabu list is referred to as the \emph{short term memory} of a search algorithm. A searches \emph{intermediate term memory} refer to the factors that direct the process to promising areas (expected payoff) of the search space inferred from past experience, referred to as intensification. Examples of such intermediate memory processes include the inclusion of common components from recently visited good quality solutions, adapting the tabu tenures when locally optimal solutions are located, and restarting the search (clearing the tabu list) with a locally optimal solution. \emph{Long term memory} processes are used to counter the intensification of intermediate memory by encouraging exploration in the search space, referred to as diversification. Examples include the use of penalties per component that are increased with the frequency of inclusion, and changing the acceptance criteria to allow non-improving moves.

Finally, so-called aspiration criteria may be used to permit neighbourhood moves on the tabu list (overcome the tabu). A standard aspiration criteria is to accept a tabu move if is better than a the current working position or better than the best solution found by the search to date.

\subsection{Tabu Search for the Traveling Salesman Problem}
This guide applies the tabu search to a classical combinatorial optimization problem called the Traveling Salesman Problem (TSP). The general class of problem involves finding an order to visit a set of cities to minimize the overall distance traveled.

The \texttt{Berlin52TSP} class provides a definition of a TSP instance called Berlin 52 (with 52 nodes or cities). The class is large given that it has the city coordinates hard coded in the \texttt{COORDINATES} class constant, as well as the optimal permutation in the \texttt{OPTIMAL\_TOUR} constant. The \texttt{evaluate(permutation)} expects a valid permutation of 52 non-repeating cities to visit as an array of integers (between 1 and 52 inclusively) and is evaluated as the rounded Euclidean distance of the cycle.

\begin{lstlisting}
class Berlin52TSP
  OPTIMAL_TOUR = [1,49,32,45,19,41,8,9,10,43,33,51,11,52,14,13,47,26,
    27,28,12,25,4,6,15,5,24,48,38,37,40,39,36,35,34,44,46,16,29,50,20,
    23,30,2,7,42,21,17,3,18,31,22]
        
  COORDINATES = [[565, 575],[25, 185],[345, 750],[945, 685],[845, 655],
  [880, 660],[25, 230],[525, 1000],[580, 1175],[650, 1130],[1605, 620], 
  [1220, 580],[1465, 200],[1530, 5],[845, 680],[725, 370],[145, 665],
  [415, 635],[510, 875], [560, 365],[300, 465],[520, 585],[480, 415],
  [835, 625],[975, 580],[1215, 245],[1320, 315],[1250, 400],[660, 180],
  [410, 250],[420, 555],[575, 665],[1150, 1160],[700, 580],[685, 595],
  [685, 610],[770, 610],[795, 645],[720, 635],[760, 650],[475, 960],
  [95, 260],[875, 920],[700, 500],[555, 815],[830, 485],[1170, 65],
  [830, 610],[605, 625],[595, 360],[1340, 725],[1740, 245]]
  
  attr_reader :num_cities

  def initialize()
    @num_cities = COORDINATES.length        
    @optimal_tour_length = evaluate(OPTIMAL_TOUR) # calculate
  end
  
  def evaluate(permutation)
    dist = 0    
    permutation.each_with_index do |c1, i|
      c2 = (i==@num_cities-1) ? permutation[0] : permutation[i+1] 
      dist += euc_2d(COORDINATES[c1-1], COORDINATES[c2-1])
    end
    return dist
  end
  
  def euc_2d(c1, c2)
    # As defined in TSPLIB'95 (EUC_2D)
    Math::sqrt((c1[0] - c2[0])**2 + (c1[1] - c2[1])**2).round
  end

  def is_optimal?(scoring)
    scoring == optimal_score
  end

  def optimal_score
    @optimal_tour_length
  end
  
  # true if s1 is better score than s2
  def is_better?(s1, s2)
    s1 < s2 # minimizing
  end
end
\end{lstlisting}

The \texttt{Solution} class provides a container for candidate solutions, with an immutable \texttt{@data} instance variable for problem specific solution representation and a mutable \texttt{@score} instance variable to hold its costing. The overridden \texttt{to\_s} method provides a pretty string representation of the solution for display on the command line. Important to this example is the overridden \texttt{==(other)} method that is used to test equality between solution objects. Here, the method is specialized assuming the \texttt{@data} holds an array, and compares two \texttt{Solution} objects based on the equality of the permutations they hold. Permutation comparison is not dependent on the order the permutation is recorded, so equality is checked in both directly and with one of the permutations reversed. This equality method is used to assess whether a solution has already been added to the tabu list (later on). Finally, a mutable \texttt{@count} instance variable is maintained to use in the frequency a tabu solution is desired (later). 

\begin{lstlisting}
class Solution
  attr_reader :data
  attr_accessor :score, :count
  
  def initialize(data)
    @data = data
    @score = 0.0/0.0 # NaN
    @count = 0
  end
  
  def ==(other)
    # permutation is direction independant
    @data == other.data or @data.reverse == other.data
  end
  
  def to_s
    "[#{@data.inspect}] (#{@score})"
  end    
end
\end{lstlisting}

The \texttt{TabuSearchAlgorithm} class offers an implementation of a iterative 2-opt search algorithm with a tabu list and aspiration criteria for a given TSP instance. The \texttt{initialize(max\_iterations)} constructor records the maximum number of iterations the algorithm will execute and sets additional tabu search specific configuration. The \texttt{@tabu\_tenure} instance variable defines the the size of the tabu list under the assumption that one candidate is added to the list each iteration. The \texttt{@aspiration\_frequency} instance variable defines the maximum number of requests that may be made for a tabu solution on the tabu list before it may be reused (prematurely liberated from the tabu list).

The \texttt{search(problem)} provides the entry point for the search for a given TSP instance. A random starting point is used to initialize the search provided by the \texttt{generate\_initial\_solution(problem)} method. The algorithm loop generally involves generating neighbouring candidate solutions one at a time, evaluating a solution, and using it as the current working position greedily if it has an improved problem specific scoring. Neighboring candidate solutions are generated by calling the \texttt{two\_opt\_solution(solution)} method for a given current working point. This standard local search for the TSP was chosen for its speed, effectiveness, and simplicity and involves selecting two break points in a permutation and reversing the sub-sequence between the points.

The generation of neighboring candidate solutions loops until a candidate solution is generated that is not already on the tabu list. Importantly, this loop increments the number of times each tabu solution is requested, liberating a specific candidate solution if it's request frequency exceeds \texttt{@aspiration\_frequency}, providing a basic aspiration criteria. The tabu list is maintained at the bottom of the main algorithm loop, first adding the current candidate solution regardless of whether it is accepted or not, and then whittling down the \texttt{tabu\_list} to \texttt{@tabu\_tenure} when required (treating the \texttt{tabu\_list} array like a first-in-last-out FILO queue).

\begin{lstlisting}
class TabuSearchAlgorithm
  attr_accessor :max_iterations
  attr_reader :best_solution
  
  def initialize(max_iterations)
    @max_iterations = max_iterations
    @tabu_tenure = 100 # maximum number of solutions on the list
    @aspiration_frequency = 5 # max tries to visit a taboo before allowing
  end
  
  # execute a search on the provided problem
  def search(problem)
    tabu_list = [] # prep the tabu list
    # random starting point
    current = generate_initial_solution(problem)
    tabu_list << current # make candidate taboo
    evaluate_candidate(current, problem)
    curr_it = 0       
    begin
      # generate new neighbour that is not tabo
      selection_ok = true
      begin
        selection_ok = true # optimism
        candidate = two_opt_solution(current) # generate
        # check tabu
        if tabu_list.include?(candidate)
          other = tabu_list.find {|t| t==candidate}
          if (other.count+=1) >= @aspiration_frequency
            puts "> we have an aspiration!!!"
            tabu_list.delete_if {|t| t==candidate} # ensure no duplicates
          else
            puts "> rejected tabu"
            selection_ok = false 
          end
        end
      end until selection_ok
      evaluate_candidate(candidate, problem) # evaluate
      # manage short term memory
      tabu_list << candidate # make candidate taboo
      tabu_list.delete_at(0) while tabu_list.length > @tabu_tenure
      # greedy acceptance
      current = candidate if problem.is_better?(candidate.score, current.score)
      curr_it += 1
    end until should_stop?(curr_it, problem)
    return @best_solution
  end
  
  def should_stop?(curr_it, problem)
    (curr_it >= @max_iterations) or problem.is_optimal?(best_solution.score)
  end
  
  def generate_initial_solution(problem)
    all = Array.new(problem.num_cities) {|i| (i+1)}
    permutation = Array.new(all.length) {|i| all.delete_at(rand(all.length))}
    return Solution.new(permutation)
  end

  def two_opt_solution(solution)
    perm = Array.new(solution.data) # copy
    # select a sub-sequence    
    c1, c2 = rand(perm.length), rand(perm.length)
    c2 = rand(perm.length) while c1 == c2
    # ensure c1 is low and c2 is high
    c1, c2 = c2, c1 if c2 < c1
    # reverse sub-sequence
    perm[c1...c2] = perm[c1...c2].reverse
    return Solution.new(perm)
  end

  def evaluate_candidate(solution, problem)
    solution.score = problem.evaluate(solution.data)
    # keep track of the best solution found
    if @best_solution.nil? or 
      problem.is_better?(solution.score, @best_solution.score)
      @best_solution = solution
      puts " > new best: #{solution.score}"               
    end
  end
end
\end{lstlisting}

The algorithm is demonstrated by first initializing the global random number generator to 1 to ensure the same sequence of random numbers on each execution (useful for testing). The problem is created and algorithm is initialized with a maximum of 2000 iterations. The search is executed by passing the problem instance to the algorithms search function, and after the execution of the run the best solution is displayed.

\begin{lstlisting}
srand(1) # set the random number seed to 1
algorithm = TabuSearchAlgorithm.new(2000) # limit to 2000 iterations 
problem = Berlin52TSP.new # create a problem
best = algorithm.search(problem) # execute the search
puts "Best Solution: #{best}" # display the best solution
\end{lstlisting}

During the execution of the algorithm, the scoring of the best solution is displayed each time a new best is located. The loop of the main algorithm displays a message each time a tabu solutions aspiration criteria is triggered or when a generated candidate solution is denied based on its existence in the tabu list. You will observe that as the search gets closer to the global optimum, the number of candidates denied increases and a number of liberation events occur as the aspiration criteria are triggered.

There is plenty of room for extension on this example. One may change the configuration of the tabu search, increasing or decreasing both the tabu tenure as well as the aspiration frequency. Also experimentation of the search algorithm with and without a tabu list may be interesting to see if and when the short term memory of this tabu search example are useful. Alternative representations may be used for the tabu list such as the selection of cities to swap in the local search. Alternatively, the local search may be adjusted to construct tours in a step-wise (per-city) manner and maintain tabu information per-edge in an adjacency matrix. This matrix would then discourage or prevent the use of specific edges during solution construction. Finally, additional memory processes may be introduced such as search-restarts with the best found solution for intermediate memory, and the solution acceptance criteria may be made less greedy to allow non-improving moves to be taken.

\subsection{Further Reading}
This section lists some additional resources.

\begin{itemize}
	\item Fred Glover and Manuel Laguna, Tabu Search. Springer, 1997
	\item F. Glover, Tabu Search: Part I. INFORMS Journal on Computing 1(3), pages 190-206, 1989
	\item F. Glover, Tabu Search: Part II. INFORMS Journal on Computing 2(1), pages 4-32, 1990
	\item Fred Glover and Eric Taillard, A user's guide to tabu search. Annals of Operations Research 41(1), pages 1-28, 1993
	\item F. Glover, Tabu Search: A Tutorial. INTERFACES 20(4), pages 74-94, 1990
	\item Tabu Search Applied to TSP. In Johannes Josef Schneider and Scott Kirkpatrick, Stochastic Optimization, Springer, pages 441-447. 2006
\end{itemize}